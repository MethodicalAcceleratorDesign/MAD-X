%\documentclass[a4paper,11pt]{article}
%%\usepackage{ulem}
%%\usepackage{a4wide}
%%\usepackage[dvipsnames,svgnames]{xcolor}
%%\usepackage[pdftex]{graphicx}
%%\title{PTC\_PRINTFRAMES}
%%\usepackage{hyperref}
% commands generated by html2latex


%%\begin{document}
%%\begin{center}
   %%EUROPEAN ORGANIZATION FOR NUCLEAR RESEARCH   
%%\includegraphics{http://cern.ch/madx/icons/mx7_25.gif}

\section{PTC\_PRINTFRAMES}
%%\end{center}
%   ##########################################################              

%   ##########################################################              

%   ##########################################################              

%   ##########################################################              


\subsubsection{   USER MANUAL   }
%   ##########################################################              


\paragraph{SYNOPSIS}
\begin{verbatim}

PTC_PRINTFRAMES, 
file = [s, none] ,
format = [s, text] ; 

\end{verbatim}
%   ##########################################################              


\textbf{ Description }\\

  Print to a specified file PTC geometry of a layout.   \\

\textbf{ Example }\\

\href{http://cern.ch/frs/mad-X_examples/ptc_madx_interface/eplacement/eplacement.madx}{   Dog leg chicane } with some elements displaced with help of ptc\_eplacement:   
%   ##########################################################              

\paragraph{ Command parameters and switches }
\begin{description}
	\item[\textbf{ file }] \textit{ string, }

 Specifies name of the file.   
	\item[\textbf{ format }] \textit{ string, default "text"}

 Format of geometry. Currently two formats are accepted:       
\begin{description}
	\item[ text ] 

 Prints a simple text file.           
	\item[ rootmacro ] 

 Creates \href{http://root.cern.ch}{ root } macro that produces 3D display of the geometry.           
\end{description}
\end{description}

\paragraph{ \_\_\_\_  }
%  ############################################################ 

%  ############################################################ 

%  ############################################################ 

% 
% <h3> PROGRAMMERS MANUAL </h3>
% 
% <p> 
% The command is implemented pro_ptc_knob function in madxn.c and 
% by subroutine xxxx in madx_ptc_xxx.f90.
% <p>
% Sopecified range is resolved with help of get_range command. Number of the element in the current sequence
% is resolved and passed as the parameter to the fortran routine. It allows to resolve uniquely the corresponding
% element in the PTC layout.
% <p>
% 
% 


%%\end{document}
