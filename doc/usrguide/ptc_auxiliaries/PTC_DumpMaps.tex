%\documentclass[a4paper,11pt]{article}
%%\usepackage{ulem}
%%\usepackage{a4wide}
%%\usepackage[dvipsnames,svgnames]{xcolor}
%%\usepackage[pdftex]{graphicx}
%%\title{PTC\_DumpMaps}
%%\usepackage{hyperref}
% commands generated by html2latex


%%\begin{document}
%%\begin{center}
   %%EUROPEAN ORGANIZATION FOR NUCLEAR RESEARCH   
%%\includegraphics{http://cern.ch/madx/icons/mx7_25.gif}

\section{PTC\_DUMPMAPS}
%%\end{center}
%   ##########################################################              

%   ##########################################################              

%   ##########################################################              

%   ##########################################################              


\subsubsection{   USER MANUAL   }
%   ##########################################################              


\paragraph{SYNOPSIS}
\begin{verbatim}

PTC_DUMPMAPS, 
file     = [s, ptcmaps, ptcmaps];
\end{verbatim}
  implemented by subroutine ptc\_dumpmaps() in madx\_ptc\_module.f90   
%   ##########################################################              


\paragraph{ Description } PTC\_DUMPMAPS dumps linear part of the map for each element  of the layout into specified file.  
%   ##########################################################              


\paragraph{ Command parameters and switches }
\begin{description}
	\item[\textbf{file}] \textit{ string, default value "ptcmaps",  default value if value explicitly not specified is "ptcmaps"}

 Specifies name of the file to which the matrices are dumped to.   
\end{description}
%  ############################################################ 

%  ############################################################ 

%  ############################################################ 


\subsubsection{ PROGRAMMERS MANUAL }   The command is implemented by subroutine ptc\_dumpmaps() in madx\_ptc\_module.f90. The matrix for a single element is obtained by tracking identity map through an element, that is initialized for each element by adding identity map to the reference particle. For the elements that change reference momentum (i.e. traveling wave cavity)  it is tracked to the end of the following marker, that has updated reference momentum. Hence, each cavity must be followed by a marker. If it is not, setcavities subroutine detects error and stops the program.  

%%\end{document}
