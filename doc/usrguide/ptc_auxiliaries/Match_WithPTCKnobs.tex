%\documentclass[a4paper,11pt]{article}
%%\usepackage{ulem}
%%\usepackage{a4wide}
%%\usepackage[dvipsnames,svgnames]{xcolor}
%%\usepackage[pdftex]{graphicx}
%%\title{Match using PTC knobs}
%%\usepackage{hyperref}
% commands generated by html2latex


%%\begin{document}
%%\begin{center}
   %%EUROPEAN ORGANIZATION FOR NUCLEAR RESEARCH   
%%\includegraphics{http://cern.ch/madx/icons/mx7_25.gif}

\section{Matching with PTC knobs}
%%\end{center}
%   ##########################################################              

%   ##########################################################              

%   ##########################################################              

%   ##########################################################              


\subsubsection{   USER MANUAL   }    This matching procedure takes advantage of the parametric results that  are accessible with PTC. Namely,  parameters occuring in the matching constrains are obtained as functions (polynomials) of the matching variables. In other words, each variable is a knob in PTC calculation. Evaluation of the polynomials is relatively fast comparing to the regular PTC calculation what makes findining the minimum with the parametrized constraints very fast. 

 However, the algorithm is not faster in a general case: 
\begin{enumerate}
	\item  The calculation time dramatically increases with number of parameters and at some point penalty rising from       this overcomes the gain we get from the fast polynomial evaluation.   
	\item  A parametric result is an approximation that is valid only around        the nominal parameter values.    
\end{enumerate}

 The algorithm: \\
 
  1. Buffer the key commands (ptc\_varyknob, constraint, ptc\_setswitch, ptc\_twiss or ptc\_normal, etc)  appearing between  match, useptcknobs=true; and any of matching actions calls (migrad,lmdif,jacobian, etc)
  
  2.  When matching action appears,  
\\
\begin{quotation}
      a) set "The Current Variables Values" (TCVV) to zero
      
      b) perform THE LOOP, i.e. points 3-17 
\\
\end{quotation}
 3.  Prepare PTC environment (ptc\_createuniverse, ptc\_createlayout) 
\\ 4.  Set the user defined knobs (with ptc\_knob). 
\\ 5.  Set TCVV using ptc\_setfieldcomp command 
\\ 6.  Run a PTC command (twiss or normal) 
\\ 7.  Run a runtime created script that performs a standard matching;     all the user defined knobs are variables of this matching. 
\\ 8.  Evaluate constraints expressions to get the matching function vector (I) 
\\ 9.  Add the matched values to TCVV 
\\ 10. End PTC session (run ptc\_end) 
\\ 11. If the matched values are not close enough to zeroes then goto 3 
\\ 12. Prepare PTC environment (ptc\_createuniverse, ptc\_createlayout) 
\\ 13. Set TCVV using ptc\_setfieldcomp command 
\\   ( --- please note that knobs are not set in this case  -- ) 
\\ 14. Run a PTC command (twiss or normal)  
\\ 15. Evaluate constraints expressions to get the matching function vector (II) 
\\ 16. Evaluate a penalty function that compares matching function vectors (I) and (II)     See points 7 and 14 
\\ 17  If the matching function vectors are not similar to each other      within requested precision then goto 3 
\\ 18. Print TCVV, which are the matched values. 
\\
%   ##########################################################              


\paragraph{SYNOPSIS}
\begin{verbatim}

MATCH, use_ptcknobs=true;

PTC_VARYKNOB: 
  initial = [s, none] , 
  element = [s, none] , 
  kn    = [i, -1], 
  ks    = [i, -1], 
  exactmatch = [l, true, true], 
  trustrange    = [r, 0.1],  
  step     = [r, 0.0], 
  lower    = [r, -1.e20],
  upper    = [r,  1.e20]; 

END_MATCH;
\end{verbatim}

  For the user convenience the limits are specified in the MAD-X units (k1,k2, etc). This also applies to dipolar field where the user must specify limits of k0=angle/path\_lengh.  This guarantees concistency in treatment of normal and skew dipol components.  

 Important: Note that inside the code skew magnets are represented only by  normal component and tilt, so the nominal skew component is always zero.  Inside PTC tilt can not become a knob, while skew component can.  Remember about this fact when setting the limits of skew components in the matching.  When the final results are exported back to MAD-X, they are converted back to the "normal" state, so the nominal skew compoment is zero and tilt and  normal component are modified accordingly.    

 trustrange - defines the range the expansion is trusted \\
%   ##########################################################              

\textbf{ Description }\\



\textbf{ Example }\\
\href{http://cern.ch/frs/mad-X_examples/ptc_madx_interface/matchknobs/.madx}{  dog leg chicane }:   
%   ##########################################################              

%  ############################################################ 

%  ############################################################ 

%  ############################################################ 

% 
% <h3> PROGRAMMERS MANUAL </h3>
% 
% <p> 
% The command is implemented pro_PTC_SETKNOBVALUE function in madxn.c and 
% 
% 


%%\end{document}
