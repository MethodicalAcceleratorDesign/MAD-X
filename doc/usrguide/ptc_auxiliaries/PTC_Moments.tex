%%\title{PTC\_MOMENTS}

\section{PTC\_MOMENTS}

\begin{verbatim}
PTC_MOMENTS, 
    no = [i, 1], 
    xdistr   = [s, gauss, gauss], 
    ydistr   = [s, gauss, gauss], 
    zdistr   = [s, gauss, gauss], 
\end{verbatim}

Calculates moments previously selected with the
\href{PTC_SelectMoment.html}{ptc\_select\_moment} command.  It uses maps
saved by the ptc\_twiss command, hence, the savemaps switch of
ptc\_twiss must be set to true (default) to be able to calculate
moments.  \\ 

{\bf  Command parameters and switches }
\begin{itemize}
   \item {\bf no}=integer (Default: 1)\\
     order of the calculation, maximally twice the order of the last
     twiss.
   
   \item {\bf xdistr, ydistr, zdistr}=string (Default: gauss)\\  
     defines the distribution in x, y and z dimension respectively 
     \begin{enumerate}
	\item {\bf gauss} - Gaussian
	\item {\bf flat5} - flat distribution in the first of
          variables (dp over p) of a given dimension and Delta Dirac in
          the second one (T)  
	\item {\bf flat56} - flat rectangular distribution 
     \end{enumerate}
\end{itemize}

{\bf Examples}\\
\href{http://cern.ch/frs/mad-X_examples/ptc_madx_interface/moments/moments.madx}{ATF2}


% <h3> PROGRAMMERS MANUAL </h3>
% 
% <p> 
% The command is implemented pro_ptc_SELECT function in madxn.c and 
% by subroutine xxxx in madx_ptc_xxx.f90.
% <p>
% Sopecified range is resolved with help of get_range command. Number of the element in the current sequence
% is resolved and passed as the parameter to the fortran routine. It allows to resolve uniquely the corresponding
% element in the PTC layout.
% <p>
