\chapter{File Handling Statements}
\label{chap:files}

\section{ASSIGN}
\label{sec:assign}

\begin{verbatim}
assign, echo="file_name", truncate;
\end{verbatim} 
where "file\_name" is the name of an output file, or "terminal" and
truncate specifies if the file must be truncated when opened (ignored
for terminal). This allows switching the echo stream to a file or back,
but only for the commands value, show, and print. Allows easy
composition of future MAD-X input files. A real life example (always the
same) is to be found under \href{foot.html}{footprint example}.  

\section{CALL}
\label{sec:call}
\begin{verbatim}
call, file = "file_name";
\end{verbatim} 
where "file\_name"  is the name of an input file. This file will be read
until a "return;" statement, or until end\_of\_file; it may contain any
number of calls itself, and so on to any depth.  



\section{PRINT}
\label{sec:print}
\begin{verbatim}
print, text = "string";
\end{verbatim} 
prints the quoted text string to the current output file (see ASSIGN
above). The text can be edited with the help of a
\href{special.html#macro}{macro statement}. For more details, see
there. [TBC]  


\section{PRINTF}
\label{sec:printf}
\begin{verbatim}
printf, text = "string", value = expr{,expr};
\end{verbatim}
prints the numerical values specified in the value field, formatted
according to the string format provided in the text field, to the
current output file.  
The string format can take numeric C or MAD format specifiers for double
real values; integer and string formats are not supported. 
The maximum number of values that can be printed in one
statement is limited to 100. 

If the number of format specifiers given in the string is higher 
than the number of values in the value field, undefined values are printed 
where they are not explicitly provided. 
If the number of format specifiers given in the string is lower 
than the number of values in the value field, the values that 
do not correspond to a format specifier are ignored. 


Example:
\begin{verbatim}
a = 1.2; b = 3.4/0.3; c := 0.8*a/b;
printf, text = "String with floats a=%f, b=%.3g, text and MAD float c=%F;", 
        value = a, b, c;
printf, text = "More specifiers than values: %f, %.3g, %F", value = a, b;
printf, text = "More values than specifiers: %f, %.3g"    , value = a, b, c;
\end{verbatim}
produces the following output:
\begin{verbatim}
String with floats a=1.200000, b=11.3, text and MAD float c=     0.08470588235;
More specifiers than values: 1.200000, 11.3,   6.953222976e-310
More values than specifiers: 1.200000, 11.3
\end{verbatim}

Note that printf, like print, produces output that cannot be read back
by MAD-X. For output that can be read back by MAD-X, use the command
{\tt VALUE} or TFS tables.

Note also that a percent sign (\%) can be printed using the format
\verb|text="%%"|. 


\section{REMOVEFILE}
\label{sec:removefile}
\begin{verbatim}
removefile, file = "file_name";
\end{verbatim} 
remove the file "file\_name" from disk. It is more portable than  
\begin{verbatim}
system("rm filename"); // Unix specific
\end{verbatim}

\section{RENAMEFILE}
\label{sec:renamefile}
\begin{verbatim}
renamefile, file = "file_name", name = "new_file_name";
\end{verbatim} 
rename the file "file\_name" to "new\_file\_name" on the disk. It is more
portable than  
\begin{verbatim}
system("mv file_name new_file_name"); // Unix specific
\end{verbatim}


\section{RETURN}
\label{Sec:return}
\begin{verbatim}
return;
\end{verbatim} 
ends reading from a "called" file; if encountered in the standard input
file, it ends the program execution.  

%% EOF
