%\documentclass[a4paper,11pt]{article}
%%\usepackage{ulem}
%%\usepackage{a4wide}
%%\usepackage[dvipsnames,svgnames]{xcolor}
%%\usepackage[pdftex]{graphicx}
%%\title{Twiss Module}
%  Changed by: Chris ISELIN, 27-Jan-1997 

%  Changed by: Frank Schmidt, 11-Jul-2002 

%  Changed by: Hans Grote, 15-Jan-2003 

%  Changed by: Frank Schmidt, 06-APR-2003 

%  IMG ISMAP SRC="http://cern.ch/Frank.Schmidt/dynap/icons/at_work.gif"height=90 Under construction and not yet reliable!!!!

%%\usepackage{hyperref}
% commands generated by html2latex


%%\begin{document} %%EUROPEAN ORGANIZATION FOR NUCLEAR RESEARCH 
%%\includegraphics{http://cern.ch/madx/icons/mx7_25.gif}

\subsection{Twiss Module}
% p

% H4

% IMG ISMAP SRC="http://cern.ch/Frank.Schmidt/dynap/icons/at_work.gif"height=90

% Under construction and not yet reliable!!!!

% /A

% /H4

% /P
 The TWISS command causes computation of the \href{../Introduction/bibliography.html#courant}{[Courant and Snyder]}\href{../Introduction/tables.html#linear}{linear lattice functions}, and optionally of the \href{../Introduction/tables.html#chrom}{chromatic functions}. The coupled functions are calculated in the sense of \href{../Introduction/bibliography.html#edwards}{[Edwards and Teng]}. For the uncoupled cases they reduce to the C and S functions. It operates on the working beam line defined in the latest \href{../control/general.html#use}{USE} command. One can also specify either a \href{../Introduction/sequence.html}{SEQUENCE}="sequence\_name" or a \href{../Introduction/line.html}{LINE}="line\_name" on the TWISS command. Moreover, one can restrict the TWISS calculation to a desired \href{../Introduction/ranges.html#range}{RANGE}. 

The relative energy error DELTAP may be entered in one of the 2 forms 
\begin{verbatim}
DELTAP=real{,real}
DELTAP=initial:final:step
\end{verbatim} The first form lists several numbers, which may be general expressions, separated by commas. The second form specifies an initial value, a final value, and a step, which must be constant expressions, separated by colons. 

Examples: 
\begin{verbatim}
DELTAP=0.001                    ! a single value
DELTAP=0.001,0.005              ! two values
DELTAP=0.001:0.007:0.002        ! four values
\end{verbatim} If DELTAP is missing, MAD-X uses the value 0.0. 

Further attributes of the TWISS statements are: 
\begin{itemize}
	\item CHROM: A logical flag. If set, MAD-X also computes the     \href{../Introduction/tables.html#chrom}{chromatic     functions}. 

\textit{Please note that this option is needed for a proper calculation of the chromaticities in the presence of coupling!}
	\item FILE: If FILE="file\_name" appears MAD-X writes a full TFS Twiss table \href{../Introduction/select.html#tfs}{Example TFS Twiss table} on the disk file "file\_name". FILE alone is equivalent to FILE="twiss": 
	\item TABLE (overrides SAVE): MAD-X creates a full \href{../Introduction/tables.html#linear}{Twiss table} in memory and gives it the name TWISS, unless TABLE="table\_name" appears on the command, then it is called \href{../Introduction/label.html}{table\_name}. This table includes linear lattice functions as well as the chromatic functions for all positions. An important new feature of MAD-X is the possibility to access entries of tables and in particular the twiss table (see \href{../Introduction/expression.html#table}{table access}).   
	\item CENTRE: This flag enforces the calculation of the \href{../Introduction/tables.html#linear}{linear lattice functions} at the center of the element instead of the end of it. \textit{ Mind you that since this is inside the element the closed orbit includes the misalignment of the element.}
	\item RMATRIX: If this flag is used the the one-turn map at the location of every element is calculated and prepared for storage in the TWISS table. Using the \href{../Introduction/select.html}{SELECT} command and using the column RE, RE11...RE16...RE61...RE66 these components will be added to the TWISS table, i.e. with "column, RE" and "column, REij" one gets all or the component "ij" respectively.   
	\item SECTORMAP: This flag initiates the calculation of a sector map as described at: \href{../Introduction/sectormap.html}{SECTORMAP}.   
	\item SECTORFILE: Used to write SECTORMAPs to the file SECTORFILE="file\_name", if missing the output of SECTORMAP will go to the file "sectormap" with the format as found in \href{../Introduction/sectormap.html}{SECTORMAP}.   
	\item KEEPORBIT: The keeporbit attribute (with an optional name, keeporbit="name") stores the orbit under this name at the start, and at all monitors.   
	\item USEORBIT: The useorbit attribute (with an optional name, useorbit="name") uses the start value provided for the closed orbit search; the values at the monitors are used by the threader.   
	\item \textit{ COUPLE (obsolete)} : This MAD8 option can no longer be set since TWISS in MAD-X is always calculated in coupled mode. MAD-X computes the coupled functions in the sense of \href{../Introduction/bibliography.html#edwards}{[Edwards and Teng]}. For the uncoupled cases they reduce to the C and S functions.   
	\item \textit{ Twiss calculation is 4D only!} : The Twiss command will calculate an approximate 6D closed orbit when the accelerator structure includes an active \href{../Introduction/cavity.html}{cavity}. However, the calcuation of the Twiss parameters are 4D only. This may result in apparently non-closure of the beta values in the plane with non-zero dispersion. The full 6D Twiss parameters can be calculated with the \href{../ptc_twiss/ptc_twiss.html}{ptc\_twiss} command.   
	\item RIPKEN: This flags will calculate the Ripken-Mais TWISS parameters (beta11, beta12, beta21, beta22, alfa11, alfa12, alfa21, alfa22, gama11, gamma12, gamm21 and gamm22) using betx, bety, alfx, alfy, gamax, gamay, R11, r12, R21 and R22 as input. 
\end{itemize}
 The tables are suitable for \href{../plot/plot.html}{plot}.   After a successful TWISS run MAD-X creates an implicit \href{../Introduction/tables.html#summ}{table of summary parameters} named "summ" which includes tunes, chromaticities etc (Please note that the \href{../Introduction/tables.html#chrom}{chrom} option is needed  for a proper calculation of the chromaticities in the presence of coupling!) versus the selected values of DELTAP. Notice that in MAD-X DELTAP is converted in PT, which is used as longitudinal variable. Dispersive and chromatic functions are hence derivatives with respects to PT( see \href{../Introduction/tables.html#summ}{table}). These summary parameters can later be accessed via the \href{../Introduction/expression.html#table}{table access} function using the aforementionned implicit table named "summ". There is no way to change the name of this summary table. 

\subsection{\href{periodic}{Twiss Parameters for a Period}} The simplest form of the TWISS command is 
\begin{verbatim}
TWISS, DELTAP=real{,value},CHROM,
       TABLE=table_name;
\end{verbatim} It computes the periodic solution for the specified beam line for all values of DELTAP entered (or for DELTAP = 0, if none is entered). 

Example: 
\begin{verbatim}
USE,period=OCT;
TWISS,DELTAP=0.001,CHROM;
\end{verbatim} This example computes the periodic solution for the linear lattice and chromatic functions for the beam line OCT. The DELTAP value used is 0.001. Apart from saving computing time, it is equivalent to the command sequence 
\begin{verbatim}
RING: LINE=(4*(OCT,-OCT));
      USE,period=RING;
      TWISS,DELTAP=0.001,CHROM;
\end{verbatim}

\subsection{\href{line}{Initial Values from a Periodic Line}} It is possible to track the lattice functions starting with the periodic solution for another beam line. If this is desired the TWISS command takes the form 
\begin{verbatim}
TWISS, DELTAP=real{,value},LINE=beam-line,
       MUX=real,MUY=real,
       TABLE=table_name;
\end{verbatim} No other attributes should appear in the command. For each value of DELTAP MAD-X first searches for the periodic solution for the beam line mentioned in LINE=beam-line: The result is used as an initial condition for the lattice function tracking. 

Example: 
\begin{verbatim}
CELL:   LINE=(...);
INSERT: LINE=(...);
        USE,period=INSERT;
        TWISS,LINE=CELL,DELTAP=0.0:0.003:0.001,CHROM,FILE;
\end{verbatim} For four values of DELTAP the following happens: First MAD-X finds the periodic solution for the beam line CELL: Then it uses this solution as initial conditions for tracking the lattice functions of the beam line CELL: Output is also written on the file TWISS: 

If any of the beam lines was defined with formal arguments, actual arguments must be provided: 
\begin{verbatim}
CELL(SF,SD): LINE=(...);
INSERT(X):   LINE=(...);
             USE,period=INSERT;
             TWISS,LINE=CELL(SF1,SD1);
\end{verbatim}

\subsection{\href{initial}{Given Initial Values}} Initial values for \href{../Introduction/tables.html#linear}{linear lattice functions} and \href{../Introduction/tables.html#chrom}{chromatic functions} may also be numerical. Initial values can be specified on the TWISS command: 


\begin{verbatim}
TWISS,   BETX=real,ALFX=real,MUX=real,
         BETY=real,ALFY=real,MUY=real,
         DX=real,DPX=real,DY=real,DPY=real,
         X=real,PX=real,Y=real,PY=real,
         T=real,PT=real,
         WX=real,PHIX=real,DMUX=real,
         WY=real,PHIY=real,DMUY=real,
         DDX=real,DDY=real,DDPX=real,DDPY=real,
         R11=real,R12=real,R21=real,R22=real,  !coupling matrix
         TABLE=table_name,
         TOLERANCE=real,
         DELTAP=real:real:real;
\end{verbatim} All initial values for \href{../Introduction/tables.html#linear}{linear lattice functions} and \href{../Introduction/tables.html#chrom}{chromatic functions} are permitted, but BETX and BETY are required. Moreover, a \href{beta0}{beta0} block can be added as filled by the \href{../control/general.html#savebeta}{savebeta} command or see below. The lattice parameters are taken from this block, but will be overwritten by explicitly stated lattice parameters. As entered, the initial conditions cannot depend on DELTAP, and can thus be correct only for one such value. 

\subsection{\href{tolerance}{Tolerance}} This value defines the maximum closed orbit error of all six orbit components during the closed orbit search. The default value is 1.e-6. The value is only valid for the current twiss command; a permanent value can be entered via the \href{../control/general.html#coguess}{COGUESS} command. 

\subsection{\href{savebeta}{SAVEBETA: Save Lattice Parameters}} Initial lattice parameters can be transfered for later commands, in particular for twiss or the \href{../match/match.html}{match module}, by using the \href{../control/general.html#savebeta}{savebeta} command sequence. 

%\textit{ It should be mentioned that parameters can be also accessed from tables using the \href{../Introduction/expression.html#table}{table access} function.}
It should be mentioned that parameters can be also accessed from tables using the \href{../Introduction/expression.html#table}{table access} function.


\begin{verbatim}
USE,period=...;
SAVEBETA,LABEL=name,PLACE=place,SEQUENCE=s_name;
TWISS,...;
\end{verbatim} When reaching the \href{../control/general.html#place}{place} in the sequence "s\_name" during execution of TWISS, MAD-X will save a \hyperlink{beta0}{beta0} block with the \href{../Introduction/label.html}{label} name: This block is filled with the values of all lattice parameters in place. 

Example 1: 
\begin{verbatim}
USE,period=CELL;
SAVEBETA,LABEL=END,PLACE=#E,SEQUENCE=CELL;
TWISS;
USE,period=INSERT;
TWISS,BETA0=END;
\end{verbatim} This first example calculates the \hyperlink{periodic}{periodic solution} of the line CELL, and then track lattice parameters through INSERT, using all end conditions (including orbit) in CELL to start. 

Example 2: 
\begin{verbatim}

USE,period=CELL;
SAVEBETA,LABEL=END,PLACE=#E,SEQUENCE=CELL;
TWISS;
USE,period=INSERT;
TWISS,BETX=END-$>$BETY,BETY=END-$>$BETX;
\end{verbatim}

This is similar to the first example,but the beta functions are interchanged (overwritten).  

\subsection{\href{beta0cmd}{BETA0: Set Initial Lattice Parameters}} Initial lattice parameters can be set 'manually' for later commands, in particular for twiss or the \href{../match/match.html}{match module}, by using the BETA0 command attached to a label. 

Example 3: 
\begin{verbatim}

INITIAL: BETA0, BETX=BX0,ALFX=0.0,MUX=0.0,BETY=BY0,ALFY=0.0,DX=DX0,DPX=0.0;
TWISS,BETA0=INITIAL;
\end{verbatim}

Example 4: 
\begin{verbatim}

INITIAL: BETA0, BETX=BX0,ALFX=0.0,MUX=0.0,BETY=BY0,ALFY=0.0,DX=DX0,DPX=0.0;
TWISS,BETX=INITIAL-$>$BETY,BETY=INITIAL-$>$BETX;
\end{verbatim}

\line(1,0){300}

\href{http://cern.ch/Frank.Schmidt/frs_sign.html}{frs}, 06-Apr-2003. Revised in February 2007. 

%%\end{document}
