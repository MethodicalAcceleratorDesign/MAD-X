%%\title{SODD}
%  Changed by: E. T. d'Amico, 8-Sep-2004 

%%\usepackage{hyperref}
% commands generated by html2latex


%%\begin{document}
%%\begin{center}
 %%EUROPEAN ORGANIZATION FOR NUCLEAR RESEARCH 
%%\includegraphics{http://cern.ch/madx/icons/mx7_25.gif}

\subsection{SODD}
%%\end{center}

 This command will execute the Second Order Detuning and Distortion as described in the paper of J. Bengtsson and J. Irwin  "Analytical Calculation of Smear and Tune Shift " (SSC-232, February 1990), on the beam line defined by the last USE command followed by a TWISS command. It is based on the stand-alone program written by Frank Schmidt in November 1998 - January 1999 who also extended the analytical computation to the second order distortion (cfr. Beam Physics Note 60 F. Schmidt "SODD: A physics Guide").  It consists of three parts: 


%	\item 

\paragraph{Subroutine detune (launched by the attribute detune)} It calculates the detuning function terms in first and second order in the strength of the multipoles. If the attribute print\_at\_end has been set, the following two files  (and the corresponding madx tables) are created : 

\textit{detune\_1\_end} containing five columns : 

 1) 'multipole order', 2) '(hor., ver. plane => (1/2)',  3) 'hor. or ver. detuning', 4) 'order of horizontal invariant', 5) 'order of vertical invariant'. 


\textit{detune\_2\_end}  containing five columns : 

 1) 'first multipole order', 2) 'second multipole order',  3) 'horizontal detuning', 4) 'order of horizontal invariant', 5)'order of vertical invariant'.  
 If the attribute print\_all has been set, the following two files  (and the corresponding madx tables) are created : 

\textit{detune\_1\_all}  containing  five columns :  

 1) 'multipole order', 2) '(hor., ver. plane => (1/2)',  3) 'hor. or ver. detuning', 4) 'order of horizontal invariant', 5)'order of vertical invariant'. 
 
\textit{detune\_2\_all}  containing five columns :  

 1) 'first multipole order', 2) 'second multipole order',  3) 'horizontal detuning', 4) 'order of horizontal invariant', 5) 'order of vertical invariant'. 



%	\item 

\paragraph{Subroutine distort1 (launched by the attribute distort1)}  It calculates the distortion function and the Hamiltonian terms in first order in the strength of the multipoles. If the attribute print\_at\_end has been set, the two files  (and the corresponding madx tables) are created : 

\textit{distort\_1\_F\_end} containing eight columns : 

 1) 'multipole order', 2) 'cosine part of distortion', 3) 'sine part of distortion', 4) 'amplitude of distortion', 5) 'j', 6) 'k', 7) 'l', 8) 'm'. 

\textit{distort\_1\_H\_end}  containing eight columns : 

 1) 'multipole order', 2) 'cosine part of Hamiltonian', 3) 'sine part of Hamiltonian', 4) 'amplitude of Hamiltonian', 5) 'j', 6) 'k', 7) 'l', 8) 'm'.  
 If the attribute print\_all has been set, the following two files  (and the corresponding madx tables) are created : 

\textit{distort\_1\_F\_all} containing eleven columns :  

 1) 'multipole order', 2) 'appearance number in position range', 3) 'number of resonance', 4) 'position', 5) 'cosine part of distortion', 6) 'sine part of distortion', 7) 'amplitude of distortion', 8) 'j', 9) 'k', 10) 'l', 11) 'm'. 

\textit{distort\_1\_H\_all}  containing eleven columns : 

 1) 'multipole order', 2) 'appearance number in position range, 3) 'number of resonance', 4) 'position', 5) 'cosine part of Hamiltonian', 6) 'sine part of Hamiltonian', 7) 'amplitude of Hamiltonian', 8) 'j', 9) 'k', 10) 'l', 11) 'm'.  



%	\item 

\paragraph{Subroutine distort2 (launched by the attribute distort2)}  It calculates the distortion function and Hamiltonian terms in second order in the strength of the multipoles. If the attribute print\_at\_end has been set, the following two files  (and the corresponding madx tables) are created : 

\textit{distort\_2\_F\_end} containing nine columns : 

 1) 'first multipole order',2) 'second multipole order',  3) 'cosine part of distortion', 4) 'sine part of distortion', 5) 'amplitude of distortion', 6) 'j', 7) 'k', 8) 'l', 9) 'm'. 
 
\textit{distort\_2\_H\_end}  containing nine columns :  

 1) 'first multipole order', 2) 'second multipole order',  3) 'cosine part of Hamiltonian', 4) 'sine part of Hamiltonian', 5) 'amplitude of Hamiltonian', 6) 'j', 7) 'k', 8) 'l', 9) 'm'.  


 N. B. The first row of every file is a header containing the names of the columns. This row is absent in the internal tables. 


%	\item 

\paragraph{\href{sodd}{SODD}}
\begin{verbatim}

sodd,
detune=logical,
distort1=logical,
distort2=logical,
start_stop = start,stop
multipole_order_range = fist,last
noprint = logical
print_all = logical
print_at_end = logical
nosixtrack  = logical
\end{verbatim} where the parameters have the following meaning: 
\begin{itemize}
	\item detune : logical, default=false. If true, the detune subroutine is executed. 
	\item distort1 : logical, default=false. If true, the distort1 subroutine is executed. 
	\item distort2 : logical, default=false. If true, the distort2 subroutine is executed. 
	\item start\_stop : longitudinal interval of the beam line (in m). start and stop should be given as real numbers. 
	\item multipole\_order\_range : the lowest and the largest multipole order which will be taken in account. first and last should be given as integers. 
	\item noprint : logical, default=false. If true, no file or internal table will be created to keep the results. In this case the attributes print\_all or print\_at\_end have no effect. 
	\item print\_all : logical, default=false. If true, the files and internal tables containing results at each multipole will be generated. 
	\item print\_at\_end : logical, default=false. If true, the files and internal tables containing results at the end of the position range will be generated. 
	\item nosixtrack  : logical, default=false. If true, the input file fc.34 will not be generated internally by invoking the conversion routine of sixtrack and the user should provide it before the execution of the sodd command. 

	\item A more detailed description can be found in  \href{http://cern.ch/madx/doc/ab-note-2004-069}{AB-note-2004-069}\href{http://xwho.web.cern.ch/xwho/people/show/6175}{damico}, September 10, 2004 
\end{itemize}

%%\end{document}
