%%\title{Conventions}
%  Changed by: Chris ISELIN, 27-Mar-1997 
%  Changed by: Hans Grote, 25-Sep-2002 

\chapter{Conventions}

The accelerator and/or beam line to be studied is described as a
sequence of beam elements placed sequentially along a reference
orbit. The reference orbit is the path of a charged particle having the
central design momentum of the accelerator through idealised magnets
with no fringe fields (see \hyperlink{local}{Figure 1}). 


The reference orbit consists of a series of straight line segments and
circular arcs. It is defined under the assumption that all elements are
perfectly aligned. The accompanying tripod of the reference orbit spans
a local curvilinear right handed coordinate system \textit{(x,y,s)} The
local \textit{s}-axis is the tangent to the reference orbit. The two
other axes are perpendicular to the reference orbit and are labelled
\textit{x} (in the bend plane) and \textit{y} (perpendicular to the bend
plane).  


%% \begin{itemize}
%% 	\item \href{closed_orbit.html}{Closed Orbit}
%% 	\item \href{global_system.html}{Global Reference System}
%% 	\item \href{local_system.html}{Local Reference System}
%% 	\item \href{sign_convent.html}{Sign Conventions for Magnetic Fields}
%% 	\item \href{tables.html}{Variable}
%% \begin{itemize}
%% 	\item \href{tables.html#canon}{Canonical Variables Describing Orbits}
%% 	\item \href{tables.html#normal}{Normalised Variables and other Derived Quantities}
%% \end{itemize}
%% 	\item \href{mad_units.html}{Physical Units}
%% \end{itemize}


\begin{figure}[h!]
  \centering
	\includegraphics{figures/local_reference.png}
  \caption{\href{local}{Local Reference System}}%{\textbf{Figure 1:} Local Reference System}}
\end{figure}

%%\title{the mad program}
%  Changed by: Chris ISELIN, 24-Jan-1997 
%  Changed by: Hans Grote, 10-Jun-2002 

\section{Closed Orbit}

Due to various errors like misalignment errors, field errors, fringe
fields etc., the closed orbit does not coincide with the reference
orbit. It also changes with the momentum error. The closed orbit is
described with respect to the reference orbit, using the local
reference system (\textit{x, y, s}). It is evaluated including any
nonlinear effects.  

MAD also computes the betatron and synchrotron oscillations with respect
to the closed orbit. Results are given in the local (\textit{x, y,
  s})-system defined by the reference orbit. 


%\href{http://www.cern.ch/Hans.Grote/hansg_sign.html}{hansg}, January 24, 1997 



%%\title{Global Reference System}
%  Changed by: Chris ISELIN, 17-Jul-1997 
%  Changed by: Hans Grote, 10-Jun-2002 

\section{Global Reference System}

The \hyperlink{global}{global reference orbit} of the accelerator is
uniquely defined by the sequence of physical elements. The local
reference system (\textit{x}, \textit{y}, \textit{s}) may thus be
referred to a global Cartesian coordinate system (\textit{X},
\textit{Y}, \textit{Z}) (see \hyperlink{global}{Figure 1}). The
positions between beam elements are numbered 0,...,i,...n. The local
reference system  (\textit{x$_i$, y$_i$, s$_i$}) at position \textit{i},
i.e. the displacement and direction of the reference orbit with respect
to the system (\textit{X}, \textit{Y}, \textit{Z}) are defined by three
displacements  (\textit{X$_i$}, \textit{Y$_i$}, \textit{Z$_i$}) and
three angles (\textit{Theta$_i$}, \textit{Phi$_i$}, \textit{Psi$_i$})

The above quantities are defined more precisely as follows:  
\begin{itemize}
   \item X: Displacement of the local origin in \textit{X}-direction. 
   \item Y: Displacement of the local origin in \textit{Y}-direction. 
   \item Z: Displacement of the local origin in \textit{Z}-direction. 
   \item \href{theta}{THETA}: Angle of rotation (azimuth) about the
     global \textit{Y}-axis, between the global \textit{Z}-axis and the
     projection of the reference orbit onto the (\textit{Z},
     \textit{X})-plane. A positive angle THETA forms a right-hand screw
     with the \textit{Y}-axis. 
   \item \href{phi}{PHI}: Elevation angle, i.e. the angle between the
     reference orbit and its projection onto the (\textit{Z},
     \textit{X})-plane. A positive angle PHI correspond to increasing
     \textit{Y}. If only horizontal bends are present, the reference
     orbit remains in the (\textit{Z}, \textit{X})-plane. In this case
     PHI is always zero. 
   \item \href{psi}{PSI}: Roll angle about the local \textit{s}-axis,
     i.e. the angle between the intersection (\textit{x}, \textit{y})-
     and (\textit{Z}, \textit{X})-planes and the local
     \textit{x}-axis. A positive angle PSI forms a right-hand screw with
     the \textit{s}-axis. 
\end{itemize} 

The angles (THETA, PHI, PSI) are \textbf{not} the Euler angles. The
reference orbit starts at the origin and points by default in the
direction of the positive \textit{Z}-axis. The initial local axes
(\textit{x}, \textit{y}, \textit{s})  coincide with the global axes
(\textit{X}, \textit{Y}, \textit{Z}) in this order. The six quantities
(X$_0$, Y$_0$, Z$_0$, THETA$_0$, PHI$_0$, PSI$_0$) thus all have zero
initial values by default. The program user may however specify
different initial conditions.  

Internally the displacement is described by a vector \textit{V} and the
orientation by a unitary matrix \textit{W}. The column vectors of
\textit{W} are the unit vectors spanning  the local coordinate axes in
the order (\textit{x, y, s}). \textit{V} and \textit{W} have the values:  


%%\includegraphics{null}
%VW
\[
V =
 \begin{pmatrix}
  X \\
  Y \\
  Z
 \end{pmatrix}
, \quad\quad
W=\Theta\Phi\Psi
\]

 where 

%%\includegraphics{null}
%PhiThetaPsi
\[
\Theta =
 \begin{pmatrix}
  \cos \theta  & 0 &  \sin \theta \\
  0            & 1 &  0 \\
  -\sin \theta & 0 &  \cos \theta
 \end{pmatrix}
, \quad
\Phi =
 \begin{pmatrix}
  1 & 0          &  0 \\
  0 & \cos \phi  &  \sin \phi \\
  0 & -\sin \phi &  \cos \phi
 \end{pmatrix}
, \quad
\Psi =
 \begin{pmatrix}
  \cos \psi &  -\sin \psi & 0 \\
  \sin \psi &  \cos \psi  & 0 \\
  0	    &	0	  & 1 
 \end{pmatrix}
.
\]

The reference orbit should be closed and it should not be twisted. This
means that the displacement of the local reference system must be
periodic with the revolution frequency of the accelerator, while the
position angles must be periodic modulo(2 pi) with the revolution
frequency. If PSI is not periodic module(2 pi), coupling effects are
introduced. When advancing through a beam element, MAD computes
\textit{V$_i$} and \textit{W$_i$} by the recurrence relations  

\textit{V$_i$ = W$_i-1$R$_i$ + V$_i-1$}, \textit{W$_i$ = w$_i-1$S$_i$}. 

The vector \textit{R$_i$} is the displacement and the matrix
\textit{S$_i$} is the rotation of the local reference system  at the
exit of the element \textit{i} with respect to the entrance  of the same
element. The values of \textit{R$_i$} and \textit{S$_i$} are listed in
the:   \href{local_system.html#straight}{straight reference system} for
each physical element type.  
%%%\begin{center}
%\href{global}{
%%%\includegraphics{null}}
%
%\textbf{Figure 1:} Global Reference System 
%%%\end{center}
\begin{figure}[h!]
  \centering
	\includegraphics{figures/global.png}
  \caption{\href{global}{Global Reference System }}%{\textbf{Figure 1:} Local Reference System}}
\end{figure}

%\href{http://www.cern.ch/Hans.Grote/hansg_sign.html}{hansg}, January 24, 1997 


%%\title{Local Reference System}
%  Changed by: Chris ISELIN, 17-Jul-1997 
%  Changed by: Hans Grote, 25-Sep-2002 

\section{Local Reference Systems}

\subsection{\href{straight}{Reference System for Straight Beam
    Elements}} 
In straight elements the local reference system is simply translated by
the length of the element along the local \textit{s}-axis. This is true
for  
\begin{itemize}
   \item \href{drift.html}{Drift space}, 
   \item \href{quadrupole.html}{Quadrupole}, 
   \item \href{sextupole.html}{Sextupole}, 
   \item \href{octupole.html}{Octupole}, 
   \item \href{solenoid.html}{Solenoid}, 
   \item \href{crabcavity.html}{CRABCAVITY}, 
   \item \href{cavity.html}{RF cavity}, 
   \item \href{separator.html}{Electrostatic separator}, 
   \item \href{kickers.html}{Closed orbit corrector}, 
   \item \href{monitors.html}{Beam position monitor}. 
\end{itemize} 

The corresponding \textit{R}, \textit{S} are 
%%\includegraphics{null}
%RSstraight
\[
R =
 \begin{pmatrix}
  0 \\
  0 \\
  L
 \end{pmatrix}
, \quad
S =
 \begin{pmatrix}
  1 & 0 &  0 \\
  0 & 1 &  0 \\
  0 & 0 &  1
 \end{pmatrix}
.
\]

A rotation of the element about the \textit{S}-axis has no effect on
\textit{R} and \textit{S}, since the rotations of the reference system
before and after the element cancel.  
%%\begin{center}
%%\includegraphics{null}
\begin{figure}[H]
  \centering
	\includegraphics{figures/ref_straight.png}
  \caption{Reference System for Straight Beam Elements}
%\\\textbf{Figure 1:} Reference System for Straight Beam Elements 
\end{figure}


\subsection{\href{rbend}{Reference System for Bending Magnets}}
\href{bend.html}{Bending magnets} have a curved reference orbit. For
both rectangular and sector bending magnets  


%%\includegraphics{null}
%RSbend
\[
R =
 \begin{pmatrix}
  \rho\,(\cos \alpha - 1) \\
  0 \\
  \rho\,\sin \alpha
 \end{pmatrix}
, \quad
S =
 \begin{pmatrix}
  \cos \alpha & 0 &  -\sin \alpha \\
  0 & 1 &  0 \\
  \sin \alpha & 0 &  \cos \alpha
 \end{pmatrix}
,
\]

where alpha is the bend angle. A positive bend angle represents a bend
to the right, i.e. towards negative \textit{x} values. For sector
bending magnets, the bend radius is given by rho, and for rectangular
bending magnets it has the value  

 rho = \textit{L} / 2 sin(alpha/2). 

If the magnet is rotated about the \textit{s}-axis by an angle psi,
\textit{R} and \textit{S} are transformed by  

\textit{R}* = \textit{T R}, \textit{S}* = \textit{T S T$^-1$}. 

where \textit{T} is the orthogonal rotation matrix 


%%\includegraphics{null}
%Trot
\[
T =
 \begin{pmatrix}
  \cos \psi &  -\sin \psi & 0 \\
  \sin \psi &  \cos \psi  & 0 \\
  0	    &	0	  & 1 
 \end{pmatrix}
.
\]

The special value psi = pi/2 represents a bend down.  

%%\begin{center}
%\href{rbend}{
%%%\includegraphics{null}}
\begin{figure}[H]
  \centering
	\includegraphics{figures/ref_rbend.png}
  \caption{Reference System for Rectangular Bends; The signs of the pole-face rotations are positive as shown.}
%\\\textbf{Figure 2:} Reference System for Rectangular Bends; The signs of the pole-face rotations are positive as shown. 
\end{figure}

%%\includegraphics{null}}
\begin{figure}[H]
  \centering
	\includegraphics{figures/ref_sbend.png}
  \caption{Reference System for Sector Bends; The signs of the pole-face rotations are positive as shown. }
%\\\textbf{Figure 3:}  Reference System for Sector Bends; The signs of the pole-face rotations are positive as shown. 
\end{figure}


\subsection{Elements which do not Change the Local Reference}
\href{marker.html}{MARKER} elements do not  affect the reference orbit. They are ignored for geometry calculations.  

%\href{http://www.cern.ch/Hans.Grote/hansg_sign.html}{hansg}, January 24, 1997 


% add other files to the end of this file

%%%\title{DRIFT}
%  Changed by: Chris ISELIN, 24-Jan-1997 
%  Changed by: Hans Grote, 30-Sep-2002 

\section{Drift Space}
\label{sec:drift}

\begin{verbatim}
label: DRIFT,L=real;
\end{verbatim} 

A drift space has one real attribute: 
\begin{itemize}
   \item L: The drift length (default: 0 m) 
\end{itemize} 

Examples: 
\begin{verbatim}
DR1: DRIFT, L = 1.5;
DR2: DRIFT, L = DR1[L];
\end{verbatim} 

The length of DR2 will always be equal to the length of DR1. The
\href{../Introduction/local_system.html#straight}{straight reference
  system} for a drift space is a cartesian coordinate system.  

%\href{http://www.cern.ch/Hans.Grote/hansg_sign.html}{hansg}, January 24, 1997 

%%%\title{QUADRUPOLE}
%  Changed by: Chris ISELIN, 27-Jan-1997 
%  Changed by: Hans Grote, 30-Sep-2002 
%  Changed by: Frank Schmidt, 28-Aug-2003 
%  Changed by: Andrea Latina, 6-May-2013 

\section{Quadrupole}

\begin{verbatim}
label: QUADRUPOLE, L = real, K1 = real, K1S = real, TILT = real;
\end{verbatim}    

A QUADRUPOLE has four real attributes:     
\begin{itemize}
   \item L: The quadrupole length (default: 0 m). 
   \item K1: The normal quadrupole coefficient \\        
     \textit{K}$_1$ = 1/(\textit{B} rho) ($\partial$\textit{B$_y$}/$\partial$\textit{x}).\\ 
     The default is 0 m**(-2). A positive normal quadrupole strength
     implies horizontal focussing of positively charged particles.  
   \item K1S: The skew quadrupole coefficient \\        
     \textit{K}$_{1s}$ = 1/(2 \textit{B} rho)
     ($\partial$\textit{B$_x$}/$\partial$\textit{x} -
     $\partial$\textit{B$_y$}/$\partial$\textit{y})\\  
     where (x,y) is now a coordinate system rotated by -45$^o$ around s
     with respect to the normal one. The default is 0  m**(-2). A
     positive skew quadrupole strength implies defocussing (!) of
     positively charged particles in the (x,s) plane rotated by 45$^o$
     around s (particles in this plane have x = y $>$ 0). 
   \item TILT: The roll angle about the longitudinal axis (default: 0
     rad, i.e. a normal quadrupole). A positive angle represents a
     clockwise rotation. A TILT=pi/4 turns a positive normal quadrupole
     into a negative skew quadrupole.          

\textbf{ Please note that contrary to MAD8 one has to
  specify the desired TILT angle, otherwise it is taken as
  0 rad. This was needed to avoid the confusion in MAD8
  about the actual meaning of the TILT attribute for
  various elements. } 

    \item THICK: If this flag is set to 1 the quadrupole will be tracked
      through as a thick-element, instead of being converted into
      thin-lenses.  
\end{itemize}

\textbf{ Note also that K$_1$/K$_{1s}$ can be considered as
  the normal or skew quadrupole components of the magnet on
  the bench, while the TILT attribute can be considered as an
  tilt alignment error in the machine. In fact, a positive
  K$_1$ with a tilt=0 is equivalent to a positive K$_{1s}$
  with positive tilt=+pi/4. } 

Example: 
\begin{verbatim}
QF: QUADRUPOLE, L = 1.5, K1 = 0.001, THICK = 1;
\end{verbatim}     

The \href{local_system.html#straight}{straight reference system} for
a quadrupole is a cartesian coordinate system.

%\href{http://www.cern.ch/Hans.Grote/hansg_sign.html}{hansg},
%\href{http://www.cern.ch/Frank.Schmidt/frs_sign.html}{frs},
%\href{https://phonebook.cern.ch/phonebook/?id=PE525753}{al},       May 6, 2013 

%%%\title{SEXTUPOLE}
%  Changed by: Chris ISELIN, 27-Jan-1997 
%  Changed by: Hans Grote, 30-Sep-2002 
%  Changed by: Frank Schmidt, 28-Aug-2003 

\section{Sextupole}
\label{sec:sextupole}

\begin{verbatim}
label: SEXTUPOLE, L = real, K2 = real, K2S = real, TILT = real;
\end{verbatim} 

A SEXTUPOLE has four real attributes: 
\begin{itemize}
    \item L: The sextupole length (default: 0 m). 
    \item K2: The normal sextupole coefficient \\
      \textit{K}$_2$ = 1/(\textit{B} rho)
      ($\partial$$^2$\textit{B$_y$}/$\partial$ \textit{x}$^2$). \\       
      (default: 0 m**(-3)). 
    \item K2S: The skew sextupole coefficient \\
%% 2013-Jul-05  17:58:30  ghislain: error reported by Christian Carli
%      \textit{K}$_{2S}$ = 1/(2 \textit{B} rho)
%      ($\partial$$^2$\textit{B$_x$}/$\partial$ \textit{x}$^2$ -
%      $\partial$$^2$\textit{B$_y$}/$\partial$ \textit{y}$^2$).   
      \textit{K}$_{2S}$ = 1/(\textit{B} rho)
      ($\partial$$^2$\textit{B$_x$}/$\partial$ \textit{x}$^2$)
      \\
      where (x,y) is now a coordinate system rotated by -30$^o$ around s with
      respect to the normal one. (default: 0 m**(-3)). A positive skew
      sextupole strength implies defocussing (!) of positively charged
      particles in the (x,s) plane rotated by 30$^o$ around s (particles in
      this plane have x $>$ 0, y $>$ 0).  


    \item TILT: The roll angle about the longitudinal axis (default: 0
      rad, i.e. a normal sextupole). A positive angle represents a
      clockwise rotation. A TILT=pi/6 turns a positive normal sextupole
      into a negative skew sextupole.
      
      \textbf{  Please note that contrary to MAD8 one has to specify the
        desired TILT angle, otherwise it is taken as 0 rad. This was needed to
        avoid the confusion in MAD8 about the actual meaning of the TILT
        attribute for various elements. } 
\end{itemize}

\textbf{  Note also that K$_2$/K$_{2s}$ can be considered as the normal
  or skew sextupole components of the magnet on the bench, while the
  TILT attribute can be considered as an tilt alignment error in the
  machine. In fact, a positive K$_2$ with a tilt=0 is equivalent to a
  positive K$_{2s}$ with positive tilt=+pi/6.  } 

Example: 
\begin{verbatim}
S: SEXTUPOLE, L = 0.4, K2 = 0.00134;
\end{verbatim} 

The \href{local_system.html#straight}{straight reference system} for a
sextupole is a cartesian coordinate system.   

%\href{http://www.cern.ch/Hans.Grote/hansg_sign.html}{hansg}, 
%\href{http://www.cern.ch/Frank.Schmidt/frs_sign.html}{frs}, August 28, 2003  

%%%\title{the mad program}
%  Changed by: Chris ISELIN, 27-Jan-1997 
%  Changed by: Hans Grote, 30-Sep-2002 
%  Changed by: Frank Schmidt, 28-Aug-2003 

\section{Octupole}

\begin{verbatim}
label: OCTUPOLE, L = real, K3 = real, K3S = real, TILT = real;
\end{verbatim} 

An OCTUPOLE has four real attributes: 
\begin{itemize}
   \item L: The octupole length (default: 0 m). 

   \item K3: The normal octupole coefficient \\
     \textit{K}$_3$ = 1/(\textit{B} rho)
     ($\partial$$^3$\textit{B$_y$}/$\partial$\textit{x}$^3$). \\ 
     (default: 0 m**(-4)). 

   \item K3S: The skew octupole coefficient 
%% 2013-Jul-05  18:00:59  ghislain: to be checked wrt error reported on
%% K2S for sextupoles 
     \textit{K}$_3S$ = 1/(2 \textit{B} rho)
     ($\partial$$^3$\textit{B$_x$}/$\partial$\textit{x}$^3$ -
     $\partial$$^3$\textit{B$_y$}/$\partial$\textit{y}$^3$). \\
     where (x,y) is now a coordinate system rotated by -22.5$^o$ around
     s with respect to the normal one. (default: 0 m**(-4)). A positive
     skew octupole strength implies defocussing (!) of positively
     charged particles in the (x,s) plane rotated by 22.5$^o$ around s
     (particles in this plane have x $>$ 0, y $>$ 0).  

   \item TILT: The roll angle about the longitudinal axis (default: 0
     rad, i.e. a normal octupole). A positive angle represents a
     clockwise rotation. A TILT=pi/8 turns a positive normal octupole
     into a negative skew octupole.  

     \textbf{  Please note that contrary to MAD8 one has to specify the
       desired TILT angle, otherwise it is taken as 0 rad. This was
       needed to avoid the confusion in MAD8 about the actual meaning of
       the TILT attribute for various elements. }

\end{itemize}

\textbf{  Note also that K$_3$/K$_3s$ can be considered as the normal or
  skew quadrupole components of the magnet on the bench, while the TILT
  attribute can be considered as an tilt alignment error in the
  machine. In fact, a positive K$_3$ with a tilt=0 is equivalent to a
  positive K$_3s$ with positive tilt=+pi/8. } 

Example: 
\begin{verbatim}
O3: OCTUPOLE, L = 0.3, K3 = 0.543;
\end{verbatim} 

The \href{local_system.html#straight}{straight reference system} for a
octupole is a cartesian coordinate system. Octupoles are normally
treated as thin lenses, except when tracking by Lie-algebraic methods.   

%\href{http://www.cern.ch/Hans.Grote/hansg_sign.html}{hansg}, 
%\href{http://www.cern.ch/Frank.Schmidt/frs_sign.html}{frs}, August 28, 2003  

%%%\title{SOLENOID}
%  Changed by: Chris ISELIN, 27-Jan-1997 
%  Changed by: Hans Grote, 30-Sep-2002 
%  Changed by: Alexander Koschik, 16-May-2006 

\section{Solenoid}
\label{sec:solenoid}

\texttt{label: SOLENOID, L = real, KS = real;           } (\textbf{thick} version) 
\\
\texttt{label: SOLENOID, L = 0,    KS = real, KSI=real; } (\textbf{thin} version) 

A SOLENOID has two or three real attributes: 
\begin{itemize}
   \item L: The length of the solenoid (default: 0 m) 
   \item KS: The solenoid strength \textit{K$_s$} (default: 0
     rad/m). For positive KS and positive particle charge, the solenoid
     field points in the direction of increasing \textit{s}.  
   \item KSI: The solenoid integrated strength \textit{K$_s$*L}
     (default: 0 rad).  This additional attribute is needed only when
     using the thin solenoid,  where \textit{L=0}!     
   \item \textit{ KNL \& KSL:  Take note that one can specify multipole
     coefficients but they have no effect in MAD-X proper but are used
     for solenoids with multipoles in PTC.} 
\end{itemize}

Example: 
\begin{verbatim}
SOLO: SOLENOID, L = 2., KS = 0.001;
THINSOLO: SOLENOID, L = 0, KS = 0.001, KSI = 0.002;
\end{verbatim}

The \href{local_system.html#straight}{straight reference system} for a
solenoid is a cartesian coordinate system. 
 
%\href{http://www.cern.ch/Hans.Grote/hansg_sign.html}{hansg}, January 27, 1977 

%%%\title{CRABCAVITY}
%  Added by: R. Calaga, Sep 2010 
%  Edited by: A. Latina, Jun 2013

\section{Crab Cavity}


\begin{verbatim}
label: CRABCAVITY, L = real, VOLT = real, LAG = real, FREQ = real,
                   RV1 = integer, RV2 = integer, RV3 = integer, RV4 = integer, 
                   RPH1 = integer, RPH2 = integer, 
                   LAGF = real, HARMON = integer;                 
\end{verbatim} 

%  BETRF=real, PG=real,
%  FREQ=real, SHUNT=real, TFILL=real; 

A CRABCAVITY has five real attributes and seven integer attributes: 

\begin{itemize}
  \item L: The length of the cavity (default: 0 m) 

  \item VOLT: The peak RF voltage (default: 0 MV). 

  \item LAG: The initial phase lag [2pi] (default: 0). 

  \item FREQ: The RF frequency [MHz] (no default). \\
    {\bf Note that if the RF frequency is not given, it is computed from the
    harmonic number and the revolution frequency \textit{f$_0$} as before. 
    However, for deflecting structures this makes no sense, and the 
    frequency is mandatory.} 

  \item RV1: Number of initial turns with zero voltage (default: 0). 
  \item RV2: Number of turns to ramp voltage from zero to nominal value (default: 0). 
  \item RV3: Number of turns with nominal voltage (default: 0). 
  \item RV4: Number of turns to ramp voltage from nominal value to zero (default: 0).  

  \item LAGF: Value of the final crab RF phase lag [2pi] (default: 0).

  \item RPH1: Number of initial turns with nominal phase (default: 0). 
  \item RPH2: Number of turns to ramp phase [2pi] from nominal to
    specified value \\ (default:~0). 

  \item HARMON: The harmonic number \textit{h} (no default). \\
    Only if the frequency is not given. 

% \item BETRF: RF coupling factor (default: 0).
% \item PG: The RF power per cavity (default: 0 MW).
% \item SHUNT: The relative shunt impedance (default: 0 MOhm/m).
% \item TFILL: The filling time of the cavity T<sub>fill</sub> (default: 0 microseconds). 
%\item EPHASE: Value of the final crab RF phase [2pi] with respect to  nominal value (default: 0). 
\end{itemize}


{\bf Caveats:}
\begin{itemize}
   \item \textit{ Please take note, that the following MAD8 attributes:
     BETRF, PG, SHUNT and TFILL are currently not implemented in MAD-X!}
   \item \textit{ Note that crab cavities are only implemented for
     tracking  purposes. \\ TWISS will ignore any effect of the crab cavity.  
% as well that twiss is 4D only. As a consequence the TWISS
% parameters in the plane of non-zero dispersion may not close as
% expected. Therefore, it is best to perform TWISS in 4D only, i.e. with
% cavities switched off. If 6D is needed one has to use the
% 
% <a href="../ptc_twiss/ptc_twiss.html">ptc_twiss</a> command. 
}
\end{itemize} 

%A cavity requires the particle energy (\href{beam.html#energy}{ENERGY})
%and the particle charge (\href{beam.html#charge}{CHARGE}) to be set by
%a \href{beam.html}{BEAM} command before any calculation is performed. 

Before any calculation is performed with a CRABCAVITY, the particle
energy (\href{beam.html#energy}{ENERGY}) and the particle charge
(\href{beam.html#charge}{CHARGE}) must be set by a
\href{beam.html}{BEAM} command.   

Then the effect of the CRABCAVITY on particle coordinates during tracking is
\\
\\ delta(\textit{px})  = VOLT * sin( PHI - OMEGA * t) 
\\ delta(\textit{E})  = -  VOLT * OMEGA * x * cos(PHI - OMEGA * t) 
\\ 
\\ where PHI =  2 $\pi$ * (LAG - HARMON * \textit{f$_0$ t}), 
\\ and OMEGA = 2 $\pi$ * FREQ / \textit{c}
\\
% delta(<i>E</i>) = VOLT * 
% sin(2 pi * (LAG - HARMON * <i>f<sub>0</sub> t</i>)). 


Example: 
\begin{verbatim}
BEAM, PARTICLE = PROTON, ENERGY = 7000.0;
CAVITY:  CRABCAVITY, L = 10.0, VOLT = 5.0, LAG = 0.0, FREQ = 400,
         RV1 = 0, RV2 = 50, RV3 = 1000, RV4 = 50, 
         RPH1 = 100, RPH2 = 500, LAGF = 0.125;
\end{verbatim} 

The \href{local_system.html#straight}{straight reference system} for a
cavity is a cartesian coordinate system.  
 
%\href{http://www.cern.ch/rcalaga}{R. Calaga}, September 2010

%%%\title{RFCAVITY}
%  Changed by: Chris ISELIN, 27-Jan-1997 
%  Changed by: Hans Grote, 30-Sep-2002 

\section{RF Cavity}
\label{sec:rf_cavity}


\begin{verbatim}
label: RFCAVITY, L = real, VOLT = real, LAG = real, HARMON = integer, FREQ = real;                  
\end{verbatim} 

%  HARMON=integer, BETRF=real,PG=real,
%                  FREQ=real,SHUNT=real,TFILL=real; 


An RFCAVITY has eight real attributes and one integer attribute: 
\begin{itemize}
   \item L: The length of the cavity (DEFAULT: 0 m) 
   \item VOLT: The peak RF voltage (DEFAULT: 0 MV). The effect of the cavity is \\
     delta(\textit{E}) = VOLT * sin(2 pi * (LAG - HARMON * \textit{f$_0$ t})). 
   \item LAG: The phase lag [2pi] (DEFAULT: 0). 
   \item FREQ: The frequency [MHz] (no DEFAULT). Note that if the RF
     frequency is not given, it is computed from the harmonic number and
     the revolution frequency \textit{f$_0$} as before. However, for
     accelerating structures this makes no sense, and the frequency is
     mandatory.  
   \item HARMON: The harmonic number \textit{h} (no DEFAULT). Only if
     the frequency is not given.  
   \item \textit{ Please take note, that the following MAD8 attributes:
     BETRF, PG, SHUNT and TFILL are currently not implemented in MAD-X!}    
%  \item BETRF: RF coupling factor (DEFAULT: 0).
%  \item PG: The RF power per cavity (DEFAULT: 0 MW).
%  \item SHUNT: The relative shunt impedance (DEFAULT: 0 MOhm/m).
%  \item TFILL: The filling time of the cavity $T_{fill}$ (DEFAULT: 0 microseconds). 

   \item \textit{ Note as well that twiss is 4D only. As a consequence
     the TWISS parameters in the plane of non-zero dispersion may not
     close as expected. Therefore, it is best to perform TWISS in 4D
     only, i.e. with cavities switched off. If 6D is needed one has to
     use the \href{../ptc_twiss/ptc_twiss.html}{ptc\_twiss} command. } 
\end{itemize}  

The RFCAVITY has attributes that will only become active in PTC: 
\begin{itemize}
   \item n\_bessel (DEFAULT: 0): \\
     Transverse focussing effects are typically ignored in the cavity in
     MAD-X or even PTC. This effect is being calculated to order n\_bessel,
     with n\_bessel=0 disregarding this effect and with a correct treatment
     when n\_bessel goes to infinty.
   \item no\_cavity\_totalpath (DEFAULT: no\_cavity\_totalpath=false): \\
     flag to choose if in a cavity the transit time factor is considered
     (no\_cavity\_totalpath=false) or if the particle is kept on the
     crest of RF voltage (no\_cavity\_totalpath=true).  
\end{itemize}  

A cavity requires the particle energy (\href{beam.html#energy}{ENERGY})
and the particle charge (\href{beam.html#charge}{CHARGE}) to be set by a
\href{beam.html}{BEAM} command before any calculations are performed.  

 Example: 
\begin{verbatim}
BEAM, PARTICLE = ELECTRON, ENERGY = 50.0;
CAVITY: RFCAVITY, L = 10.0, VOLT = 150.0, LAG = 0.0, HARMON = 31320;
\end{verbatim} 

The \href{local_system.html#straight}{straight reference system} for a
cavity is a cartesian coordinate system.  

%\href{http://www.cern.ch/Hans.Grote/hansg_sign.html}{hansg}, January 24, 1997 

%%%\title{ELSEPARATOR}
%  Changed by: Chris ISELIN, 27-Jan-1997 
%  Changed by: Hans Grote, 30-Sep-2002 
%  Changed by: Frank Schmidt, 28-Aug-2003 

\section{ELSEPARATOR: Electrostatic Separator}

\begin{verbatim}
label: ELSEPARATOR, L = real, EX = real, EY = real, TILT = real;
\end{verbatim} 

An ELSEPARATOR (electrostatic separator) has four real attributes: 
\begin{itemize}
   \item L: The length of the separator (default: 0 m). 
   \item EX: The horizontal electric field strength (default: 0 MV/m). 
     A positive field increases \textit{p$_x$} for positive particles.  
   \item EY: The vertical electric field strength (default: 0 MV/m). 
     A positive field increases \textit{p$_y$} for positive particles.  
   \item TILT: The roll angle about the longitudinal axis (default: 0
     rad). A positive angle represents a clockwise of the electrostatic
     separator.  
\end{itemize} 

A separator requires the particle energy
(\href{beam.html#energy}{ENERGY}) and the particle charge
(\href{beam.html#charge}{CHARGE}) to be set by a \href{beam.html}{BEAM}
command before any calculations are performed.  

Example: 
\begin{verbatim}
BEAM,PARTICLE = POSITRON, ENERGY = 50.0;
SEP: ELSEPARATOR, L = 5.0, EY = 0.5;
\end{verbatim} 

The \href{local_system.html#straight}{straight reference system} for a
separator is a cartesian coordinate system.   

%\href{http://www.cern.ch/Hans.Grote/hansg_sign.html}{hansg}, 
%\href{http://www.cern.ch/Frank.Schmidt/frs_sign.html}{frs}, August 28, 2003  

%%%\title{KICK, HKICK, VKICK}
%  Changed by: Chris ISELIN, 27-Jan-1997 

%  Changed by: Hans Grote, 30-Sep-2002 

%  Changed by: Frank Schmidt, 28-Aug-2003 

%  Changed by: Werner Herr, 22-May-2007 

%%\usepackage{hyperref}
% commands generated by html2latex


%%\begin{document}
%%\begin{center}
 %%EUROPEAN ORGANIZATION FOR NUCLEAR RESEARCH 
%%\includegraphics{http://cern.ch/madx/icons/mx7_25.gif}

\subsection{Closed Orbit Correctors}
%%\end{center}  
Three types of closed orbit correctors are available: 
\begin{itemize}
	\item \href{hkick}{HKICKER}, a corrector for the horizontal plane, 
	\item \href{vkick}{VKICKER}, a corrector for the vertical plane, 
	\item \href{kick}{KICKER}, a corrector for both planes. 
\end{itemize}
\begin{verbatim}

label:   HKICKER, L=real,KICK=real,TILT=real;
label:   VKICKER, L=real,KICK=real,TILT=real;
label:   KICKER,  L=real,HKICK=real,VKICK=real,TILT=real;
\end{verbatim} The type KICKER should not be used when an orbit corrector kicks only in one plane. 
\\
\\ The attributes have the following meaning: 
\begin{itemize}
	\item L: The length of the closed orbit corrector (default: 0 m). 
	\item KICK: The kick angle for either horizontal or vertical correctors. (default: 0 rad). 
	\item HKICK: The horizontal kick angle for a corrector in both planes (default: 0 rad). 
	\item VKICK: The vertical kick angle for a corrector in both planes (default: 0 rad). 
	\item TILT: The roll angle about the longitudinal axis (default: 0 rad). A positive angle represents a clockwise rotation of the kicker. 
\end{itemize} A positive kick increases \textit{p$_x$} or \textit{p$_y$} respectively. This means that a positive horizontal kick bends to the left,  i.e. to positive x which is opposite of what is true for bends.  
\\ It should be noted that the kick values assigned to an orbit corrector like above are not overwritten by an orbit correction using the CORRECT command. Instead the kicks computed by an orbit correction and the assigned values are added when the correctors are used. 

 Examples: 
\begin{verbatim}

HK1:   HKICKER, KICK=0.001;
VK3:   VKICKER, KICK=0.0005;
VK4:   VKICKER, KICK:=AVK4;
KHV1:  KICKER,  HKICK=0.001,VKICK=0.0005;
KHV2:  KICKER,  HKICK:=AKHV2H,VKICK:=AKHV2V;
\end{verbatim} The assignment in the form of a deferred expression has the advantage that the values can be assigned and/or modified at any time (and matched !). 
\\ The \href{local_system.html#straight}{straight reference system} for an orbit corrector is a Cartesian coordinate system.  

 Please note that there is a new feature introduced by Stefan Sorge from GSI. Here his decription:

The elements KICKER, HKICKER, and VKICKER can also be used as  an exciter providing a sinusoidal momentum kick. The usage in this case is  



xykick: KICKER, SINKICK=integer, SINPEAK=real, SINTUNE=real, SINPHASE=real;  

xkick : HKICKER, SINKICK=integer, SINPEAK=real, SINTUNE=real, SINPHASE=real;  

ykick : VKICKER, SINKICK=integer, SINPEAK=real, SINTUNE=real, SINPHASE=real;  

 where a sinusoidal momentum kick dpz as a function of the  revolution number n given by  

 dpz(n)=SINPEAK * sin(2*PI*SINTUNE*n + SINPHASE), pz=px,py  

 is provided. So, the variables are 



SINKICK - integer, must be set to 1 to switch on the sinusoidal signal,        default: 0.  

SINPEAK - amplitude of the bending angle (rad), default: 0 rad.  

SINTUNE - frequency of the signal times the revolution frequency.        Hence, the phase per revolution is 2*PI*SINTUNE, default: 0.  

SINPHASE - initial phase, default: 0 rad.  

 KICKER generates a kick in horizontal and a kick vertical direction,  where both are synchron, HKICKER generates a horizontal kick,  and VKICKER generates a vertical kick.  

 The momentum kick of a kicker has only a single frequency. An element  having a finite bandwidth can approximately created by defining  thin kickers with all amplitudes SINPEAK, frequencies SINTUNE, and  initial phases SINPHASE desired and putting them at the same position s in  the accelerator.  

 From S.Sorge@gsi.de  

\href{http://www.cern.ch/Hans.Grote/hansg_sign.html}{hansg}, \href{http://www.cern.ch/Frank.Schmidt/frs_sign.html}{frs}, August 28, 2003  

%%\end{document}

%%%\title{Monitors}
%  Changed by: Chris ISELIN, 27-Jan-1997 
%  Changed by: Hans Grote, 30-Sep-2002 
%  Changed by: G. Roy, 17 Oct 2013: added PLACEHOLDER

\section{Beam Position Monitors}
\label{sec:monitors}

A beam monitor acts on the beam like a drift space. In addition it
serves to record the beam position for closed orbit corrections. Four
different types of beam position monitors are recognised:  

\begin{itemize}
   \item \href{hmon}{HMONITOR}. Monitor for the horizontal beam position, 
   \item \href{vmon}{VMONITOR}. Monitor for the vertical beam position, 
   \item \href{mon}{MONITOR}. Monitor for both horizontal and vertical beam position. 
   \item \href{inst}{INSTRUMENT}. A place holder for any type of beam
     instrumentation. Optically it behaves like a drift space; it
     returns \emph{no beam observation}. It represent a class of
     elements which is completely independent from drifts and monitors.  
   \item \href{plac}{PLACEHOLDER}. A place holder for any type of
     element. Internally it is equivalent to an INSTRUMENT: optically it
     behaves as a drift space, it returns \emph{no beam observation}. It
     represent a class of elements which is completely independent from
     drifts and monitors. 
\end{itemize}

\begin{verbatim}
label: HMONITOR,    L = real;
label: VMONITOR,    L = real;
label: MONITOR,     L = real;
label: INSTRUMENT,  L = real;
label: PLACEHOLDER,  L = real;
\end{verbatim} 

A beam position monitor has one real attribute: 
\begin{itemize}
   \item L: The length of the monitor (default: 0 m). If the length is
     different from zero, the beam position is recorded in the centre of
     the monitor.  
\end{itemize} 

Examples: 
\begin{verbatim}
MH: HMONITOR, L = 1;
MV: VMONITOR;
\end{verbatim} 

The \href{local_system.html#straight}{straight reference system} for a
monitor is a cartesian coordinate system.  

%\href{http://www.cern.ch/Hans.Grote/hansg_sign.html}{hansg}, June 17, 2002 


%%\title{Sign Conventions}
%  Changed by: Chris ISELIN, 17-Jul-1997 
%  Changed by: Hans Grote, 10-Jun-2002 

\section{Sign Conventions for Magnetic Fields}
\label{sec:signconvention}

The MAD program uses the following Taylor expansion for the field on the
mid-plane \textit{y}=0, described in
\href{bibliography.html#slac75}{SLAC-75}:  


%%\includegraphics{null}
%Taylor_field
$$
B_y(x,0)=\sum_{n=0}^{\infty} \frac{B_n\,x^n}{n!}
$$

Note the factorial in the denominator. The field coefficients have the following meaning: 
\begin{itemize}
   \item \textit{B}$_0$: Dipole field, with a positive value in the
     positive \textit{y} direction; a positive field bends a positively
     charged particle to the right.  
   \item \textit{B}$_1$: Quadrupole coefficient \\
%     \textit{B}$_1$ = (del \textit{B$_y$} / del \textit{x});\\ 
     \( \textit{B}_1 = ( \partial \textit{B}_y / \partial \textit{x} ) \);\\ 
     a positive value corresponds to horizontal focussing of a
     positively charged particle. 
   \item \textit{B}$_2$: Sextupole coefficient \\
%     \textit{B}$_2$ =  (del$^2$\textit{B$_y$} / del \textit{x}$^2$). 
     \( \textit{B}_2 =  ( \partial^2 \textit{B}_y / \partial \textit{x}^2 ) \). 
   \item \textit{B}$_3$: Octupole coefficient \\ 
%     \textit{B}$_3$ =  (del$^3$\textit{B$_y$} / del \textit{x}$^3$). 
     \( \textit{B}_3 =  ( \partial^3 \textit{B}_y / \partial \textit{x}^3 ) \). 
   \item \ldots
\end{itemize} 

Using this expansion and the curvature \textit{h} of the reference
orbit, the longitudinal component of the vector potential to order 4 is:  
%%\includegraphics{null}
%Taylor_A_s
% this is problematic, {align}, {eqnarray} do not work 
% EDIT : Laurent fixed that.
\[
\begin{aligned}
A_x =
&+ B_0\,\Big(x-\frac{hx^2}{2(1+hx)}\Big)&
&+ B_1\,\Big(\frac{1}{2}(x^2-y^2) - \frac{h}{6}x^3 + \frac{h^2}{24}(4x^4-y^4)+\cdots\Big) \\
&+ B_2\,\Big(\frac{1}{6}(x^3-3xy^2) - \frac{h}{24}(x^4-y^4)+\cdots \Big)&
&+ B_3\,\Big(\frac{1}{24}(x^4-6x^2y^2+y^4) \cdots \Big)+\cdots
\end{aligned}
\]

Taking \(\vec{B} = \nabla \times \vec{A}\) in curvilinear coordinates,
the field components can be computed as  

%%\includegraphics{null}
%Taylor_B
\[
\begin{aligned}
B_x(x,y) =
&+ B_1\,\Big(y+\frac{h^2}{6}y^3+\cdots\Big)&  &  \\
&+ B_2\,\Big(xy - \frac{h^3}{6}y^3+\cdots \Big)&+B_3\,\Big(\frac{1}{6}(3x^2y-y^3)+ \cdots \Big)+\cdots\\
B_y(x,y)=
&+ B_0 & + B_1\,\Big(x-\frac{h}{2}y^2+\frac{h^2}{2}xy^2+\cdots \Big)\\
&+ B_2\,\Big(\frac{1}{2}(x^2-y^2)-\frac{h}{2}xy^2+\cdots \Big) & + B_3\,\Big(\frac{1}{6}(x^3-3xy^2)+ \cdots \Big)+\cdots
\end{aligned}
\]

It can be easily verified that both \(\nabla \times \vec{\textit{B}}\)
and \(\nabla . \vec{\textit{B}}\) are zero to the order of the
\(\textit{B}_3\) term.  

Introducing the magnetic rigidity \(\textit{B}\rho = p_s / q\) as the
momentum of the particle divided by its charge, the multipole
coefficients are computed as   
%\textit{K$_n$} = \textit{e B$_n$ / p$_s$} =  \textit{B$_n$ / B} rho. 
\[ \textit{K}_n = \textit{q B}_n / \textit{p}_s  =  \textit{B}_n / \textit{B} \rho \] 


%\href{http://www.cern.ch/Hans.Grote/hansg_sign.html}{hansg}, June 17, 2002 


%%\title{Variables in MAD}
%  Changed by: Chris ISELIN, 17-Jul-1997 
%  Changed by: Hans Grote, 10-Jun-2002 

\section{Variables}

\subparagraph{ For each variable the physical units are listed in square brackets. }

\subsection{\href{canon}{Canonical Variables Describing Orbits}} 
MAD uses the following canonical variables to describe the motion of particles: 
\begin{itemize}
   \item X: Horizontal position \textit{x} of the (closed) orbit,
     referred to the ideal orbit [m].    
   \item PX: Horizontal canonical momentum \textit{p$_x$} of the
     (closed) orbit referred to the ideal orbit, divided by the
     reference momentum: PX = \textit{p$_x$ / p$_0$}, [1].   
   \item Y: Vertical position \textit{y} of the (closed) orbit, referred
     to the ideal orbit [m].   
   \item PY: Vertical canonical momentum \textit{p$_y$} of the (closed)
     orbit referred to the ideal orbit, divided by the reference
     momentum: PY = \textit{p$_x$ / p$_0$}, [1].   
   \item T: Velocity of light times the negative time difference with
     respect to the reference particle: T = \textit{ - c t}, [m]. A
     positive T means that the particle arrives ahead of the reference
     particle.   
   \item PT: Energy error, divided by the reference momentum times the
     velocity of light: PT = delta(\textit{E}) / \textit{p$_s$ c},
     [1]. This value is only non-zero when synchrotron motion is
     present. It describes the deviation of the particle from the orbit
     of a particle with the momentum error DELTAP.   
   \item DELTAP: Difference of the reference momentum and the design
     momentum, divided by the reference momentum: DELTAP =
     delta(\textit{p}) / \textit{p$_0$}, [1]. This quantity is used to
     \href{defects.html}{normalize} all element strengths.   
\end{itemize} 

The independent variable is: 
\begin{itemize}
	\item \href{s}{S}: Arc length \textit{s} along the reference
          orbit, [m].   
\end{itemize} 

In the limit of fully relativistic particles (gamma $\gg$ 1, \textit{v
  = c}, \textit{p c = E}), the variables T, PT used here agree with the
longitudinal variables used in
\href{bibliography.html#transport}{[TRANSPORT]}. This means that T
becomes the negative path length difference, while PT becomes the
fractional momentum error. The reference momentum \textit{p$_s$} must be
constant in order to keep the system canonical.  

\subsection{\href{normal}{Normalised Variables and other Derived Quantities}}
\begin{itemize}
	\item XN: The normalised horizontal displacement\\
          XN = \textit{x$_n$} = Re(\textit{E$_1$$^T$ S Z}), [sqrt(m)]. 
	\item PXN: The normalised horizontal transverse momentum \\    
          PXN = \textit{x$_n$} = Im(\textit{E$_1$$^T$ S Z}), [sqrt(m)].     
	\item WX: The horizontal Courant-Snyder invariant \\
          WX = sqrt(\textit{x$_n$$^2$ + p$_{xn}$$^2$}), [m].    
	\item PHIX: The horizontal phase \\    
          PHIX = - atan(\textit{p$_{xn}$ / x$_n$}) / 2 pi [1].      
	\item YN: The normalised vertical displacement \\
          YN = \textit{x$_n$} = Re(\textit{E$_2$$^T$ S Z}), [sqrt(m)].     
	\item PYN: The normalised vertical transverse momentum \\ 
          PYN = \textit{x$_n$} = Im(\textit{E$_2$$^T$ S Z}), [sqrt(m)].     
	\item WY: The vertical Courant-Snyder invariant \\
          WY = sqrt(\textit{y$_n$$^2$ + p$_{yn}$$^2$}), [m].     
	\item PHIY: The vertical phase \\ 
          PHIY = - atan(\textit{p$_{yn}$ / y$_n$}) / 2 pi [1].     
	\item TN: The normalised longitudinal displacement \\     
          TN = \textit{x$_n$} = Re(\textit{E$_3$$^T$ S Z}), [sqrt(m)].     
	\item PTN: The normalised longitudinal transverse momentum \\    
          PTN = \textit{x$_n$} = Im(\textit{E$_3$$^T$ S Z}), [sqrt(m)].     
	\item WT: The longitudinal invariant \\    
          WT = sqrt(\textit{t$_n$$^2$ + p$_{tn}$$^2$}), [m].     
	\item PHIT: The longitudinal phase \\    
          PHIT = + atan(\textit{p$_{tn}$ / t$_n$}) / 2 pi [1].     
\end{itemize} 

In the above formulas \textit{Z} is the phase space vector
\textit{Z = ( x, p$_x$, y, p$_y$, t, p$_t$)$^T$}, the vectors
\textit{E$_i$} are the three complex eigenvectors and  
the matrix \textit{S} is the ``symplectic unit matrix'' 
%%\includegraphics{../equations/S_matrix.gif}
%S_matrix
\[
S =
 \begin{pmatrix}
  0 & 1 & 0 & 0 & 0 & 0 \\
  -1 & 0 & 0 & 0 & 0 & 0 \\
  0 & 0 & 0 & 1 & 0 & 0 \\
  0 & 0 & -1 & 0 & 0 & 0 \\
  0 & 0 & 0 & 0 & 0 & 1 \\
  0 & 0 & 0 & 0 & -1 & 0 \\
 \end{pmatrix}
\]


\subsection{\href{linear}{Linear Lattice Functions (Optical Functions)}} 

Several MAD commands refer to linear lattice functions. Since MAD uses
the canonical momenta (\textit{p$_x$}, \textit{p$_y$}) instead of the
slopes (\textit{x}', \textit{y}'), their definitions differ slightly
from those in \href{bibliography.html#courant}{[Courant and
    Snyder]}. Notice that in MAD-X PT substitutes DELTAP as longitudinal
variable. Dispersive and chromatic functions are hence derivatives with
respects to PT. Being PT=BETA*DELTAP, where BETA is the relativistic
Lorentz factor, those functions must be multiplied by BETA a number of
time equal to the order of the derivative. 

The linear lattice functions are known to MAD under the following names:
\begin{itemize}
	\item BETX: Amplitude function beta$_\textit{x}$, [m].   
	\item ALFX: Correlation function alpha$_\textit{x}$, [1]:\\     
          ALFX = alpha$_\textit{x}$ = - 1/2 * (del beta$_\textit{x}$ / del \textit{s}).     
	\item MUX: Phase function mu$_\textit{x}$, [2pi]:\\
          MUX = mu$_\textit{x}$ = integral (d\textit{s} / beta$_\textit{x}$).     
	\item DX: Dispersion \textit{D$_x$} of \textit{x}, [m]:\\
          DX = \textit{D$_x$} = (del \textit{x} / del PT).     
	\item DPX: Dispersion \textit{D$_px$} of \textit{p$_x$}, [1]:\\
          DPX = \textit{D$_px$} = (del \textit{p$_x$} / del PT) / \textit{p$_s$}.     
	\item BETY: Amplitude function beta$_\textit{y}$, [m].   
	\item ALFY: Correlation function alpha$_\textit{y}$, [1].\\
          ALFY = alpha$_\textit{y}$ = - 1/2 * (del beta$_\textit{y}$ / del \textit{s}).     
	\item MUY: Phase function mu$_\textit{y}$, [2pi].\\
          MUY = mu$_\textit{y}$ = integral (d\textit{s} / beta$_\textit{y}$).     
	\item DY: Dispersion \textit{D$_y$} of \textit{y}, [m]:\\
          DY = \textit{D$_y$} = (del \textit{y} / del PT).     
	\item DPY: Dispersion \textit{D$_px$} of \textit{p$_x$}, [1]:\\
          DPY = \textit{D$_py$} = (del \textit{p$_y$} / del PT) / \textit{p$_s$}.     
	\item R11, R12, R21, R22: Coupling Matrix     
	\item ENERGY: The total energy per particle in GeV. If given, it
          must be greater then the particle mass.
\end{itemize}

%  The TWISS table also defines the following expressions which 
%  can be used in plots:
% \begin{itemize}
%   \item  GAMX = (1 + ALFX*ALFX) / BETX, 
%   \item  GAMY = (1 + ALFY*ALFY) / BETY, 
%   \item  SIGX = SQRT(BETX * EX), the vertical r.m.s. half-width of the beam, 
%   \item  SIGY = SQRT(BETY * EY), the vertical r.m.s. half-height of the beam. 
% \end{itemize}


\subsection{\href{chrom}{Chromatic Functions}} 
Several MAD commands refer to the chromatic functions. 

Because MAD-X uses the (\textit{p$_x$}, \textit{p$_y$}) coordinates
instead of the slopes (\textit{x}', \textit{y}'), the definition of the
chromatic functions in MAD-X differ slightly from those in
\href{bibliography.html#montague}{[Montague]}.  

Notice also that in MAD-X, PT substitutes DELTAP as longitudinal
variable. Dispersive and chromatic functions are hence derivatives with
respects to PT. Since we have PT=BETA*DELTAP, where BETA is the
relativistic Lorentz factor, those functions must be multiplied by BETA
a number of time equal to the order of the derivative. The chromatic
functions are known to MAD under the following names:  

\textit{Please note that this option is needed for a proper calculation
  of the chromaticities in the presence of coupling!} 

\begin{itemize}
   \item WX: Chromatic amplitude function \textit{W$_x$}, [1]:\\
     WX = \textit{W$_x$} = sqrt(\textit{a$_x$$^2$ + b$_x$$^2$}),\\
     \textit{a$_x$} = (del beta$_\textit{x}$ / del PT) / beta$_\textit{x}$,\\
     \textit{b$_x$} = (del alpha$_\textit{x}$ / del PT) -
     (alpha$_\textit{x}$ / beta$_\textit{x}$) * (del beta$_\textit{x}$ /
     del PT).      
   \item PHIX: Chromatic phase function Phi$_\textit{x}$, [2pi]:\\
     PHIX = Phi$_\textit{x}$ = atan(\textit{a$_x$ / b$_x$}).     
   \item DMUX: Chromatic derivative of phase function mu$_\textit{x}$, [2pi]:\\
     DMUX = (del mu$_\textit{x}$ / del PT).     
   \item DDX: Chromatic derivative of dispersion \textit{D$_x$}, [m]:\\
     DDX = 1/2 * (del$^2$\textit{x} / del PT$^2$).     
   \item DDPX: Chromatic derivative of dispersion \textit{D$_px$}, [1]:\\
     DDPX = 1/2 * (del$^2$\textit{p$_x$} / del PT$^2$) / \textit{p$_s$}.     
   \item WY: Chromatic amplitude function \textit{W$_y$}, [1]:\\
     WY = \textit{W$_y$} = sqrt(\textit{a$_y$$^2$ + b$_y$$^2$}),\\     
     \textit{a$_y$} = (del beta$_\textit{y}$ / del PT) / beta$_\textit{y}$,\\     
     \textit{b$_y$} = (del alpha$_\textit{y}$ / del PT) -
     (alpha$_\textit{y}$ / beta$_\textit{y}$) * (del beta$_\textit{y}$ /
     del PT).     
   \item PHIY: Chromatic phase function Phi$_\textit{y}$, [2pi]:\\     
     PHIY = Phi$_\textit{y}$ = atan(\textit{a$_y$ / b$_y$}).     
   \item DMUY: Chromatic derivative of phase function mu$_\textit{y}$, [2pi]:\\     
     DMUY = (del mu$_\textit{y}$ / del PT).     
   \item DDY: Chromatic derivative of dispersion \textit{D$_y$}, [m]:\\     
     DDY = 1/2 * (del$^2$\textit{y} / del PT$^2$).     
   \item DDPY: Chromatic derivative of dispersion \textit{D$_py$}, [1]:\\ 
     DDPY = 1/2 * (del$^2$\textit{p$_y$} / del PT$^2$) / \textit{p$_s$}.     
\end{itemize}

\subsection{\href{summ}{Variables in the SUMM Table}} 
After a successful TWISS command a summary table is created which
contains the following variables:  
\begin{itemize}
   \item LENGTH: The length of the machine, [m].     
   \item ORBIT5: The T (= \textit{c t}, [m]) component of the closed orbit.     
   \item ALFA: The momentum compaction alpha$_p$, [1].     
   \item GAMMATR: The transition energy gamma$_transition$, [1].     
   \item Q1: The horizontal tune \textit{Q$_1$} [1].     
   \item DQ1: The horizontal chromaticity dq$_\textit{1}$, [1]:\\     
     DQ1 = dq$_\textit{1}$ = (del \textit{Q$_1$} / del PT).     
   \item BETXMAX: The largest horizontal beta$_\textit{x}$, [m].     
   \item DXMAX: The largest horizontal dispersion [m].     
   \item DXRMS: The r.m.s. of the horizontal dispersion [m].     
   \item XCOMAX: The maximum of the horizontal closed orbit deviation [m].     
   \item XRMS: The r.m.s. of the horizontal closed orbit deviation [m].     
   \item Q2: The vertical tune \textit{Q$_2$} [1].     
   \item DQ2: The vertical chromaticity dq$_\textit{2}$, [1]:\\     
     DQ2 = dq$_\textit{2}$ = (del \textit{Q$_2$} / del PT).     
   \item BETYMAX: The largest vertical beta$_\textit{y}$, [m].     
   \item DYMAX: The largest vertical dispersion [m].     
   \item DYRMS: The r.m.s. of the vertical dispersion [m].     
   \item YCOMAX: The maximum of the vertical closed orbit deviation [m].     
   \item YCORMS: The r.m.s. of the vertical closed orbit deviation [m].     
   \item DELTAP: Energy difference, divided by the reference
     momentum times the velocity of light, [1]:\\
     DELTAP = delta(\textit{E}) / \textit{p$_s$ c}.
\end{itemize} 

Notice that in MAD-X PT substitutes DELTAP as longitudinal
variable. Dispersive and chromatic functions are hence derivatives with
respects to PT. Being PT=BETA*DELTAP, where BETA is the relativistic
Lorentz factor, those functions must be multiplied by BETA a number of
time equal to the order of the derivative.  

\subsection{\href{track}{Variables in the TRACK Table}} 
The command RUN writes tables with the following variables: 
\begin{itemize}
   \item X: Horizontal position \textit{x} of the orbit, referred to the
     ideal orbit [m].    
   \item PX: Horizontal canonical momentum \textit{p$_x$} of the orbit
     referred to the ideal orbit, divided by the reference momentum.    
   \item Y: Vertical position \textit{y} of the orbit, referred to the
     ideal orbit [m].    
   \item PY: Vertical canonical momentum \textit{p$_x$} of the orbit
     referred to the ideal orbit, divided by the reference momentum.    
   \item T: Velocity of light times the negative time difference with
     respect to the reference particle, [m]. A positive T means that the
     particle arrives ahead of the reference part icle.   
   \item PT: Energy difference, divided by the reference momentum times
     the velocity of light, [1].    
\end{itemize} 

When tracking Lyapunov companions (not yet implemented), the TRACK table
defines the following dependent expressions:  
\begin{itemize}
   \item DISTANCE: the relative Lyapunov distance between the two
     particles.    
   \item LYAPUNOV: the estimated Lyapunov Exponent.   
   \item LOGDIST: the natural logarithm of the relative distance.   
   \item LOGTURNS: the natural logarithm of the turn number.   
\end{itemize}

%\href{http://www.cern.ch/Hans.Grote/hansg_sign.html}{hansg}, January 24, 1997. Revised in February 2007.


%%\title{Physical Units}
%  Changed by: Chris ISELIN, 27-Jan-1997 

%  Changed by: Hans Grote, 30-Sep-2002 

%%\usepackage{hyperref}
% commands generated by html2latex


%%\begin{document}
%%\begin{center}
 %%EUROPEAN ORGANIZATION FOR NUCLEAR RESEARCH 
%%\includegraphics{http://cern.ch/madx/icons/mx7_25.gif}

\subsection{Physical Units}
%%\end{center} 
 Throughout the computations MAD uses international (SI, Syst\`eme International) units. These units are summarised in the \hyperlink{table}{Units table}. 

%\href{table}{
\begin{table}[h]
\begin{center}
{\textbf{Table 1:} Physical Units}
\\
\begin{tabular}{l | l}
Length                  & m (metres) \\ 
Angle                   & rad (radians) \\ 
Quadrupole coefficient  & m**(-2) \\ 
Multipole coefficient, 2n poles   & m**(-n) \\ 
Electric voltage        & MV (Megavolts) \\ 
Electric field strength & MV/m \\ 
Frequency               & MHz (Megahertz) \\ 
Phase angles            & 2 pi \\ 
Particle energy         & GeV \\ 
Particle mass           & GeV/c**2 \\ 
Particle momentum       & GeV/c \\ 
Beam current            & A (Amperes) \\ 
Particle charge         & e (elementary charges) \\ 
Impedances              & MOhm (Megohms) \\ 
Emittances              & pi m mrad \\ 
RF power                & MW (Megawatts) \\ 
Higher mode loss factor & V/pc
% if you want to use \caption the tabular must be enclosed in
% table or figure environment
% \caption{\textbf{Table 1:} Physical Units}
\end{tabular}
\end{center}
\end{table}
%}


\href{http://www.cern.ch/Hans.Grote/hansg_sign.html}{hansg}, June 17, 2002 

%%\end{document}


%\href{http://www.cern.ch/Hans.Grote/hansg_sign.html}{hansg}, May 8, 2001 
