%%\title{RBEND, SBEND}
%  Changed by: Chris ISELIN, 17-Jul-1997 
%  Changed by: Hans Grote, 30-Sep-2002 
%  Changed by: Frank Schmidt, 28-Aug-2003 

\section{Bending Magnets}

Two different type keywords are recognised for bending magnets, they are
distinguished only by the reference system used:  
\begin{itemize}
   \item \href{rbend}{RBEND} is a rectangular bending magnet. It has
     parallel pole faces and is based on a curved
     \href{local_system.html#rbend}{rbend reference system};  \textbf{
       its length is the straight length as in the Figure but internally
       the arc length is being used.  - to define an RBEND with the arc
       length as length (straight line shorter than input - for
       compatibility with MAD8 version up to version 8.23.06 including),
       the option RBARC=FALSE has to be set.} 
   \item \href{sbend}{SBEND} is a sector bending magnet. Its pole faces
     meet at the centre of curvature of the curved
     \href{local_system.html#sbend}{sbend reference system}.  
\end{itemize} 

They are defined by the commands: 
\begin{verbatim}
SBEND, L = real, ANGLE = real, TILT = real, 
       K0 = real, K0S = real, K1 = real, K2 = real, 
       E1 = real, E2 = real, FINT = real, FINTX = real, 
       HGAP = real, H1 = real, H2 = real;

RBEND, L = real, ANGLE = real, TILT = real, 
       K0 = real, K0S = real, K1 = real, K2 = real, 
       E1 = real, E2 = real, FINT = real, FINTX = real,
       HGAP = real, H1 = real, H2 = real,
       ADD_ANGLE='array';
\end{verbatim} 

For both types, the following first-order attributes are permitted: 
\begin{itemize}
   \item L: The length of the magnet (default: 0 m). For a rectangular
     magnet the length is measured along a straight line  as in the
     Figure (internally the arc length is used), while for a  sector
     magnet it is the arc length of the reference orbit. \textbf{ To
       define an RBEND with the arc length (shorter straight length),
       the option RBARC=FALSE has to be set.} 

   \item ANGLE: The bend angle (default: 0 rad). A positive bend angle
     represents a bend to the right, i.e. towards negative \texttt{x}
     values.  
   
   \item ADD\_ANGLE: An array of (maximum 5) bending angles for multiple
     passes. See \texttt{add\_pass} option of the
     \href{sequence.html}{sequence} command.  

   \item TILT: The roll angle about the longitudinal axis (default: 0
     rad, i.e. a horizontal bend). A positive angle represents a
     clockwise rotation. A TILT=pi/2 turns a horizontal into a vertical
     bend, i.e. a positive bend ANGLE denotes a deflection
     down. \textbf{ Please note that contrary to MAD8 one has to specify
       the desired TILT angle, otherwise it is taken as 0 rad. This was
       needed to avoid the confusion in MAD8 about the actual meaning of
       the TILT attribute for various elements. } 

   \item \textbf{Please take note that K$_0$ and K$_0s$ are left in the
     data base but are no longer used for the MAP of the bends (but see
     below for what K$_0$ is being used), instead ANGLE and TILT are
     used exclusively. We believe that this will allow for a clearer and
     unambiguous definition, in particular in view of the upcoming
     integration of MAD-X with PTC which will allow a more general
     definition of bends.  However, it is required to specify k0 to
     assign RELATIVE field errors to a bending magnet since k0 is used
     for the normalization and NOT the ANGLE. (see EFCOMP).} 


% -<PRE>K0: 
% -The horizontal dipole coefficient 
% -<i>K<sub>0</sub> = (1 / B rho) B<sub>y</sub> </i>. 
% -The default is 0 m<sup>-1</sup>. 
% -A positive dipole strength is equivalent to a positive ANGLE which has the 
% -precedence over K<sub>0</sub>; however, if ANGLE is omitted,  
% -K<sub>0</sub>*L is used instead. 
% -K0S: 
% -The vertical dipole coefficient 
% -<i>K<sub>0s</sub> = (1 / B rho) B<sub>x</sub> </i>. 
% -The default is 0 m<sup>-1</sup>. A positive K0S deflects the closed 
% -orbit to positive y values.</PRE> 

   \item K1: The quadrupole coefficient \\
     \textit{K$_1$ = (1 / B rho) (del B$_y$ / del x)}. \\
     The default is 0 m$^{-2}$. A positive quadrupole strength implies
     horizontal focussing of positively charged particles. 

   \item E1: The rotation angle for the entrance pole face (default: 0 rad). 
   \item E2: The rotation angle for the exit pole face (default: 0 rad). 
   \item FINT: The field integral whose default value is 0. 
   \item FINTX: Allows (FINTX \textgreater 0)to set FINT at the element
     exit different from its entry value. In particular useful to switch
     it off (FINTX=0). 

   \item HGAP: The half gap of the magnet (default: 0 m). 
\end{itemize} 

The pole face rotation angles are referred to the magnet model for
\href{local_system.html#rbend}{rectangular bend} and
\href{local_system.html#sbend}{sector bend} respectively. The quantities
FINT and HGAP specify the finite extent of the fringe fields as defined
in \href{bibliography.html#slac75}{[SLAC-75]} There they are defined as
follows:  


%%\includegraphics{../equations/fint_hgap.gif}
\[
FINT=\int_{-\infty}^\infty \frac{B_y(s)(B_0-B_y(s))}{g \cdot B_0^2}\,\mathrm{d}s ,\quad\quad g=2\cdot HGAP.
\]

The default values of zero corresponds to the hard-edge approximation,
i.e. a rectangular field distribution. For other approximations, enter
the correct value of the half gap, and one of the following values for
FINT:  
\begin{verbatim}
Linear Field drop-off                     1/6
Clamped "Rogowski" fringing field         0.4
Unclamped "Rogowski" fringing field       0.7
"Square-edged" non-saturating magnet      0.45
\end{verbatim} 

Entering the keyword FINT alone sets the integral to 0.5. This is a
reasonable average of the above values.  The following second-order
attributes are permitted:  
\begin{itemize}
    \item K2: The sextupole coefficient \textit{K$_2$ = (1 / B rho)
      (del$^2$ B$_y$ / del x$^2$)}.  
    \item H1: The curvature of the entrance pole face (default: 0
      m$^{-1}$).  
    \item H2: The curvature of the exit pole face (default: 0
      m$^{-1}$). A positive pole face curvature induces a negative
      sextupole component; i.e. for positive H1 and H2 the centres of
      curvature of the pole faces are placed inside the magnet.  
\end{itemize} 

Examples: 
\begin{verbatim}
BR: RBEND, L = 5.5, ANGLE = +0.001;           // Deflection to the right
BD: SBEND, L = 5.5, K0S = +0.001/5.5;         // Deflection up
BL: SBEND, L = 5.5, K0 = -0.001/5.5;          // Deflection to the left
BU: SBEND, L = 5.5, K0S = -0.001;             // Deflection down
\end{verbatim}

%\href{http://www.cern.ch/Hans.Grote/hansg_sign.html}{hansg}, 
%\href{http://www.cern.ch/Frank.Schmidt/frs_sign.html}{frs}, August 28, 2003  
