%%\title{APERTURE AND TOLERANCES}
%  Changed by: Mark HAYES, 19-Sep-2002 
%  Changed by: Ivar Waarum, 24-Feb-2005 
 
\chapter{APERTURE}
\label{chap:aperture}

\section{Defining aperture}
\label{sec:def_aper}
The aperture for a particular element or class of elements can be set in MAD-X 
at the time of definition or instantiation of the element or class.

During tracking the particle excursion is then checked against the defined aperture and 
the particle is lost if it is outside the defined aperture.

The aperture can be specified for any element or class of elements, 
with the exception of drift spaces. 

The definition of the aperture takes the following form and parameters:\\
{\tt ..., APERTYPE=string,  APERTURE=\{values\}}

%% A new feature of MAD-X is the ability to set an aperture for a
%% particular  element, or parent of a set of elements. This removes the
%% need of placing a collimator next to every element to do aperture
%% tracking.  The aperture of any elements can be specified (excepts
%% drifts) by the use of the following parameters:  

\begin{madlist}
  \ttitem{APERTYPE} defines the aperture type from a set of seven
  preselected types:
    \begin{madlist}
      \ttitem{CIRCLE}
      \ttitem{RECTANGLE} 
      \ttitem{ELLIPSE} 
      \ttitem{RECTCIRCLE}, a superposition of a CIRCLE and
        a RECTANGLE 
      \ttitem{LHCSCREEN}, an alias for RECTCIRCLE
      \ttitem{RECTELLIPSE}, a superposition of an ELLIPSE and a
        RECTANGLE
      \ttitem{RACETRACK}, a circle exploded in four with
      quartercircles connected by straight lines.  
    \end{madlist}
    or from a provided file.
  \ttitem{APERTURE} is an array of values, the number and meaning 
of which depends on the APERTYPE.  
\end{madlist}
Note that the {\tt MARGUERITE} aperture type (two RECTCIRCLEs crossing
at right angle) is no longer supported. 

%http://en.wikibooks.org/wiki/LaTeX/Tables#Text_wrapping_in_tables
\begin{tabular}{|l | c | p{9cm}|}
\hline 
\textbf{APERTYPE} & \textbf{\# of values} & \textbf{meaning of
  values} \\  
\hline
CIRCLE & 1 &  radius of circle \\ 
\hline
ELLIPSE & 2 & horizontal an vertical semi-axes of ellipse \\  
\hline
RECTANGLE & 2 & half width and half height of rectangle\\ 
\hline
LHCSCREEN & 3 & half width and half height of rectangle, radius of circle\\  
\hline
MARGUERITE & 3 & half width and half height of rectangle, radius of circle\\  
\hline
RECTELLIPSE & 4 & half width and half height of rectangle, horizontal
and vertical semi-axes of ellipse \\  
\hline
RACETRACK & 3 & horizontal and vertical offsets (g and s) of the circle center,
radius (r) of the circular part (see figure below)\\  
\hline
filename & 0 & where the file contains a list of x and y coordinates
outlining the shape. This option is only supported by the aperture
module, see below. \\  
\hline
\end{tabular}

{\bf Examples}\\
The following sets an ELLIPTICAL aperture for the main
dipoles for the LHC.
\begin{verbatim}
MB: SBEND, L=l.MB, APERTYPE=ELLIPSE, APERTURE={0.02202,0.02202};
\end{verbatim} 

When the aperture is specified in a specific file the only aperture
parameter is the filename: 
\begin{verbatim}
MB: SBEND, L=l.MB, APERTYPE=myfile;
\end{verbatim} 

where "myfile" contains the list of x-y coordinates defining the
aperture shape:
\begin{verbatim}
x0   y0
xi   yi
...
xn   yn
\end{verbatim}

{\bf Notes}
\begin{itemize}
   \item There are inconsistencies in the parameter definition for the
     different aperture types. This is historical and is kept for
     backwards compatibility. Pay some attention to the parameters you
     introduce! 
   \item When \href{../makethin/makethin.html}{MAKETHIN} is called all
     thin slices inherit the aperture of their respective original thick
     lens version.  
   \item When the SIXTRACK command is called (see the SixTrack converter
     module \href{../c6t/c6t.html}{C6T}) the apertures are ignored by
     default. To convert the apertures as well the APERTURE flag has to
     be set in C6T.  
   \item  Aperture parameters are like all parameters and are inherited
     by derived elements. Like other parameters they can also be overridden by
     the derived elements if necessary.  
\end{itemize}

The APERTYPE and the APERTUREs themselves can be conveniently added to
the TWISS table (see \href{../twiss/twiss.html}{Twiss Module}) by using
the \href{select.html}{SELECT} command. For example the command:   
\begin{verbatim}
select, flag=twiss, clear;
select, flag=twiss, column= name,s,betx,alfx,mux,bety,alfy,muy,
                            apertype,aper_1,aper_2;
\end{verbatim}
and a subsequent TWISS command will put the aperture information
together with the specified TWISS parameters into the TWISS table.   

\section{Defining aperture tolerances}
A parameter closely connected to the aperture is the sum of the
mechanical and alignment tolerances. The mechanical tolerance is the
maximal error margin of errors in the element body which causes a
decrease of aperture, and the alignment tolerance is a mislignment of
the element in the accelerator, which also causes a decrease of
aperture. The tolerance is given in the transverse plane as a racetrack,
like in the picture below. 
\\
\includegraphics[width=450px]{Introduction/tolerance.jpg}
%\includegraphics{Introduction/tolerance.jpg align=center width=450}
\\ 
A tolerance can be assigned to each element in a MAD-X sequence as a vector: 
\begin{verbatim}
Syntax: APER_TOL = {r, g, s};

MB : SBEND, L := l.MB, APER_TOL={1.5, 1.1, 0};
\end{verbatim}

\section{APERTURE command}
{\bf The APERTURE module was developped specifically for the LHC.\\ 
Default parameter values are LHC values and the physics 
in APERTURE module assumes ultra-relativistic particles (v\(\approx\)c).} 

The APERTURE module computes the n1 values for a piece of machine. 
Each element is sliced into thick subelements at given intervals, and
the available aperture is computed at the end of each slice. 
The computation is based on the last Twiss table, so it is important to
run the 
\href{../twiss/twiss.html}{Twiss} and aperture commands on the same
period or sequence, see the aperture example below. Also showed in the
example is how n1 values can be \href{../plot/plot.html}{plotted}.   

The minimum n1 for each element is written to the last Twiss table, to
allow for \href{../match/match.html}{matching} by aperture.   
	
\begin{verbatim}
APERTURE, RANGE=range,
          EXN=real, EYN=real, DQF=real, DPARX=real, DPARY=real, 
          BETAQFX=real, BBEAT=real, DP=real, 
          COR=real,  NCO=integer, 
          HALO={real,real,real,real}, HALOFILE=filename,
          INTERVAL=real, SPEC=real, NOTSIMPLE=logical, 
          TRUEPROFILE=filename, OFFSETELEM=filename, 
          FILE=filename;  
\end{verbatim} 
where the parameters have the following meaning: 
\begin{madlist}
   \ttitem{RANGE} \href{../Introduction/ranges.html}{Range} given by
     elements. Default = \#s/\#e  
   \ttitem{EXN} Normalised horizontal emittance. Default = 3.75*e-6  
   \ttitem{EYN} Normalised vertical emittance. Default = 3.75*e-6 
   \ttitem{DQF} Peak linear dispersion [m]. Default = 2.086 
   \ttitem{DPARX} Fractional horizontal parasitic dispersion. Default = 0.273 
   \ttitem{DPARY} Fractional vertical parasitic dispersion. Default = 0.273 
   \ttitem{BETAQFX} Beta x in standard qf [m]. Default = 170.25 
   \ttitem{BBEAT} Beta beating coefficient applying to beam size. Default = 1.1 
   \ttitem{DP} Bucket edge at the current beam energy. Default = 0.0015 
   \ttitem{COR} Maximum radial closed orbit uncertainty [m]. Default = 0.004 
   \ttitem{NCO} Number of azimuth for radial scan. Default = 5 
   \ttitem{HALO} Halo parameters: \{n, r, h, v\}. n is the radius of the
     primary halo,  r is the radial part of the secondary halo, h and v
     is the horizontal and  vertical cuts in the secondary halo. Default
     = \{6, 8.4, 7.3, 7.3\}  
   \ttitem{HALOFILE} Input file with halo polygon coordinates. Will
     suppress  an eventual halo parameter. Default = none  
   \ttitem{INTERVAL} Approximate length in meters between
     measurements. Actual value:  nslice = nodelength/interval, nslice
     is rounded down to closest integer,  interval =
     nodelength/nslice. Default = 1.0  
   \ttitem{SPEC} Aperture spec, for plotting only. Gives the spec line in
     the plot. Default = 0.0  
   \ttitem{NOTSIMPLE} Use only if one or more beamscreens in the range are
     considered not to  be a "simply connex". Since all MAD-X apertypes
     are simply connex, this is only possible  if an input file with
     beam screen coordinates are given. See below for a graphical
     example. Default = false.  
   \ttitem{TRUEPROFILE} A file containing a list of magnets, and for each
     magnet a list of horizontal and vertical deviations from the ideal
     magnet axis. These values may come from measurements done on the
     magnet. See below for example. Default = none.  
   \ttitem{OFFSETELEM} A file containing a reference point in the machine,
     and a list of magnets with their offsets from this point described
     as a parabola. See below for example. Default = none. \\
     {\bf Note that the reference point should be within the range of
       elements given for the offsets to be taken into account.}
   \ttitem{FILE} Output file with aperture table. Default = none 
\end{madlist}


{\bf Important note regarding emittances:}\\
The APERTURE module sets the actual emittances from the normalized values
given as input using the definition\\
{\tt ex = exn / gamma}\\
while the BEAM command in MAD-X uses the definition\\
{\tt ex = exn / (4 * beta * gamma)}


\section{Not simply connex beam pipe profiles} 
Methodically, the algorithm for finding the largest possible halo is
fairly simple. The distance from halo centre to the first apex (i = 0)
in the halo is calculated (l\_i), and the equation for a line going
through these points is derived. This line is then compared with all
lines making the pipe polygon to find their respective intersection
coordinates. The distance h\_i between halo centre and intersection are
then divided by l\_i, to find the maximal ratio of enlargement, as seen
below. This procedure is then repeated for all apexes i in the halo
polygon, and the smallest ratio  of all apexes is the maximal
enlargement ratio for this halo to just touch the pipe at this
particular longitudinal position. 
\\
\includegraphics[width=420px]{Introduction/notsimple0.jpg}
%\includegraphics{notsimple0.jpg align=center width=420}
\\  
There is one complication to this solution; polygons which are not
simple connexes. (Geometrical definition of ``simply connex'': A figure
in which any two points can be connected by a line segment, with all
points on the segment inside the figure.) The figure below shows what
happens when a beam pipe polygon is not a simple connex. The halo is
expanded in such a way that it overlaps the external polygon in the area
where the latter is dented inwards. 
\\
\includegraphics[width=420px]{Introduction/notsimple1.jpg}
%%\includegraphics{notsimple1.jpg align=center width=420}
\\  
To make the module able to treat all kinds of polygons,
\textit{notsimple} must be activated. With this option activated, apexes
are strategically added to the halo polygon wherever the beam pipe
polygon might have an inward dent. This is done by drawing a line from
halo centre to each apex on the pipe polygon. An apex with its
coordinates on the intersection point line-halo is added to a table of
halo polygon apexes. The result is that the halo polygon has a few
``excessive'' points on straight sections, but the algorithm used for
expansion will now never miss a dent in the beam pipe. The use of the
notsimple option significantly increases computation time. 
\\
\includegraphics[width=420px]{Introduction/notsimple2.jpg}
%%\includegraphics{notsimple2.jpg align=center width=420}
\\

\section{Trueprofile file syntax}
This file contains magnet names, and their associated displacements of
the axis for  an arbitrary number of S, where So is the start of the
magnet and Sn the end. The interval between each S must be regular, and
X and Y  must be given in meters. The magnet name must be identical to
how it appears in the  sequence. The displacements are only valid for
this particular magnet, and cannot be  assigned to a family of
magnets. n1 is calculated for a number of slices determinated by the
number of Si. 
\\
{\bf Layout of file:}
\begin{verbatim}
magnet.name1
So   X   Y
Si   X   Y
Si   X   Y
Sn   X   Y

magnet.name2
So   X   Y
Si   X   Y
Si   X   Y
Sn   X   Y

etc.
\end{verbatim}

{\bf Example of file:}
\begin{verbatim}
!This is the start of the file.
!Comments are made with exclamation marks.

mb.a14r1.b1
0        0.0002        0.000004
7.15     1.4e-5        0.3e-3
14.3     0.0000000032  4e-6

!further comments can of course be added

mq.22r1.b1
0      0.3e-5     1.332e-4
1.033  0.00034    0.3e-9
2.066  0          0.00e-2
3.1    4.232e-4   0.00000003

!This is the end of the file.
\end{verbatim}

\section{OFFSETELEM file syntax}
This file contains parameters describing how certain elements are
actually located in space with respect to a given reference element in
the machine.  

The basis for this file is a pair of coordinate systems, \{s,x\} and \{s,y\} 
with the origin at the reference point. The units for s, x and y are
meters.

The actual location of the magnetic axis of a given element can be
described, in each plane, as a parabola defined with three coefficients: 
\begin{verbatim}
X_act(s) = DDX_OFF * s^2  +  DX_OFF * s  +  X_OFF
Y_act(s) = DDY_OFF * s^2  +  DY_OFF * s  +  Y_OFF
\end{verbatim} 

The reference position for the element, X\_ref(s) and Y\_ref(s), is calculated 
by MAD-X via an internal call to the \href{../survey/survey.html}{SURVEY}
module, taking the reference element as the origin. 

The resulting offset, in each plane, which is taken into account in the aperture calculation, 
is the difference between reference position and actual position: 
\begin{verbatim}
X_offset(s) = X_ref(s) - X_act(s) 
Y_offset(s) = Y_ref(s) - Y_act(s)
\end{verbatim} 

The offsets are only calculated for elements for which actual positions 
have been given through the OFFSETELEM file mechanism. 

The file must be given in TFS format according to the following template, 
with mandatory header and any number of data lines, one per element. 

\begin{verbatim}
@ NAME             %06s "OFFSET" 
@ TYPE             %06s "OFFSET" 
@ REFERENCE        %10s "reference-element-name" 
* NAME          S_IP       X_OFF     DX_OFF    DDX_OFF    Y_OFF    DY_OFF     DDY_OFF
$ %s            %le        %le       %le       %le        %le      %le        %le
"elementname"	real       real      real      real       real     real       real
\end{verbatim}

Note that the column S\_IP is actually not used, and the values are ignored, 
but the column and values are parsed nevertheless and must be present. 
The absence of this column will trigger an error. 

{\bf Example:}
\begin{verbatim}
@ NAME             %06s "OFFSET" 
@ TYPE             %06s "OFFSET" 
@ REFERENCE        %10s "mq.12r1.b1" 
* NAME          S_IP       X_OFF     DX_OFF    DDX_OFF    Y_OFF    DY_OFF     DDY_OFF
$ %s            %le        %le       %le       %le        %le      %le        %le
"mq.12r1.b1"	0.0        -3.0      -2.56545   0.0       0.0      -2.3443666 0.0
"mcbxa.3r2"     0.0        -0.84     32.443355  0.3323    0.0      32.554363  0.2522
\end{verbatim}

%A python script to convert a file from the old V.3.XX format to the new V4.xx can be found at :
%/afs/cern.ch/eng/lhc/optics/V6.503/aperture/convert_offsets.py
%usage : convert_offsets.py filename

As an example we see in the picture below how the horizontal axes of elements m1 and m2 
does not coincide with the reference trajectory.
\\
\includegraphics[width=450px]{Introduction/offsetelem.jpg}
%%\includegraphics{offsetelem.jpg align=center width=780}
\\ 

Note that prior to MAD-X version 4, the layout of the file was different and not formatted as TFS file:

\begin{verbatim}
reference-element-name

elementname
DDX_OFF   DX_OFF   X_OFF
DDY_OFF   DY_OFF   Y_OFF
\end{verbatim} 

{\bf Example:} 
\begin{verbatim}
!comment can be given anywhere with an exclamation mark

mq.12r1.b1

!Then we give a list of elements and their displacement 
!w.r.t. the reference point.
mcbxa.3l2
0   -2.56545   -3
0   -2.3443666  0

!The next element use the same reference point.
!Elements offset w.r.t. another point must be given in another file,
!together with the new reference point.
mcbxa.3r2
0.3323  32.443355 -0.84
0.2522  32.554363 0.0

!This is the end of the file.
\end{verbatim}


%% This file contains coordinates describing how certain elements are
%% displaced w.r.t. a  given reference point in the machine. It might be
%% used with elements in insertions, or other special-purpose elements that
%% has a magnet axis which does not coincide with the reference
%% trajectory. We operate with two coordinate system, s,x and s,y, where
%% the reference point is the origin and the actual element axis is
%% described as a parabola with coefficients A, B and C. For each element
%% we give two sets of coefficients, one for horizontal displacement and
%% one for vertical:  
%% \begin{verbatim}
%% X_offs(s) = Ax*s^2 + Bx*s + Cx 
%% \end{verbatim}
%% and 
%% \begin{verbatim}
%% Y_offs(s) = Ay*s^2 + By*s + Cy
%% \end{verbatim}.
%% The coordinate systems are in meters. 

%% {\bf Layout of file: --- FOR MADX VERSION 3.XX AND OLDER ONLY--- }
%% \begin{verbatim}
%% reference.point

%% magnet.name1
%% Ax   Bx   Cx
%% Ay   By   Cy

%% magnet.name2
%% Ax   Bx   Cx
%% Ay   By   Cy

%% etc.
%% \end{verbatim}

%% {\bf Example of file:}
%% \begin{verbatim}
%% !This is the start of the file.
%% !First we give a reference point. The origin of the 
%% !coordinate system will be at the START of this element.

%% mq.12r1.b1

%% !Then we give a list of elements and their displacement 
%% !w.r.t. the reference point.

%% mcbxa.3l2
%% 0   -2.56545   -3
%% 0   -2.3443666  0

%% !The next nodes use the same reference point.
%% !Elements offset w.r.t. another point must be given in another file,
%% !together with the new reference point.

%% mcbxa.3r2
%% 0.3323  32.443355 -0.84
%% 0.2522  32.554363 0.0

%% !This is the end of the file.
%% \end{verbatim}

%% {\bf Layout of file: --- FOR MADX VERSION 4.XX ONWARDS : now TFS format --- }\\ 
%% note that variable names changes with : Ax -\textgreater DDX\_OFF,   Bx
%% -\textgreater DX\_OFF,  Cx -\textgreater X\_OFF, same for Y The column
%% S\_IP is useless but mandatory (!). It results from a
%% misunderstanding. Content is ignored. In a future version, it will be
%% suppressed (but will not induce an error if present).  

%% % I aligned below lines by hand, do not touch them
%% \begin{verbatim}
%% @ NAME             %06s "OFFSET" 
%% @ TYPE             %06s "OFFSET" 
%% @ REFERENCE        %10s "mq.12r1.b1" 
%% * NAME         S_IP   X_OFF  DX_OFF     DDX_OFF   Y_OFF  DY_OFF      DDY_OFF
%% $ %s    %le    %le    %le    %le        %le       %le    %le'
%% "mq.12r1.b1"   0.0   -3.0    -2.56545   0.0       0.0    -2.3443666  0.0
%% "mcbxa.3r2"    0.0   -0.84   32.443355  0.3323    0.0    32.554363   0.2522
%% \end{verbatim}

%% %% A python script to convert a file from the old V.3.XX 
%% %% format to the new V4.xx can be found at :
%% %% /afs/cern.ch/eng/lhc/optics/V6.503/aperture/convert_offsets.py
%% %% usage : convert_offsets.py filename

%% As an example we see in the picture below how the horizontal axes of
%% elements m1 and m2 does not coincide with the reference trajectory.  
%% \\
%% \includegraphics[width=450px]{Introduction/offsetelem.jpg}
%% %%\includegraphics{offsetelem.jpg align=center width=780}
%% \\ 
%% The X\_ref(s) and Y\_ref(s) of the reference trajectory are
%% calculated via an internal call to the
%% \href{../survey/survey.html}{Survey} module. X\_offs(s) and Y\_offs(s)
%% are derived from the coefficients given in the file. The resulting  
%% \begin{verbatim}
%% X_tot(s) = X_ref(s) - X_offs(s)
%% \end{verbatim} 
%% and 
%% \begin{verbatim}
%% Y_tot(s) = Y_ref(s) - Y_offs(s)
%% \end{verbatim} 
%% are taken into account in the aperture calculations.  


\section{Aperture command example}
The aperture module needs a Twiss table to operate on. It is important
not to USE another period or sequence between the Twiss and aperture
module calls, else aperture looses its table. One can choose the ranges
for Twiss and aperture freely, they need not be the same.  

\begin{verbatim}
use, period=lhcb1;
select, flag=twiss,range=mb.a14r1.b1/mb.a17r1.b1,column=keyword,name,
parent,k0l,k1l,s,betx,bety,n1;
twiss, file=twiss.b1.data, betx=beta.ip1, bety=beta.ip1, x=+x.ip1, 
y=+y.ip1, py=+py.ip1;
plot,haxis=s,vaxis=betx,bety,colour=100;

select, flag=aperture, column=name,n1,x,dy;
aperture, range=mb.b14r1.b1/mb.a17r1.b1, spec=5.235;
plot,table=aperture,noline,vmin=0,vmax=10,haxis=s,vaxis=n1,spec,
on_elem,style=100;
\end{verbatim}

The \href{../Introduction/select.html}{select} command can be  used to
choose which columns to print in the output file.   
\\ Column names: name, n1, n1x\_m, n1y\_m, apertype, aper\_1, aper\_2,
aper\_3, aper\_4, rtol, xtol, ytol, s, betx, bety, dx, dy, x, y, on\_ap,
on\_elem, spec  

n1 is the maximum beam size in sigma, while n1x\_m and n1y\_m is the n1
values in si-units in the x- and y-direction.  

aper\_\# means for all apertypes but racetrack:
\\ aper\_1 = half width rectangle
\\ aper\_2 = half heigth rectangle
\\ aper\_3 = half horizontal axis ellipse (or radius if circle)
\\ aper\_4 = half vertical axis ellipse

For racetrack, the aperture parameters will have the same meaning as the
tolerances: 
\\ aper\_1 and xtol = horizontal displacement of radial part 
\\ aper\_2 and ytol = vertical displacement of radial part 
\\ aper\_3 and rtol = radius 
\\ aper\_4 = not used 

On\_elem indicates whether the node is an element or a drift, and on\_ap
whether it has a valid aperture. The Twiss parameters are the
interpolated  values used for aperture computation.  

When one wants to plot the aperture, the table=aperture parameter is
necessary. The normal line of hardware symbols along the top is not
compatible with the aperture table, so it is best to include
noline. Plot instead the column on\_elem along the vaxis to have a
simple picture of the hardware. Spec can be used for giving a limit
value for n1, to have something to compare with on the plot. This
example  provides a plot,  

\includegraphics[width=450px]{Introduction/aperexample.jpg}
%%\includegraphics{aperexample.jpg align=center width=740}

where we see the n1, beta functions and the hardware symbolized by
on\_elem.     


% Ivar Waarum, 24.02.05  -  Mark Hayes, 19.06.02 

