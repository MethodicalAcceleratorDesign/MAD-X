%%\title{Physical Units}
%  Changed by: Chris ISELIN, 27-Jan-1997 

%  Changed by: Hans Grote, 30-Sep-2002 

%%\usepackage{hyperref}
% commands generated by html2latex


%%\begin{document}
%%\begin{center}
 %%EUROPEAN ORGANIZATION FOR NUCLEAR RESEARCH 
%%\includegraphics{http://cern.ch/madx/icons/mx7_25.gif}

\subsection{Physical Units}
%%\end{center} 
 Throughout the computations MAD uses international (SI, Syst\`eme International) units. These units are summarised in the \hyperlink{table}{Units table}. 

%\href{table}{
\begin{table}[h]
\begin{center}
{\textbf{Table 1:} Physical Units}
\\
\begin{tabular}{l | l}
Length                  & m (metres) \\ 
Angle                   & rad (radians) \\ 
Quadrupole coefficient  & m**(-2) \\ 
Multipole coefficient, 2n poles   & m**(-n) \\ 
Electric voltage        & MV (Megavolts) \\ 
Electric field strength & MV/m \\ 
Frequency               & MHz (Megahertz) \\ 
Phase angles            & 2 pi \\ 
Particle energy         & GeV \\ 
Particle mass           & GeV/c**2 \\ 
Particle momentum       & GeV/c \\ 
Beam current            & A (Amperes) \\ 
Particle charge         & e (elementary charges) \\ 
Impedances              & MOhm (Megohms) \\ 
Emittances              & pi m mrad \\ 
RF power                & MW (Megawatts) \\ 
Higher mode loss factor & V/pc
% if you want to use \caption the tabular must be enclosed in
% table or figure environment
% \caption{\textbf{Table 1:} Physical Units}
\end{tabular}
\end{center}
\end{table}
%}


\href{http://www.cern.ch/Hans.Grote/hansg_sign.html}{hansg}, June 17, 2002 

%%\end{document}
