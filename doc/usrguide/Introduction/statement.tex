%%\title{the mad program}
%  Changed by: Chris ISELIN, 27-Mar-1997 

%  Changed by: Hans Grote, 25-Sep-2002 

%%\usepackage{hyperref}
% commands generated by html2latex


%%\begin{document}
%%\begin{center}
 %%EUROPEAN ORGANIZATION FOR NUCLEAR RESEARCH 
%%\includegraphics{http://cern.ch/madx/icons/mx7_25.gif}

\subsection{Statements} and 

\subsection{Comments}
%%\end{center} 
 Input for MAD-X follows in broad lines the new \href{http://cern.ch/mad9}{MAD-9} format, i.e. free format with commas "," as separators; however, outside \{...\} enclosures blanks may be used as separators. Blank input lines do not affect program execution. The input is not case sensitive except for strings enclosed in " ". 

 The input file consists of a sequence of commands, also known as statements. A statement may occupy any number of  input lines. It must be terminated by a semicolon, except if it contains a block of statements itself, like in 
\begin{verbatim}

if (a < 3) {a=b^2; b=2*b+4;}
\end{verbatim}

Several statements may be placed on the same line.
When a "!" or "//" is found on an input line,
the remaining characters of the line are skipped.
A line "/*" starts a comment region, it ends with a "*/" line.
The general format for a command is (items enclosed in /rep/ ... /rep/
can be repeated any number of times, including zero):

\begin{verbatim}

label: keyword /rep/,attribute/rep/ ;
\end{verbatim}
It has three parts:

\begin{itemize}
	\item A \href{label.html}{label}
is required for a definition statement.
It gives a name to the stored command.

	\item A \href{keyword.html}{keyword}
identifies the action desired.

	\item The \href{attribute HREF=attribute.html}{attributes}
are normally entered in the form
"attribute-name=attribute-value"
and serve to define data for the command, where:

\begin{itemize}
	\item \href{label.html}{attribute-name} selects the attribute,

	\item \href{attribute.html}{attribute-value} gives it a value.

\end{itemize}
\end{itemize}
If a value has to be assigned to an attribute, the
attribute name is mandatory.
For logical attributes it is sufficient to enter the name only.
The attribute is then given a default value taken from the
command dictionary.



Example: TILT attribute for various magnets.



The command attributes can have one of the following types:

\begin{itemize}
	\item \href{string.html}{String attribute},

	\item \href{logical.html}{Logical attribute},

	\item \href{integer.html}{Integer attribute},

	\item \href{real.html}{Real attribute},

	\item \href{expression.html}{Expression},

	\item \href{select.html}{Range selection},

\end{itemize}
Any integer or real attribute can be replaced by
a \href{expression.html}{real expression}; expressions are
normally deferred (see 
\href{expression.html#defer}{deferred expression}), except in the
definition of constants where they are evaluated immediately.
When a command has a \href{label.html}{label},
MAD-X keeps it in memory.
This allows repeated execution of the same command
by just entering EXEC label. This construct may be nested.
For an exhaustive list of valid declarations of constants or variables
see \href{declarations.html}{declarations}.


\href{http://www.cern.ch/Hans.Grote/hansg_sign.html}{hansg},
May 8, 2001
%\end{verbatim}


% add other files to the end of this file

%%%\title{Identifiers}
%  Changed by: Chris ISELIN, 24-Jan-1997 
%  Changed by: Hans Grote, 10-Jun-2002 

%\subsection{Identifiers or Labels}

\subsection{Keywords}

A keyword begins with a letter and consists of letters and digits. 

The MAD-X keywords are protected; using one of them as a label results
in a fatal error.   

% \href{http://www.cern.ch/Hans.Grote/hansg_sign.html}{hansg}, May 8, 2001 


%%%\title{Variable Declarations}
%  Changed by: Chris ISELIN, 24-Jan-1997 
%  Changed by: Hans Grote, 10-Jun-2002 

\subsection{Variable Declarations}
\label{subsec:var_declarations}

In the following, "=" means that the variable at the left receives the
current value of the expression at right, but does not depend on it any
further, whereas ":=" makes it depend on this expression, i.e. every
time the expression changes, the variable is re-evaluated, except for
"const" variables.  

\begin{verbatim}
var = expression;
var := expression;
real var = expression;        // identical
real var := expression;       // to above
int var = expression;         // truncated if expression is real
int var := expression;
const var = expression;
const var := expression;
const real var = expression;        // identical
const real var := expression;       // to above
const int var = expression;         // truncated if expression is real
const int var := expression;
\end{verbatim}

%\href{http://www.cern.ch/Hans.Grote/hansg_sign.html}{hansg}, May 8, 2001 


%\input{Introduction/select}


%%\end{document}
