%%\title{Integer Attributes}
%  Changed by: Chris ISELIN, 24-Jan-1997 

%  Changed by: Hans Grote, 10-Jun-2002 

%%\usepackage{hyperref}
% commands generated by html2latex


%%\begin{document}
%%\begin{center}
 %%EUROPEAN ORGANIZATION FOR NUCLEAR RESEARCH 
%%\includegraphics{http://cern.ch/madx/icons/mx7_25.gif}

\subsection{Integer Attributes}
%%\end{center}  
An integer attribute usually denotes a count. Example: 
\begin{verbatim}

myline:line=(-3*(a,b,c));
\end{verbatim} In this case, a litteral integer is requested; however, in the following 
\begin{verbatim}

rfc:rfcavity,harmon=12345;
\end{verbatim} or 
\begin{verbatim}

rfc:rfcavity,harmon=num;
\end{verbatim} "num" may be an integer variable, a real variable, or an expression  (in the two latter cases, the value is truncated).
\\
\href{http://www.cern.ch/Hans.Grote/hansg_sign.html}{hansg}, May 8, 2001 

%%\end{document}
