%%\title{Parameters}
%  Changed by: Hans Grote, 17-Jun-2002 


\chapter{Parameter Statements}
\label{chap:parameter}

\section{\href{relation}{Relations between Variable Parameters}}
 A relation is established between variables by the statement 
\begin{verbatim}
parameter-name = expression;
\end{verbatim} 
or 
\begin{verbatim}
parameter-name := expression;
\end{verbatim} 
The first form evaluates the expression at the right immediately and
assigns its value to the parameter. The second form assigns the value by
evaluating  the expression at right every time the parameter is actually
used. This holds as well for element parameters - beware ! If you want
to modify e.g. the strength of a quadrupole later (e.g. in a match,  or
by entering a new value for a parameter on which it depends) then the
defition has to be  
\begin{verbatim}
qd:quadrupole,k1:= ak1;
\end{verbatim} 
and not 
\begin{verbatim}
qd:quadrupole,k1 = ak1;
\end{verbatim} 
In the latter case, k1 will be set to the current value of ak1, and will
not change when ak1 changes.  

Parameters not yet defined have the value zero. 

Example: 
\begin{verbatim}
gev = 100;
beam,energy=gev;
\end{verbatim} 
the parameter on the left may appear on the right as well: 
\begin{verbatim}
x = x+1;
\end{verbatim} 
Increases the value of x by 1. As a result, the SET statement of MAD-8
is no longer necessary and is not implemented.  


Circular definitions are allowed in the first form:
\begin{verbatim}
a = b + 2;
b = a * b;
\end{verbatim}
However, circular definitions in the second form are forbidden:
\begin{verbatim}
a := b + 2;
b := a * b;
\end{verbatim}
will result in an error.


\section{\href{par_output}{VALUE: Output of Parameters}}
The VALUE statement
\begin{verbatim}
VALUE = expression;
\end{verbatim}
or
\begin{verbatim}
VALUE = expression1, expression2, ...;
\end{verbatim}
evaluates the current value of "expression" resp. "expression1" etc. and
prints the result on the standard output file.

Example:
\begin{verbatim}
p1 = 5;
p2 = 7;
value,p1*p2-3;
\end{verbatim}
After echoing the command, this prints:
\begin{verbatim}
p1*p2-3 = 32       ;
\end{verbatim}

Another example:
\begin{verbatim}
option,-warn;
while (x 
\end{verbatim}
?????

After echoing the command, this prints:
\begin{verbatim}
x = 100        ;       
x^2 = 10000      ;       
log10(x) = 2      ;            
\end{verbatim}

%\href{http://www.cern.ch/Hans.Grote/hansg_sign.html}{hansg} 11.9.2000

