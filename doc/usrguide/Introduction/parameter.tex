%%\title{Parameters}
%  Changed by: Hans Grote, 17-Jun-2002 


\chapter{Parameter Statements}
\label{chap:parameter}

\section{\href{relation}{Relations between Variable Parameters}}
 A relation is established between variables by one of two statements 
\madbox{
parameter-name = \= expression; \\
parameter-name := \> expression;
}
The first form evaluates the expression on the right immediately and
assigns its value to the parameter. The second form assigns the value by
evaluating  the expression on the right every time the parameter is actually
used. This holds as well for element parameters. 

Attention! If you want
to modify e.g. the strength of a quadrupole later (e.g. in a match,  or
by entering a new value for a parameter on which it depends) then the
defition has to be  
\madxmp{QD: QUADRUPOLE, K1 := ak1;}
and not 
\madxmp{QD: QUADRUPOLE, K1 = ak1;}
In the latter case, K1 will be set to the current value of ak1 at the
time of declaration, and will not change when ak1 later changes.  

Parameters not yet defined have a zero value.

Example: 
\madxmp{
gev = 100; \\
BEAM, ENERGY=gev;
}
the parameter on the left may appear on the right as well: 
\madxmp{x = x+1;}
increases the value of x by 1. As a result, the SET statement of
\madeight is no longer necessary and is not implemented in \madx.  


Circular definitions are allowed in the first form of relations
\madxmp{
a = b + 2; \\
b = a * b;
}
However, circular definitions in the second form or relations are
forbidden: 
\madxmp{
a := b + 2; \\
b := a * b;
}
will result in an error.


\section{\href{par_output}{VALUE: Output of Parameters}}
The VALUE statement
\madbox{
VALUE = expression {, expression2};
}
evaluates the current value of all listed expressions, and
prints the result on the standard output file.

Example:
\madxmp{
p1 = 5; \\
p2 = 7; \\
VALUE, p1*p2-3;
}
After echoing the command, this prints:
\madxmp{
p1*p2-3 = 32       ;
}

%% Another example:
%% \begin{verbatim}
%% option,-warn;
%% while (x 
%% \end{verbatim}
%% ?????

%% After echoing the command, this prints:
%% \begin{verbatim}
%% x = 100        ;       
%% x^2 = 10000      ;       
%% log10(x) = 2      ;            
%% \end{verbatim}

%\href{http://www.cern.ch/Hans.Grote/hansg_sign.html}{hansg} 11.9.2000

