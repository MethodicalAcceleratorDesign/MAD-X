%%\title{CRABCAVITY}
%  Added by: R. Calaga, Sep 2010 
%  Edited by: A. Latina, Jun 2013

\section{Crab Cavity}


\begin{verbatim}
label: CRABCAVITY, L = real, VOLT = real, LAG = real, FREQ = real,
                   RV1 = integer, RV2 = integer, RV3 = integer, RV4 = integer, 
                   RPH1 = integer, RPH2 = integer, 
                   LAGF = real, HARMON = integer;                 
\end{verbatim} 

%  BETRF=real, PG=real,
%  FREQ=real, SHUNT=real, TFILL=real; 

A CRABCAVITY has five real attributes and seven integer attributes: 

\begin{itemize}
  \item L: The length of the cavity (default: 0 m) 

  \item VOLT: The peak RF voltage (default: 0 MV). 

  \item LAG: The initial phase lag [2pi] (default: 0). 

  \item FREQ: The RF frequency [MHz] (no default). \\
    {\bf Note that if the RF frequency is not given, it is computed from the
    harmonic number and the revolution frequency \textit{f$_0$} as before. 
    However, for deflecting structures this makes no sense, and the 
    frequency is mandatory.} 

  \item RV1: Number of initial turns with zero voltage (default: 0). 
  \item RV2: Number of turns to ramp voltage from zero to nominal value (default: 0). 
  \item RV3: Number of turns with nominal voltage (default: 0). 
  \item RV4: Number of turns to ramp voltage from nominal value to zero (default: 0).  

  \item LAGF: Value of the final crab RF phase lag [2pi] (default: 0).

  \item RPH1: Number of initial turns with nominal phase (default: 0). 
  \item RPH2: Number of turns to ramp phase [2pi] from nominal to
    specified value \\ (default:~0). 

  \item HARMON: The harmonic number \textit{h} (no default). \\
    Only if the frequency is not given. 

% \item BETRF: RF coupling factor (default: 0).
% \item PG: The RF power per cavity (default: 0 MW).
% \item SHUNT: The relative shunt impedance (default: 0 MOhm/m).
% \item TFILL: The filling time of the cavity T<sub>fill</sub> (default: 0 microseconds). 
%\item EPHASE: Value of the final crab RF phase [2pi] with respect to  nominal value (default: 0). 
\end{itemize}


{\bf Caveats:}
\begin{itemize}
   \item \textit{ Please take note, that the following MAD8 attributes:
     BETRF, PG, SHUNT and TFILL are currently not implemented in MAD-X!}
   \item \textit{ Note that crab cavities are only implemented for
     tracking  purposes. \\ TWISS will ignore any effect of the crab cavity.  
% as well that twiss is 4D only. As a consequence the TWISS
% parameters in the plane of non-zero dispersion may not close as
% expected. Therefore, it is best to perform TWISS in 4D only, i.e. with
% cavities switched off. If 6D is needed one has to use the
% 
% <a href="../ptc_twiss/ptc_twiss.html">ptc_twiss</a> command. 
}
\end{itemize} 

%A cavity requires the particle energy (\href{beam.html#energy}{ENERGY})
%and the particle charge (\href{beam.html#charge}{CHARGE}) to be set by
%a \href{beam.html}{BEAM} command before any calculation is performed. 

Before any calculation is performed with a CRABCAVITY, the particle
energy (\href{beam.html#energy}{ENERGY}) and the particle charge
(\href{beam.html#charge}{CHARGE}) must be set by a
\href{beam.html}{BEAM} command.   

Then the effect of the CRABCAVITY on particle coordinates during tracking is
\\
\\ delta(\textit{px})  = VOLT * sin( PHI - OMEGA * t) 
\\ delta(\textit{E})  = -  VOLT * OMEGA * x * cos(PHI - OMEGA * t) 
\\ 
\\ where PHI =  2 $\pi$ * (LAG - HARMON * \textit{f$_0$ t}), 
\\ and OMEGA = 2 $\pi$ * FREQ / \textit{c}
\\
% delta(<i>E</i>) = VOLT * 
% sin(2 pi * (LAG - HARMON * <i>f<sub>0</sub> t</i>)). 


Example: 
\begin{verbatim}
BEAM, PARTICLE = PROTON, ENERGY = 7000.0;
CAVITY:  CRABCAVITY, L = 10.0, VOLT = 5.0, LAG = 0.0, FREQ = 400,
         RV1 = 0, RV2 = 50, RV3 = 1000, RV4 = 50, 
         RPH1 = 100, RPH2 = 500, LAGF = 0.125;
\end{verbatim} 

The \href{local_system.html#straight}{straight reference system} for a
cavity is a cartesian coordinate system.  
 
%\href{http://www.cern.ch/rcalaga}{R. Calaga}, September 2010
