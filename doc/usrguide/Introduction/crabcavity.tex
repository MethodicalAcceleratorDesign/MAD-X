%%\title{CRABCAVITY}
%  Added by: R. Calaga, Sep 2010 

%%\usepackage{hyperref}
% commands generated by html2latex


%%\begin{document}
%%\begin{center}
 %%EUROPEAN ORGANIZATION FOR NUCLEAR RESEARCH 
%%\includegraphics{http://cern.ch/madx/icons/mx7_25.gif}

\subsection{CRAB Cavity}
%%\end{center}
%  HARMON=integer, BETRF=real,PG=real,
%                  FREQ=real,SHUNT=real,TFILL=real; 
\begin{verbatim}


label: CRABCAVITY, L=real,VOLT=real,LAG=real,FREQ=real,
rv1=integer, rv2=integer, rv3=integer, rv4=integer, 
rph1=integer, rph2=integer,lagf=real;
                  

\end{verbatim} An CRABCAVITY has ten real attributes and seven integer attributes: 
\begin{itemize}
	\item L: The length of the cavity (default: 0 m) 
	\item VOLT: The peak RF voltage (default: 0 MV). The effect of the cavity is 

 delta(\textit{px})  = VOLT * sin( $\phi$ - \&omega * t) 
\\ delta(\textit{E})  = -  VOLT * \&omega * x * cos($\phi$ - \&omega * t) 
\\ (where, \&phi =  sin(2 $\pi$ * (LAG - HARMON * \textit{f$_0$ t}))) 
\\
% - delta(<i>E</i>) = VOLT * 
% sin(2 pi * (LAG - HARMON * <i>f<sub>0</sub> t</i>)). 



	\item LAG: The initial phase lag [2pi] (default: 0). 
	\item FREQ: The frequency [MHz] \textit{fenergy}(no default). Note that if the RF frequency is not given, it is computed from the harmonic number and the revolution frequency \textit{f$_0$} as before. However, for deflecting structures this makes no sense,  and the frequency is mandatory. 
\\
	\item RV1: Number of initial turns with zero voltage(default: 0). 
	\item RV2: Number of turns to ramp voltage from zero to nominal(default: 0). 
	\item RV3: Number of turns with nominal voltage (default: VOLT). 
	\item RV4: Number of turns to ramp voltage from nominal to zero(default: 0).  
\\
	\item RPH1: Number of initial turns with nominal phase (default: 0). 
	\item EPHASE: Value of the final crab RF phase [2pi] with respect to  nominal value (default: 0). 
	\item RPH2: Number of turns to ramp phase [2pi] from nominal to specified value(default: 0). 
\\
	\item HARMON: The harmonic number \textit{h} (no default). Only if the frequency is not given. 
%  <li>BETRF:
% RF coupling factor (default: 0).
% <li>PG:
% The RF power per cavity (default: 0 MW).
% <li>SHUNT:
% The relative shunt impedance (default: 0 MOhm/m).
% <li>TFILL:
% The filling time of the cavity T<sub>fill</sub>
% (default: 0 microseconds). 

	\item \textit{ Please take note, that the following MAD8 attributes: BETRF, PG, SHUNT and TFILL are currently not implemented in MAD-X!}
	\item \textit{ Note that crab cavities are only implemented for tracking  purposes. TWISS will ignore any effect of the crab cavity.  
%  as well that twiss is 4D only. As a consequence the TWISS
% parameters in the plane of non-zero dispersion may not close as
% expected. Therefore, it is best to perform TWISS in 4D only, i.e. with
% cavities switched off. If 6D is needed one has to use the
% 
% <a href="../ptc_twiss/ptc_twiss.html">ptc_twiss</a> command. 
}
\end{itemize} A cavity requires the particle energy (\href{beam.html#energy}{ENERGY}) and the particle charge (\href{beam.html#charge}{CHARGE}) to be set by a \href{beam.html}{BEAM} command before any calculations are performed. 

 Example: 
\begin{verbatim}

BEAM, PARTICLE=PROTON, ENERGY=7000.0;
CAVITY:  CRABCAVITY, L=10.0, VOLT=5.0, LAG=0.0, FREQ=400,
rv1=0, rv2=50, rv3=1000, rv4=50, rph1=100, rph2=500,lagf=0.125;

\end{verbatim} The \href{local_system.html#straight}{straight reference system} for a cavity is a cartesian coordinate system. 

\href{http://www.cern.ch/rcalaga}{R. Calaga}, September 2010 

%%\end{document}
