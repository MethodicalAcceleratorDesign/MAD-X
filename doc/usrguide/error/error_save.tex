%%\title{ESAVE}
%  Changed by: Hans Grote, 13-Sep-2000 
%  Changed by: Werner Herr, 19-Jun-2002 
%  Changed by: Hans Grote, 30-Sep-2002 

\section{Saving and reloading errors to and from tables and files} 
\label{sec:error_save}
% ESAVE: Save Machine Imperfections and read back from file}

\subsection{Writing errors to a file}

\begin{verbatim}
ESAVE, FILE=string;
\end{verbatim}

This command saves a table of errors assigned to elements on a file,
using a format which can be read in again to obtain the same
results. This allows dumping the errors and reloading them after a new
USE command. The range for these elements has to be specified. An error
save is requested by the statement  

Example: 
\begin{verbatim}
SELECT, FLAG=ERROR, RANGE=range, CLASS=name, PATTERN=string;
ESAVE, FILE=err.file;
\end{verbatim} 
and elements selected by the  \href{../Introduction/select.html}{SELECT}
command are saved to the file.  


To save the errors of all elements to a file, one can use: 
\begin{verbatim}
SELECT, FLAG = ERROR, FULL;                                    
ESAVE, FILE = err.file;
\end{verbatim}

{\bf Please note: in case of field errors, the absolute errors are
  saved and not relative errors. } 

\subsection{Reading errors from a table or file}

To assign errors from a file is not a priori straightforward. It may be
required to re-assign existing errors after a \textbf{USE} command was
executed (which deletes all errors attached to a sequence).  

Errors stored in the form of an internal table (\textit{errtab}) can  be
directly attached to the appropriate positions in the sequence with the
command:  

\begin{verbatim}
SETERR, TABLE=errtab;
\end{verbatim}
The table \textit{errtab} can be generated internally or from an
external file (\textit{errfile}) with the generic command READMYTABLE.  
 

The command sequence: 
\begin{verbatim}
READMYTABLE, file=errfile, table=errtab;
SETERR, TABLE=errtab;
\end{verbatim}
reads the file \textit{errfile} into the table \textit{errtab} and the
command SETERR attaches the errors to the elements in the active
sequence.  

The file \textit{errfile} can be produced by a preceding ESAVE command
or any other utility. It should follow the format of a file generated
with ESAVE (see example program). 

{\bf Please note:}
\begin{enumerate}
   \item To assign correctly the errors from the file to the elements in
     the sequence, all elements must have individual names, otherwise an
     identification is not possible. Elements in the file not identified
     in the active sequence are ignored.  
   \item Errors are assigned to ALL elements found in the file and the
     FLAG=ERROR is set. Therefore the number of elements selected
     corresponding to a command like:  
     \\ SELECT, FLAG=ERROR,...;
     \\ can be different after the execution of SETERR. 
\end{enumerate}

%\href{http://consult.cern.ch/xwho/people/1808}{Werner Herr} 18.6.2002 

