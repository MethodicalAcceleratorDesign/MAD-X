%%\title{EALIGN}
%  Changed by: Hans Grote, 13-Sep-2000 

%  Changed by: Werner Herr, 19-Jun-2002 

%  Changed by: Hans Grote, 19-Jun-2002 

%  Changed by: Werner Herr, 24-Jul-2002 

%  Changed by: Werner Herr, 02-Sep-2002 

%  Changed by: Hans Grote, 25-Sep-2002 

%%\usepackage{hyperref}
% commands generated by html2latex


%%\begin{document}

\section{EALIGN: Define Misalignments}  Alignment errors are defined by the EALIGN command. The misalignments refer to the \href{../Introduction/local_system.html}{local reference system} for a perfectly aligned machine. Possible misalignments are displacements along the three coordinate axes, and rotations about the coordinate axes. Alignment errors can be assigned to all beam elements except drift spaces. The effect of misalignments is treated in a linear approximation. A \href{read HREF=../Introduction/monitors.html}{Beam position monitor} can be given read errors in both horizontal and vertical planes. Monitor errors (MREX, MREY, MSCALX and MSCALY) are ignored for all other elements. Each new EALIGN statement replaces the misalignment errors for all elements in its range, unless EOPTION,ADD=TRUE has been entered. 

 Alignment errors are defined by the statement 
\begin{verbatim}

SELECT,FLAG=ERROR,RANGE=range,CLASS=name,PATTERN=string;
EALIGN, DX=real,DY=real,DS=real, 
        DPHI=real,DTHETA=real,DPSI=real, 
        MREX=real,MREY=real,
        MSCALX=real,MSCALY=real,
        AREX=real,AREY=real;
\end{verbatim} and elements are now selected by the \href{../Introduction/select.html}{SELECT} command. The attributes are: 

DX: The misalignment in the \textit{x}-direction for the entry of the beam element (default: 0 m). 
\\ DX$>$0 displaces the element in the positive \textit{x}-direction 

DY: The misalignment in the \textit{y}-direction for the entry of the beam element (default: 0 m). 
\\ DY$>$0 displaces the element in the positive \textit{y}-direction 

DS: The misalignment in the \textit{s}-direction for the entry of the beam element (default: 0 m). 
\\ DS$>$0 displaces the element in the positive \textit{s}-direction 

DPHI: The rotation around the \textit{x}-axis. 
\\ A positive angle gives a greater \textit{x}-coordinate for the exit than for the entry (default: 0 rad). 

DTHETA: The rotation around the \textit{y}-axis according to the right hand rule (default: 0 rad). 

DPSI: The rotation around the \textit{s}-axis according to the right hand rule (default: 0 rad). 

MREX: The horizontal read error for a monitor. This is ignored if the element is not a monitor 
\\ If MREX$>$0 the reading for \textit{x} is too high (default: 0 m). 

MREY: The vertical read error for a monitor. This is ignored if the element is not a monitor 
\\ If MREY$>$0, the reading for \textit{y} is too high (default: 0 m). 

AREX: The misalignment in the \textit{x}-direction for the entry of an aperture limit (default: 0 m). 
\\ AREX$>$0 displaces the element in the positive \textit{x}-direction 

AREY: The misalignment in the \textit{y}-direction for the entry of an aperture limit (default: 0 m). 
\\ AREY$>$0 displaces the element in the positive \textit{y}-direction 

MSCALX: The relative horizontal scaling error for a monitor. This is ignored if the element is not a monitor. 
\\ If MSCALX$>$0 the reading for \textit{x} is too high (default: 0). A value of 0.5 implies the actual reading is multiplied by 1.5. 

MSCALY: The relative vertical scaling error for a monitor. This is ignored if the element is not a monitor.  
\\ If MSCALY$>$0 the reading for \textit{y} is too high (default: 0). A value of -0.3 implies the actual reading is multiplied by 0.7. 
\\
\\
\\ Example: 
\begin{verbatim}

SELECT,FLAG=ERROR,CLASS=MQ;                  
EALIGN,DX=0.002,DY=0.0004*RANF(),DPHI=0.0002*GAUSS();
\end{verbatim} Assigns alignment errors to all elements of class MQ.           
\\
\begin{verbatim}

SELECT,FLAG=ERROR,PATTERN="QF.*";            
EALIGN,DX=0.001*TGAUSS(2.5),DY=0.0001*RANF();
\end{verbatim} Assigns alignment errors to all elements starting with "QF". TGAUSS(2.5) means a Gaussian distribution cut at 2.5 sigma. 
\\

%\href{xsdisp}{
\includegraphics{figures/xs_align.png}

\textbf{Figure 1:} Example of Misplacement in the (\textit{x, s})-plane. 

%\href{xydisp}{
\includegraphics{error/dpsi.png}

\textbf{Figure 2:} Example of Misplacement in the (\textit{x, y})-plane. 

%\href{ysdisp}{
\includegraphics{figures/ys_align.png}

\textbf{Figure 3:} Example of Misplacement in the (\textit{y, s})-plane. 

%\href{monitor}{
\includegraphics{figures/monitor_read.png}

\textbf{Figure 4:} Example of Read Errors in a monitor 
\\
\\
\\
\\\href{http://consult.cern.ch/xwho/people/1808}{Last updated:} 02.9.2002 \\\href{http://consult.cern.ch/xwho/people/1808}{Werner Herr} 18.6.2002 

%%\end{document}
