%%\title{Thin-Lens Tracking Module (thintrack)}
%  Created by: Andre VERDIER, 21-Jun-2002 
%  Changed by: Andre Verdier, 26-Jun-2002 
%  Changed by: Alexander Koschik, 07-Mar-2006 
%  Changed by: Alexander Koschik, 29-Mar-2006 
%  Changed by: Alexander Koschik, 02-Feb-2007 

\chapter{Thin-Lens Tracking} % Module (thintrack)}

The \textbf{thin-lens tracking module} of \madx performs element per
element tracking of one or several particle trajectories in the last
\href{../control/general.html#use}{\texttt{use}}d sequence.  
 
%  either for single passage (option <var class="option">onepass</var>)
%  or for many turns (default option).  

Only thin elements are allowed (apart from the element \texttt{drift}),
which guarantees the symplecticity of the coordinate transformation. Any
lattice can be converted into a "thin element" lattice by invoking the
\href{../makethin/makethin.html}{\texttt{makethin}} command. 

Several commands are actually required to complete a tracking run:

\begin{verbatim}
TRACK, onepass, deltap=real, dump; 
  ...
  START, x=real, px=real, y=real, py=real, t=real, pt=real;  
  ...
  RUN, turns=integer;
  ...
ENDTRACK;
\end{verbatim}

{\bf Commands:}
%TRACK, DELTAP=double, ONEPASS, DUMP, ONETABLE, FILE= string;  ! (MAD-X version 1)
%TRACK, DELTAP=double, ONEPASS, DAMP, QUANTUM, DUMP, ONETABLE, FILE=string; ! (MAD-X version 2)
%TRACK, DELTAP=double, ONEPASS, DAMP, QUANTUM, DUMP, APERTURE,ONETABLE, RECLOSS, FILE= string; ! (MADX version 3)
\begin{verbatim}
TRACK, DELTAP=real, ONEPASS, DAMP, QUANTUM, DUMP, APERTURE, ONETABLE, RECLOSS, FILE= string;
...
START, X=real, PX=real, Y=real, py=real, t=real, pt=real;
START, FX=double, PHIX=double, FY=double, PHIY=double, FT=double, PHIT=double;  
...
OBSERVE, PLACE=string;  
...
RUN, MAXAPER=double_array, TURNS=integer, FFILE=integer;  
...

ENDTRACK;
\end{verbatim}

{\bf Description}  \\
The \texttt{TRACK} command initiates trajectory tracking by entering the 
thin-lens tracking module. Several options can be specified, the most 
important being \texttt{dump}, \texttt{deltap} and
\texttt{aperture}. 
          
%  The options <var class="option">damp</var> and 
%  <var class="option">quantum</var> are not available in MAD-X versions 1.xx. 

Inside the block \texttt{TRACK}-\texttt{ENDTRACK} a series 
of initial trajectory coordinates can be specified by the \texttt{START} 
command (as many commands as trajectories). This will be usually done in a 
\texttt{while}-loop. \textbf{Note} that the coordinates are either 
\textbf{canonical} coordinates or \textbf{action-angle} variables!




For usual tracking (single/multi-turn), all coordinates are specified
with respect to the actual closed orbit (possibly off-momentum, with
magnet errors) and \textbf{NOT} with respect to the reference orbit. 

%  For <var class="option">onepass</var> tracking, the coordinates are specified with respect 
%                to the reference orbit. 
            
If the option \texttt{onepass} is used, the coordinates are specified
with respect to the reference orbit. The name "onepass" might be
misleading: Still tracking can be single- or multi-turn!   
            
The tracking is actually started with the \texttt{RUN} command, where
the option \texttt{turns} defines for how many turns the particles will
be tracked in the given sequence. 
          
If the option \texttt{dump} is used, the particle coordinates are
written to files at each turn. The output files are named
automatically. The name given by the user is followed by
.obsnnnn(observation point), followed by .pnnnn(particle number). Hence
filenames look like \texttt{track.obs0001.p0001}.  

Tracking is terminated by the command \texttt{ENDTRACK}.
          
{\bf Options:} 
	  
\begin{tabular}{c p{5cm} p{3cm} c}
  \hline 
  \textbf{Option} & \textbf{Meaning} & \textbf{Default Value} &
  \textbf{Value Type} \\  
  \hline
  DELTAP & relative momentum offset for reference closed orbit (switched
  off for onepass) &  0.0 & double \\  
  \hline
  ONEPASS & the sequence is treated as transfer line (no stability test,
  ie. no closed-orbit search) & .FALSE.= closed-orbit search & logical
  \\  
  \hline
  DAMP & introduce synchrotron damping (needs RF cavity, RADIATE in
  BEAM)  & .FALSE.= no damping & logical \\  
  \hline
  QUANTUM & introduce quantum excitation via random number generator and
  tables for photon emission & .FALSE.= no excitation & logical \\  
  \hline
  DUMP & write the particle coordinates in files (names generated
  automatically)  & .FALSE.= no file generated & logical \\  
  \hline
  APERTURE & particle is lost if its trajectory is outside the aperture
  of the current
  element. \hyperlink{track:remarks:aperture:notes}{Notes}. & .FALSE.=
  no aperture check & logical \\  
  \hline
  ONETABLE & write all particle coordinates in a single file & .FALSE.=
  one file per particle & logical \\  
  \hline
  RECLOSS & create a table named "trackloss" in memory with lost
  particles' coordinates & .FALSE.= no table & logical \\  
  \hline
  FILE & name for the track table   & "track", "trackone" & string \\ 
  \hline
  UPDATE & parameter update per turn   & .FALSE.= no update & string \\  
  \hline
\end{tabular}
          
{\bf Remarks}\\
\emph{IMPORTANT:} If an RF cavity has a no zero voltage, synchrotron
oscillations are automatically included. If tracking with constant
momentum is desired, then the voltage of the RF cavities has to be set
to zero. If an RF cavity has a no zero voltage and DELTAP is non zero, 
tracking is done with synchrotron oscillations around an off-momentum
closed orbit.
          
          
\href{track:remarks:deltap:notes}{DELTAP}
Defining a non-zero \texttt{deltap} results in a change of the beam
momentum/energy without changing the magnetic properties in the
sequence. This leads to a new closed orbit, the off-momentum closed
orbit. Particle coordinates are then given with respect to this new
closed orbit, unless the option \texttt{onepass} is used!


\href{track:remarks:onepass:notes}{ONEPASS}
If the option \texttt{onepass} is used, no closed orbit is searched,
which also means that no stability test is done. Use this option if you
want to get the particles' coordinates with respect to the reference
orbit rather than the closed orbit. Unfortunately the name is
misleading, but for backwards compatibility it is kept. "onepass" does
\textbf{NOT} restrict the tracking to one turn only! 
          
\href{track:remarks:aperture:notes}{APERTURE}
\begin{itemize}
    \item If the \texttt{aperture} option is applied, the \texttt{apertype} 
      and \texttt{aperture} information of each element in the sequence
      is used to check whether the particle is lost or not. For further
      information on the definition of apertures and different aperture
      types, see the documentation of the
      \href{../Introduction/aperture.html}{\texttt{APERTURE}} module. 
      
    \item In case no aperture information was specified for an element, 
      the following procedure will currently take place:
      \\
      $\rightarrow$ No aperture definition for element $\rightarrow$ 
      Default apertype/aperture assigned (currently this is 
      
      \texttt{apertype= circle, aperture = \{0\}}) 
      \\ $\rightarrow$  
      If tracking with \texttt{aperture} is used and an
      element with 
      \texttt{apertype= circle} AND 
      \texttt{aperture= \{0\}} 
      is encountered, then the first value of the \texttt{maxaper} vector
      is assigned as the circle's radius (no permanent assignment!). 
      See option \hyperlink{run}{\texttt{maxaper}} for 
      the default values. 
      \\ $\Rightarrow$
      Hence even if no aperture information is specified by the user for
      certain elements, default values will be used! 
\end{itemize}
          
\href{track:remarks:recloss:notes}{RECLOSS}
Traditionally, when a particle is lost on the aperture, this information
is written to stdout. To allow more flexible tracking studies, the lost
particles' coordinates and further information can also be saved in a
table in memory. Usually one would save this table to a file using the
\texttt{WRITE} command after the tracking run has finished. The
following information is available in the TFS table "trackloss": 
          
\begin{itemize}
   \item Particle ID (number)
   \item Turn number
   \item Particle coordinates (x,px,y,py,t,pt)
   \item Longitudinal position in the machine (s)
   \item Beam energy
   \item Element name, where the particle is lost
\end{itemize}
          
\href{track:remarks:update:notes}{UPDATE}
Changed behaviour for time variation in tracking. Use
track command option 'update' (e.g.: 'track, onepass,
update;') to use the following additions: 

\begin{itemize}
   \item  Introduced special variable ('tr\$turni') that can be
     used in expressions like 'KICK:= sin(tr\$turni)' and is updated at
     each turn during tracking. 
     
   \item  Introduced special macro ('tr\$macro') that can be
     user-defined ('tr\$macro(turn): macro = \{whatever
     depending on turnnumber;\};') and is executed/updated at each turn
     during tracking. (Macro is necessary e.g. for table access.)
     
\end{itemize}

\section{START}          
\begin{verbatim}
START, X=double, PX=double, Y=double, py=double, t=double, pt=double;
START, FX=double, PHIX=double, FY=double, PHIY=double, FT=double, PHIT=double;  
\end{verbatim}
          
{\bf Description} \\
After the \texttt{TRACK} command, a series of initial trajectory
coordinates has to be given by means of a \texttt{START} command (as
many commands as trajectories). The coordinates can be either
\href{../Introduction/tables.html#canon}{\textbf{canonical}}
coordinates,  
\\
\\\textbf{START, X= double, PX= double, Y= double, PY= double, T= double, PT= double; }
\\
\\
or \textbf{action-angle} coordinates,
\\
\\\textbf{START, FX= double, PHIX= double, FY= double, PHIY= double, FT= double, PHIT= double; }
\\
\\
For this case the normalised amplitudes are expressed in number 
of r.m.s. beam size F$_X$, F$_Y$, F$_T$ (the actions being computed with
the  emittances in the \texttt{BEAM} command) \textbf{in each mode
  plane}. The phases are PHI$_X$, PHI$_Y$ and PHI$_T$ expressed in
radian. In the uncoupled case, we have in the plane mode labelled z,  
\\
\\
Z = F$_z$ sqrt(E$_z$) cos(PHI$_z$),    
P$_z$= F$_z$ sqrt(E$_z$) sin(PHI$_z$), 
\\
\\ 
where E$_z$ is the r.m.s. emittance in the plane Z.

{\bf Options}
\begin{tabular}{p{3cm} cccc}
   \hline 
   \textbf{Option} & \textbf{Meaning} & \textbf{Default Value} & \textbf{Value Type} & \textbf{Unit} \\ 
   \hline
   X, PX, Y, PY, T, PT & canonical coordinates & 0.0 & double & m \\ 
   \hline
   FX, PHIX, FY, PHIY, FT, PHIT & action-angle coordinates & 0.0 & double & rad \\ 
   \hline
\end{tabular}

{\bf Remarks} 
\begin{itemize}
   \item For usual tracking (single/multi-turn), all coordinates are
     specified with respect to the actual closed orbit (possibly
     off-momentum, with magnet errors) and \textbf{NOT} with respect to
     the reference orbit. 
   \item If the option \texttt{onepass} is used, the coordinates are
     specified with respect to the reference orbit. The name "onepass"
     might be misleading: Still tracking can be single- or multi-turn!   
     %  For <var class="option">onepass</var> tracking, the coordinates
     %  are specified with respect to the reference orbit.  
            
\end{itemize}
          

\section{OBSERVE}
\begin{verbatim}
OBSERVE, PLACE=string;  
\end{verbatim}


{\bf Description} \\
Coordinates can be recorded at places that have names.  Such observation
points are specified by the command \texttt{OBSERVE} (as many commands
as places). The output files are named automatically. The name given by
the user is followed by .obsnnnn(observation point), followed by
.pnnnn(particle number). Hence filenames look like
\texttt{track.obs0001.p0001}.  
     
{\bf Options} 
\begin{tabular}{cccc}
   \hline 
   \textbf{Option} & \textbf{Meaning} & \textbf{Default Value} & \textbf{Value Type} \\ 
   \hline
   PLACE & name of the observation point &  & string \\ 
   \hline   
\end{tabular}

{\bf Remarks}\\ 
If no \texttt{OBSERVE} command is given, but the \texttt{dump} option in
the \texttt{TRACK} command is used, the particles trajectory coordinates
are still recorded. The observation point is then the starting point of
the sequence. 
     

\section{RUN}
\begin{verbatim}
RUN, MAXAPER=double_array, TURNS=integer, FFILE=integer;  
\end{verbatim}

  
{\bf Description}\\
The actual tracking itself is launched by the \texttt{RUN} command. Via
the option \texttt{turns} the user can specify how many turns will be
tracked. 
     
{\bf Options} 
     
\begin{tabular}{cccc}
   \hline 
   \textbf{Option} & \textbf{Meaning} & \textbf{Default Value} & \textbf{Value Type} \\ 
   \hline
   MAXAPER & upper limits for the six coordinates & \{0.1, 0.01, 0.1, 0.01, 1.0, 0.1\} & double array \\ 
   \hline
   TURNS & number of turns  & 1 & integer \\ 
   \hline
   FFILE & periodicity for printing coordinates  & 1 & integer \\ 
   \hline   
\end{tabular}

{\bf Remarks}\\ 
The limits defined by the \texttt{maxaper} option are only being taken
into account if the \texttt{aperture} option of the \texttt{TRACK}
command is used. 
     

\section{Remarks}
\begin{itemize}
   \item Plotting is possible in \madx, however it can also be done
     externally by using the files created by \texttt{TRACK}. 

   \item The following internal tables are created while tracking:\\
     \texttt{tracksumm}, 
     \texttt{trackloss}, and 
     \texttt{trackone} or 
     \texttt{track.obs\$\$\$\$.p\$\$\$\$} 
     (depending on option \texttt{onetable}). \\
     These internal tables can be accessed via the
     \href{../Introduction/expression.html#table}{\texttt{table}}-access
     functions.      
\end{itemize}

{\bf See Also:}\\
\href{../Introduction/aperture.html}{\texttt{APERTURE}}, 
\href{../makethin/makethin.html}{\texttt{MAKETHIN}}



%\href{http://consult.cern.ch/xwho/people/74251}{A. Koschik},  February  2007
