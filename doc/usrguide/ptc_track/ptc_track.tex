%\documentclass[a4paper,11pt]{article}
%%\usepackage{ulem}
%%\usepackage{a4wide}
%%\usepackage[dvipsnames,svgnames]{xcolor}
%%\usepackage[pdftex]{graphicx}
%%\title{Thick-Lens Tracking Module (PTC-TRACK Module)}
%  Created by: Valery KAPIN, 06-Apr-2006 

%  Changed by: ____________, ___________ 

% 
% div.Section1
% 	{page:Section1;}
% span.SpellE
% 	{}
% h4
% 	{margin-right:0cm;
% 	margin-left:0cm;
% 	text-indent:35.45pt;
% 	font-size:12.0pt;
% 	font-family:"Times New Roman";
% 	font-weight:bold}
% table.MsoTableGrid
% 	{border:1.0pt solid windowtext;
% 	text-indent:35.45pt;
% 	font-size:10.0pt;
% 	font-family:"Times New Roman";
% 	}
% 

%%\usepackage{hyperref}
% commands generated by html2latex


%%\begin{document}
%%\begin{center}

   %%EUROPEAN ORGANIZATION FOR NUCLEAR RESEARCH
   
%%\includegraphics{http://cern.ch/madx/icons/mx7_25.gif}

\section{Thick-Lens Tracking Module 
\\
  (PTC-TRACK Module)}
%%\end{center}


 The \textbf{PTC-TRACK module }[\hyperlink{V._Kapin}{a}] is the symplectic 
 thick-lens tracking facility in MAD-X [\hyperlink{F._Schmidt}{b}]. It is based 
 on PTC library written by E.Forest [\hyperlink{E._Forest}{c}]. The 
 commands of this module are described below, optional parameters are 
 denoted by square brackets ([ ]). Prior to using this module 
 the active beam line must be selected by means of a
 \href{../control/general.html#use}{USE} command. 
 The general \href{../ptc_general/ptc_general.html}{PTC 
 environment} must also be initialized. 

\line(1,0){300}

\begin{description}
	\item[Synopsis] \textit{
\texttt{PTC\_CREATE\_UNIVERSE;}}\textit{
\texttt{PTC\_CREATE\_LAYOUT, model=integer,method=integer, 
   nst=integer, [exact];}}
\texttt{..........................
\\PTC\_START, .....;
\\..........................
\\PTC\_OBSERVE,....;
\\..........................
\\PTC\_TRACK, .....;
\\..........................
\\PTC\_TRACK\_END;}
\texttt{..............................}\textit{
\texttt{PTC\_END; }}
\end{description}
\begin{description}
	\item[Commands]
	\item{\textbf{PTC\_START}, \textbf{
\\ x}=double, 
     \textbf{px}=double, \textbf{y}=double, \textbf{py}=double,
   \textbf{t}=double, \textbf{pt}=double,
\\ \textbf{fx}=double, 
     \textbf{phix}=double, \textbf{fy}=double,
   \textbf{phiy}=double, \textbf{ft}=double, \textbf{phit}=double 
   ;}

	\item[Description] To start particle tracking, a series of initial trajectory 
     coordinates has to be given by means of \texttt{PTC\_START 
       }command (as many commands as trajectories). It 
     must be done before the \texttt{\hyperlink{PTC_TRACK}{PTC\_TRACK}}command. The coordinates can be either
       \href{../Introduction/tables.html#canon}{canonical coordinates} (\textbf{x}, 
       \textbf{px}, \textbf{y}, \textbf{py},
     \textbf{t}, \textbf{pt}) or action-angle coordinates (\textbf{fx}, \textbf{phix}, 
       \textbf{fy}, \textbf{phiy}, \textbf{ft}, \textbf{phit}), which 
     are expressed by the normalized amplitude, \textit{F}$_\textit{z }$and the phase, 
       \textit{$\Phi$}$_\textit{z}$ 
     for the \textit{z}-th mode plane (\textit{z}=\{\textit{x},\textit{y},\textit{t}\}). 
     The actions are computed with the values of the emittances, \textit{F}$_\textit{z}$, 
     which must be specified in the preceding
       \href{../Introduction/beam.html}{BEAM} command. 
       \textit{F}$_\textit{z}$ are expressed in 
     number of r.m.s. beam sizes and
       \textit{$\Phi$}$_\textit{z}$ 
     are expressed in radians.
	\item[Options] 
	\text{ \\}	
	

\begin{tabular}{cc p{1.5cm} p{1.5cm}}
\hline 
\textbf{Option} & \textbf{Meaning} & \textbf{Default Value} & \textbf{Value Type} \\ 
\hline
X, PX, Y, PY, T, PT & canonical coordinates & 0.0 & double \\ 
\hline
FX, PHIX, FY, PHIY, FT, PHIT & action-angle coordinates & 0.0 & double \\ 
\hline
\end{tabular}

	\item[Remarks] 1. If the option \texttt{\hyperlink{CLOSED_ORBIT}{closed\_orbit}}in the \texttt{\hyperlink{PTC_TRACK}{PTC\_TRACK}} 
     command is active (see below), all coordinates are specified with 
     respect to the actual closed orbit (possibly off-momentum with 
     magnet errors) and NOT with respect to the reference orbit. If 
     the option \texttt{\hyperlink{CLOSED_ORBIT}{
       closed\_orbit}}is absent, then coordinates are specified with respect 
     to the reference orbit.2. In the uncoupled case, the canonical 
     and the action-angle variables are related with equations
       \textit{
\\
       z}=
     \textit{F$_z$}(\textit{E$_z$})$^1/2$cos(\textit{$\Phi$}$_\textit{z}$),
       \textit{p$_z$}= \textit{F$_z$}(\textit{E$_z$})$^1/2$sin(\textit{$\Phi$}$_\textit{z}$).

3. The use of the action-angle coordinates 
       requires the option \texttt{\hyperlink{CLOSED_ORBIT}{closed\_orbit}}in the \texttt{\hyperlink{PTC_TRACK}{PTC\_TRACK}} 
       command. 

4. If both the canonical and the action-angle coordinates are 
     given in the \texttt{PTC\_START }command, they 
     are summed after conversion of the action-angle coordinates to 
     the canonical ones.
\end{description}
\begin{description}
	\item[\textbf{PTC\_OBSERVE, 
\\ place=}string; 
     ] 
	\item[Description] Besides of the beginning of the beam-line, one can 
       define an additional observation points along the 
       machine. Subsequent \texttt{PTC\_TRACK }
       command will then record the tracking data on all these 
       observation points. 
	\item[Option] 
	\text{ \\}	

\begin{tabular}{c p{5cm} c}
\hline 
\textbf{Option} & \textbf{Meaning} & \textbf{Value Type} \\ 
\hline
PLACE & name of observation point (markers are very much preferred) & string \\ 
\hline
\end{tabular}

	\item[Remarks] 1. The first observation point at the beginning of 
       the beam-line is marked as \textbf{"start"}. 
       
       2. It is recommended to use
       \href{../Introduction/label.html}{labels} of
       \href{../Introduction/marker.html}{markers} in order to avoid usage observations at the ends of 
     thick elements.
\\
\\
       3. The data at the observation points other than at \textbf{
       "start"} can be produced by two different means: 
\\
       a) traditional (\href{../thintrack/thintrack.html}{MADX}) element-by-element tracking (use 
       option \hyperlink{ELEMENT_BY_ELEMENT}{element\_by\_element});
       
       b) coordinate transformation from \textbf{"start"} to the 
       respective observation point using high-order PTC 
       transfer maps 
\\
       (required option \texttt{\hyperlink{CLOSED_ORBIT}{closed\_orbit}}; 
       turned off options \hyperlink{RADIATION}{radiation} 
       and \hyperlink{ELEMENT_BY_ELEMENT}{element\_by\_element}).
       
\end{description}

\begin{description}
	\item {\textbf{\href{PTC_TRACK}{PTC\_TRACK}, 
\\ deltap=}double\textbf{, icase=}integer\textbf{, closed\_orbit, 
   element\_by\_element, turns=}integer\textbf{, 
\\ dump, onetable, maxaper=double array, norm=}integer\textbf{, 
   norm\_out, 
\\ file[=}string\textbf{], extension=}string\textbf{, ffile=}integer\textbf{,
\\ radiation, radiation\_model1, radiation\_energy\_loss, 
   radiation\_quadr,
\\ beam\_envelope, space\_charge;}}

	\item[Description] The \texttt{PTC\_TRACK} command initiates 
     trajectory tracking by entering the thick-lens tracking module. 
     Several options can be specified, the most important are 
     presented in table "Basic Options". There are also switches to 
     use special modules for particular tasks. They are presented in 
     the table "Special Switches".The tracking can be done element-by-element using the option
       \hyperlink{ELEMENT_BY_ELEMENT}{element-by-element}, or 
     "turn-by-turn" (default) with coordinate transformations over the whole 
     turn. Tracking 
     is done in parallel, i.e. the coordinates of all particles are 
     transformed through each beam element (option
       \hyperlink{ELEMENT_BY_ELEMENT}{element-by-element}) or 
       over full turns.The particle is lost if its trajectory is outside the boundaries 
       as specified by \hyperlink{MAXAPER}{maxaper} option. 
       In PTC, there is a continuous check, if the particle 
       trajectories stays within the aperture limits. 
\\
       The Normal Form calculations (required option
       \hyperlink{CLOSED_ORBIT}{closed\_orbit}) is controlled 
       by \hyperlink{NORM_NO}{norm\_no} and
       \hyperlink{NORM_OUT}{norm\_out} 
     are used.
	\item[Basic Options] 
	\text{ \\}	
		
\begin{longtable}{l p{5cm} p{2cm} p{2cm}}
\hline 
\textbf{Option} & \textbf{Meaning} & \textbf{Default Value} & \textbf{Value}\\
\hline
ICASE & user-defined dimensionality of the phase-space (4, 5 or 6). & 4 & integer \\ 
\hline
DELTAP & relative momentum offset for reference closed orbit (used for 5D case ONLY).  & 0.0 & double \\ 
\hline
CLOSED\_ORBIT & switch to turn on the closed orbit calculation & .FALSE. & logical \\ 
\hline
ELEMENT\_BY\_ELEMENT & switch from the default turn-by-turn tracking to the element-by-element tracking. & .FALSE. & logical \\ 
\hline
TURNS & number of turns to be tracked. & 1 & integer \\ 
\hline
DUMP & enforces writing of particle coordinates to formatted text files & .FALSE. & logical \\ 
\hline
ONETABLE & writing all particle coordinates to a single file  & .FALSE. & logical \\ 
\hline
MAXAPER & upper limits for the particle coordinates.  &\{0.1,0.01,0.1, 0.01,1.0,0.1\} & double,
array (1:6) \\ 
\hline
NORM\_NO & order of the Normal Form & 1 & integer \\ 
\hline
NORM\_OUT & switch to transform canonical variables to action-angle variables & .FALSE. & logical \\ 
\hline
multirows{2}{*}{FILE} 
& 
	\begin{tabular}{ll}
	omitted & no output written to a file
	\end{tabular}
 &  &  \\ 
& 	
	\begin{tabular}{ll}
	present &  	file name for printing
the track tables.
	\end{tabular}
 & track & string \\ 
\hline
EXTENSION & the extension of filename for the track table, e.g., txt, doc etc & none & logical \\ 
\hline
FFILE & printing coordinates after every FFILE turns & 1 & integer \\ 
\hline
\end{longtable}

	\item[Remarks] \textbf{ICASE}:has a highest 
       priority over other options: 
\\
  a) RF cavity with non-zero voltage will be ignored for
       \hyperlink{ICASE}{icase}=4, 5;
\\
  b) A non-zero \hyperlink{DELTAP}{deltap} will be ignored 
       for \hyperlink{ICASE}{icase}=4, 6.
\\
       However, if RF cavity has the voltage set to zero and 
       for \hyperlink{ICASE}{icase}=6, the code sets
       \hyperlink{ICASE}{icase}=4.
\\
\\\textbf{DELTAP: }is 
       ignored for \hyperlink{ICASE}{icase}=6, but the option
       \href{../ptc_general/ptc_general.html}{offset\_deltap} of the command
       \texttt{PTC\_CREATE\_LAYOUT} may 
       be used, if 
       the reference particle should 
       have an momentum off-set as specified by
       \href{../ptc_general/ptc_general.html}{offset\_deltap}.
\\\textbf{CLOSED\_ORBIT} : It must 
       be used for closed rings only. This option allows to 
       switch ON 
\\
       the Normal Form analysis, if required. If CLOSED\_ORBIT is off, the sequence is 
     treated as a transfer line.
\\
\\\textbf{NORM\_NO=1}: makes the 
       Normal Form linear (always true for MAD8/X).
\\
\\\textbf{FILE}: The output file endings are:
       .obsnnnn(observation 
     point), followed by .pnnnn 
     (particle number), 
\\
       if the \hyperlink{ONETABLE}{onetable} option is not 
       used.
\end{description}
\begin{description}
	\item[Special Switches] 
	\text{ \\}	
	
\begin{tabular}{l p{5cm} p{2cm} p{2cm}}
\hline 
\textbf{Option} & \textbf{Meaning} & \textbf{Default Value} & \textbf{Value Type} \\ 
\hline
RADIATION & turn on the synchrotron radiation calculated by an internal procedure of PTC & .FALSE. & logical \\ 
\hline
RADIATION\_MODEL1 & switch to turn on the radiation according to the method given in the Ref. [\hyperlink{G.J._Roy}{d}] & .FALSE. & logical \\ 
\hline
RADIATION\_ENERGY\_LOSS & adds the energy loss for \hyperlink{RADIATION_MODEL1}{radiation\_model1} & .FALSE. & logical \\ 
\hline
RADIATION\_QUADR & adds the radiation in quadrupoles. It supplements either\hyperlink{RADIATION}{radiation}, \hyperlink{RADIATION_MODEL1}{radiation\_model1} & .FALSE. & logical \\ 
\hline
BEAM\_ENVELOPE & turn on the calculations of the beam envelope with PTC  & .FALSE. & logical \\ 
\hline
SPACE\_CHARGE (under construction) & turn on the simulations of the space charge forces between particles.  & .FALSE. & logical \\ 
\hline
\end{tabular}

	\item[Remarks] \textbf{1. RADIATION: }Has precedence
       \hyperlink{RADIATION_MODEL1}{radiation model1.}
\\
\\\textbf{2. }\textbf{RADIATION\_MODEL1}: Additional 
       module by F. Zimmermann. The model simulates 
       quantum excitation via a random number generator and tables for 
     photon emission. It can be used only with the element-by-element 
     tracking (option \hyperlink{ELEMENT_BY_ELEMENT}{element-by-element}).
\\
\\\textbf{3. RADIATION\_ENERGY\_LOSS}: Of use for
       \hyperlink{RADIATION_MODEL1}{radiation\_model1}.
\\
\\\textbf{4. BEAM\_ENVELOPE:} It requires the options
       \hyperlink{RADIATION}{radiation} and \hyperlink{ICASE}{icase}=6.
\\
\\\textbf{5. SPACE\_CHARGE:} This option 
       is under construction and is reserved for future use.
\end{description}
\begin{description}
	\item[\textbf{PTC\_TRACK\_END;}] 

	\item[Description] The \texttt{PTC\_TRACK\_END }command terminate 
     the command lines related to the PTC\_TRACK module.
\end{description}
\begin{description}
	\item[TRACKSUMM table] 
The starting and final 
     canonical coordinates are collected in the internal table "tracksumm" (printed to the file with
     \href{../control/general.html#write}{WRITE} 
   command).
\end{description}
\begin{description}
	\item[Examples] Several examples are found on the
       \href{http://cern.ch/frs/mad-X_examples/ptc_track}{
    here}.
	\item[The typical tasks ] 
 The following table 
  facilitates the choice of the correct options for a number of tasks.
    
\begin{tabular}{cccccc}
\hline 
\textbf{Option} & \textbf{1} & \textbf{2} & \textbf{3} & \textbf{4} & \textbf{5 } \\ 
\hline
CLOSED\_ORBIT & - & - & + & + & + \\ 
\hline
ELEMENT\_BY\_ELEMENT & - & + & - & + & - \\ 
\hline
PTC\_START, X, PX, ... & + & + & + & + & + \\ 
\hline
PTC\_START, FX, PHIX, $\backslash$85 & -  & - & + & + & + \\ 
\hline
NORM\_NO & - & - & $>$1 & $>$1 & $>$1 \\ 
\hline
NORM\_OUT & - & - & + & - & + \\ 
\hline
PTC\_OBSERVE & - & + & + & + & - \\ 
\hline
RADIATION & - & - & - & - & + \\ 
\hline
RADIATION\_MODEL1 & - & - & - & - & - \\ 
\hline
RADIATION\_ENERGY\_LOSS & - & - & - & - & - \\ 
\hline
RADIATION\_QUAD & - & - & - & - & +/- \\ 
\hline
BEAM\_ENVELOPE & - & - & - & - & - \\ 
\hline
SPACE\_CHARGE & - & - & - & - & - \\ 
\hline

\end{tabular}

1) The tracking of a beam-line with default parameters.2) As $\backslash$931)$\backslash$94, but with element-by-element tracking and an output 
   at observation points. 3) Tracking in a closed ring with closed orbit search and the 
   Normal Forms calculations. 
\\
   Both canonical and action-angle 
   input/output coordinates are possible. Output at observation 
   points is produced via PTC maps. 4) Similar to "3)" except that output at observation points is 
   created by element-by-element tracking.5) The with PTC radiation.

	\item[References for PTC-TRACK] 
\end{description}
\begin{enumerate}
	\item \href{V._Kapin}{V. Kapin} and F. Schmidt, $\backslash$93PTC modules for MAD-X code$\backslash$94, to be published as CERN internal note by the end of 
   2006
	\item \href{F._Schmidt}{F. Schmidt}, "`\href{http://cern.ch/madx/doc/MPPE012.pdf}{MAD-X PTC Integration}'', 
   Proc. of the 2005 PAC Conference in Knoxville, USA, pp.1272.
	\item \href{E._Forest}{E. Forest}, F. Schmidt and E. McIntosh, 
   $\backslash$93Introduction to the Polymorphic Tracking Code$\backslash$94, KEK report 2002-3, July 
   2002
	\item \href{G.J._Roy}{G.J. Roy}, $\backslash$93A new method for the simulation of 
   synchrotron radiation in particle tracking codes$\backslash$94, Nuclear Instruments \& 
   Methods in Phys. Res., Vol. A298, 1990, pp. 128-133.
\end{enumerate}
\begin{description}
	\item[See Also] \href{../tracking/tracking.html}{Overview of 
   MAD-X Tracking Modules},
   \href{../ptc_general/ptc_general.html}{PTC 
   Set-up Parameters},\texttt{}\href{../thintrack/thintrack.html}{thintrack},
   \href{http://cern.ch/frs/mad-X_examples/ptc_track}{PTC-TRACK 
   Examples}.
\end{description}

\line(1,0){300}

\href{mailto:kapin@itep.ru}{
 V.Kapin}(ITEP) and \href{mailto:Frank.Schmidt@cern.ch}{
 F.Schmidt}, July 2005; revised in April, 2006

%%\end{document}
