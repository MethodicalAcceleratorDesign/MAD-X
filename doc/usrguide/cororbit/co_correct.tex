%%\title{CORRECT}
%  Changed by: Hans Grote, 13-Sep-2000 
%  Changed by: Werner Herr, 19-Jun-2002 
%  Changed by: Werner Herr, 02-Sep-2002 
%  Changed by: Werner Herr, 01-Oct-2002 
%  Changed by: Werner Herr, 01-May-2003 
%  Changed by: Werner Herr, 05-April-2004 
%  Changed by: Werner Herr, 01-Dec-2004 
%  Changed by: Werner Herr, 22-Oct-2008 

\section{CORRECT: Orbit Correction}  

The CORRECT statement makes a complete closed orbit  or trajectory
correction using the \textbf{computed} values at the monitors  from the
Twiss table.   

The CORRECT command has the following format (not all possible options
included, some options are valid only for special algorithms):  

\begin{verbatim}
CORRECT, ORBIT=myorbit,MODEL=mymodel,TARGET=mytarget,
         FLAG=ring,MODE=lsq,  
         MONERROR=integer,MONON=real,MONSCALE=real,
         PLANE=x,COND=integer,RESOUT=integer,
         CLIST=file1,MLIST=file2; 
\end{verbatim} 

The command CORRECT is set up with defaults which should allow a
reasonable correction for most cases with a minimum of required options
(see Example 1 below).  

The orbit correction must always be preceded by TWISS commands  which
generate Twiss tables. The most recent Twiss table is assumed to contain
the optical parameters and the distorted orbits. 

The options used in the CORRECT command are: 

\begin{itemize}
   \item FLAG: FLAG can be "ring" or "line", either a circular machine
     or a trajectory is corrected.   
     \\ Default flag is "ring". 

   \item MODE: MODE defines the method to be used for corrections. 
     \\ Available modes are LSQ, MICADO and SVD.  The first performs a
     least squares minimization using all available correctors. The mode
     SVD uses a Singular Value Decomposition to compute a correction
     using all available correctors. The latter can also be used to
     condition the response matrix for the modes LSQ or MICADO (using
     COND=1). It is highly recommended to precede a LSQ correction by a
     SVD conditioning (set COND=1).  
     \\ The mode MICADO is a "best kick" algorithm. Naive use or using
     it with a large number of correctors (see option NCORR) can give
     unexpected results. To avoid the creation of local bumps, it is
     recommended to precede a MICADO correction by a SVD conditioning
     (set COND=1).  
     \\ Default mode is MICADO.            

   \item PLANE: If this attribute is x, only the horizontal correction
     is made; if it is y, only the vertical correction is made. (This
     differs from the MAD8 implementation).  
     \\ Default plane is horizontal. 

   \item COND: When COND is 1, a Singular Value Decomposition is
     performed and  the response matrix CONDitioned to avoid linearly
     dependent correctors. This can be used to avoid creation of
     artificial bumps during a LSQ or MICADO correction (requires some
     computing time).  Please note: this option is not robust since it
     depends on parameters which control the determination of singular
     values and redundant correctors. These can be set with the commands
     SNGVAL and SNGCUT. Both parameters depend on the machine and may
     need adjustment. Default values are adjusted to large machines and
     "reasonable" performance for smaller machines.  
     \\

   \item NCORR: Only used by the MICADO algorithm. Defines the number of
     correctors to be used, unless set to 0 in which case all available
     correctors are used.  
     \\ Default is 0 (all available correctors). 

   \item SNGVAL:  Used to set the threshold for finding singular values
     with the COND command. (Hint: smaller number finds fewer singular
     values).  
     \\ Use with care ! 
     \\ Default is 2.0 

   \item SNGCUT:  Used to set the threshold for finding redundant
     correctors with the COND command. (Hint: larger number finds fewer
     redundant correctors).  
     \\ Use with extreme care ! 
     \\ Default is 50.0 

   \item MONERROR: When MONERROR is 1, the alignment errors on monitors
     assigned  by \href{../error/error_align.html}{EALIGN} MREX and MREY
     are taken into account, otherwise they are ignored.  
     \\ Default is 0. 

   \item MONSCALE: When MONSCALE is 1, the scaling errors on monitors
     assigned  by \href{../error/error_align.html}{EALIGN} MSCALX and
     MSCALY are taken into account, otherwise they are ignored.  
     \\ Default is 0. 

   \item MONON: MONON takes a real number between 0.0 and 1.0. It
     determines the number of available monitors. If the command is
     given, each monitor is considered valid with a probability
     MONON. In the average a fraction (1.0 - MONON) of the monitors will
     be disabled for the correction, i.e. they are considered  not
     existing.  This allows to study the effect of missing monitors.  
     \\ Default is 1.0 (100 \%). 

   \item CORRLIM:  A limit on the maximum corrector strength can be
     given and a WARNING is issued if it is exceeded by one or more
     correctors.  Please note: the strengths computed by the correction
     algorithms are NOT limited, only a warning is printed ! 
     \\ Default is 1.0 mrad. \\
\end{itemize}

Normally the last active table provides the orbit to be
corrected and the model for the correction. This can be overwritten
by the appropriate options. Optionally, these tables can be given
names like in:  TWISS, TABLE=name; (see documentation on TWISS
command). To use these named tables, one of the following optional
parameters must be  used:  

\begin{itemize}
   \item ORBIT: When this parameter is given, the orbit to be corrected
     is taken from a named table. The default is the last (named or
     unnamed) Twiss table.  

   \item MODEL: When this parameter is given, the model for the
     correction is taken from a named Twiss table. The default is the
     last (named or unnamed) Twiss table.  

   \item TARGET: When this parameter is given, the correction is made to
     a named target orbit, pre-computed with a TWISS command. Default is
     correction to the zero orbit.  

   \item EXTERN (default: false): When false, the ORBIT and TARGET table
     are assumed to be computed by MAD with a previous twiss
     command. When set to true, that option allows to use twiss tables
     imported from an external file (with the readmytable command), for
     example to use measured BPM data. In that case, the imported twiss
     table is allowed to contain coordinate data only at the location of
     the monitors.  
\end{itemize}

Example of use of CORRECT to reproduce a measured orbit: 
\begin{verbatim}
! To have a refererence optical model
twiss, table=twiss_ref;

! The bpm.tsv is a reduced Twiss file containing only lines for the BPMs
readmytable, file="bpm.tsv", table="twiss_bpm";

! correct orbit using external measurements
correct, flag=ring, mode=micado, ncorr=5, cond=1 ,plane=x, extern,
         model=twiss_ref, orbit=twiss_ref, target=twiss_bpm, 
         error=1.0e-21;
\end{verbatim}


Two attributes affect the printing of tables and results: 
\begin{itemize}
   \item CLIST=\textbf{file}: Corrector settings (in units of rad)
     before and after correction printed to \textbf{file} 

   \item MLIST=\textbf{file}: Monitor readings (in units of m) before
     and after correction printed to \textbf{file} 

   \item RESOUT: This command outputs the results for all monitors and
     all correctors in a computer readable format if its integer
     argument is larger than 0. The argument is added to the
     output. Useful to analyze runs with loops to produce large
     statistics. 
     \\\textbf{ATTENTION: May produce gigantic outputs for large
       machines.} 
     \\

   \item TWISSUM:  If the argument of twissum is larger than 0, it
     prints maximum orbit and r.m.s. for both planes taken from the
     Twiss summary table in computer readable form. Allows to analyze
     orbits etc. at elements that are not monitors or correctors. The
     argument is added to the output.  Only for output: no correction is
     made, all other commands are ignored.  
\end{itemize}

Obsolete commands or options:
\begin{verbatim}
ITERATE, ITERMAX             /* Done with loop feature in MAD commands */
THREADER, THRTOL, WRORBIT    /* Not part of orbit correction module */
M1LIST, M2LIST               /* Replaced by MLIST */
C1LIST, C2LIST               /* Replaced by CLIST */
GETORBIT, PUTORBIT           /* Replaced by generic TFS access */
GETKICK, PUTKICK             /* Replaced by generic TFS access */
\end{verbatim}

\subsection{EXAMPLES} 

for complete MAD input files see section on examples:

Example 1: correct orbit in horizontal plane, taken from most recent
Twiss table, using default algorithm (MICADO)
\begin{verbatim}
 CORRECT, PLANE = x; 
\end{verbatim}

Example 2: no correction, only output of Twiss summary 
\begin{verbatim}
CORRECT, TWISSUM = 1; 
\end{verbatim}

Example 3: correct orbit in horizontal plane, corrector and monitor
output on table 
\begin{verbatim}
CORRECT, PLANE = x, MODE = lsq, CLIST = corr.out, MLIST = mon.out;   
\end{verbatim}

Example 4: correct orbit in horizontal plane, use alignment and scaling
errors, 15\% of orbit correctors faulty
\begin{verbatim}
CORRECT, PLANE = x, MONERROR = 1, MONSCALE = 1, MONON = 0.85; 
\end{verbatim}

%\href{http://consult.cern.ch/xwho/people/1808}{Last updated:} 22.10.2008 

