%%%\title{Range Selection}
%  Changed by: Chris ISELIN, 27-Jan-1997 
%  Changed by: Hans Grote, 15-Jan-2003 


%%%%\title{Range Selection}
%  Changed by: Chris ISELIN, 27-Jan-1997 
%  Changed by: Hans Grote, 15-Jan-2003 


%%%%\title{Range Selection}
%  Changed by: Chris ISELIN, 27-Jan-1997 
%  Changed by: Hans Grote, 15-Jan-2003 


%%%%\title{Range Selection}
%  Changed by: Chris ISELIN, 27-Jan-1997 
%  Changed by: Hans Grote, 15-Jan-2003 


%\input{control/control}
%\input{control/foot}
%\include{control/general}
%\include{control/special}


\section{Control Statements}

MAD-X consists of a core program, and modules for specific tasks such as
twiss parameter calculation, matching, thin lens tracking, and so on.  
 
The statements listed here are those executed by the program core. They
deal with the I/O, element and sequence declaration, sequence
manipulation, statement flow control (e.g. IF, WHILE), MACRO
declaration, saving sequences onto files in MAD-X or MAD-8 format, and
so on.  


%% \subsection{Program flow control}
%% \begin{itemize}
%% 	\item \href{special.html#if}{IF}
%% 	\item \href{special.html#elseif}{ELSEIF}
%% 	\item \href{special.html#else}{ELSE}
%% 	\item \href{special.html#while}{WHILE}
%% 	\item \href{special.html#macro}{MACRO}
%% \end{itemize}


%% \subsection{General control}
%% \begin{itemize}
%% 	\item \href{general.html#assign}{ASSIGN}
%% 	\item \href{general.html#call}{CALL}
%% 	\item \href{general.html#coguess}{COGUESS}
%% 	\item \href{general.html#create}{CREATE}
%% 	\item \href{general.html#dumpsequ}{DUMPSEQU}
%% 	\item \href{general.html#exec}{EXEC}
%% 	\item \href{general.html#exit}{EXIT}
%% 	\item \href{general.html#fill}{FILL}
%% 	\item \href{general.html#help}{HELP}
%% 	\item \href{general.html#option}{OPTION}
%% 	\item \href{general.html#print}{PRINT}
%% 	\item \href{general.html#quit}{QUIT}
%% 	\item \href{general.html#readtable}{READTABLE}
%% 	\item \href{general.html#removefile}{REMOVEFILE}
%% 	\item \href{general.html#renamefile}{RENAMEFILE}
%% 	\item \href{general.html#return}{RETURN}
%% 	\item \href{general.html#save}{SAVE}
%% 	\item \href{general.html#savebeta}{SAVEBETA}
%% 	\item \href{../Introduction/select.html}{SELECT}
%% 	\item \href{../Introduction/set.html}{SET}
%% 	\item \href{general.html#show}{SHOW}
%% 	\item \href{general.html#stop}{STOP}
%% 	\item \href{general.html#system}{SYSTEM}
%% 	\item \href{general.html#tabstring}{TABSTRING}
%% 	\item \href{general.html#title}{TITLE}
%% 	\item \href{general.html#use}{USE}
%% 	\item \href{general.html#value}{VALUE}
%% 	\item \href{general.html#write}{WRITE}
%% \end{itemize}



%% \subsection{Beam specification}
%% \begin{itemize}
%% 	\item \href{../Introduction/beam.html}{BEAM}
%% 	\item \href{../Introduction/resbeam.html}{RESBEAM}
%% \end{itemize}



%% \subsection{PLOT}
%% \begin{itemize}
%% 	\item \href{../plot/plot.html}{PLOT}
%% 	\item \href{../plot/plot.html#resplot}{RESPLOT}
%% 	\item \href{../plot/plot.html#setplot}{SETPLOT}
%% \end{itemize}


%% \subsection{\href{seqedit.html}{Sequence editing}}
%% \begin{itemize}
%% 	\item \href{seqedit.html#seqedit}{SEQEDIT}
%% 	\item \href{seqedit.html#flatten}{FLATTEN}
%% 	\item \href{seqedit.html#install}{INSTALL}
%% 	\item \href{seqedit.html#move}{MOVE}
%% 	\item \href{seqedit.html#remove}{REMOVE}
%% 	\item \href{seqedit.html#cycle}{CYCLE}
%% 	\item \href{seqedit.html#reflect}{REFLECT}
%% 	\item \href{seqedit.html#endedit}{ENDEDIT}
%% \end{itemize}

%% \href{http://www.cern.ch/Hans.Grote/hansg_sign.html}{hansg}, June 17, 2002 

\input{control/special}
\input{control/general}
\input{Introduction/beam}
\input{Introduction/resbeam}
\input{Introduction/bv_flag}
%\input{plot/plot}


%%%%\title{Range Selection}
%  Changed by: Chris ISELIN, 27-Jan-1997 
%  Changed by: Hans Grote, 10-Jun-2002 

\paragraph{Real life example for IF statements, and MACRO usage}


\begin{verbatim}

! Creates a footprint for head-on + parasitic collisions at IP1+5 
! of lhc.6.5; both lhcb1 (for tracking) and lhcb2 (to define the
! beam-beam elements, i.e. weak-strong) are used; there are flags to
! select head-on, left, and right parasitic separately at all IPs.
! The bunch spacing can be given in nanosec and automatically creates
! the beam-beam interaction points at the correct positions.
! It is important to set the correct BEAM parameters, i.e. number
! of particles, emittances, bunch length, energy.

!--- For completeness, all files needed by this job are copied
!    to the local directory ldb. The links to the the originals
!    in offdb (official database) are commented out.

Option,  warn,info,echo;
!System,
"ln -fns /afs/cern.ch/eng/sl/MAD-X/dev/test_suite/foot/V3.01.01 ldb";
!system,"ln -fns /afs/cern.ch/eng/lhc/optics/V6.4 offdb";
Option, -echo,-info,warn;
SU=1.0;
call, file = "ldb/V6.5.seq";
call,file="ldb/slice_new.madx";
Option, echo,info,warn;

!+++++++++++++++++++++++++ Step 1 +++++++++++++++++++++++
! 	define beam constants
!++++++++++++++++++++++++++++++++++++++++++++++++++++++++

b_t_dist = 25.e-9;                  !--- bunch distance in [sec]
b_h_dist = clight * b_t_dist / 2 ;  !--- bunch half-distance in [m]
ip1_range = 58.;                     ! range for parasitic collisions
ip5_range = ip1_range;
ip2_range = 60.;
ip8_range = ip2_range;

npara_1 = ip1_range / b_h_dist;     ! # parasitic either side
npara_2 = ip2_range / b_h_dist;
npara_5 = ip5_range / b_h_dist;
npara_8 = ip8_range / b_h_dist;

value,npara_1,npara_2,npara_5,npara_8;

 eg   =  7000;
 bg   =  eg/pmass;
 en   = 3.75e-06;
 epsx = en/bg;
 epsy = en/bg;

Beam, particle = proton, sequence=lhcb1, energy = eg,
          sigt=      0.077     , 
          bv = +1, NPART=1.1E11, sige=      1.1e-4, 
          ex=epsx,   ey=epsy;

Beam, particle = proton, sequence=lhcb2, energy = eg,
          sigt=      0.077     , 
          bv = -1, NPART=1.1E11, sige=      1.1e-4, 
          ex=epsx,   ey=epsy;

beamx = beam%lhcb1->ex;   beamy%lhcb1 = beam->ey;
sigz  = beam%lhcb1->sigt; sige = beam%lhcb1->sige;

!--- split5, 4d
long_a= 0.53 * sigz/2;
long_b= 1.40 * sigz/2;
value,long_a,long_b;

ho_charge = 0.2;

!+++++++++++++++++++++++++ Step 2 +++++++++++++++++++++++
! 	slice, flatten sequence, and cycle start to ip3
!++++++++++++++++++++++++++++++++++++++++++++++++++++++++

use,sequence=lhcb1;
makethin,sequence=lhcb1;
!save,sequence=lhcb1,file=lhcb1_thin_new_seq;
use,sequence=lhcb2;
makethin,sequence=lhcb2;
!save,sequence=lhcb2,file=lhcb2_thin_new_seq;
!stop;

option,-warn,-echo,-info;
call,file="ldb/V6.5.thin.coll.str";
option,warn,echo,info;

! keep sextupoles
ksf0=ksf; ksd0=ksd;
use,period=lhcb1;
select,flag=twiss.1,column=name,x,y,betx,bety;
twiss,file;
plot,haxis=s,vaxis=x,y,colour=100,noline;

use,period=lhcb2;
select,flag=twiss.2,column=name,x,y,betx,bety;
twiss,file;
plot,haxis=s,vaxis=x,y,colour=100,noline;
seqedit,sequence=lhcb1;
flatten;
endedit;

seqedit,sequence=lhcb1;
cycle,start=ip3.b1;
endedit;

seqedit,sequence=lhcb2;
flatten;
endedit;

seqedit,sequence=lhcb2;
cycle,start=ip3.b2;
endedit;

bbmarker: marker;  /* for subsequent remove */


!+++++++++++++++++++++++++ Step 3 +++++++++++++++++++++++
! 	define the beam-beam elements
!++++++++++++++++++++++++++++++++++++++++++++++++++++++++
!
!===========================================================
! read macro definitions
option,-echo;
call,file="ldb/bb.macros";
option,echo;

!
!===========================================================
!   this sets CHARGE in the head-on beam-beam elements. 
!   set +1 * ho_charge   for parasitic on, 0 for off

 on_ho1  = +1 * ho_charge; ! ho_charge depends on split
 on_ho2  = +0 * ho_charge; ! because of the "by hand" splitting
 on_ho5  = +1 * ho_charge;
 on_ho8  = +0 * ho_charge;

!
!===========================================================
!   set CHARGE in the parasitic beam-beam elements. 
!   set +1 for parasitic on, 0 for off
 on_lr1l = +1;
 on_lr1r = +1;
 on_lr2l = +0;
 on_lr2r = +0;
 on_lr5l = +1;
 on_lr5r = +1;
 on_lr8l = +0;
 on_lr8r = +0;

!
!===========================================================
!   define markers and savebetas
assign,echo=temp.bb.install;
!--- ip1
if (on_ho1  0)
{
  exec, mkho(1);
  exec, sbhomk(1);
}
if (on_lr1l  0 || on_lr1r  0)
{
  n=1; ! counter
  while (n  0 || on_lr1l  0)
{
  n=1; ! counter
  while (n  0)
{
  exec, mkho(5);
  exec, sbhomk(5);
}
if (on_lr5l  0 || on_lr5r  0)
{
  n=1; ! counter
  while (n  0 || on_lr5l  0)
{
  n=1; ! counter
  while (n  0)
{
  exec, mkho(2);
  exec, sbhomk(2);
}
if (on_lr2l  0 || on_lr2r  0)
{
  n=1; ! counter
  while (n  0 || on_lr2l  0)
{
  n=1; ! counter
  while (n  0)
{
  exec, mkho(8);
  exec, sbhomk(8);
}
if (on_lr8l  0 || on_lr8r  0)
{
  n=1; ! counter
  while (n  0 || on_lr8l  0)
{
  n=1; ! counter
  while (n  0)
{
exec, inho(mk,1);
}
if (on_lr1l  0 || on_lr1r  0)
{
  n=1; ! counter
  while (n  0 || on_lr1l  0)
{
  n=1; ! counter
  while (n  0)
{
exec, inho(mk,5);
}
if (on_lr5l  0 || on_lr5r  0)
{
  n=1; ! counter
  while (n  0 || on_lr5l  0)
{
  n=1; ! counter
  while (n  0)
{
exec, inho(mk,2);
}
if (on_lr2l  0 || on_lr2r  0)
{
  n=1; ! counter
  while (n  0 || on_lr2l  0)
{
  n=1; ! counter
  while (n  0)
{
exec, inho(mk,8);
}
if (on_lr8l  0 || on_lr8r  0)
{
  n=1; ! counter
  while (n  0 || on_lr8l  0)
{
  n=1; ! counter
  while (n betx) / 0.0007999979093;
value,on_sep2;
!===========================================================
!   define bb elements
assign,echo=temp.bb.install;
!--- ip1
if (on_ho1  0)
{
exec, bbho(1);
}
if (on_lr1l  0)
{
  n=1; ! counter
  while (n  0)
{
  n=1; ! counter
  while (n  0)
{
exec, bbho(5);
}
if (on_lr5l  0)
{
  n=1; ! counter
  while (n  0)
{
  n=1; ! counter
  while (n  0)
{
exec, bbho(2);
}
if (on_lr2l  0)
{
  n=1; ! counter
  while (n  0)
{
  n=1; ! counter
  while (n  0)
{
exec, bbho(8);
}
if (on_lr8l  0)
{
  n=1; ! counter
  while (n  0)
{
  n=1; ! counter
  while (n  0)
{
exec, inho(bb,1);
}
if (on_lr1l  0)
{
  n=1; ! counter
  while (n  0)
{
  n=1; ! counter
  while (n  0)
{
exec, inho(bb,5);
}
if (on_lr5l  0)
{
  n=1; ! counter
  while (n  0)
{
  n=1; ! counter
  while (n  0)
{
exec, inho(bb,2);
}
if (on_lr2l  0)
{
  n=1; ! counter
  while (n  0)
{
  n=1; ! counter
  while (n  0)
{
exec, inho(bb,8);
}
if (on_lr8l  0)
{
  n=1; ! counter
  while (n  0)
{
  n=1; ! counter
  while (n  footprint";
stop;
\end{verbatim}

\paragraph{\href{macro}{Real life example of MACRO definitions}}

\begin{verbatim}

bbho(nn): macro = {
!--- macro defining head-on beam-beam elements; nn = IP number
print, text="bbipnnl2: beambeam, sigx=sqrt(rnnipnnl2->betx*epsx),";
print, text="          sigy=sqrt(rnnipnnl2->bety*epsy),";
print, text="          xma=rnnipnnl2->x,yma=rnnipnnl2->y,";
print, text="          charge:=on_honn;";
print, text="bbipnnl1: beambeam, sigx=sqrt(rnnipnnl1->betx*epsx),";
print, text="          sigy=sqrt(rnnipnnl1->bety*epsy),";
print, text="          xma=rnnipnnl1->x,yma=rnnipnnl1->y,";
print, text="          charge:=on_honn;";
print, text="bbipnn:   beambeam, sigx=sqrt(rnnipnn->betx*epsx),";
print, text="          sigy=sqrt(rnnipnn->bety*epsy),";
print, text="          xma=rnnipnn->x,yma=rnnipnn->y,";
print, text="          charge:=on_honn;";
print, text="bbipnnr1: beambeam, sigx=sqrt(rnnipnnr1->betx*epsx),";
print, text="          sigy=sqrt(rnnipnnr1->bety*epsy),";
print, text="          xma=rnnipnnr1->x,yma=rnnipnnr1->y,";
print, text="          charge:=on_honn;";
print, text="bbipnnr2: beambeam, sigx=sqrt(rnnipnnr2->betx*epsx),";
print, text="          sigy=sqrt(rnnipnnr2->bety*epsy),";
print, text="          xma=rnnipnnr2->x,yma=rnnipnnr2->y,";
print, text="          charge:=on_honn;";
};

mkho(nn): macro = {
!--- macro defining head-on markers; nn = IP number
print, text="mkipnnl2: bbmarker;";
print, text="mkipnnl1: bbmarker;";
print, text="mkipnn:   bbmarker;";
print, text="mkipnnr1: bbmarker;";
print, text="mkipnnr2: bbmarker;";
};

inho(xx,nn): macro = {
!--- macro installing bb or markers for head-on beam-beam (split into 5)
print, text="install, element= xxipnnl2, at=-long_b, from=ipnn;";
print, text="install, element= xxipnnl1, at=-long_a, from=ipnn;";
print, text="install, element= xxipnn,   at=1.e-9,   from=ipnn;";
print, text="install, element= xxipnnr1, at=+long_a, from=ipnn;"; 
print, text="install, element= xxipnnr2, at=+long_b, from=ipnn;"; 
};

sbhomk(nn): macro = {
!--- macro to create savebetas for ho markers
print, text="savebeta, label=rnnipnnl2, place=mkipnnl2;";
print, text="savebeta, label=rnnipnnl1, place=mkipnnl1;";
print, text="savebeta, label=rnnipnn,   place=mkipnn;";
print, text="savebeta, label=rnnipnnr1, place=mkipnnr1;";
print, text="savebeta, label=rnnipnnr2, place=mkipnnr2;";    
};

mkl(nn,cc): macro = {
!--- macro to create parasitic bb marker on left side of ip nn; cc = count
print, text="mkipnnplcc: bbmarker;";
};

mkr(nn,cc): macro = {
!--- macro to create parasitic bb marker on right side of ip nn; cc = count
print, text="mkipnnprcc: bbmarker;";
};

sbl(nn,cc): macro = {
!--- macro to create savebetas for left parasitic
print, text="savebeta, label=rnnipnnplcc, place=mkipnnplcc;";
};

sbr(nn,cc): macro = {
!--- macro to create savebetas for right parasitic
print, text="savebeta, label=rnnipnnprcc, place=mkipnnprcc;";
};

inl(xx,nn,cc): macro = {
!--- macro installing bb and markers for left side parasitic beam-beam
print, text="install, element= xxipnnplcc, at=-cc*b_h_dist, from=ipnn;";
};

inr(xx,nn,cc): macro = {
!--- macro installing bb and markers for right side parasitic beam-beam
print, text="install, element= xxipnnprcc, at=cc*b_h_dist, from=ipnn;";
};

bbl(nn,cc): macro = {
!--- macro defining parasitic beam-beam elements; nn = IP number
print, text="bbipnnplcc: beambeam, sigx=sqrt(rnnipnnplcc->betx*epsx),";
print, text="          sigy=sqrt(rnnipnnplcc->bety*epsy),";
print, text="          xma=rnnipnnplcc->x,yma=rnnipnnplcc->y,";
print, text="          charge:=on_lrnnl;";
};

bbr(nn,cc): macro = {
!--- macro defining parasitic beam-beam elements; nn = IP number
print, text="bbipnnprcc: beambeam, sigx=sqrt(rnnipnnprcc->betx*epsx),";
print, text="          sigy=sqrt(rnnipnnprcc->bety*epsy),";
print, text="          xma=rnnipnnprcc->x,yma=rnnipnnprcc->y,";
print, text="          charge:=on_lrnnr;";
};
\end{verbatim}

%\href{http://www.cern.ch/Hans.Grote/hansg_sign.html}{hansg}, June 17, 2002 


%%%%\title{Range Selection}
%  Changed by: Chris ISELIN, 27-Jan-1997 
%  Changed by: Hans Grote, 16-Jan-2003 

\section{General Control Statements}

\subsection{ASSIGN}
\begin{verbatim}
assign, echo="file_name", truncate;
\end{verbatim} 
where "file\_name" is the name of an output file, or "terminal" and
truncate specifies if the file must be truncated when opened (ignored
for terminal). This allows switching the echo stream to a file or back,
but only for the commands value, show, and print. Allows easy
composition of future MAD-X input files. A real life example (always the
same) is to be found under \href{foot.html}{footprint example}.  

\subsection{CALL}
\begin{verbatim}
call, file = "file_name";
\end{verbatim} 
where "file\_name"  is the name of an input file. This file will be read
until a "return;" statement, or until end\_of\_file; it may contain any
number of calls itself, and so on to any depth.  

%% 2013-Jul-11  17:23:00  ghislain: I propose to move COGUESS to the
%% orbit correction part of the manual
\subsection{COGUESS}
\label{subsec:general_coguess}
\begin{verbatim}
coguess, tolerance = double, 
         x = double, px = double, 
         y = double, py = double, 
         t = double, pt = double;
\end{verbatim} 
sets the required convergence precision in the closed orbit search
("tolerance", see as well Twiss command
\href{../twiss/twiss.html#tolerance}{tolerance}).  

The other parameters define a first guess for all future closed orbit
searches in case they are different from zero.  

\subsection{CREATE}
\begin{verbatim}
create, table = table, column = var1, var2,_name,...;
\end{verbatim} 
creates a table with the specified variables as columns. This table can
then be \hyperlink{fill}{fill}ed, and finally one can
\hyperlink{write}{write} it in TFS format. The attribute "\_name" adds
the element name to the table at the specified column, this replaces the
undocumented "withname" attribute that was not always working properly.  

See the \href{../Introduction/select.html#ucreate}{user table I}
example; 
or an example of joining 2 tables of different length in one table
including the element name:
\href{../Introduction/select.html#screate}{user table II} 

\subsection{DELETE}
\begin{verbatim}
delete, sequence = s_name, table = t_name;
\end{verbatim} 
deletes a sequence with name "s\_name" or a table with name "t\_name"
from memory. The sequence deletion is done without influence on other
sequences that may have elements that were in the deleted sequence.  

%% 2013-Jul-11  17:24:21  ghislain: I propose to move DUMPSEQU to the
%% sequence edition and manipulation part of the manual
\subsection{DUMPSEQU}
\begin{verbatim}
dumpsequ, sequence = s_name, level = integer;
\end{verbatim} 
Actually a debug statement, but it may come handy at certain
occasions. Here "s\_name" is the name of an expanded (i.e. USEd)
sequence. The amount of detail is controlled by "level":  
\begin{verbatim}
level = 0:    print only the cumulative node length = sequence length
      > 0:    print all node (element) names except drifts
      > 2:    print all nodes with their attached parameters
      > 3:    print all nodes, and their elements with all parameters
\end{verbatim}


\subsection{EXEC}
\begin{verbatim}
exec, label;
\end{verbatim} 
Each statement may be preceded by a label; it is then stored and can be
executed again with "exec, label;" any number of times; the executed
statement may be another "exec", etc.; however, the major usage of this
statement is the execution of a \href{special.html#macro}{macro}.  

\subsection{EXIT}
\begin{verbatim}
exit;
\end{verbatim} 
ends the program execution. 

\subsection{FILL} 
Every command 
\begin{verbatim}
fill, table = table;
\end{verbatim} 
adds a new line with the current values of all column variables into the
user table \hyperlink{create}{create}d beforehand. This table one can
then \hyperlink{write}{write} in TFS format.  See as well the
\href{../Introduction/select.html#ucreate}{user table} example.  

\subsection{OPTION}
\label{subsec:general_option}
\begin{verbatim}
option, flag { = true | = false };
option, flag | -flag;
\end{verbatim} 
sets an option as given in "flag"; the part in curly brackets is
optional: if only the name of the option is given, then the option will
be set true (see second line); a "-" sign preceding the name sets it to
"false".  

Example: 
\begin{verbatim}
option, echo = true;
option, echo;
\end{verbatim} 
are identical, ditto 
\begin{verbatim}
option, echo = false;
option, -echo;
\end{verbatim} 

The available options are: 
\begin{verbatim}
  name           default meaning if true
  ====           ======= ===============
  echo            true   echoes the input on the standard output file
  warn            true   issues warnings
  info            true   issues informations
  debug           false  issues debugging information
  trace           false  prints the system time after each command
  verify          false  issues a warning if an undefined variable is used
  tell            false  prints the current value of all options
  reset           false  resets all options to their defaults
  no_fatal_stop   false  Prevents madx from stopping in case of a fatal error. 
                         Use at your own risk.

  rbarc           true   converts the RBEND straight length into the arc 
                         length
  thin_foc        true   if false suppresses the 1(rho**2) focusing of thin 
                         dipoles
  bborbit         false  the closed orbit is modified by beam-beam kicks
  sympl           false  all element matrices are symplectified in Twiss
  twiss_print     true   controls whether the twiss command produces output.
\end{verbatim} 

The option "rbarc" is implemented for backwards compatibility with MAD-8
up to version 8.23.06 included; in this version, the RBEND length was
just taken as the arc length of an SBEND with inclined pole faces,
contrary to the MAD-8 manual.  


\subsection{PRINT}
\begin{verbatim}
print, text = "...";
\end{verbatim} 
prints the text to the current output file (see ASSIGN above). The text
can be edited with the help of a  \href{special.html#macro}{macro
  statement}. For more details, see there.  


\subsection{QUIT}
\begin{verbatim}
quit;
\end{verbatim} 
ends the program execution. 


\subsection{READTABLE}
\begin{verbatim}
readtable, file = "file_name";
\end{verbatim} 
reads a TFS file containing a MAD-X table back into memory. This table
can then be manipulated as any other table, i.e. its values can be
accessed, it can be plotted, written out again etc.  


\subsection{READMYTABLE}
\label{subsec:general_readmy}
\begin{verbatim}
readmytable, file = "file_name", table = internalname;
\end{verbatim} 
reads a TFS file containing a MAD-X table back into memory. This table
can then be manipulated as any other table, i.e. its values can be
accessed, it can be plotted, written out again etc. 

An internal name for
the table can be freely assigned while for the command READTABLE it is
taken from the information section of the table itself. This feature
allows to store multiple tables of the same type in memory without
overwriting existing ones.   


\subsection{REMOVEFILE}
\begin{verbatim}
removefile, file = "file_name";
\end{verbatim} 
remove the file "file\_name" from disk. It is more portable than  
\begin{verbatim}
system("rm filename"); // Unix specific
\end{verbatim}


\subsection{RENAMEFILE}
\begin{verbatim}
renamefile, file = "file_name", name = "new_file_name";
\end{verbatim} 
rename the file "file\_name" to "new\_file\_name" on the disk. It is more
portable than  
\begin{verbatim}
system("mv file_name new_file_name"); // Unix specific
\end{verbatim}

%% 2013-Jul-11  17:24:21  ghislain: I propose to move RESBEAM to the
%% beam declaration part of the manual
\subsection{RESBEAM}
\begin{verbatim}
resbeam, sequence = sequence_name;
\end{verbatim} 
resets the default values of the beam belonging to sequence sequence\_name, or
of the default beam if no sequence is given.  


\subsection{RETURN}
\begin{verbatim}
return;
\end{verbatim} 
ends reading from a "called" file; if encountered in the standard input
file, it ends the program execution.  


\subsection{SAVE}
\label{subsec:general_save}
\begin{verbatim}
save, sequence = sequ1, sequ2, ..., file = "file_name", beam, bare;
\end{verbatim} 
saves the sequence(s) specified with all variables and elements needed
for their expansion, onto the file "file\_name". 

{\bf Warning:} If quotes are used for
the "file\_name", capital and low characters are kept as specified, if they
are omitted the "filename" will have lower characters only. 

Example:
\begin{verbatim}
save, sequence = lhc, file = "Test_One";
\end{verbatim}
saves the lhc sequence to a file name Test\_One on disk, while
\begin{verbatim}
save, sequence = lhc, file = Test_One;
\end{verbatim}
saves the lhc sequence to a file name test\_one on disk.

The optional
flag can have the value "mad8" (without the quotes), in which case the
sequence(s) is/are saved in MAD-8 input format.  

The flag "beam" is optional; when given, all beams belonging to the
sequences specified are saved at the top of the save file.  

The parameter "sequence" is optional; when omitted, all sequences are
saved.  

However, it is not advisable to use "save" without the "sequence" option
unless you know what you are doing. This practice will avoid spurious
saved entries.    Any number of "select,flag=save" commands may precede
the SAVE command. In that case, the names of elements, variables, and
sequences must match the pattern(s) if given, and in addition the
elements must be of the class(es) specified. See here for a
\href{../Introduction/select.html#save_select}{SAVE with SELECT}
example.  

It is important to note that the precision of the output of the save
command depends on the output precision. Details about default
precisions and how to adjust those precisions can be found at the
\href{../Introduction/set.html#Format}{SET Format} instruction page.   
 
The attribute 'bare' allows to save just the sequence without the
element definitions nor beam information. This allows to re-read in a
sequence with might otherwise create a stop of the program. This is
particularly useful to turn a line into a sequence to seqedit
it. 

Example:  
\begin{verbatim}
tl3:line=(ldl6,qtl301,mqn,qtl301,ldl7,qtl302,mqn,qtl302,ldl8,ison);
DLTL3 : LINE=(delay, tl3);
use, period=dltl3;

save,sequence=dltl3,file=t1,bare; // new parameter "bare": only sequ. saved
call,file=t1; // sequence is read in and is now a "real" sequence
// if the two preceding lines are suppressed, seqedit will print a warning
// and else do nothing
use, period=dltl3;
twiss, save, betx=bxa, alfx=alfxa, bety=bya, alfy=alfya;
plot, vaxis=betx, bety, haxis=s, colour:=100;
SEQEDIT, SEQUENCE=dltl3;
  remove,element=cx.bhe0330;
  remove,element=cd.bhe0330;
ENDEDIT;

use, period=dltl3;
twiss, save, betx=bxa, alfx=alfxa, bety=bya, alfy=alfya;
\end{verbatim}


\subsection{SAVEBETA}
\label{subsec:general_savebeta}
\begin{verbatim}
savebeta, label = label, place = place, sequence = sequence_name;
\end{verbatim} 
marks a place named "place" in an expanded sequence "sequence\_name"; 
at the next TWISS command execution, a
\href{../twiss/twiss.html#beta0}{beta0} 
block will be saved at that place with the label "label". This is done
only once; in order to get a new beta0 block there, one has to re-issue
the command. The contents of the beta0 block can then be used in other
commands, e.g. TWISS and MATCH.  

Example (after sequence expansion): 
\begin{verbatim}
savebeta, label = sb1, place = mb[5], sequence = fivecell;
twiss;
show, sb1;
\end{verbatim} 
will save and show the beta0 block parameters at the end (!) of the
fifth element of type mb in the sequence.  


\subsection{SELECT} %select</a}{SELECT}
\begin{verbatim}
select, flag = flag, range = range, class = class, pattern = pattern,
        slice = integer, column =s1, s2, s3,..,sn, sequence=sequence_name,
        full, clear;
\end{verbatim} 
selects one or several elements for special treatment in a subsequent
command. All selections for a given command remain valid until "clear"
is specified; the selection criteria on the same command are logically
ANDed; the selection criteria on different SELECT statements logically
ORed.   

 Example: 
\begin{verbatim}
select, flag = error, class = quadrupole, range = mb[1]/mb[5];
select, flag = error, pattern = "^mqw.*";
\end{verbatim} 
selects all quadrupoles in the range mb[1] to mb[5], and all elements
(in the whole sequence) the name of which starts with "mqw", for
treatment by the error module.  

"flag" can be one of the following: 
\begin{itemize}
   \item seqedit: selection of elements for the
     \href{seqedit.html}{seqedit} module.  
   \item error: selection of elements for the
     \href{../error/error.html}{error} assignment module.  
   \item makethin: selection of elements for the
     \href{../makethin/makethin.html}{makethin} module that
     converts the sequence into one with thin elements only.  
   \item sectormap: selection of elements for the
     \href{../Introduction/sectormap.html}{sectormap} output file
     from the Twiss module.  
   \item table: here "table" is a table name such as twiss, track
     etc., and the rows and columns to be written are selected.  
\end{itemize} 

For the RANGE, CLASS, PATTERN, FULL, and CLEAR parameters
see \href{../Introduction/select.html}{SELECT}.  

"slice" is only used by \href{../makethin/makethin.html}{makethin} and
prescribes the number of slices into which the selected elements have to
be cut (default = 1).  

"column" is only valid for tables and decides the selection of columns
to be written into the TFS file. The "name" argument is special in that
it refers to the actual name of the selected element. For an example,
see \href{../Introduction/select.html}{SELECT}.  


\subsection{SHOW}
\begin{verbatim}
show, command;
\end{verbatim} 
prints the "command" (typically "beam", "beam\%sequ", or an element
name), with the actual value of all its parameters.  


\subsection{STOP}
\begin{verbatim}
stop;
\end{verbatim} 
ends the program execution. 


\subsection{SYSTEM}
\begin{verbatim}
system, "string";
\end{verbatim} 
transfers the string in quotes to the system for execution.  

Example: 
\begin{verbatim}
system,"ln -s /afs/cern.ch/user/u/user/public/some/directory short";
\end{verbatim}


\subsection{TABSTRING}
Note: this is not a command and should appear in the variables section
\begin{verbatim}
tabstring(arg1,arg2,arg3)
\end{verbatim}  
The "string function" tabstring(arg1,arg2,arg3) with exactly  three
arguments; arg1 is a table name (string), arg2 is a column name
(string), arg3 is a row number (integer), count starts at 0. The
function can be used in any context where a string appears; in case
there is no match, it returns \_void\_.  


\subsection{TITLE}
\begin{verbatim}
title, "title";
\end{verbatim} 
inserts the string in quotes as title in various tables and plots.  


\subsection{USE}
\label{subsec:general_use}
\begin{verbatim}
use, period = sequence_name, range = range, survey;
\end{verbatim} 
expands the sequence with name "sequence\_name", or a part of it as specified
in the \href{../Introduction/ranges.html#range}{range}. The
\texttt{survey} option plugs the survey data into the sequence elements
nodes on the first pass (see \href{../survey/survey.html}{survey}).  


\subsection{VALUE}
\begin{verbatim}
value, exp1, exp2,...;
\end{verbatim} 
prints the actual values of the expressions given. 

Example: 
\begin{verbatim}
a = clight/1000.;
value, a, pmass, exp(sqrt(2));
\end{verbatim} 
results in 
\begin{verbatim}
a = 299792.458         ;
pmass = 0.938271998        ;
exp(sqrt(2)) = 4.113250379        ;
\end{verbatim}


\subsection{WRITE}
\label{subsec:general_write}
\begin{verbatim}
write, table = table, file = "file_name";
\end{verbatim} 
writes the table "table" onto the file "file\_name"; only the rows and
columns of a preceding \verb+select, flag = table,...;+ are written. If no select
has been issued for this table, the file will only contain the
header. If the FILE argument is omitted, the table is written to
standard output.  


%\href{http://www.cern.ch/Hans.Grote/hansg_sign.html}{hansg}, June 17, 2002 

%%%%\title{Range Selection}
%  Changed by: Chris ISELIN, 27-Jan-1997 
%  Changed by: Hans Grote, 30-Sep-2002 

\subsection{Program Flow Statements}

\begin{itemize}
	\item IF
\begin{verbatim}

if (logical_expression) {statement 1; statement 2; ...; statement n; }
\end{verbatim}
\href{logical}{ where "logical\_expression" } is one of 
\begin{verbatim}

expr1 oper expr2
expr11 oper1 expr12 && expr21 oper2 expr22
expr11 oper1 expr12 || expr21 oper2 expr22
\end{verbatim} 
and oper one of 
\begin{verbatim}

==          ! equal
<>          ! not equal
<           ! less than
>           ! greater than
<=          ! less than or equal
>=          ! greater than or equal
\end{verbatim} 
The expressions are arithmetic expressions of type real. The statements
in the curly brackets are executed if the logical expression is true.  


	\item ELSEIF%elseif}{ELSEIF}
\begin{verbatim}

elseif (logical_expression) {statement 1; statement 2; ...; statement n; }
\end{verbatim} 
Only possible (in any number) behind an IF, or another ELSEIF; is
executed if  logical\_expression is true, and if none of the preceding
IF or ELSEIF logical conditions was true.  


	\item ELSE%else}{ELSE}
\begin{verbatim}

else {statement 1; statement 2; ...; statement n; }
\end{verbatim} 
Only possible (once) behind an IF, or an ELSEIF; is executed if
logical\_expression is true, and if none of the preceding IF or ELSEIF
logical conditions was true.  

For a real life example, see \href{foot.html}{ELSE example}. 


	\item WHILE
\begin{verbatim}

while (logical_condition){statement 1; statement 2; ...; statement n; }
\end{verbatim}  
executes the statements in curly brackets while the logical\_expression
is true. A simple example (in case you have forgotten the first ten
factorials) would be  
\begin{verbatim}

option,-info;   ! otherwise you get redifiniton warnings
n=1; m=1;
while (n <= 10)
{
  m = m * n;  value, m;
  n = n + 1;
};
\end{verbatim}

For a real life example, see \href{foot.html}{WHILE example}.

	\item MACRO

\begin{verbatim}

label: macro = {statement 1; statement 2; ...; statement n; };
label(arg1,...,argn): macro = {statement 1; statement 2; ...; statement n; };
\end{verbatim} 
The first form allows the execution of a group of statements via a
single command:  
\begin{verbatim}

exec, label;
\end{verbatim} 
will execute the statements in curly brackets exactly once. This command
can be issued any number of times.  

The second form allows to replace strings anywhere inside the statements
in curly brackets by other strings, or integer numbers prior to
execution. This is a powerful construct and should be handled with care.  

Simple example: 
\begin{verbatim}

option,-echo,-info;  ! otherwise the output is somewhat confusing
simple(xx,yy): macro = { xx = yy^2 + xx; value, xx;};
a = 3;
b = 5;
exec, simple(a,b);
\end{verbatim}

Somewhat more tricky (a "\$" in front of an argument means that the
truncated integer value of this argument is used for replacement, rather
than the argument string itself).  
\begin{verbatim}

tricky(xx,yy,zz): macro = {mzz.yy: xx, l = 1.yy, kzz = k.yy;};
n=0;
while (n < 3)
{
  n = n+1;
  exec,tricky(quadrupole,$n,1);
  exec,tricky(sextupole,$n,2);
};
\end{verbatim} 
Whereas the actual use of the preceding example is NOT recommended,
a real life example, showing the full power (!) of macros is to be
found under \href{foot.html}{macro usage} for the usage, and
under \href{foot.html#macro}{macro definition} for the
definition.


Beware of the following rules:

	\item Generally speaking: \textit{ special constructs } like IF, WHILE,
MACRO will only allow one level of inclusion of another \textit{
special construct }.

	\item  Macros must not be called with numbers, but with strings
(i.e. variable names in case of numerical values), i.e. 



NOT

\begin{verbatim}

exec,thismacro($99,$129);
\end{verbatim}
BUT

\begin{verbatim}

n1=99; n2=219;
exec,thismacro($n1,$n2);
\end{verbatim}

\end{itemize}

%\href{http://www.cern.ch/Hans.Grote/hansg_sign.html}{hansg}, June 17, 2002




\section{Control Statements}

MAD-X consists of a core program, and modules for specific tasks such as
twiss parameter calculation, matching, thin lens tracking, and so on.  
 
The statements listed here are those executed by the program core. They
deal with the I/O, element and sequence declaration, sequence
manipulation, statement flow control (e.g. IF, WHILE), MACRO
declaration, saving sequences onto files in MAD-X or MAD-8 format, and
so on.  


%% \subsection{Program flow control}
%% \begin{itemize}
%% 	\item \href{special.html#if}{IF}
%% 	\item \href{special.html#elseif}{ELSEIF}
%% 	\item \href{special.html#else}{ELSE}
%% 	\item \href{special.html#while}{WHILE}
%% 	\item \href{special.html#macro}{MACRO}
%% \end{itemize}


%% \subsection{General control}
%% \begin{itemize}
%% 	\item \href{general.html#assign}{ASSIGN}
%% 	\item \href{general.html#call}{CALL}
%% 	\item \href{general.html#coguess}{COGUESS}
%% 	\item \href{general.html#create}{CREATE}
%% 	\item \href{general.html#dumpsequ}{DUMPSEQU}
%% 	\item \href{general.html#exec}{EXEC}
%% 	\item \href{general.html#exit}{EXIT}
%% 	\item \href{general.html#fill}{FILL}
%% 	\item \href{general.html#help}{HELP}
%% 	\item \href{general.html#option}{OPTION}
%% 	\item \href{general.html#print}{PRINT}
%% 	\item \href{general.html#quit}{QUIT}
%% 	\item \href{general.html#readtable}{READTABLE}
%% 	\item \href{general.html#removefile}{REMOVEFILE}
%% 	\item \href{general.html#renamefile}{RENAMEFILE}
%% 	\item \href{general.html#return}{RETURN}
%% 	\item \href{general.html#save}{SAVE}
%% 	\item \href{general.html#savebeta}{SAVEBETA}
%% 	\item \href{../Introduction/select.html}{SELECT}
%% 	\item \href{../Introduction/set.html}{SET}
%% 	\item \href{general.html#show}{SHOW}
%% 	\item \href{general.html#stop}{STOP}
%% 	\item \href{general.html#system}{SYSTEM}
%% 	\item \href{general.html#tabstring}{TABSTRING}
%% 	\item \href{general.html#title}{TITLE}
%% 	\item \href{general.html#use}{USE}
%% 	\item \href{general.html#value}{VALUE}
%% 	\item \href{general.html#write}{WRITE}
%% \end{itemize}



%% \subsection{Beam specification}
%% \begin{itemize}
%% 	\item \href{../Introduction/beam.html}{BEAM}
%% 	\item \href{../Introduction/resbeam.html}{RESBEAM}
%% \end{itemize}



%% \subsection{PLOT}
%% \begin{itemize}
%% 	\item \href{../plot/plot.html}{PLOT}
%% 	\item \href{../plot/plot.html#resplot}{RESPLOT}
%% 	\item \href{../plot/plot.html#setplot}{SETPLOT}
%% \end{itemize}


%% \subsection{\href{seqedit.html}{Sequence editing}}
%% \begin{itemize}
%% 	\item \href{seqedit.html#seqedit}{SEQEDIT}
%% 	\item \href{seqedit.html#flatten}{FLATTEN}
%% 	\item \href{seqedit.html#install}{INSTALL}
%% 	\item \href{seqedit.html#move}{MOVE}
%% 	\item \href{seqedit.html#remove}{REMOVE}
%% 	\item \href{seqedit.html#cycle}{CYCLE}
%% 	\item \href{seqedit.html#reflect}{REFLECT}
%% 	\item \href{seqedit.html#endedit}{ENDEDIT}
%% \end{itemize}

%% \href{http://www.cern.ch/Hans.Grote/hansg_sign.html}{hansg}, June 17, 2002 

%%%\title{Range Selection}
%  Changed by: Chris ISELIN, 27-Jan-1997 
%  Changed by: Hans Grote, 30-Sep-2002 

\subsection{Program Flow Statements}

\begin{itemize}
	\item IF
\begin{verbatim}

if (logical_expression) {statement 1; statement 2; ...; statement n; }
\end{verbatim}
\href{logical}{ where "logical\_expression" } is one of 
\begin{verbatim}

expr1 oper expr2
expr11 oper1 expr12 && expr21 oper2 expr22
expr11 oper1 expr12 || expr21 oper2 expr22
\end{verbatim} 
and oper one of 
\begin{verbatim}

==          ! equal
<>          ! not equal
<           ! less than
>           ! greater than
<=          ! less than or equal
>=          ! greater than or equal
\end{verbatim} 
The expressions are arithmetic expressions of type real. The statements
in the curly brackets are executed if the logical expression is true.  


	\item ELSEIF%elseif}{ELSEIF}
\begin{verbatim}

elseif (logical_expression) {statement 1; statement 2; ...; statement n; }
\end{verbatim} 
Only possible (in any number) behind an IF, or another ELSEIF; is
executed if  logical\_expression is true, and if none of the preceding
IF or ELSEIF logical conditions was true.  


	\item ELSE%else}{ELSE}
\begin{verbatim}

else {statement 1; statement 2; ...; statement n; }
\end{verbatim} 
Only possible (once) behind an IF, or an ELSEIF; is executed if
logical\_expression is true, and if none of the preceding IF or ELSEIF
logical conditions was true.  

For a real life example, see \href{foot.html}{ELSE example}. 


	\item WHILE
\begin{verbatim}

while (logical_condition){statement 1; statement 2; ...; statement n; }
\end{verbatim}  
executes the statements in curly brackets while the logical\_expression
is true. A simple example (in case you have forgotten the first ten
factorials) would be  
\begin{verbatim}

option,-info;   ! otherwise you get redifiniton warnings
n=1; m=1;
while (n <= 10)
{
  m = m * n;  value, m;
  n = n + 1;
};
\end{verbatim}

For a real life example, see \href{foot.html}{WHILE example}.

	\item MACRO

\begin{verbatim}

label: macro = {statement 1; statement 2; ...; statement n; };
label(arg1,...,argn): macro = {statement 1; statement 2; ...; statement n; };
\end{verbatim} 
The first form allows the execution of a group of statements via a
single command:  
\begin{verbatim}

exec, label;
\end{verbatim} 
will execute the statements in curly brackets exactly once. This command
can be issued any number of times.  

The second form allows to replace strings anywhere inside the statements
in curly brackets by other strings, or integer numbers prior to
execution. This is a powerful construct and should be handled with care.  

Simple example: 
\begin{verbatim}

option,-echo,-info;  ! otherwise the output is somewhat confusing
simple(xx,yy): macro = { xx = yy^2 + xx; value, xx;};
a = 3;
b = 5;
exec, simple(a,b);
\end{verbatim}

Somewhat more tricky (a "\$" in front of an argument means that the
truncated integer value of this argument is used for replacement, rather
than the argument string itself).  
\begin{verbatim}

tricky(xx,yy,zz): macro = {mzz.yy: xx, l = 1.yy, kzz = k.yy;};
n=0;
while (n < 3)
{
  n = n+1;
  exec,tricky(quadrupole,$n,1);
  exec,tricky(sextupole,$n,2);
};
\end{verbatim} 
Whereas the actual use of the preceding example is NOT recommended,
a real life example, showing the full power (!) of macros is to be
found under \href{foot.html}{macro usage} for the usage, and
under \href{foot.html#macro}{macro definition} for the
definition.


Beware of the following rules:

	\item Generally speaking: \textit{ special constructs } like IF, WHILE,
MACRO will only allow one level of inclusion of another \textit{
special construct }.

	\item  Macros must not be called with numbers, but with strings
(i.e. variable names in case of numerical values), i.e. 



NOT

\begin{verbatim}

exec,thismacro($99,$129);
\end{verbatim}
BUT

\begin{verbatim}

n1=99; n2=219;
exec,thismacro($n1,$n2);
\end{verbatim}

\end{itemize}

%\href{http://www.cern.ch/Hans.Grote/hansg_sign.html}{hansg}, June 17, 2002


%%%\title{Range Selection}
%  Changed by: Chris ISELIN, 27-Jan-1997 
%  Changed by: Hans Grote, 16-Jan-2003 

\section{General Control Statements}

\subsection{ASSIGN}
\begin{verbatim}
assign, echo="file_name", truncate;
\end{verbatim} 
where "file\_name" is the name of an output file, or "terminal" and
truncate specifies if the file must be truncated when opened (ignored
for terminal). This allows switching the echo stream to a file or back,
but only for the commands value, show, and print. Allows easy
composition of future MAD-X input files. A real life example (always the
same) is to be found under \href{foot.html}{footprint example}.  

\subsection{CALL}
\begin{verbatim}
call, file = "file_name";
\end{verbatim} 
where "file\_name"  is the name of an input file. This file will be read
until a "return;" statement, or until end\_of\_file; it may contain any
number of calls itself, and so on to any depth.  

%% 2013-Jul-11  17:23:00  ghislain: I propose to move COGUESS to the
%% orbit correction part of the manual
\subsection{COGUESS}
\label{subsec:general_coguess}
\begin{verbatim}
coguess, tolerance = double, 
         x = double, px = double, 
         y = double, py = double, 
         t = double, pt = double;
\end{verbatim} 
sets the required convergence precision in the closed orbit search
("tolerance", see as well Twiss command
\href{../twiss/twiss.html#tolerance}{tolerance}).  

The other parameters define a first guess for all future closed orbit
searches in case they are different from zero.  

\subsection{CREATE}
\begin{verbatim}
create, table = table, column = var1, var2,_name,...;
\end{verbatim} 
creates a table with the specified variables as columns. This table can
then be \hyperlink{fill}{fill}ed, and finally one can
\hyperlink{write}{write} it in TFS format. The attribute "\_name" adds
the element name to the table at the specified column, this replaces the
undocumented "withname" attribute that was not always working properly.  

See the \href{../Introduction/select.html#ucreate}{user table I}
example; 
or an example of joining 2 tables of different length in one table
including the element name:
\href{../Introduction/select.html#screate}{user table II} 

\subsection{DELETE}
\begin{verbatim}
delete, sequence = s_name, table = t_name;
\end{verbatim} 
deletes a sequence with name "s\_name" or a table with name "t\_name"
from memory. The sequence deletion is done without influence on other
sequences that may have elements that were in the deleted sequence.  

%% 2013-Jul-11  17:24:21  ghislain: I propose to move DUMPSEQU to the
%% sequence edition and manipulation part of the manual
\subsection{DUMPSEQU}
\begin{verbatim}
dumpsequ, sequence = s_name, level = integer;
\end{verbatim} 
Actually a debug statement, but it may come handy at certain
occasions. Here "s\_name" is the name of an expanded (i.e. USEd)
sequence. The amount of detail is controlled by "level":  
\begin{verbatim}
level = 0:    print only the cumulative node length = sequence length
      > 0:    print all node (element) names except drifts
      > 2:    print all nodes with their attached parameters
      > 3:    print all nodes, and their elements with all parameters
\end{verbatim}


\subsection{EXEC}
\begin{verbatim}
exec, label;
\end{verbatim} 
Each statement may be preceded by a label; it is then stored and can be
executed again with "exec, label;" any number of times; the executed
statement may be another "exec", etc.; however, the major usage of this
statement is the execution of a \href{special.html#macro}{macro}.  

\subsection{EXIT}
\begin{verbatim}
exit;
\end{verbatim} 
ends the program execution. 

\subsection{FILL} 
Every command 
\begin{verbatim}
fill, table = table;
\end{verbatim} 
adds a new line with the current values of all column variables into the
user table \hyperlink{create}{create}d beforehand. This table one can
then \hyperlink{write}{write} in TFS format.  See as well the
\href{../Introduction/select.html#ucreate}{user table} example.  

\subsection{OPTION}
\label{subsec:general_option}
\begin{verbatim}
option, flag { = true | = false };
option, flag | -flag;
\end{verbatim} 
sets an option as given in "flag"; the part in curly brackets is
optional: if only the name of the option is given, then the option will
be set true (see second line); a "-" sign preceding the name sets it to
"false".  

Example: 
\begin{verbatim}
option, echo = true;
option, echo;
\end{verbatim} 
are identical, ditto 
\begin{verbatim}
option, echo = false;
option, -echo;
\end{verbatim} 

The available options are: 
\begin{verbatim}
  name           default meaning if true
  ====           ======= ===============
  echo            true   echoes the input on the standard output file
  warn            true   issues warnings
  info            true   issues informations
  debug           false  issues debugging information
  trace           false  prints the system time after each command
  verify          false  issues a warning if an undefined variable is used
  tell            false  prints the current value of all options
  reset           false  resets all options to their defaults
  no_fatal_stop   false  Prevents madx from stopping in case of a fatal error. 
                         Use at your own risk.

  rbarc           true   converts the RBEND straight length into the arc 
                         length
  thin_foc        true   if false suppresses the 1(rho**2) focusing of thin 
                         dipoles
  bborbit         false  the closed orbit is modified by beam-beam kicks
  sympl           false  all element matrices are symplectified in Twiss
  twiss_print     true   controls whether the twiss command produces output.
\end{verbatim} 

The option "rbarc" is implemented for backwards compatibility with MAD-8
up to version 8.23.06 included; in this version, the RBEND length was
just taken as the arc length of an SBEND with inclined pole faces,
contrary to the MAD-8 manual.  


\subsection{PRINT}
\begin{verbatim}
print, text = "...";
\end{verbatim} 
prints the text to the current output file (see ASSIGN above). The text
can be edited with the help of a  \href{special.html#macro}{macro
  statement}. For more details, see there.  


\subsection{QUIT}
\begin{verbatim}
quit;
\end{verbatim} 
ends the program execution. 


\subsection{READTABLE}
\begin{verbatim}
readtable, file = "file_name";
\end{verbatim} 
reads a TFS file containing a MAD-X table back into memory. This table
can then be manipulated as any other table, i.e. its values can be
accessed, it can be plotted, written out again etc.  


\subsection{READMYTABLE}
\label{subsec:general_readmy}
\begin{verbatim}
readmytable, file = "file_name", table = internalname;
\end{verbatim} 
reads a TFS file containing a MAD-X table back into memory. This table
can then be manipulated as any other table, i.e. its values can be
accessed, it can be plotted, written out again etc. 

An internal name for
the table can be freely assigned while for the command READTABLE it is
taken from the information section of the table itself. This feature
allows to store multiple tables of the same type in memory without
overwriting existing ones.   


\subsection{REMOVEFILE}
\begin{verbatim}
removefile, file = "file_name";
\end{verbatim} 
remove the file "file\_name" from disk. It is more portable than  
\begin{verbatim}
system("rm filename"); // Unix specific
\end{verbatim}


\subsection{RENAMEFILE}
\begin{verbatim}
renamefile, file = "file_name", name = "new_file_name";
\end{verbatim} 
rename the file "file\_name" to "new\_file\_name" on the disk. It is more
portable than  
\begin{verbatim}
system("mv file_name new_file_name"); // Unix specific
\end{verbatim}

%% 2013-Jul-11  17:24:21  ghislain: I propose to move RESBEAM to the
%% beam declaration part of the manual
\subsection{RESBEAM}
\begin{verbatim}
resbeam, sequence = sequence_name;
\end{verbatim} 
resets the default values of the beam belonging to sequence sequence\_name, or
of the default beam if no sequence is given.  


\subsection{RETURN}
\begin{verbatim}
return;
\end{verbatim} 
ends reading from a "called" file; if encountered in the standard input
file, it ends the program execution.  


\subsection{SAVE}
\label{subsec:general_save}
\begin{verbatim}
save, sequence = sequ1, sequ2, ..., file = "file_name", beam, bare;
\end{verbatim} 
saves the sequence(s) specified with all variables and elements needed
for their expansion, onto the file "file\_name". 

{\bf Warning:} If quotes are used for
the "file\_name", capital and low characters are kept as specified, if they
are omitted the "filename" will have lower characters only. 

Example:
\begin{verbatim}
save, sequence = lhc, file = "Test_One";
\end{verbatim}
saves the lhc sequence to a file name Test\_One on disk, while
\begin{verbatim}
save, sequence = lhc, file = Test_One;
\end{verbatim}
saves the lhc sequence to a file name test\_one on disk.

The optional
flag can have the value "mad8" (without the quotes), in which case the
sequence(s) is/are saved in MAD-8 input format.  

The flag "beam" is optional; when given, all beams belonging to the
sequences specified are saved at the top of the save file.  

The parameter "sequence" is optional; when omitted, all sequences are
saved.  

However, it is not advisable to use "save" without the "sequence" option
unless you know what you are doing. This practice will avoid spurious
saved entries.    Any number of "select,flag=save" commands may precede
the SAVE command. In that case, the names of elements, variables, and
sequences must match the pattern(s) if given, and in addition the
elements must be of the class(es) specified. See here for a
\href{../Introduction/select.html#save_select}{SAVE with SELECT}
example.  

It is important to note that the precision of the output of the save
command depends on the output precision. Details about default
precisions and how to adjust those precisions can be found at the
\href{../Introduction/set.html#Format}{SET Format} instruction page.   
 
The attribute 'bare' allows to save just the sequence without the
element definitions nor beam information. This allows to re-read in a
sequence with might otherwise create a stop of the program. This is
particularly useful to turn a line into a sequence to seqedit
it. 

Example:  
\begin{verbatim}
tl3:line=(ldl6,qtl301,mqn,qtl301,ldl7,qtl302,mqn,qtl302,ldl8,ison);
DLTL3 : LINE=(delay, tl3);
use, period=dltl3;

save,sequence=dltl3,file=t1,bare; // new parameter "bare": only sequ. saved
call,file=t1; // sequence is read in and is now a "real" sequence
// if the two preceding lines are suppressed, seqedit will print a warning
// and else do nothing
use, period=dltl3;
twiss, save, betx=bxa, alfx=alfxa, bety=bya, alfy=alfya;
plot, vaxis=betx, bety, haxis=s, colour:=100;
SEQEDIT, SEQUENCE=dltl3;
  remove,element=cx.bhe0330;
  remove,element=cd.bhe0330;
ENDEDIT;

use, period=dltl3;
twiss, save, betx=bxa, alfx=alfxa, bety=bya, alfy=alfya;
\end{verbatim}


\subsection{SAVEBETA}
\label{subsec:general_savebeta}
\begin{verbatim}
savebeta, label = label, place = place, sequence = sequence_name;
\end{verbatim} 
marks a place named "place" in an expanded sequence "sequence\_name"; 
at the next TWISS command execution, a
\href{../twiss/twiss.html#beta0}{beta0} 
block will be saved at that place with the label "label". This is done
only once; in order to get a new beta0 block there, one has to re-issue
the command. The contents of the beta0 block can then be used in other
commands, e.g. TWISS and MATCH.  

Example (after sequence expansion): 
\begin{verbatim}
savebeta, label = sb1, place = mb[5], sequence = fivecell;
twiss;
show, sb1;
\end{verbatim} 
will save and show the beta0 block parameters at the end (!) of the
fifth element of type mb in the sequence.  


\subsection{SELECT} %select</a}{SELECT}
\begin{verbatim}
select, flag = flag, range = range, class = class, pattern = pattern,
        slice = integer, column =s1, s2, s3,..,sn, sequence=sequence_name,
        full, clear;
\end{verbatim} 
selects one or several elements for special treatment in a subsequent
command. All selections for a given command remain valid until "clear"
is specified; the selection criteria on the same command are logically
ANDed; the selection criteria on different SELECT statements logically
ORed.   

 Example: 
\begin{verbatim}
select, flag = error, class = quadrupole, range = mb[1]/mb[5];
select, flag = error, pattern = "^mqw.*";
\end{verbatim} 
selects all quadrupoles in the range mb[1] to mb[5], and all elements
(in the whole sequence) the name of which starts with "mqw", for
treatment by the error module.  

"flag" can be one of the following: 
\begin{itemize}
   \item seqedit: selection of elements for the
     \href{seqedit.html}{seqedit} module.  
   \item error: selection of elements for the
     \href{../error/error.html}{error} assignment module.  
   \item makethin: selection of elements for the
     \href{../makethin/makethin.html}{makethin} module that
     converts the sequence into one with thin elements only.  
   \item sectormap: selection of elements for the
     \href{../Introduction/sectormap.html}{sectormap} output file
     from the Twiss module.  
   \item table: here "table" is a table name such as twiss, track
     etc., and the rows and columns to be written are selected.  
\end{itemize} 

For the RANGE, CLASS, PATTERN, FULL, and CLEAR parameters
see \href{../Introduction/select.html}{SELECT}.  

"slice" is only used by \href{../makethin/makethin.html}{makethin} and
prescribes the number of slices into which the selected elements have to
be cut (default = 1).  

"column" is only valid for tables and decides the selection of columns
to be written into the TFS file. The "name" argument is special in that
it refers to the actual name of the selected element. For an example,
see \href{../Introduction/select.html}{SELECT}.  


\subsection{SHOW}
\begin{verbatim}
show, command;
\end{verbatim} 
prints the "command" (typically "beam", "beam\%sequ", or an element
name), with the actual value of all its parameters.  


\subsection{STOP}
\begin{verbatim}
stop;
\end{verbatim} 
ends the program execution. 


\subsection{SYSTEM}
\begin{verbatim}
system, "string";
\end{verbatim} 
transfers the string in quotes to the system for execution.  

Example: 
\begin{verbatim}
system,"ln -s /afs/cern.ch/user/u/user/public/some/directory short";
\end{verbatim}


\subsection{TABSTRING}
Note: this is not a command and should appear in the variables section
\begin{verbatim}
tabstring(arg1,arg2,arg3)
\end{verbatim}  
The "string function" tabstring(arg1,arg2,arg3) with exactly  three
arguments; arg1 is a table name (string), arg2 is a column name
(string), arg3 is a row number (integer), count starts at 0. The
function can be used in any context where a string appears; in case
there is no match, it returns \_void\_.  


\subsection{TITLE}
\begin{verbatim}
title, "title";
\end{verbatim} 
inserts the string in quotes as title in various tables and plots.  


\subsection{USE}
\label{subsec:general_use}
\begin{verbatim}
use, period = sequence_name, range = range, survey;
\end{verbatim} 
expands the sequence with name "sequence\_name", or a part of it as specified
in the \href{../Introduction/ranges.html#range}{range}. The
\texttt{survey} option plugs the survey data into the sequence elements
nodes on the first pass (see \href{../survey/survey.html}{survey}).  


\subsection{VALUE}
\begin{verbatim}
value, exp1, exp2,...;
\end{verbatim} 
prints the actual values of the expressions given. 

Example: 
\begin{verbatim}
a = clight/1000.;
value, a, pmass, exp(sqrt(2));
\end{verbatim} 
results in 
\begin{verbatim}
a = 299792.458         ;
pmass = 0.938271998        ;
exp(sqrt(2)) = 4.113250379        ;
\end{verbatim}


\subsection{WRITE}
\label{subsec:general_write}
\begin{verbatim}
write, table = table, file = "file_name";
\end{verbatim} 
writes the table "table" onto the file "file\_name"; only the rows and
columns of a preceding \verb+select, flag = table,...;+ are written. If no select
has been issued for this table, the file will only contain the
header. If the FILE argument is omitted, the table is written to
standard output.  


%\href{http://www.cern.ch/Hans.Grote/hansg_sign.html}{hansg}, June 17, 2002 

%%\title{BEAM}
%  Changed by: Hans Grote, 30-Sep-2002 

\section{BEAM: Set Beam Parameters}

Many commands in MAD-X require the setting of various quantities related
to the beam in the machine. Therefore, MAD-X will stop with a fatal
error if an attempt is made to expand (USE) a sequence for which no BEAM
command has been issued before.  

The quantities are entered by a BEAM command: 
\begin{verbatim}
BEAM, PARTICLE=name,MASS=real,CHARGE=real,
      ENERGY=real,PC=real,GAMMA=real,
      EX=real,EXN=real,EY=real,EYN=real,
      ET=real,SIGT=real,SIGE=real,
      KBUNCH=integer,NPART=real,BCURRENT=real,
      BUNCHED=logical,RADIATE=logical,BV=integer,SEQUENCE=name;
\end{verbatim} 

{\bf Warning:} BEAM updates, i.e. it replaces attributes explicitely
mentioned, but does not return to default values for others! To reset to
\href{resbeam.html#defaults}{beam value defaults},  use
\href{resbeam.html}{RESBEAM}.

The particle restmass and \href{charge}{charge} are defined by:
\begin{itemize}
   \item \href{particle}{PARTICLE}: The name of particles in the
     machine. MAD knows the restmass and the charge for the
     following particles:  
     \begin{itemize}
	  \item POSITRON: The particles are positrons (MASS=\textit{m$_e$}, CHARGE=1), 
	  \item ELECTRON: The particles are electrons (MASS=\textit{m$_e$}, CHARGE=-1), 
	  \item PROTON: The particles are protons (MASS=\textit{m$_p$}, CHARGE=1), 
	  \item ANTIPROTON: The particles are anti-protons (MASS=\textit{m$_p$}, CHARGE=-1). 
	  \item POSMUON: The particles are positive muons (MASS=\textit{m$_mu$}, CHARGE=1), 
	  \item NEGMUON: The particles are negative muons (MASS=\textit{m$_mu$}, CHARGE=-1). 
     \end{itemize}
\end{itemize} 

Therefore neither restmass nor charge can be modified for these
predefined particles. On the other hand, for ions and all other user
defined particles the name, restmass, and charge can be entered
independently.  

By default the total particle energy is 1 GeV. A different value can be 
defined by one of the following: 
 
\begin{itemize}
   \item \href{energy}{ENERGY}: The total energy per particle in
     GeV. If given, it must be greater then the particle restmass.  
   \item \href{pc}{PC}: The momentum per particle in GeV/c. If
     given, it must be greater than zero.  
   \item \href{gamma}{GAMMA}: The ratio between total energy and
     rest energy of the particles: GAMMA = \textit{E / m$_0$}. If
     given, it must be greater than one. If the restmass is changed
     a new value for the energy should be entered. Otherwise the
     energy remains unchanged, and the momentum PC and the quantity
     GAMMA are recalculated . 
\end{itemize}  

The emittances are defined by: 
\begin{itemize}
   \item \href{ex}{EX}: The horizontal emittance \textit{E$_x$} (default: 1 m). 
   \item \href{ey}{EY}: The vertical emittance \textit{E$_y$} (default: 1 m). 
   \item \href{et}{ET}: The longitudinal emittance \textit{E$_t$} (default: 1 m). 
\end{itemize}  

The emittances can be replaced by the normalised emittances and the
energy spread:  
\begin{itemize}
   \item \href{exn}{EXN}: The normalised horizontal emittance [m]:
     \textit{E$_xn$} = 4 (GAMMA$^2$ - 1)$^{1/2}$\textit{E$_x$} (ignored
     if \textit{E$_x$} is given).  
   \item \href{eyn}{EYN}: The normalised vertical emittance [m]:
     \textit{E$_yn$} = 4 (GAMMA$^2$ - 1)$^{1/2}$\textit{E$_y$} (ignored
     if \textit{E$_x$} is given).  
   \item \href{sigt}{SIGT}: The bunch length \textit{c}
     sigma(\textit{t}) in [m].  
   \item \href{sige}{SIGE}: The \emph{relative} energy spread
     sigma(\textit{E})/\textit{E} in [1].  
\end{itemize} 

Certain commands compute the synchrotron tune \textit{Q$_s$} from the RF
cavities. If \textit{Q$_s$} is non-zero, the relative energy spread and
the bunch length are  \\
sigma(\textit{E}) / \textit{p$_0$ c =  (2 pi Q$_s$ E$_t$ / ETA C)$^{1/2}$}, 

\textit{c} sigma(\textit{t}) = (ETA C E$_t$ / 2 pi Q$_s$)$^{1/2}$. 

\textit{C} is the machine circumference, and 

\textit{ETA} = GAMMA$^{-2}$ - GAMMA(transition)$^{-2}$. 

The order of precedence in the parameter evaluation is given below: 
\begin{verbatim}
    particle->(mass+charge)
    energy->pc->gamma->beta
    ex->exn
    ey->eyn
    current->npart
    et->sigt->sige
\end{verbatim} 

where any item to the left takes precendence over the others. 

Finally, the BEAM command accepts 
\begin{itemize}
   \item \href{kbunch}{KBUNCH}: The number of particle bunches in the
     machine (default: 1).  
   \item \href{npart}{NPART}: The number of particles per bunch (default: 0). 
   \item \href{bcurrent}{BCURRENT}: The bunch current (default: 0 A). 
   \item \href{bunched}{BUNCHED}: A logical flag. If set, the beam is
     treated as bunched whenever this makes sense.  
   \item \href{radiate}{RADIATE}: A logical flag. If set, synchrotron
     radiation is considered in all bipolar magnets.  
   \item \href{bv}{BV}: an integer specifying the direction of the
     particle movement in a beam line; either +1 (default), or -1. For a
     detailed explanation see under \href{bv_flag.html}{bv flag}.  
   \item \href{sequence}{SEQUENCE}: this attaches the beam command to a
     specific sequence; if the name is omitted, the BEAM command refers
     to the default beam always present. Sequences without attached beam
     use this default beam. When updating a beam, the corresponding
     sequence name, if any, must always be mentioned.  
\end{itemize} 

The BEAM command changes only the parameters entered. The command
\href{resbeam.html}{RESBEAM} resets all beam data to their
\href{resbeam.html#defaults}{beam value defaults}.  

Examples: 
\begin{verbatim}
BEAM, PARTICLE = ELECTRON, ENERGY = 50, EX = 1.E-6, EY = 1.E-8, SIGE = 1.E-3;
 ...
BEAM, RADIATE;
 ...
RESBEAM;
BEAM, EX = 2.E-5, EY = 3.E-7, SIGE = 4.E-3;
\end{verbatim} 

The first command selects electrons, and sets energy and emittances. The
second one turns on synchrotron radiation. The last two select positrons
(by default), set the energy to 1 GeV (default), clear the synchrotron
radiation flag, and set the emittances to the values entered.  

Some program modules of MAD-X may also store data into a beam data
block. Expressions may refer to data in this beam data block using the
notation  
\begin{verbatim}
BEAM->attribute-name
\end{verbatim} 
or 
\begin{verbatim}
BEAM%sequence->attribute-name.
\end{verbatim} 

This notation refers to the value of attribute-name found in the default
BEAM resp. the beam belonging to the sequence given. This can be used
for receiving or using values, e.g. 

\begin{verbatim}
value, beam%lhcb2->bv;
\end{verbatim} 
or for storing values in the beam (this does NOT trigger an update of dependent variables !), e.g. 
\begin{verbatim}
beam->charge=-1;
\end{verbatim} 

The current values in the BEAM bank can be obtained by the command
\begin{verbatim}
show,beam;
\end{verbatim}
resp.
\begin{verbatim}
show,beam%sequence;
\end{verbatim}


%\href{http://www.cern.ch/Hans.Grote/hansg_sign.html}{hansg} 11.9.2000 

%%\title{DRIFT}
%  Changed by: Chris ISELIN, 24-Jan-1997 
%  Changed by: Hans Grote, 10-Jun-2002 

\section{RESBEAM: reset beam defaults}
\label{sec:resbeam}

\begin{verbatim}
label: RESBEAM,SEQUENCE=name;
\end{verbatim} 

If the sequence name is omitted, the default beam is reset. 
\begin{verbatim}
Default BEAM Data:
PARTICLE        POSITRON
ENERGY          1 GeV
EX              1 rad m
EY              1 rad m
ET              1 rad m
KBUNCH          1
NPART           0
BCURRENT        0 A
BUNCHED         .TRUE.
RADIATE         .FALSE.
\end{verbatim}

%\href{http://www.cern.ch/Hans.Grote/hansg_sign.html}{hansg}, January 24, 1997 


%%\title{BV flag in the Beam command}
%  Changed by: Thys Risselada 29-Mar-2009 

\section{BV FLAG}

When reversing the direction ("V") of a particle in a magnetic field
("B") while keeping its charge constant, the resulting force V * B
changes sign. This is equivalent to flipping the field, but not the
direction.  

For practical reasons the properties of all elements of the LHC are
defined in the MADX input as if they apply to a clockwise proton beam
("LHC beam 1"). This allows a single definition for elements traversed
by both beams. Their effects on a beam with identical particle charge
but running in the opposite direction ("LHC beam 2") must then be
reversed inside the program.  

In MADX this may be taken into account by setting the value of the BV
attribute in the Beam commands. In the case of LHC beam 1 (clockwise)
and beam 2 (counter-clockwise), treated in MADX both as clockwise proton
beams, the Beam commands must look as follows: 

\begin{verbatim}
beam, sequence=lhcb1, particle=proton, pc=450, bv = +1;

beam, sequence=lhcb2, particle=proton, pc=450, bv = -1;
\end{verbatim}

%\href{http://www.cern.ch/Frank.Schmidt/frs_sign.html}{frs}, March 29, 2009 

%%%\title{PLOT}
%  Changed by: Chris ISELIN, 27-Jan-1997
% 

%  Changed by: Hans Grote, 25-Sep-2002 

%  Changed by:
% E. T. d'Amico, 20-Oct-2004 

%%\usepackage{hyperref}
% commands generated by html2latex


%%\begin{document}
%%\begin{center}
 %%EUROPEAN ORGANIZATION FOR NUCLEAR RESEARCH 
%%\includegraphics{http://cern.ch/madx/icons/mx7_25.gif}

\subsection{PLOT}
%%\end{center}

 Values contained in MAD-X tables can be plotted in the form column versus column, with up to four differently scaled vertical axes; furthermore, if the horizontal axis is the position "s" of the elements in a sequence, then the symbolic machine can be plotted above the curves as well. In certain conditions True interpolation inside the element is available (through calls to the Twiss module for each slice) .  The "environment" (interpolation, line thickness, annotation size, PostScript format) can be set with the \hyperlink{setplot}{setplot} command.  
\begin{itemize}
	\item \textbf{PLOT}
	
\begin{verbatim}
 plot, vaxis=vname1,vname2,..,vnamen,
vaxis1=vname1,vname2,..,vnamen, vaxis2=vname1,vname2,..,vnamen,
vaxis3=vname1,vname2,..,vnamen, vaxis4=vname1,vname2,..,vnamen,
haxis=vname, hmin=real, hmax=real, vmin=reals, vmax=reals, bars=integer,
style=integer, colour=integer, symbol=integer, noversion=logical,
interpolate=logical, noline=logical, notitle=logical, marker_plot=logical, 
range_plot=logical, table=table_name, particle=particle1,particle2,
..,particlen,
multiple=logical, title=string, range=range, file=file_name_start, 
ptc=logical, ptc_table=table_name, trackfile=table_name; 
\end{verbatim} where the parameters have the following meaning: 
\begin{itemize}
	\item vaxis: one or several variables from the table to be plotted against the (only) vertical axis.  
	\item vaxis1: one or several variables from the table to be plotted against the vertical axis number 1 (out of 4 possible ones). 
	\item vaxis2: one or several variables from the table to be plotted against the vertical axis number 2 (out of 4 possible ones). 
	\item vaxis3: one or several variables from the table to be plotted against the vertical axis number 3 (out of 4 possible ones). 
	\item vaxis4: one or several variables from the table to be plotted against the vertical axis number 4 (out of 4 possible ones). 
	\item \textit{Important: vaxis and vaxisI are exclusive in their application!}
	\item haxis: name of the horizontal variable 
	\item hmin: lower horizontal edge 
	\item hmax: upper horizontal edge; to be used, both hmin and hmax must be given.  
	\item vmin:lower edges of vertical axes, up to four numbers 
	\item vmax:upper edges of vertical axes, up to four numbers; both vmin and vmax must be given for an axis to be effective.  
	\item bars: 0 (default) or 1 - in the latter case, all curve points coming from the table are connected with the horizontal axis by vertical bars.  
	\item style: 1 (default), 2, 3, or 4: curve style, being solid, dashed, dotted, and dot-dashed; a value of 100 makes MAD-X use these four styles in turn for successive curves in the same plot. If style is 0 no curve is printed between points.  N.B. If symbol and style are null at the same time, style is forced to its default value (= 1).  
	\item colour: 1 (default), 2, 3, , or 5: colour, being black, red, green, blue, and magenta; a value of 100 makes MAD-X use these five colours in turn for successive curves. 
	\item symbol: 0 (default), 1, 2, 3, 4, or 5: none, dot, "+", "*", circle, and "x". These symbols are potted at all curve points; there size may have to be adapted (see below).  
	\item noversion: logical, default=false. If set true, the information concerning the madx version and the date are suppressed from the title. This option frees more space for the user's title.  
	\item interpolate: logical, default=false. Normally the curve points from the table are connected by straight lines; if "interpolate" is requested, then on-momentum Twiss parameters such as beta, alfa, and dispersion are interpolated with calls to the Twiss module inside each element, for all other variables splines are used to smooth the curves.  
	\item noline: logical, default=false. If s is the horizontal variable, then the machine will be plotted in symbolic form above the curve plot (except for tables having been read back into MAD-X). This may result in a thick black block if the horizontal scale is too large. "noline" allows the user to suppress the machine plotting.   
	\item notitle: logical, default=false. If true, suppresses the title line, including the information on the version and date. 
	\item marker\_plot: logical, default=false. If true, plotting is done also at the location of marker elements. This is only useful for the plotting of non-continuous functions like the "N1" from the aperture module. Beware that the PS file might became very large if this flag is invoked. 
	\item range\_plot: logical, default=false. Needed to allow to specify a plotting range also for user defined horizontal axis.  
	\item table: name of the table to be plotted from (default: twiss). If it is \textit{track}, the data to be plotted are taken from the tracking files generated for each required particle as defined by the attribute \textit{particle}. The name of this file has the following format: file name as defined by the attribute \textit{trackfile}, the observation point fixed to 1 and the particle number, e.g. \textit{testtrack.obs0001.p0003}.  If the required file has not been generated by the previous MAD-X command track, no plot is done for that particle.  The plot is obtained through the \textit{gnuplot} package.  N.B. the previous track command should contain the attribute \textit{dump}. The tracking plots appends the plots to an existing file specified via \textit{filename} appended by \textit{.ps}. The user should make sure that this file does not exist before starting a MAD-X run!
	\item particle: one or several numbers associated to the tracked particles for which the specified plot has to be displayed. 
	\item multiple: logical, default=false. If true all the curves generated for each tracked particle are put on one plot. Otherwise there will be one plot for each particle.  
	\item title: plot title string; if absent, the last overall title is used; if no such overall title as well, the sequence name is used.  
	\item range: horizontal plot \href{../Introduction/ranges.html}{range} given by elements. 
	\item file\_name: start of the file name for the Postscript file(s). Only the first occurrence of such a name will be used. Default is "madx" or "madx\_track" if the \textit{table} attribute is track.  Depending on the format (.ps or .eps, see below) the plots will either all be written into one file file\_name.ps, or one per plot into file\_name01.eps, file\_name02.eps, etc.  
	\item ptc: logical, default=false. If set true, the data to be plotted are taken from the table defined by the attribute \textit{ptc\_table} which is expected to be generated previously by the ptc package. The data belong to the column identified by one of the names set in the definition of the ptc twiss table. Interpolation is not available and the attribute \textit{interpolate} has no effect.  
	\item ptc\_table: name of the ptc twiss table to be plotted from (default: ptc\_twiss) 
	\item trackfile: first part of the name of the files containing tracking data for each particle (default: track) 
\end{itemize}


	\item \textbf{SETPLOT}
\begin{verbatim}
 setplot,
post=integer,font=integer, lwidth=real,xsize=real,ysize=real,
ascale=real, lscale=real, sscale=real, rscale=real; \end{verbatim} where the parameters have the following meaning: 
\begin{itemize}
	\item post: default = 1. If =1, makes one PostScript file (.ps) with all plots; if =2, makes one Encapsulated PostSscript file (.eps) per plot.  
	\item font: there are two defaults: 1 for screen plotting: this uses characters made from polygons; -1 for PostScript files; this is Times-Italic. There are various fonts available for positive and negative integers, best to be tried out, since they will look different on different systems anyway. GhostView will show strange vertical axis annotations, but the printed versions are normally OK.  
	\item lwidth: default = 1. Allows the user to set the curve line width.  Depends on the system as well, so to be tried out.  
	\item xsize: bounding box size for PostScript, default=27 cm.  
	\item ysize: bounding box size for PostScript, default=19 cm.  
	\item ascale: annotation character height scale factor, default=1.  
	\item lscale: axis label character height scale factor, default=1.  
	\item sscale: curve symbol (see above) scale factor, default=1.  
	\item rscale: axis text character height scale factor, default=1.  
\end{itemize}


	\item \textbf{RESPLOT}
\begin{verbatim}
 resplot; \end{verbatim} resets all defaults for the setplot command.  
\end{itemize}\href{http://www.cern.ch/Hans.Grote/hansg_sign.html}{hansg}, June 17, 2002, rdemaria \href{http://cern.ch/rdemaria}{rdemaria}, September 2007. 

%%\end{document}



%%%%\title{Range Selection}
%  Changed by: Chris ISELIN, 27-Jan-1997 
%  Changed by: Hans Grote, 10-Jun-2002 

\paragraph{Real life example for IF statements, and MACRO usage}


\begin{verbatim}

! Creates a footprint for head-on + parasitic collisions at IP1+5 
! of lhc.6.5; both lhcb1 (for tracking) and lhcb2 (to define the
! beam-beam elements, i.e. weak-strong) are used; there are flags to
! select head-on, left, and right parasitic separately at all IPs.
! The bunch spacing can be given in nanosec and automatically creates
! the beam-beam interaction points at the correct positions.
! It is important to set the correct BEAM parameters, i.e. number
! of particles, emittances, bunch length, energy.

!--- For completeness, all files needed by this job are copied
!    to the local directory ldb. The links to the the originals
!    in offdb (official database) are commented out.

Option,  warn,info,echo;
!System,
"ln -fns /afs/cern.ch/eng/sl/MAD-X/dev/test_suite/foot/V3.01.01 ldb";
!system,"ln -fns /afs/cern.ch/eng/lhc/optics/V6.4 offdb";
Option, -echo,-info,warn;
SU=1.0;
call, file = "ldb/V6.5.seq";
call,file="ldb/slice_new.madx";
Option, echo,info,warn;

!+++++++++++++++++++++++++ Step 1 +++++++++++++++++++++++
! 	define beam constants
!++++++++++++++++++++++++++++++++++++++++++++++++++++++++

b_t_dist = 25.e-9;                  !--- bunch distance in [sec]
b_h_dist = clight * b_t_dist / 2 ;  !--- bunch half-distance in [m]
ip1_range = 58.;                     ! range for parasitic collisions
ip5_range = ip1_range;
ip2_range = 60.;
ip8_range = ip2_range;

npara_1 = ip1_range / b_h_dist;     ! # parasitic either side
npara_2 = ip2_range / b_h_dist;
npara_5 = ip5_range / b_h_dist;
npara_8 = ip8_range / b_h_dist;

value,npara_1,npara_2,npara_5,npara_8;

 eg   =  7000;
 bg   =  eg/pmass;
 en   = 3.75e-06;
 epsx = en/bg;
 epsy = en/bg;

Beam, particle = proton, sequence=lhcb1, energy = eg,
          sigt=      0.077     , 
          bv = +1, NPART=1.1E11, sige=      1.1e-4, 
          ex=epsx,   ey=epsy;

Beam, particle = proton, sequence=lhcb2, energy = eg,
          sigt=      0.077     , 
          bv = -1, NPART=1.1E11, sige=      1.1e-4, 
          ex=epsx,   ey=epsy;

beamx = beam%lhcb1->ex;   beamy%lhcb1 = beam->ey;
sigz  = beam%lhcb1->sigt; sige = beam%lhcb1->sige;

!--- split5, 4d
long_a= 0.53 * sigz/2;
long_b= 1.40 * sigz/2;
value,long_a,long_b;

ho_charge = 0.2;

!+++++++++++++++++++++++++ Step 2 +++++++++++++++++++++++
! 	slice, flatten sequence, and cycle start to ip3
!++++++++++++++++++++++++++++++++++++++++++++++++++++++++

use,sequence=lhcb1;
makethin,sequence=lhcb1;
!save,sequence=lhcb1,file=lhcb1_thin_new_seq;
use,sequence=lhcb2;
makethin,sequence=lhcb2;
!save,sequence=lhcb2,file=lhcb2_thin_new_seq;
!stop;

option,-warn,-echo,-info;
call,file="ldb/V6.5.thin.coll.str";
option,warn,echo,info;

! keep sextupoles
ksf0=ksf; ksd0=ksd;
use,period=lhcb1;
select,flag=twiss.1,column=name,x,y,betx,bety;
twiss,file;
plot,haxis=s,vaxis=x,y,colour=100,noline;

use,period=lhcb2;
select,flag=twiss.2,column=name,x,y,betx,bety;
twiss,file;
plot,haxis=s,vaxis=x,y,colour=100,noline;
seqedit,sequence=lhcb1;
flatten;
endedit;

seqedit,sequence=lhcb1;
cycle,start=ip3.b1;
endedit;

seqedit,sequence=lhcb2;
flatten;
endedit;

seqedit,sequence=lhcb2;
cycle,start=ip3.b2;
endedit;

bbmarker: marker;  /* for subsequent remove */


!+++++++++++++++++++++++++ Step 3 +++++++++++++++++++++++
! 	define the beam-beam elements
!++++++++++++++++++++++++++++++++++++++++++++++++++++++++
!
!===========================================================
! read macro definitions
option,-echo;
call,file="ldb/bb.macros";
option,echo;

!
!===========================================================
!   this sets CHARGE in the head-on beam-beam elements. 
!   set +1 * ho_charge   for parasitic on, 0 for off

 on_ho1  = +1 * ho_charge; ! ho_charge depends on split
 on_ho2  = +0 * ho_charge; ! because of the "by hand" splitting
 on_ho5  = +1 * ho_charge;
 on_ho8  = +0 * ho_charge;

!
!===========================================================
!   set CHARGE in the parasitic beam-beam elements. 
!   set +1 for parasitic on, 0 for off
 on_lr1l = +1;
 on_lr1r = +1;
 on_lr2l = +0;
 on_lr2r = +0;
 on_lr5l = +1;
 on_lr5r = +1;
 on_lr8l = +0;
 on_lr8r = +0;

!
!===========================================================
!   define markers and savebetas
assign,echo=temp.bb.install;
!--- ip1
if (on_ho1  0)
{
  exec, mkho(1);
  exec, sbhomk(1);
}
if (on_lr1l  0 || on_lr1r  0)
{
  n=1; ! counter
  while (n  0 || on_lr1l  0)
{
  n=1; ! counter
  while (n  0)
{
  exec, mkho(5);
  exec, sbhomk(5);
}
if (on_lr5l  0 || on_lr5r  0)
{
  n=1; ! counter
  while (n  0 || on_lr5l  0)
{
  n=1; ! counter
  while (n  0)
{
  exec, mkho(2);
  exec, sbhomk(2);
}
if (on_lr2l  0 || on_lr2r  0)
{
  n=1; ! counter
  while (n  0 || on_lr2l  0)
{
  n=1; ! counter
  while (n  0)
{
  exec, mkho(8);
  exec, sbhomk(8);
}
if (on_lr8l  0 || on_lr8r  0)
{
  n=1; ! counter
  while (n  0 || on_lr8l  0)
{
  n=1; ! counter
  while (n  0)
{
exec, inho(mk,1);
}
if (on_lr1l  0 || on_lr1r  0)
{
  n=1; ! counter
  while (n  0 || on_lr1l  0)
{
  n=1; ! counter
  while (n  0)
{
exec, inho(mk,5);
}
if (on_lr5l  0 || on_lr5r  0)
{
  n=1; ! counter
  while (n  0 || on_lr5l  0)
{
  n=1; ! counter
  while (n  0)
{
exec, inho(mk,2);
}
if (on_lr2l  0 || on_lr2r  0)
{
  n=1; ! counter
  while (n  0 || on_lr2l  0)
{
  n=1; ! counter
  while (n  0)
{
exec, inho(mk,8);
}
if (on_lr8l  0 || on_lr8r  0)
{
  n=1; ! counter
  while (n  0 || on_lr8l  0)
{
  n=1; ! counter
  while (n betx) / 0.0007999979093;
value,on_sep2;
!===========================================================
!   define bb elements
assign,echo=temp.bb.install;
!--- ip1
if (on_ho1  0)
{
exec, bbho(1);
}
if (on_lr1l  0)
{
  n=1; ! counter
  while (n  0)
{
  n=1; ! counter
  while (n  0)
{
exec, bbho(5);
}
if (on_lr5l  0)
{
  n=1; ! counter
  while (n  0)
{
  n=1; ! counter
  while (n  0)
{
exec, bbho(2);
}
if (on_lr2l  0)
{
  n=1; ! counter
  while (n  0)
{
  n=1; ! counter
  while (n  0)
{
exec, bbho(8);
}
if (on_lr8l  0)
{
  n=1; ! counter
  while (n  0)
{
  n=1; ! counter
  while (n  0)
{
exec, inho(bb,1);
}
if (on_lr1l  0)
{
  n=1; ! counter
  while (n  0)
{
  n=1; ! counter
  while (n  0)
{
exec, inho(bb,5);
}
if (on_lr5l  0)
{
  n=1; ! counter
  while (n  0)
{
  n=1; ! counter
  while (n  0)
{
exec, inho(bb,2);
}
if (on_lr2l  0)
{
  n=1; ! counter
  while (n  0)
{
  n=1; ! counter
  while (n  0)
{
exec, inho(bb,8);
}
if (on_lr8l  0)
{
  n=1; ! counter
  while (n  0)
{
  n=1; ! counter
  while (n  footprint";
stop;
\end{verbatim}

\paragraph{\href{macro}{Real life example of MACRO definitions}}

\begin{verbatim}

bbho(nn): macro = {
!--- macro defining head-on beam-beam elements; nn = IP number
print, text="bbipnnl2: beambeam, sigx=sqrt(rnnipnnl2->betx*epsx),";
print, text="          sigy=sqrt(rnnipnnl2->bety*epsy),";
print, text="          xma=rnnipnnl2->x,yma=rnnipnnl2->y,";
print, text="          charge:=on_honn;";
print, text="bbipnnl1: beambeam, sigx=sqrt(rnnipnnl1->betx*epsx),";
print, text="          sigy=sqrt(rnnipnnl1->bety*epsy),";
print, text="          xma=rnnipnnl1->x,yma=rnnipnnl1->y,";
print, text="          charge:=on_honn;";
print, text="bbipnn:   beambeam, sigx=sqrt(rnnipnn->betx*epsx),";
print, text="          sigy=sqrt(rnnipnn->bety*epsy),";
print, text="          xma=rnnipnn->x,yma=rnnipnn->y,";
print, text="          charge:=on_honn;";
print, text="bbipnnr1: beambeam, sigx=sqrt(rnnipnnr1->betx*epsx),";
print, text="          sigy=sqrt(rnnipnnr1->bety*epsy),";
print, text="          xma=rnnipnnr1->x,yma=rnnipnnr1->y,";
print, text="          charge:=on_honn;";
print, text="bbipnnr2: beambeam, sigx=sqrt(rnnipnnr2->betx*epsx),";
print, text="          sigy=sqrt(rnnipnnr2->bety*epsy),";
print, text="          xma=rnnipnnr2->x,yma=rnnipnnr2->y,";
print, text="          charge:=on_honn;";
};

mkho(nn): macro = {
!--- macro defining head-on markers; nn = IP number
print, text="mkipnnl2: bbmarker;";
print, text="mkipnnl1: bbmarker;";
print, text="mkipnn:   bbmarker;";
print, text="mkipnnr1: bbmarker;";
print, text="mkipnnr2: bbmarker;";
};

inho(xx,nn): macro = {
!--- macro installing bb or markers for head-on beam-beam (split into 5)
print, text="install, element= xxipnnl2, at=-long_b, from=ipnn;";
print, text="install, element= xxipnnl1, at=-long_a, from=ipnn;";
print, text="install, element= xxipnn,   at=1.e-9,   from=ipnn;";
print, text="install, element= xxipnnr1, at=+long_a, from=ipnn;"; 
print, text="install, element= xxipnnr2, at=+long_b, from=ipnn;"; 
};

sbhomk(nn): macro = {
!--- macro to create savebetas for ho markers
print, text="savebeta, label=rnnipnnl2, place=mkipnnl2;";
print, text="savebeta, label=rnnipnnl1, place=mkipnnl1;";
print, text="savebeta, label=rnnipnn,   place=mkipnn;";
print, text="savebeta, label=rnnipnnr1, place=mkipnnr1;";
print, text="savebeta, label=rnnipnnr2, place=mkipnnr2;";    
};

mkl(nn,cc): macro = {
!--- macro to create parasitic bb marker on left side of ip nn; cc = count
print, text="mkipnnplcc: bbmarker;";
};

mkr(nn,cc): macro = {
!--- macro to create parasitic bb marker on right side of ip nn; cc = count
print, text="mkipnnprcc: bbmarker;";
};

sbl(nn,cc): macro = {
!--- macro to create savebetas for left parasitic
print, text="savebeta, label=rnnipnnplcc, place=mkipnnplcc;";
};

sbr(nn,cc): macro = {
!--- macro to create savebetas for right parasitic
print, text="savebeta, label=rnnipnnprcc, place=mkipnnprcc;";
};

inl(xx,nn,cc): macro = {
!--- macro installing bb and markers for left side parasitic beam-beam
print, text="install, element= xxipnnplcc, at=-cc*b_h_dist, from=ipnn;";
};

inr(xx,nn,cc): macro = {
!--- macro installing bb and markers for right side parasitic beam-beam
print, text="install, element= xxipnnprcc, at=cc*b_h_dist, from=ipnn;";
};

bbl(nn,cc): macro = {
!--- macro defining parasitic beam-beam elements; nn = IP number
print, text="bbipnnplcc: beambeam, sigx=sqrt(rnnipnnplcc->betx*epsx),";
print, text="          sigy=sqrt(rnnipnnplcc->bety*epsy),";
print, text="          xma=rnnipnnplcc->x,yma=rnnipnnplcc->y,";
print, text="          charge:=on_lrnnl;";
};

bbr(nn,cc): macro = {
!--- macro defining parasitic beam-beam elements; nn = IP number
print, text="bbipnnprcc: beambeam, sigx=sqrt(rnnipnnprcc->betx*epsx),";
print, text="          sigy=sqrt(rnnipnnprcc->bety*epsy),";
print, text="          xma=rnnipnnprcc->x,yma=rnnipnnprcc->y,";
print, text="          charge:=on_lrnnr;";
};
\end{verbatim}

%\href{http://www.cern.ch/Hans.Grote/hansg_sign.html}{hansg}, June 17, 2002 


%%%%\title{Range Selection}
%  Changed by: Chris ISELIN, 27-Jan-1997 
%  Changed by: Hans Grote, 16-Jan-2003 

\section{General Control Statements}

\subsection{ASSIGN}
\begin{verbatim}
assign, echo="file_name", truncate;
\end{verbatim} 
where "file\_name" is the name of an output file, or "terminal" and
truncate specifies if the file must be truncated when opened (ignored
for terminal). This allows switching the echo stream to a file or back,
but only for the commands value, show, and print. Allows easy
composition of future MAD-X input files. A real life example (always the
same) is to be found under \href{foot.html}{footprint example}.  

\subsection{CALL}
\begin{verbatim}
call, file = "file_name";
\end{verbatim} 
where "file\_name"  is the name of an input file. This file will be read
until a "return;" statement, or until end\_of\_file; it may contain any
number of calls itself, and so on to any depth.  

%% 2013-Jul-11  17:23:00  ghislain: I propose to move COGUESS to the
%% orbit correction part of the manual
\subsection{COGUESS}
\label{subsec:general_coguess}
\begin{verbatim}
coguess, tolerance = double, 
         x = double, px = double, 
         y = double, py = double, 
         t = double, pt = double;
\end{verbatim} 
sets the required convergence precision in the closed orbit search
("tolerance", see as well Twiss command
\href{../twiss/twiss.html#tolerance}{tolerance}).  

The other parameters define a first guess for all future closed orbit
searches in case they are different from zero.  

\subsection{CREATE}
\begin{verbatim}
create, table = table, column = var1, var2,_name,...;
\end{verbatim} 
creates a table with the specified variables as columns. This table can
then be \hyperlink{fill}{fill}ed, and finally one can
\hyperlink{write}{write} it in TFS format. The attribute "\_name" adds
the element name to the table at the specified column, this replaces the
undocumented "withname" attribute that was not always working properly.  

See the \href{../Introduction/select.html#ucreate}{user table I}
example; 
or an example of joining 2 tables of different length in one table
including the element name:
\href{../Introduction/select.html#screate}{user table II} 

\subsection{DELETE}
\begin{verbatim}
delete, sequence = s_name, table = t_name;
\end{verbatim} 
deletes a sequence with name "s\_name" or a table with name "t\_name"
from memory. The sequence deletion is done without influence on other
sequences that may have elements that were in the deleted sequence.  

%% 2013-Jul-11  17:24:21  ghislain: I propose to move DUMPSEQU to the
%% sequence edition and manipulation part of the manual
\subsection{DUMPSEQU}
\begin{verbatim}
dumpsequ, sequence = s_name, level = integer;
\end{verbatim} 
Actually a debug statement, but it may come handy at certain
occasions. Here "s\_name" is the name of an expanded (i.e. USEd)
sequence. The amount of detail is controlled by "level":  
\begin{verbatim}
level = 0:    print only the cumulative node length = sequence length
      > 0:    print all node (element) names except drifts
      > 2:    print all nodes with their attached parameters
      > 3:    print all nodes, and their elements with all parameters
\end{verbatim}


\subsection{EXEC}
\begin{verbatim}
exec, label;
\end{verbatim} 
Each statement may be preceded by a label; it is then stored and can be
executed again with "exec, label;" any number of times; the executed
statement may be another "exec", etc.; however, the major usage of this
statement is the execution of a \href{special.html#macro}{macro}.  

\subsection{EXIT}
\begin{verbatim}
exit;
\end{verbatim} 
ends the program execution. 

\subsection{FILL} 
Every command 
\begin{verbatim}
fill, table = table;
\end{verbatim} 
adds a new line with the current values of all column variables into the
user table \hyperlink{create}{create}d beforehand. This table one can
then \hyperlink{write}{write} in TFS format.  See as well the
\href{../Introduction/select.html#ucreate}{user table} example.  

\subsection{OPTION}
\label{subsec:general_option}
\begin{verbatim}
option, flag { = true | = false };
option, flag | -flag;
\end{verbatim} 
sets an option as given in "flag"; the part in curly brackets is
optional: if only the name of the option is given, then the option will
be set true (see second line); a "-" sign preceding the name sets it to
"false".  

Example: 
\begin{verbatim}
option, echo = true;
option, echo;
\end{verbatim} 
are identical, ditto 
\begin{verbatim}
option, echo = false;
option, -echo;
\end{verbatim} 

The available options are: 
\begin{verbatim}
  name           default meaning if true
  ====           ======= ===============
  echo            true   echoes the input on the standard output file
  warn            true   issues warnings
  info            true   issues informations
  debug           false  issues debugging information
  trace           false  prints the system time after each command
  verify          false  issues a warning if an undefined variable is used
  tell            false  prints the current value of all options
  reset           false  resets all options to their defaults
  no_fatal_stop   false  Prevents madx from stopping in case of a fatal error. 
                         Use at your own risk.

  rbarc           true   converts the RBEND straight length into the arc 
                         length
  thin_foc        true   if false suppresses the 1(rho**2) focusing of thin 
                         dipoles
  bborbit         false  the closed orbit is modified by beam-beam kicks
  sympl           false  all element matrices are symplectified in Twiss
  twiss_print     true   controls whether the twiss command produces output.
\end{verbatim} 

The option "rbarc" is implemented for backwards compatibility with MAD-8
up to version 8.23.06 included; in this version, the RBEND length was
just taken as the arc length of an SBEND with inclined pole faces,
contrary to the MAD-8 manual.  


\subsection{PRINT}
\begin{verbatim}
print, text = "...";
\end{verbatim} 
prints the text to the current output file (see ASSIGN above). The text
can be edited with the help of a  \href{special.html#macro}{macro
  statement}. For more details, see there.  


\subsection{QUIT}
\begin{verbatim}
quit;
\end{verbatim} 
ends the program execution. 


\subsection{READTABLE}
\begin{verbatim}
readtable, file = "file_name";
\end{verbatim} 
reads a TFS file containing a MAD-X table back into memory. This table
can then be manipulated as any other table, i.e. its values can be
accessed, it can be plotted, written out again etc.  


\subsection{READMYTABLE}
\label{subsec:general_readmy}
\begin{verbatim}
readmytable, file = "file_name", table = internalname;
\end{verbatim} 
reads a TFS file containing a MAD-X table back into memory. This table
can then be manipulated as any other table, i.e. its values can be
accessed, it can be plotted, written out again etc. 

An internal name for
the table can be freely assigned while for the command READTABLE it is
taken from the information section of the table itself. This feature
allows to store multiple tables of the same type in memory without
overwriting existing ones.   


\subsection{REMOVEFILE}
\begin{verbatim}
removefile, file = "file_name";
\end{verbatim} 
remove the file "file\_name" from disk. It is more portable than  
\begin{verbatim}
system("rm filename"); // Unix specific
\end{verbatim}


\subsection{RENAMEFILE}
\begin{verbatim}
renamefile, file = "file_name", name = "new_file_name";
\end{verbatim} 
rename the file "file\_name" to "new\_file\_name" on the disk. It is more
portable than  
\begin{verbatim}
system("mv file_name new_file_name"); // Unix specific
\end{verbatim}

%% 2013-Jul-11  17:24:21  ghislain: I propose to move RESBEAM to the
%% beam declaration part of the manual
\subsection{RESBEAM}
\begin{verbatim}
resbeam, sequence = sequence_name;
\end{verbatim} 
resets the default values of the beam belonging to sequence sequence\_name, or
of the default beam if no sequence is given.  


\subsection{RETURN}
\begin{verbatim}
return;
\end{verbatim} 
ends reading from a "called" file; if encountered in the standard input
file, it ends the program execution.  


\subsection{SAVE}
\label{subsec:general_save}
\begin{verbatim}
save, sequence = sequ1, sequ2, ..., file = "file_name", beam, bare;
\end{verbatim} 
saves the sequence(s) specified with all variables and elements needed
for their expansion, onto the file "file\_name". 

{\bf Warning:} If quotes are used for
the "file\_name", capital and low characters are kept as specified, if they
are omitted the "filename" will have lower characters only. 

Example:
\begin{verbatim}
save, sequence = lhc, file = "Test_One";
\end{verbatim}
saves the lhc sequence to a file name Test\_One on disk, while
\begin{verbatim}
save, sequence = lhc, file = Test_One;
\end{verbatim}
saves the lhc sequence to a file name test\_one on disk.

The optional
flag can have the value "mad8" (without the quotes), in which case the
sequence(s) is/are saved in MAD-8 input format.  

The flag "beam" is optional; when given, all beams belonging to the
sequences specified are saved at the top of the save file.  

The parameter "sequence" is optional; when omitted, all sequences are
saved.  

However, it is not advisable to use "save" without the "sequence" option
unless you know what you are doing. This practice will avoid spurious
saved entries.    Any number of "select,flag=save" commands may precede
the SAVE command. In that case, the names of elements, variables, and
sequences must match the pattern(s) if given, and in addition the
elements must be of the class(es) specified. See here for a
\href{../Introduction/select.html#save_select}{SAVE with SELECT}
example.  

It is important to note that the precision of the output of the save
command depends on the output precision. Details about default
precisions and how to adjust those precisions can be found at the
\href{../Introduction/set.html#Format}{SET Format} instruction page.   
 
The attribute 'bare' allows to save just the sequence without the
element definitions nor beam information. This allows to re-read in a
sequence with might otherwise create a stop of the program. This is
particularly useful to turn a line into a sequence to seqedit
it. 

Example:  
\begin{verbatim}
tl3:line=(ldl6,qtl301,mqn,qtl301,ldl7,qtl302,mqn,qtl302,ldl8,ison);
DLTL3 : LINE=(delay, tl3);
use, period=dltl3;

save,sequence=dltl3,file=t1,bare; // new parameter "bare": only sequ. saved
call,file=t1; // sequence is read in and is now a "real" sequence
// if the two preceding lines are suppressed, seqedit will print a warning
// and else do nothing
use, period=dltl3;
twiss, save, betx=bxa, alfx=alfxa, bety=bya, alfy=alfya;
plot, vaxis=betx, bety, haxis=s, colour:=100;
SEQEDIT, SEQUENCE=dltl3;
  remove,element=cx.bhe0330;
  remove,element=cd.bhe0330;
ENDEDIT;

use, period=dltl3;
twiss, save, betx=bxa, alfx=alfxa, bety=bya, alfy=alfya;
\end{verbatim}


\subsection{SAVEBETA}
\label{subsec:general_savebeta}
\begin{verbatim}
savebeta, label = label, place = place, sequence = sequence_name;
\end{verbatim} 
marks a place named "place" in an expanded sequence "sequence\_name"; 
at the next TWISS command execution, a
\href{../twiss/twiss.html#beta0}{beta0} 
block will be saved at that place with the label "label". This is done
only once; in order to get a new beta0 block there, one has to re-issue
the command. The contents of the beta0 block can then be used in other
commands, e.g. TWISS and MATCH.  

Example (after sequence expansion): 
\begin{verbatim}
savebeta, label = sb1, place = mb[5], sequence = fivecell;
twiss;
show, sb1;
\end{verbatim} 
will save and show the beta0 block parameters at the end (!) of the
fifth element of type mb in the sequence.  


\subsection{SELECT} %select</a}{SELECT}
\begin{verbatim}
select, flag = flag, range = range, class = class, pattern = pattern,
        slice = integer, column =s1, s2, s3,..,sn, sequence=sequence_name,
        full, clear;
\end{verbatim} 
selects one or several elements for special treatment in a subsequent
command. All selections for a given command remain valid until "clear"
is specified; the selection criteria on the same command are logically
ANDed; the selection criteria on different SELECT statements logically
ORed.   

 Example: 
\begin{verbatim}
select, flag = error, class = quadrupole, range = mb[1]/mb[5];
select, flag = error, pattern = "^mqw.*";
\end{verbatim} 
selects all quadrupoles in the range mb[1] to mb[5], and all elements
(in the whole sequence) the name of which starts with "mqw", for
treatment by the error module.  

"flag" can be one of the following: 
\begin{itemize}
   \item seqedit: selection of elements for the
     \href{seqedit.html}{seqedit} module.  
   \item error: selection of elements for the
     \href{../error/error.html}{error} assignment module.  
   \item makethin: selection of elements for the
     \href{../makethin/makethin.html}{makethin} module that
     converts the sequence into one with thin elements only.  
   \item sectormap: selection of elements for the
     \href{../Introduction/sectormap.html}{sectormap} output file
     from the Twiss module.  
   \item table: here "table" is a table name such as twiss, track
     etc., and the rows and columns to be written are selected.  
\end{itemize} 

For the RANGE, CLASS, PATTERN, FULL, and CLEAR parameters
see \href{../Introduction/select.html}{SELECT}.  

"slice" is only used by \href{../makethin/makethin.html}{makethin} and
prescribes the number of slices into which the selected elements have to
be cut (default = 1).  

"column" is only valid for tables and decides the selection of columns
to be written into the TFS file. The "name" argument is special in that
it refers to the actual name of the selected element. For an example,
see \href{../Introduction/select.html}{SELECT}.  


\subsection{SHOW}
\begin{verbatim}
show, command;
\end{verbatim} 
prints the "command" (typically "beam", "beam\%sequ", or an element
name), with the actual value of all its parameters.  


\subsection{STOP}
\begin{verbatim}
stop;
\end{verbatim} 
ends the program execution. 


\subsection{SYSTEM}
\begin{verbatim}
system, "string";
\end{verbatim} 
transfers the string in quotes to the system for execution.  

Example: 
\begin{verbatim}
system,"ln -s /afs/cern.ch/user/u/user/public/some/directory short";
\end{verbatim}


\subsection{TABSTRING}
Note: this is not a command and should appear in the variables section
\begin{verbatim}
tabstring(arg1,arg2,arg3)
\end{verbatim}  
The "string function" tabstring(arg1,arg2,arg3) with exactly  three
arguments; arg1 is a table name (string), arg2 is a column name
(string), arg3 is a row number (integer), count starts at 0. The
function can be used in any context where a string appears; in case
there is no match, it returns \_void\_.  


\subsection{TITLE}
\begin{verbatim}
title, "title";
\end{verbatim} 
inserts the string in quotes as title in various tables and plots.  


\subsection{USE}
\label{subsec:general_use}
\begin{verbatim}
use, period = sequence_name, range = range, survey;
\end{verbatim} 
expands the sequence with name "sequence\_name", or a part of it as specified
in the \href{../Introduction/ranges.html#range}{range}. The
\texttt{survey} option plugs the survey data into the sequence elements
nodes on the first pass (see \href{../survey/survey.html}{survey}).  


\subsection{VALUE}
\begin{verbatim}
value, exp1, exp2,...;
\end{verbatim} 
prints the actual values of the expressions given. 

Example: 
\begin{verbatim}
a = clight/1000.;
value, a, pmass, exp(sqrt(2));
\end{verbatim} 
results in 
\begin{verbatim}
a = 299792.458         ;
pmass = 0.938271998        ;
exp(sqrt(2)) = 4.113250379        ;
\end{verbatim}


\subsection{WRITE}
\label{subsec:general_write}
\begin{verbatim}
write, table = table, file = "file_name";
\end{verbatim} 
writes the table "table" onto the file "file\_name"; only the rows and
columns of a preceding \verb+select, flag = table,...;+ are written. If no select
has been issued for this table, the file will only contain the
header. If the FILE argument is omitted, the table is written to
standard output.  


%\href{http://www.cern.ch/Hans.Grote/hansg_sign.html}{hansg}, June 17, 2002 

%%%%\title{Range Selection}
%  Changed by: Chris ISELIN, 27-Jan-1997 
%  Changed by: Hans Grote, 30-Sep-2002 

\subsection{Program Flow Statements}

\begin{itemize}
	\item IF
\begin{verbatim}

if (logical_expression) {statement 1; statement 2; ...; statement n; }
\end{verbatim}
\href{logical}{ where "logical\_expression" } is one of 
\begin{verbatim}

expr1 oper expr2
expr11 oper1 expr12 && expr21 oper2 expr22
expr11 oper1 expr12 || expr21 oper2 expr22
\end{verbatim} 
and oper one of 
\begin{verbatim}

==          ! equal
<>          ! not equal
<           ! less than
>           ! greater than
<=          ! less than or equal
>=          ! greater than or equal
\end{verbatim} 
The expressions are arithmetic expressions of type real. The statements
in the curly brackets are executed if the logical expression is true.  


	\item ELSEIF%elseif}{ELSEIF}
\begin{verbatim}

elseif (logical_expression) {statement 1; statement 2; ...; statement n; }
\end{verbatim} 
Only possible (in any number) behind an IF, or another ELSEIF; is
executed if  logical\_expression is true, and if none of the preceding
IF or ELSEIF logical conditions was true.  


	\item ELSE%else}{ELSE}
\begin{verbatim}

else {statement 1; statement 2; ...; statement n; }
\end{verbatim} 
Only possible (once) behind an IF, or an ELSEIF; is executed if
logical\_expression is true, and if none of the preceding IF or ELSEIF
logical conditions was true.  

For a real life example, see \href{foot.html}{ELSE example}. 


	\item WHILE
\begin{verbatim}

while (logical_condition){statement 1; statement 2; ...; statement n; }
\end{verbatim}  
executes the statements in curly brackets while the logical\_expression
is true. A simple example (in case you have forgotten the first ten
factorials) would be  
\begin{verbatim}

option,-info;   ! otherwise you get redifiniton warnings
n=1; m=1;
while (n <= 10)
{
  m = m * n;  value, m;
  n = n + 1;
};
\end{verbatim}

For a real life example, see \href{foot.html}{WHILE example}.

	\item MACRO

\begin{verbatim}

label: macro = {statement 1; statement 2; ...; statement n; };
label(arg1,...,argn): macro = {statement 1; statement 2; ...; statement n; };
\end{verbatim} 
The first form allows the execution of a group of statements via a
single command:  
\begin{verbatim}

exec, label;
\end{verbatim} 
will execute the statements in curly brackets exactly once. This command
can be issued any number of times.  

The second form allows to replace strings anywhere inside the statements
in curly brackets by other strings, or integer numbers prior to
execution. This is a powerful construct and should be handled with care.  

Simple example: 
\begin{verbatim}

option,-echo,-info;  ! otherwise the output is somewhat confusing
simple(xx,yy): macro = { xx = yy^2 + xx; value, xx;};
a = 3;
b = 5;
exec, simple(a,b);
\end{verbatim}

Somewhat more tricky (a "\$" in front of an argument means that the
truncated integer value of this argument is used for replacement, rather
than the argument string itself).  
\begin{verbatim}

tricky(xx,yy,zz): macro = {mzz.yy: xx, l = 1.yy, kzz = k.yy;};
n=0;
while (n < 3)
{
  n = n+1;
  exec,tricky(quadrupole,$n,1);
  exec,tricky(sextupole,$n,2);
};
\end{verbatim} 
Whereas the actual use of the preceding example is NOT recommended,
a real life example, showing the full power (!) of macros is to be
found under \href{foot.html}{macro usage} for the usage, and
under \href{foot.html#macro}{macro definition} for the
definition.


Beware of the following rules:

	\item Generally speaking: \textit{ special constructs } like IF, WHILE,
MACRO will only allow one level of inclusion of another \textit{
special construct }.

	\item  Macros must not be called with numbers, but with strings
(i.e. variable names in case of numerical values), i.e. 



NOT

\begin{verbatim}

exec,thismacro($99,$129);
\end{verbatim}
BUT

\begin{verbatim}

n1=99; n2=219;
exec,thismacro($n1,$n2);
\end{verbatim}

\end{itemize}

%\href{http://www.cern.ch/Hans.Grote/hansg_sign.html}{hansg}, June 17, 2002




\section{Control Statements}

MAD-X consists of a core program, and modules for specific tasks such as
twiss parameter calculation, matching, thin lens tracking, and so on.  
 
The statements listed here are those executed by the program core. They
deal with the I/O, element and sequence declaration, sequence
manipulation, statement flow control (e.g. IF, WHILE), MACRO
declaration, saving sequences onto files in MAD-X or MAD-8 format, and
so on.  


%% \subsection{Program flow control}
%% \begin{itemize}
%% 	\item \href{special.html#if}{IF}
%% 	\item \href{special.html#elseif}{ELSEIF}
%% 	\item \href{special.html#else}{ELSE}
%% 	\item \href{special.html#while}{WHILE}
%% 	\item \href{special.html#macro}{MACRO}
%% \end{itemize}


%% \subsection{General control}
%% \begin{itemize}
%% 	\item \href{general.html#assign}{ASSIGN}
%% 	\item \href{general.html#call}{CALL}
%% 	\item \href{general.html#coguess}{COGUESS}
%% 	\item \href{general.html#create}{CREATE}
%% 	\item \href{general.html#dumpsequ}{DUMPSEQU}
%% 	\item \href{general.html#exec}{EXEC}
%% 	\item \href{general.html#exit}{EXIT}
%% 	\item \href{general.html#fill}{FILL}
%% 	\item \href{general.html#help}{HELP}
%% 	\item \href{general.html#option}{OPTION}
%% 	\item \href{general.html#print}{PRINT}
%% 	\item \href{general.html#quit}{QUIT}
%% 	\item \href{general.html#readtable}{READTABLE}
%% 	\item \href{general.html#removefile}{REMOVEFILE}
%% 	\item \href{general.html#renamefile}{RENAMEFILE}
%% 	\item \href{general.html#return}{RETURN}
%% 	\item \href{general.html#save}{SAVE}
%% 	\item \href{general.html#savebeta}{SAVEBETA}
%% 	\item \href{../Introduction/select.html}{SELECT}
%% 	\item \href{../Introduction/set.html}{SET}
%% 	\item \href{general.html#show}{SHOW}
%% 	\item \href{general.html#stop}{STOP}
%% 	\item \href{general.html#system}{SYSTEM}
%% 	\item \href{general.html#tabstring}{TABSTRING}
%% 	\item \href{general.html#title}{TITLE}
%% 	\item \href{general.html#use}{USE}
%% 	\item \href{general.html#value}{VALUE}
%% 	\item \href{general.html#write}{WRITE}
%% \end{itemize}



%% \subsection{Beam specification}
%% \begin{itemize}
%% 	\item \href{../Introduction/beam.html}{BEAM}
%% 	\item \href{../Introduction/resbeam.html}{RESBEAM}
%% \end{itemize}



%% \subsection{PLOT}
%% \begin{itemize}
%% 	\item \href{../plot/plot.html}{PLOT}
%% 	\item \href{../plot/plot.html#resplot}{RESPLOT}
%% 	\item \href{../plot/plot.html#setplot}{SETPLOT}
%% \end{itemize}


%% \subsection{\href{seqedit.html}{Sequence editing}}
%% \begin{itemize}
%% 	\item \href{seqedit.html#seqedit}{SEQEDIT}
%% 	\item \href{seqedit.html#flatten}{FLATTEN}
%% 	\item \href{seqedit.html#install}{INSTALL}
%% 	\item \href{seqedit.html#move}{MOVE}
%% 	\item \href{seqedit.html#remove}{REMOVE}
%% 	\item \href{seqedit.html#cycle}{CYCLE}
%% 	\item \href{seqedit.html#reflect}{REFLECT}
%% 	\item \href{seqedit.html#endedit}{ENDEDIT}
%% \end{itemize}

%% \href{http://www.cern.ch/Hans.Grote/hansg_sign.html}{hansg}, June 17, 2002 

%%%\title{Range Selection}
%  Changed by: Chris ISELIN, 27-Jan-1997 
%  Changed by: Hans Grote, 30-Sep-2002 

\subsection{Program Flow Statements}

\begin{itemize}
	\item IF
\begin{verbatim}

if (logical_expression) {statement 1; statement 2; ...; statement n; }
\end{verbatim}
\href{logical}{ where "logical\_expression" } is one of 
\begin{verbatim}

expr1 oper expr2
expr11 oper1 expr12 && expr21 oper2 expr22
expr11 oper1 expr12 || expr21 oper2 expr22
\end{verbatim} 
and oper one of 
\begin{verbatim}

==          ! equal
<>          ! not equal
<           ! less than
>           ! greater than
<=          ! less than or equal
>=          ! greater than or equal
\end{verbatim} 
The expressions are arithmetic expressions of type real. The statements
in the curly brackets are executed if the logical expression is true.  


	\item ELSEIF%elseif}{ELSEIF}
\begin{verbatim}

elseif (logical_expression) {statement 1; statement 2; ...; statement n; }
\end{verbatim} 
Only possible (in any number) behind an IF, or another ELSEIF; is
executed if  logical\_expression is true, and if none of the preceding
IF or ELSEIF logical conditions was true.  


	\item ELSE%else}{ELSE}
\begin{verbatim}

else {statement 1; statement 2; ...; statement n; }
\end{verbatim} 
Only possible (once) behind an IF, or an ELSEIF; is executed if
logical\_expression is true, and if none of the preceding IF or ELSEIF
logical conditions was true.  

For a real life example, see \href{foot.html}{ELSE example}. 


	\item WHILE
\begin{verbatim}

while (logical_condition){statement 1; statement 2; ...; statement n; }
\end{verbatim}  
executes the statements in curly brackets while the logical\_expression
is true. A simple example (in case you have forgotten the first ten
factorials) would be  
\begin{verbatim}

option,-info;   ! otherwise you get redifiniton warnings
n=1; m=1;
while (n <= 10)
{
  m = m * n;  value, m;
  n = n + 1;
};
\end{verbatim}

For a real life example, see \href{foot.html}{WHILE example}.

	\item MACRO

\begin{verbatim}

label: macro = {statement 1; statement 2; ...; statement n; };
label(arg1,...,argn): macro = {statement 1; statement 2; ...; statement n; };
\end{verbatim} 
The first form allows the execution of a group of statements via a
single command:  
\begin{verbatim}

exec, label;
\end{verbatim} 
will execute the statements in curly brackets exactly once. This command
can be issued any number of times.  

The second form allows to replace strings anywhere inside the statements
in curly brackets by other strings, or integer numbers prior to
execution. This is a powerful construct and should be handled with care.  

Simple example: 
\begin{verbatim}

option,-echo,-info;  ! otherwise the output is somewhat confusing
simple(xx,yy): macro = { xx = yy^2 + xx; value, xx;};
a = 3;
b = 5;
exec, simple(a,b);
\end{verbatim}

Somewhat more tricky (a "\$" in front of an argument means that the
truncated integer value of this argument is used for replacement, rather
than the argument string itself).  
\begin{verbatim}

tricky(xx,yy,zz): macro = {mzz.yy: xx, l = 1.yy, kzz = k.yy;};
n=0;
while (n < 3)
{
  n = n+1;
  exec,tricky(quadrupole,$n,1);
  exec,tricky(sextupole,$n,2);
};
\end{verbatim} 
Whereas the actual use of the preceding example is NOT recommended,
a real life example, showing the full power (!) of macros is to be
found under \href{foot.html}{macro usage} for the usage, and
under \href{foot.html#macro}{macro definition} for the
definition.


Beware of the following rules:

	\item Generally speaking: \textit{ special constructs } like IF, WHILE,
MACRO will only allow one level of inclusion of another \textit{
special construct }.

	\item  Macros must not be called with numbers, but with strings
(i.e. variable names in case of numerical values), i.e. 



NOT

\begin{verbatim}

exec,thismacro($99,$129);
\end{verbatim}
BUT

\begin{verbatim}

n1=99; n2=219;
exec,thismacro($n1,$n2);
\end{verbatim}

\end{itemize}

%\href{http://www.cern.ch/Hans.Grote/hansg_sign.html}{hansg}, June 17, 2002


%%%\title{Range Selection}
%  Changed by: Chris ISELIN, 27-Jan-1997 
%  Changed by: Hans Grote, 16-Jan-2003 

\section{General Control Statements}

\subsection{ASSIGN}
\begin{verbatim}
assign, echo="file_name", truncate;
\end{verbatim} 
where "file\_name" is the name of an output file, or "terminal" and
truncate specifies if the file must be truncated when opened (ignored
for terminal). This allows switching the echo stream to a file or back,
but only for the commands value, show, and print. Allows easy
composition of future MAD-X input files. A real life example (always the
same) is to be found under \href{foot.html}{footprint example}.  

\subsection{CALL}
\begin{verbatim}
call, file = "file_name";
\end{verbatim} 
where "file\_name"  is the name of an input file. This file will be read
until a "return;" statement, or until end\_of\_file; it may contain any
number of calls itself, and so on to any depth.  

%% 2013-Jul-11  17:23:00  ghislain: I propose to move COGUESS to the
%% orbit correction part of the manual
\subsection{COGUESS}
\label{subsec:general_coguess}
\begin{verbatim}
coguess, tolerance = double, 
         x = double, px = double, 
         y = double, py = double, 
         t = double, pt = double;
\end{verbatim} 
sets the required convergence precision in the closed orbit search
("tolerance", see as well Twiss command
\href{../twiss/twiss.html#tolerance}{tolerance}).  

The other parameters define a first guess for all future closed orbit
searches in case they are different from zero.  

\subsection{CREATE}
\begin{verbatim}
create, table = table, column = var1, var2,_name,...;
\end{verbatim} 
creates a table with the specified variables as columns. This table can
then be \hyperlink{fill}{fill}ed, and finally one can
\hyperlink{write}{write} it in TFS format. The attribute "\_name" adds
the element name to the table at the specified column, this replaces the
undocumented "withname" attribute that was not always working properly.  

See the \href{../Introduction/select.html#ucreate}{user table I}
example; 
or an example of joining 2 tables of different length in one table
including the element name:
\href{../Introduction/select.html#screate}{user table II} 

\subsection{DELETE}
\begin{verbatim}
delete, sequence = s_name, table = t_name;
\end{verbatim} 
deletes a sequence with name "s\_name" or a table with name "t\_name"
from memory. The sequence deletion is done without influence on other
sequences that may have elements that were in the deleted sequence.  

%% 2013-Jul-11  17:24:21  ghislain: I propose to move DUMPSEQU to the
%% sequence edition and manipulation part of the manual
\subsection{DUMPSEQU}
\begin{verbatim}
dumpsequ, sequence = s_name, level = integer;
\end{verbatim} 
Actually a debug statement, but it may come handy at certain
occasions. Here "s\_name" is the name of an expanded (i.e. USEd)
sequence. The amount of detail is controlled by "level":  
\begin{verbatim}
level = 0:    print only the cumulative node length = sequence length
      > 0:    print all node (element) names except drifts
      > 2:    print all nodes with their attached parameters
      > 3:    print all nodes, and their elements with all parameters
\end{verbatim}


\subsection{EXEC}
\begin{verbatim}
exec, label;
\end{verbatim} 
Each statement may be preceded by a label; it is then stored and can be
executed again with "exec, label;" any number of times; the executed
statement may be another "exec", etc.; however, the major usage of this
statement is the execution of a \href{special.html#macro}{macro}.  

\subsection{EXIT}
\begin{verbatim}
exit;
\end{verbatim} 
ends the program execution. 

\subsection{FILL} 
Every command 
\begin{verbatim}
fill, table = table;
\end{verbatim} 
adds a new line with the current values of all column variables into the
user table \hyperlink{create}{create}d beforehand. This table one can
then \hyperlink{write}{write} in TFS format.  See as well the
\href{../Introduction/select.html#ucreate}{user table} example.  

\subsection{OPTION}
\label{subsec:general_option}
\begin{verbatim}
option, flag { = true | = false };
option, flag | -flag;
\end{verbatim} 
sets an option as given in "flag"; the part in curly brackets is
optional: if only the name of the option is given, then the option will
be set true (see second line); a "-" sign preceding the name sets it to
"false".  

Example: 
\begin{verbatim}
option, echo = true;
option, echo;
\end{verbatim} 
are identical, ditto 
\begin{verbatim}
option, echo = false;
option, -echo;
\end{verbatim} 

The available options are: 
\begin{verbatim}
  name           default meaning if true
  ====           ======= ===============
  echo            true   echoes the input on the standard output file
  warn            true   issues warnings
  info            true   issues informations
  debug           false  issues debugging information
  trace           false  prints the system time after each command
  verify          false  issues a warning if an undefined variable is used
  tell            false  prints the current value of all options
  reset           false  resets all options to their defaults
  no_fatal_stop   false  Prevents madx from stopping in case of a fatal error. 
                         Use at your own risk.

  rbarc           true   converts the RBEND straight length into the arc 
                         length
  thin_foc        true   if false suppresses the 1(rho**2) focusing of thin 
                         dipoles
  bborbit         false  the closed orbit is modified by beam-beam kicks
  sympl           false  all element matrices are symplectified in Twiss
  twiss_print     true   controls whether the twiss command produces output.
\end{verbatim} 

The option "rbarc" is implemented for backwards compatibility with MAD-8
up to version 8.23.06 included; in this version, the RBEND length was
just taken as the arc length of an SBEND with inclined pole faces,
contrary to the MAD-8 manual.  


\subsection{PRINT}
\begin{verbatim}
print, text = "...";
\end{verbatim} 
prints the text to the current output file (see ASSIGN above). The text
can be edited with the help of a  \href{special.html#macro}{macro
  statement}. For more details, see there.  


\subsection{QUIT}
\begin{verbatim}
quit;
\end{verbatim} 
ends the program execution. 


\subsection{READTABLE}
\begin{verbatim}
readtable, file = "file_name";
\end{verbatim} 
reads a TFS file containing a MAD-X table back into memory. This table
can then be manipulated as any other table, i.e. its values can be
accessed, it can be plotted, written out again etc.  


\subsection{READMYTABLE}
\label{subsec:general_readmy}
\begin{verbatim}
readmytable, file = "file_name", table = internalname;
\end{verbatim} 
reads a TFS file containing a MAD-X table back into memory. This table
can then be manipulated as any other table, i.e. its values can be
accessed, it can be plotted, written out again etc. 

An internal name for
the table can be freely assigned while for the command READTABLE it is
taken from the information section of the table itself. This feature
allows to store multiple tables of the same type in memory without
overwriting existing ones.   


\subsection{REMOVEFILE}
\begin{verbatim}
removefile, file = "file_name";
\end{verbatim} 
remove the file "file\_name" from disk. It is more portable than  
\begin{verbatim}
system("rm filename"); // Unix specific
\end{verbatim}


\subsection{RENAMEFILE}
\begin{verbatim}
renamefile, file = "file_name", name = "new_file_name";
\end{verbatim} 
rename the file "file\_name" to "new\_file\_name" on the disk. It is more
portable than  
\begin{verbatim}
system("mv file_name new_file_name"); // Unix specific
\end{verbatim}

%% 2013-Jul-11  17:24:21  ghislain: I propose to move RESBEAM to the
%% beam declaration part of the manual
\subsection{RESBEAM}
\begin{verbatim}
resbeam, sequence = sequence_name;
\end{verbatim} 
resets the default values of the beam belonging to sequence sequence\_name, or
of the default beam if no sequence is given.  


\subsection{RETURN}
\begin{verbatim}
return;
\end{verbatim} 
ends reading from a "called" file; if encountered in the standard input
file, it ends the program execution.  


\subsection{SAVE}
\label{subsec:general_save}
\begin{verbatim}
save, sequence = sequ1, sequ2, ..., file = "file_name", beam, bare;
\end{verbatim} 
saves the sequence(s) specified with all variables and elements needed
for their expansion, onto the file "file\_name". 

{\bf Warning:} If quotes are used for
the "file\_name", capital and low characters are kept as specified, if they
are omitted the "filename" will have lower characters only. 

Example:
\begin{verbatim}
save, sequence = lhc, file = "Test_One";
\end{verbatim}
saves the lhc sequence to a file name Test\_One on disk, while
\begin{verbatim}
save, sequence = lhc, file = Test_One;
\end{verbatim}
saves the lhc sequence to a file name test\_one on disk.

The optional
flag can have the value "mad8" (without the quotes), in which case the
sequence(s) is/are saved in MAD-8 input format.  

The flag "beam" is optional; when given, all beams belonging to the
sequences specified are saved at the top of the save file.  

The parameter "sequence" is optional; when omitted, all sequences are
saved.  

However, it is not advisable to use "save" without the "sequence" option
unless you know what you are doing. This practice will avoid spurious
saved entries.    Any number of "select,flag=save" commands may precede
the SAVE command. In that case, the names of elements, variables, and
sequences must match the pattern(s) if given, and in addition the
elements must be of the class(es) specified. See here for a
\href{../Introduction/select.html#save_select}{SAVE with SELECT}
example.  

It is important to note that the precision of the output of the save
command depends on the output precision. Details about default
precisions and how to adjust those precisions can be found at the
\href{../Introduction/set.html#Format}{SET Format} instruction page.   
 
The attribute 'bare' allows to save just the sequence without the
element definitions nor beam information. This allows to re-read in a
sequence with might otherwise create a stop of the program. This is
particularly useful to turn a line into a sequence to seqedit
it. 

Example:  
\begin{verbatim}
tl3:line=(ldl6,qtl301,mqn,qtl301,ldl7,qtl302,mqn,qtl302,ldl8,ison);
DLTL3 : LINE=(delay, tl3);
use, period=dltl3;

save,sequence=dltl3,file=t1,bare; // new parameter "bare": only sequ. saved
call,file=t1; // sequence is read in and is now a "real" sequence
// if the two preceding lines are suppressed, seqedit will print a warning
// and else do nothing
use, period=dltl3;
twiss, save, betx=bxa, alfx=alfxa, bety=bya, alfy=alfya;
plot, vaxis=betx, bety, haxis=s, colour:=100;
SEQEDIT, SEQUENCE=dltl3;
  remove,element=cx.bhe0330;
  remove,element=cd.bhe0330;
ENDEDIT;

use, period=dltl3;
twiss, save, betx=bxa, alfx=alfxa, bety=bya, alfy=alfya;
\end{verbatim}


\subsection{SAVEBETA}
\label{subsec:general_savebeta}
\begin{verbatim}
savebeta, label = label, place = place, sequence = sequence_name;
\end{verbatim} 
marks a place named "place" in an expanded sequence "sequence\_name"; 
at the next TWISS command execution, a
\href{../twiss/twiss.html#beta0}{beta0} 
block will be saved at that place with the label "label". This is done
only once; in order to get a new beta0 block there, one has to re-issue
the command. The contents of the beta0 block can then be used in other
commands, e.g. TWISS and MATCH.  

Example (after sequence expansion): 
\begin{verbatim}
savebeta, label = sb1, place = mb[5], sequence = fivecell;
twiss;
show, sb1;
\end{verbatim} 
will save and show the beta0 block parameters at the end (!) of the
fifth element of type mb in the sequence.  


\subsection{SELECT} %select</a}{SELECT}
\begin{verbatim}
select, flag = flag, range = range, class = class, pattern = pattern,
        slice = integer, column =s1, s2, s3,..,sn, sequence=sequence_name,
        full, clear;
\end{verbatim} 
selects one or several elements for special treatment in a subsequent
command. All selections for a given command remain valid until "clear"
is specified; the selection criteria on the same command are logically
ANDed; the selection criteria on different SELECT statements logically
ORed.   

 Example: 
\begin{verbatim}
select, flag = error, class = quadrupole, range = mb[1]/mb[5];
select, flag = error, pattern = "^mqw.*";
\end{verbatim} 
selects all quadrupoles in the range mb[1] to mb[5], and all elements
(in the whole sequence) the name of which starts with "mqw", for
treatment by the error module.  

"flag" can be one of the following: 
\begin{itemize}
   \item seqedit: selection of elements for the
     \href{seqedit.html}{seqedit} module.  
   \item error: selection of elements for the
     \href{../error/error.html}{error} assignment module.  
   \item makethin: selection of elements for the
     \href{../makethin/makethin.html}{makethin} module that
     converts the sequence into one with thin elements only.  
   \item sectormap: selection of elements for the
     \href{../Introduction/sectormap.html}{sectormap} output file
     from the Twiss module.  
   \item table: here "table" is a table name such as twiss, track
     etc., and the rows and columns to be written are selected.  
\end{itemize} 

For the RANGE, CLASS, PATTERN, FULL, and CLEAR parameters
see \href{../Introduction/select.html}{SELECT}.  

"slice" is only used by \href{../makethin/makethin.html}{makethin} and
prescribes the number of slices into which the selected elements have to
be cut (default = 1).  

"column" is only valid for tables and decides the selection of columns
to be written into the TFS file. The "name" argument is special in that
it refers to the actual name of the selected element. For an example,
see \href{../Introduction/select.html}{SELECT}.  


\subsection{SHOW}
\begin{verbatim}
show, command;
\end{verbatim} 
prints the "command" (typically "beam", "beam\%sequ", or an element
name), with the actual value of all its parameters.  


\subsection{STOP}
\begin{verbatim}
stop;
\end{verbatim} 
ends the program execution. 


\subsection{SYSTEM}
\begin{verbatim}
system, "string";
\end{verbatim} 
transfers the string in quotes to the system for execution.  

Example: 
\begin{verbatim}
system,"ln -s /afs/cern.ch/user/u/user/public/some/directory short";
\end{verbatim}


\subsection{TABSTRING}
Note: this is not a command and should appear in the variables section
\begin{verbatim}
tabstring(arg1,arg2,arg3)
\end{verbatim}  
The "string function" tabstring(arg1,arg2,arg3) with exactly  three
arguments; arg1 is a table name (string), arg2 is a column name
(string), arg3 is a row number (integer), count starts at 0. The
function can be used in any context where a string appears; in case
there is no match, it returns \_void\_.  


\subsection{TITLE}
\begin{verbatim}
title, "title";
\end{verbatim} 
inserts the string in quotes as title in various tables and plots.  


\subsection{USE}
\label{subsec:general_use}
\begin{verbatim}
use, period = sequence_name, range = range, survey;
\end{verbatim} 
expands the sequence with name "sequence\_name", or a part of it as specified
in the \href{../Introduction/ranges.html#range}{range}. The
\texttt{survey} option plugs the survey data into the sequence elements
nodes on the first pass (see \href{../survey/survey.html}{survey}).  


\subsection{VALUE}
\begin{verbatim}
value, exp1, exp2,...;
\end{verbatim} 
prints the actual values of the expressions given. 

Example: 
\begin{verbatim}
a = clight/1000.;
value, a, pmass, exp(sqrt(2));
\end{verbatim} 
results in 
\begin{verbatim}
a = 299792.458         ;
pmass = 0.938271998        ;
exp(sqrt(2)) = 4.113250379        ;
\end{verbatim}


\subsection{WRITE}
\label{subsec:general_write}
\begin{verbatim}
write, table = table, file = "file_name";
\end{verbatim} 
writes the table "table" onto the file "file\_name"; only the rows and
columns of a preceding \verb+select, flag = table,...;+ are written. If no select
has been issued for this table, the file will only contain the
header. If the FILE argument is omitted, the table is written to
standard output.  


%\href{http://www.cern.ch/Hans.Grote/hansg_sign.html}{hansg}, June 17, 2002 

%%\title{BEAM}
%  Changed by: Hans Grote, 30-Sep-2002 

\section{BEAM: Set Beam Parameters}

Many commands in MAD-X require the setting of various quantities related
to the beam in the machine. Therefore, MAD-X will stop with a fatal
error if an attempt is made to expand (USE) a sequence for which no BEAM
command has been issued before.  

The quantities are entered by a BEAM command: 
\begin{verbatim}
BEAM, PARTICLE=name,MASS=real,CHARGE=real,
      ENERGY=real,PC=real,GAMMA=real,
      EX=real,EXN=real,EY=real,EYN=real,
      ET=real,SIGT=real,SIGE=real,
      KBUNCH=integer,NPART=real,BCURRENT=real,
      BUNCHED=logical,RADIATE=logical,BV=integer,SEQUENCE=name;
\end{verbatim} 

{\bf Warning:} BEAM updates, i.e. it replaces attributes explicitely
mentioned, but does not return to default values for others! To reset to
\href{resbeam.html#defaults}{beam value defaults},  use
\href{resbeam.html}{RESBEAM}.

The particle restmass and \href{charge}{charge} are defined by:
\begin{itemize}
   \item \href{particle}{PARTICLE}: The name of particles in the
     machine. MAD knows the restmass and the charge for the
     following particles:  
     \begin{itemize}
	  \item POSITRON: The particles are positrons (MASS=\textit{m$_e$}, CHARGE=1), 
	  \item ELECTRON: The particles are electrons (MASS=\textit{m$_e$}, CHARGE=-1), 
	  \item PROTON: The particles are protons (MASS=\textit{m$_p$}, CHARGE=1), 
	  \item ANTIPROTON: The particles are anti-protons (MASS=\textit{m$_p$}, CHARGE=-1). 
	  \item POSMUON: The particles are positive muons (MASS=\textit{m$_mu$}, CHARGE=1), 
	  \item NEGMUON: The particles are negative muons (MASS=\textit{m$_mu$}, CHARGE=-1). 
     \end{itemize}
\end{itemize} 

Therefore neither restmass nor charge can be modified for these
predefined particles. On the other hand, for ions and all other user
defined particles the name, restmass, and charge can be entered
independently.  

By default the total particle energy is 1 GeV. A different value can be 
defined by one of the following: 
 
\begin{itemize}
   \item \href{energy}{ENERGY}: The total energy per particle in
     GeV. If given, it must be greater then the particle restmass.  
   \item \href{pc}{PC}: The momentum per particle in GeV/c. If
     given, it must be greater than zero.  
   \item \href{gamma}{GAMMA}: The ratio between total energy and
     rest energy of the particles: GAMMA = \textit{E / m$_0$}. If
     given, it must be greater than one. If the restmass is changed
     a new value for the energy should be entered. Otherwise the
     energy remains unchanged, and the momentum PC and the quantity
     GAMMA are recalculated . 
\end{itemize}  

The emittances are defined by: 
\begin{itemize}
   \item \href{ex}{EX}: The horizontal emittance \textit{E$_x$} (default: 1 m). 
   \item \href{ey}{EY}: The vertical emittance \textit{E$_y$} (default: 1 m). 
   \item \href{et}{ET}: The longitudinal emittance \textit{E$_t$} (default: 1 m). 
\end{itemize}  

The emittances can be replaced by the normalised emittances and the
energy spread:  
\begin{itemize}
   \item \href{exn}{EXN}: The normalised horizontal emittance [m]:
     \textit{E$_xn$} = 4 (GAMMA$^2$ - 1)$^{1/2}$\textit{E$_x$} (ignored
     if \textit{E$_x$} is given).  
   \item \href{eyn}{EYN}: The normalised vertical emittance [m]:
     \textit{E$_yn$} = 4 (GAMMA$^2$ - 1)$^{1/2}$\textit{E$_y$} (ignored
     if \textit{E$_x$} is given).  
   \item \href{sigt}{SIGT}: The bunch length \textit{c}
     sigma(\textit{t}) in [m].  
   \item \href{sige}{SIGE}: The \emph{relative} energy spread
     sigma(\textit{E})/\textit{E} in [1].  
\end{itemize} 

Certain commands compute the synchrotron tune \textit{Q$_s$} from the RF
cavities. If \textit{Q$_s$} is non-zero, the relative energy spread and
the bunch length are  \\
sigma(\textit{E}) / \textit{p$_0$ c =  (2 pi Q$_s$ E$_t$ / ETA C)$^{1/2}$}, 

\textit{c} sigma(\textit{t}) = (ETA C E$_t$ / 2 pi Q$_s$)$^{1/2}$. 

\textit{C} is the machine circumference, and 

\textit{ETA} = GAMMA$^{-2}$ - GAMMA(transition)$^{-2}$. 

The order of precedence in the parameter evaluation is given below: 
\begin{verbatim}
    particle->(mass+charge)
    energy->pc->gamma->beta
    ex->exn
    ey->eyn
    current->npart
    et->sigt->sige
\end{verbatim} 

where any item to the left takes precendence over the others. 

Finally, the BEAM command accepts 
\begin{itemize}
   \item \href{kbunch}{KBUNCH}: The number of particle bunches in the
     machine (default: 1).  
   \item \href{npart}{NPART}: The number of particles per bunch (default: 0). 
   \item \href{bcurrent}{BCURRENT}: The bunch current (default: 0 A). 
   \item \href{bunched}{BUNCHED}: A logical flag. If set, the beam is
     treated as bunched whenever this makes sense.  
   \item \href{radiate}{RADIATE}: A logical flag. If set, synchrotron
     radiation is considered in all bipolar magnets.  
   \item \href{bv}{BV}: an integer specifying the direction of the
     particle movement in a beam line; either +1 (default), or -1. For a
     detailed explanation see under \href{bv_flag.html}{bv flag}.  
   \item \href{sequence}{SEQUENCE}: this attaches the beam command to a
     specific sequence; if the name is omitted, the BEAM command refers
     to the default beam always present. Sequences without attached beam
     use this default beam. When updating a beam, the corresponding
     sequence name, if any, must always be mentioned.  
\end{itemize} 

The BEAM command changes only the parameters entered. The command
\href{resbeam.html}{RESBEAM} resets all beam data to their
\href{resbeam.html#defaults}{beam value defaults}.  

Examples: 
\begin{verbatim}
BEAM, PARTICLE = ELECTRON, ENERGY = 50, EX = 1.E-6, EY = 1.E-8, SIGE = 1.E-3;
 ...
BEAM, RADIATE;
 ...
RESBEAM;
BEAM, EX = 2.E-5, EY = 3.E-7, SIGE = 4.E-3;
\end{verbatim} 

The first command selects electrons, and sets energy and emittances. The
second one turns on synchrotron radiation. The last two select positrons
(by default), set the energy to 1 GeV (default), clear the synchrotron
radiation flag, and set the emittances to the values entered.  

Some program modules of MAD-X may also store data into a beam data
block. Expressions may refer to data in this beam data block using the
notation  
\begin{verbatim}
BEAM->attribute-name
\end{verbatim} 
or 
\begin{verbatim}
BEAM%sequence->attribute-name.
\end{verbatim} 

This notation refers to the value of attribute-name found in the default
BEAM resp. the beam belonging to the sequence given. This can be used
for receiving or using values, e.g. 

\begin{verbatim}
value, beam%lhcb2->bv;
\end{verbatim} 
or for storing values in the beam (this does NOT trigger an update of dependent variables !), e.g. 
\begin{verbatim}
beam->charge=-1;
\end{verbatim} 

The current values in the BEAM bank can be obtained by the command
\begin{verbatim}
show,beam;
\end{verbatim}
resp.
\begin{verbatim}
show,beam%sequence;
\end{verbatim}


%\href{http://www.cern.ch/Hans.Grote/hansg_sign.html}{hansg} 11.9.2000 

%%\title{DRIFT}
%  Changed by: Chris ISELIN, 24-Jan-1997 
%  Changed by: Hans Grote, 10-Jun-2002 

\section{RESBEAM: reset beam defaults}
\label{sec:resbeam}

\begin{verbatim}
label: RESBEAM,SEQUENCE=name;
\end{verbatim} 

If the sequence name is omitted, the default beam is reset. 
\begin{verbatim}
Default BEAM Data:
PARTICLE        POSITRON
ENERGY          1 GeV
EX              1 rad m
EY              1 rad m
ET              1 rad m
KBUNCH          1
NPART           0
BCURRENT        0 A
BUNCHED         .TRUE.
RADIATE         .FALSE.
\end{verbatim}

%\href{http://www.cern.ch/Hans.Grote/hansg_sign.html}{hansg}, January 24, 1997 


%%\title{BV flag in the Beam command}
%  Changed by: Thys Risselada 29-Mar-2009 

\section{BV FLAG}

When reversing the direction ("V") of a particle in a magnetic field
("B") while keeping its charge constant, the resulting force V * B
changes sign. This is equivalent to flipping the field, but not the
direction.  

For practical reasons the properties of all elements of the LHC are
defined in the MADX input as if they apply to a clockwise proton beam
("LHC beam 1"). This allows a single definition for elements traversed
by both beams. Their effects on a beam with identical particle charge
but running in the opposite direction ("LHC beam 2") must then be
reversed inside the program.  

In MADX this may be taken into account by setting the value of the BV
attribute in the Beam commands. In the case of LHC beam 1 (clockwise)
and beam 2 (counter-clockwise), treated in MADX both as clockwise proton
beams, the Beam commands must look as follows: 

\begin{verbatim}
beam, sequence=lhcb1, particle=proton, pc=450, bv = +1;

beam, sequence=lhcb2, particle=proton, pc=450, bv = -1;
\end{verbatim}

%\href{http://www.cern.ch/Frank.Schmidt/frs_sign.html}{frs}, March 29, 2009 

%%%\title{PLOT}
%  Changed by: Chris ISELIN, 27-Jan-1997
% 

%  Changed by: Hans Grote, 25-Sep-2002 

%  Changed by:
% E. T. d'Amico, 20-Oct-2004 

%%\usepackage{hyperref}
% commands generated by html2latex


%%\begin{document}
%%\begin{center}
 %%EUROPEAN ORGANIZATION FOR NUCLEAR RESEARCH 
%%\includegraphics{http://cern.ch/madx/icons/mx7_25.gif}

\subsection{PLOT}
%%\end{center}

 Values contained in MAD-X tables can be plotted in the form column versus column, with up to four differently scaled vertical axes; furthermore, if the horizontal axis is the position "s" of the elements in a sequence, then the symbolic machine can be plotted above the curves as well. In certain conditions True interpolation inside the element is available (through calls to the Twiss module for each slice) .  The "environment" (interpolation, line thickness, annotation size, PostScript format) can be set with the \hyperlink{setplot}{setplot} command.  
\begin{itemize}
	\item \textbf{PLOT}
	
\begin{verbatim}
 plot, vaxis=vname1,vname2,..,vnamen,
vaxis1=vname1,vname2,..,vnamen, vaxis2=vname1,vname2,..,vnamen,
vaxis3=vname1,vname2,..,vnamen, vaxis4=vname1,vname2,..,vnamen,
haxis=vname, hmin=real, hmax=real, vmin=reals, vmax=reals, bars=integer,
style=integer, colour=integer, symbol=integer, noversion=logical,
interpolate=logical, noline=logical, notitle=logical, marker_plot=logical, 
range_plot=logical, table=table_name, particle=particle1,particle2,
..,particlen,
multiple=logical, title=string, range=range, file=file_name_start, 
ptc=logical, ptc_table=table_name, trackfile=table_name; 
\end{verbatim} where the parameters have the following meaning: 
\begin{itemize}
	\item vaxis: one or several variables from the table to be plotted against the (only) vertical axis.  
	\item vaxis1: one or several variables from the table to be plotted against the vertical axis number 1 (out of 4 possible ones). 
	\item vaxis2: one or several variables from the table to be plotted against the vertical axis number 2 (out of 4 possible ones). 
	\item vaxis3: one or several variables from the table to be plotted against the vertical axis number 3 (out of 4 possible ones). 
	\item vaxis4: one or several variables from the table to be plotted against the vertical axis number 4 (out of 4 possible ones). 
	\item \textit{Important: vaxis and vaxisI are exclusive in their application!}
	\item haxis: name of the horizontal variable 
	\item hmin: lower horizontal edge 
	\item hmax: upper horizontal edge; to be used, both hmin and hmax must be given.  
	\item vmin:lower edges of vertical axes, up to four numbers 
	\item vmax:upper edges of vertical axes, up to four numbers; both vmin and vmax must be given for an axis to be effective.  
	\item bars: 0 (default) or 1 - in the latter case, all curve points coming from the table are connected with the horizontal axis by vertical bars.  
	\item style: 1 (default), 2, 3, or 4: curve style, being solid, dashed, dotted, and dot-dashed; a value of 100 makes MAD-X use these four styles in turn for successive curves in the same plot. If style is 0 no curve is printed between points.  N.B. If symbol and style are null at the same time, style is forced to its default value (= 1).  
	\item colour: 1 (default), 2, 3, , or 5: colour, being black, red, green, blue, and magenta; a value of 100 makes MAD-X use these five colours in turn for successive curves. 
	\item symbol: 0 (default), 1, 2, 3, 4, or 5: none, dot, "+", "*", circle, and "x". These symbols are potted at all curve points; there size may have to be adapted (see below).  
	\item noversion: logical, default=false. If set true, the information concerning the madx version and the date are suppressed from the title. This option frees more space for the user's title.  
	\item interpolate: logical, default=false. Normally the curve points from the table are connected by straight lines; if "interpolate" is requested, then on-momentum Twiss parameters such as beta, alfa, and dispersion are interpolated with calls to the Twiss module inside each element, for all other variables splines are used to smooth the curves.  
	\item noline: logical, default=false. If s is the horizontal variable, then the machine will be plotted in symbolic form above the curve plot (except for tables having been read back into MAD-X). This may result in a thick black block if the horizontal scale is too large. "noline" allows the user to suppress the machine plotting.   
	\item notitle: logical, default=false. If true, suppresses the title line, including the information on the version and date. 
	\item marker\_plot: logical, default=false. If true, plotting is done also at the location of marker elements. This is only useful for the plotting of non-continuous functions like the "N1" from the aperture module. Beware that the PS file might became very large if this flag is invoked. 
	\item range\_plot: logical, default=false. Needed to allow to specify a plotting range also for user defined horizontal axis.  
	\item table: name of the table to be plotted from (default: twiss). If it is \textit{track}, the data to be plotted are taken from the tracking files generated for each required particle as defined by the attribute \textit{particle}. The name of this file has the following format: file name as defined by the attribute \textit{trackfile}, the observation point fixed to 1 and the particle number, e.g. \textit{testtrack.obs0001.p0003}.  If the required file has not been generated by the previous MAD-X command track, no plot is done for that particle.  The plot is obtained through the \textit{gnuplot} package.  N.B. the previous track command should contain the attribute \textit{dump}. The tracking plots appends the plots to an existing file specified via \textit{filename} appended by \textit{.ps}. The user should make sure that this file does not exist before starting a MAD-X run!
	\item particle: one or several numbers associated to the tracked particles for which the specified plot has to be displayed. 
	\item multiple: logical, default=false. If true all the curves generated for each tracked particle are put on one plot. Otherwise there will be one plot for each particle.  
	\item title: plot title string; if absent, the last overall title is used; if no such overall title as well, the sequence name is used.  
	\item range: horizontal plot \href{../Introduction/ranges.html}{range} given by elements. 
	\item file\_name: start of the file name for the Postscript file(s). Only the first occurrence of such a name will be used. Default is "madx" or "madx\_track" if the \textit{table} attribute is track.  Depending on the format (.ps or .eps, see below) the plots will either all be written into one file file\_name.ps, or one per plot into file\_name01.eps, file\_name02.eps, etc.  
	\item ptc: logical, default=false. If set true, the data to be plotted are taken from the table defined by the attribute \textit{ptc\_table} which is expected to be generated previously by the ptc package. The data belong to the column identified by one of the names set in the definition of the ptc twiss table. Interpolation is not available and the attribute \textit{interpolate} has no effect.  
	\item ptc\_table: name of the ptc twiss table to be plotted from (default: ptc\_twiss) 
	\item trackfile: first part of the name of the files containing tracking data for each particle (default: track) 
\end{itemize}


	\item \textbf{SETPLOT}
\begin{verbatim}
 setplot,
post=integer,font=integer, lwidth=real,xsize=real,ysize=real,
ascale=real, lscale=real, sscale=real, rscale=real; \end{verbatim} where the parameters have the following meaning: 
\begin{itemize}
	\item post: default = 1. If =1, makes one PostScript file (.ps) with all plots; if =2, makes one Encapsulated PostSscript file (.eps) per plot.  
	\item font: there are two defaults: 1 for screen plotting: this uses characters made from polygons; -1 for PostScript files; this is Times-Italic. There are various fonts available for positive and negative integers, best to be tried out, since they will look different on different systems anyway. GhostView will show strange vertical axis annotations, but the printed versions are normally OK.  
	\item lwidth: default = 1. Allows the user to set the curve line width.  Depends on the system as well, so to be tried out.  
	\item xsize: bounding box size for PostScript, default=27 cm.  
	\item ysize: bounding box size for PostScript, default=19 cm.  
	\item ascale: annotation character height scale factor, default=1.  
	\item lscale: axis label character height scale factor, default=1.  
	\item sscale: curve symbol (see above) scale factor, default=1.  
	\item rscale: axis text character height scale factor, default=1.  
\end{itemize}


	\item \textbf{RESPLOT}
\begin{verbatim}
 resplot; \end{verbatim} resets all defaults for the setplot command.  
\end{itemize}\href{http://www.cern.ch/Hans.Grote/hansg_sign.html}{hansg}, June 17, 2002, rdemaria \href{http://cern.ch/rdemaria}{rdemaria}, September 2007. 

%%\end{document}



%%%%\title{Range Selection}
%  Changed by: Chris ISELIN, 27-Jan-1997 
%  Changed by: Hans Grote, 10-Jun-2002 

\paragraph{Real life example for IF statements, and MACRO usage}


\begin{verbatim}

! Creates a footprint for head-on + parasitic collisions at IP1+5 
! of lhc.6.5; both lhcb1 (for tracking) and lhcb2 (to define the
! beam-beam elements, i.e. weak-strong) are used; there are flags to
! select head-on, left, and right parasitic separately at all IPs.
! The bunch spacing can be given in nanosec and automatically creates
! the beam-beam interaction points at the correct positions.
! It is important to set the correct BEAM parameters, i.e. number
! of particles, emittances, bunch length, energy.

!--- For completeness, all files needed by this job are copied
!    to the local directory ldb. The links to the the originals
!    in offdb (official database) are commented out.

Option,  warn,info,echo;
!System,
"ln -fns /afs/cern.ch/eng/sl/MAD-X/dev/test_suite/foot/V3.01.01 ldb";
!system,"ln -fns /afs/cern.ch/eng/lhc/optics/V6.4 offdb";
Option, -echo,-info,warn;
SU=1.0;
call, file = "ldb/V6.5.seq";
call,file="ldb/slice_new.madx";
Option, echo,info,warn;

!+++++++++++++++++++++++++ Step 1 +++++++++++++++++++++++
! 	define beam constants
!++++++++++++++++++++++++++++++++++++++++++++++++++++++++

b_t_dist = 25.e-9;                  !--- bunch distance in [sec]
b_h_dist = clight * b_t_dist / 2 ;  !--- bunch half-distance in [m]
ip1_range = 58.;                     ! range for parasitic collisions
ip5_range = ip1_range;
ip2_range = 60.;
ip8_range = ip2_range;

npara_1 = ip1_range / b_h_dist;     ! # parasitic either side
npara_2 = ip2_range / b_h_dist;
npara_5 = ip5_range / b_h_dist;
npara_8 = ip8_range / b_h_dist;

value,npara_1,npara_2,npara_5,npara_8;

 eg   =  7000;
 bg   =  eg/pmass;
 en   = 3.75e-06;
 epsx = en/bg;
 epsy = en/bg;

Beam, particle = proton, sequence=lhcb1, energy = eg,
          sigt=      0.077     , 
          bv = +1, NPART=1.1E11, sige=      1.1e-4, 
          ex=epsx,   ey=epsy;

Beam, particle = proton, sequence=lhcb2, energy = eg,
          sigt=      0.077     , 
          bv = -1, NPART=1.1E11, sige=      1.1e-4, 
          ex=epsx,   ey=epsy;

beamx = beam%lhcb1->ex;   beamy%lhcb1 = beam->ey;
sigz  = beam%lhcb1->sigt; sige = beam%lhcb1->sige;

!--- split5, 4d
long_a= 0.53 * sigz/2;
long_b= 1.40 * sigz/2;
value,long_a,long_b;

ho_charge = 0.2;

!+++++++++++++++++++++++++ Step 2 +++++++++++++++++++++++
! 	slice, flatten sequence, and cycle start to ip3
!++++++++++++++++++++++++++++++++++++++++++++++++++++++++

use,sequence=lhcb1;
makethin,sequence=lhcb1;
!save,sequence=lhcb1,file=lhcb1_thin_new_seq;
use,sequence=lhcb2;
makethin,sequence=lhcb2;
!save,sequence=lhcb2,file=lhcb2_thin_new_seq;
!stop;

option,-warn,-echo,-info;
call,file="ldb/V6.5.thin.coll.str";
option,warn,echo,info;

! keep sextupoles
ksf0=ksf; ksd0=ksd;
use,period=lhcb1;
select,flag=twiss.1,column=name,x,y,betx,bety;
twiss,file;
plot,haxis=s,vaxis=x,y,colour=100,noline;

use,period=lhcb2;
select,flag=twiss.2,column=name,x,y,betx,bety;
twiss,file;
plot,haxis=s,vaxis=x,y,colour=100,noline;
seqedit,sequence=lhcb1;
flatten;
endedit;

seqedit,sequence=lhcb1;
cycle,start=ip3.b1;
endedit;

seqedit,sequence=lhcb2;
flatten;
endedit;

seqedit,sequence=lhcb2;
cycle,start=ip3.b2;
endedit;

bbmarker: marker;  /* for subsequent remove */


!+++++++++++++++++++++++++ Step 3 +++++++++++++++++++++++
! 	define the beam-beam elements
!++++++++++++++++++++++++++++++++++++++++++++++++++++++++
!
!===========================================================
! read macro definitions
option,-echo;
call,file="ldb/bb.macros";
option,echo;

!
!===========================================================
!   this sets CHARGE in the head-on beam-beam elements. 
!   set +1 * ho_charge   for parasitic on, 0 for off

 on_ho1  = +1 * ho_charge; ! ho_charge depends on split
 on_ho2  = +0 * ho_charge; ! because of the "by hand" splitting
 on_ho5  = +1 * ho_charge;
 on_ho8  = +0 * ho_charge;

!
!===========================================================
!   set CHARGE in the parasitic beam-beam elements. 
!   set +1 for parasitic on, 0 for off
 on_lr1l = +1;
 on_lr1r = +1;
 on_lr2l = +0;
 on_lr2r = +0;
 on_lr5l = +1;
 on_lr5r = +1;
 on_lr8l = +0;
 on_lr8r = +0;

!
!===========================================================
!   define markers and savebetas
assign,echo=temp.bb.install;
!--- ip1
if (on_ho1  0)
{
  exec, mkho(1);
  exec, sbhomk(1);
}
if (on_lr1l  0 || on_lr1r  0)
{
  n=1; ! counter
  while (n  0 || on_lr1l  0)
{
  n=1; ! counter
  while (n  0)
{
  exec, mkho(5);
  exec, sbhomk(5);
}
if (on_lr5l  0 || on_lr5r  0)
{
  n=1; ! counter
  while (n  0 || on_lr5l  0)
{
  n=1; ! counter
  while (n  0)
{
  exec, mkho(2);
  exec, sbhomk(2);
}
if (on_lr2l  0 || on_lr2r  0)
{
  n=1; ! counter
  while (n  0 || on_lr2l  0)
{
  n=1; ! counter
  while (n  0)
{
  exec, mkho(8);
  exec, sbhomk(8);
}
if (on_lr8l  0 || on_lr8r  0)
{
  n=1; ! counter
  while (n  0 || on_lr8l  0)
{
  n=1; ! counter
  while (n  0)
{
exec, inho(mk,1);
}
if (on_lr1l  0 || on_lr1r  0)
{
  n=1; ! counter
  while (n  0 || on_lr1l  0)
{
  n=1; ! counter
  while (n  0)
{
exec, inho(mk,5);
}
if (on_lr5l  0 || on_lr5r  0)
{
  n=1; ! counter
  while (n  0 || on_lr5l  0)
{
  n=1; ! counter
  while (n  0)
{
exec, inho(mk,2);
}
if (on_lr2l  0 || on_lr2r  0)
{
  n=1; ! counter
  while (n  0 || on_lr2l  0)
{
  n=1; ! counter
  while (n  0)
{
exec, inho(mk,8);
}
if (on_lr8l  0 || on_lr8r  0)
{
  n=1; ! counter
  while (n  0 || on_lr8l  0)
{
  n=1; ! counter
  while (n betx) / 0.0007999979093;
value,on_sep2;
!===========================================================
!   define bb elements
assign,echo=temp.bb.install;
!--- ip1
if (on_ho1  0)
{
exec, bbho(1);
}
if (on_lr1l  0)
{
  n=1; ! counter
  while (n  0)
{
  n=1; ! counter
  while (n  0)
{
exec, bbho(5);
}
if (on_lr5l  0)
{
  n=1; ! counter
  while (n  0)
{
  n=1; ! counter
  while (n  0)
{
exec, bbho(2);
}
if (on_lr2l  0)
{
  n=1; ! counter
  while (n  0)
{
  n=1; ! counter
  while (n  0)
{
exec, bbho(8);
}
if (on_lr8l  0)
{
  n=1; ! counter
  while (n  0)
{
  n=1; ! counter
  while (n  0)
{
exec, inho(bb,1);
}
if (on_lr1l  0)
{
  n=1; ! counter
  while (n  0)
{
  n=1; ! counter
  while (n  0)
{
exec, inho(bb,5);
}
if (on_lr5l  0)
{
  n=1; ! counter
  while (n  0)
{
  n=1; ! counter
  while (n  0)
{
exec, inho(bb,2);
}
if (on_lr2l  0)
{
  n=1; ! counter
  while (n  0)
{
  n=1; ! counter
  while (n  0)
{
exec, inho(bb,8);
}
if (on_lr8l  0)
{
  n=1; ! counter
  while (n  0)
{
  n=1; ! counter
  while (n  footprint";
stop;
\end{verbatim}

\paragraph{\href{macro}{Real life example of MACRO definitions}}

\begin{verbatim}

bbho(nn): macro = {
!--- macro defining head-on beam-beam elements; nn = IP number
print, text="bbipnnl2: beambeam, sigx=sqrt(rnnipnnl2->betx*epsx),";
print, text="          sigy=sqrt(rnnipnnl2->bety*epsy),";
print, text="          xma=rnnipnnl2->x,yma=rnnipnnl2->y,";
print, text="          charge:=on_honn;";
print, text="bbipnnl1: beambeam, sigx=sqrt(rnnipnnl1->betx*epsx),";
print, text="          sigy=sqrt(rnnipnnl1->bety*epsy),";
print, text="          xma=rnnipnnl1->x,yma=rnnipnnl1->y,";
print, text="          charge:=on_honn;";
print, text="bbipnn:   beambeam, sigx=sqrt(rnnipnn->betx*epsx),";
print, text="          sigy=sqrt(rnnipnn->bety*epsy),";
print, text="          xma=rnnipnn->x,yma=rnnipnn->y,";
print, text="          charge:=on_honn;";
print, text="bbipnnr1: beambeam, sigx=sqrt(rnnipnnr1->betx*epsx),";
print, text="          sigy=sqrt(rnnipnnr1->bety*epsy),";
print, text="          xma=rnnipnnr1->x,yma=rnnipnnr1->y,";
print, text="          charge:=on_honn;";
print, text="bbipnnr2: beambeam, sigx=sqrt(rnnipnnr2->betx*epsx),";
print, text="          sigy=sqrt(rnnipnnr2->bety*epsy),";
print, text="          xma=rnnipnnr2->x,yma=rnnipnnr2->y,";
print, text="          charge:=on_honn;";
};

mkho(nn): macro = {
!--- macro defining head-on markers; nn = IP number
print, text="mkipnnl2: bbmarker;";
print, text="mkipnnl1: bbmarker;";
print, text="mkipnn:   bbmarker;";
print, text="mkipnnr1: bbmarker;";
print, text="mkipnnr2: bbmarker;";
};

inho(xx,nn): macro = {
!--- macro installing bb or markers for head-on beam-beam (split into 5)
print, text="install, element= xxipnnl2, at=-long_b, from=ipnn;";
print, text="install, element= xxipnnl1, at=-long_a, from=ipnn;";
print, text="install, element= xxipnn,   at=1.e-9,   from=ipnn;";
print, text="install, element= xxipnnr1, at=+long_a, from=ipnn;"; 
print, text="install, element= xxipnnr2, at=+long_b, from=ipnn;"; 
};

sbhomk(nn): macro = {
!--- macro to create savebetas for ho markers
print, text="savebeta, label=rnnipnnl2, place=mkipnnl2;";
print, text="savebeta, label=rnnipnnl1, place=mkipnnl1;";
print, text="savebeta, label=rnnipnn,   place=mkipnn;";
print, text="savebeta, label=rnnipnnr1, place=mkipnnr1;";
print, text="savebeta, label=rnnipnnr2, place=mkipnnr2;";    
};

mkl(nn,cc): macro = {
!--- macro to create parasitic bb marker on left side of ip nn; cc = count
print, text="mkipnnplcc: bbmarker;";
};

mkr(nn,cc): macro = {
!--- macro to create parasitic bb marker on right side of ip nn; cc = count
print, text="mkipnnprcc: bbmarker;";
};

sbl(nn,cc): macro = {
!--- macro to create savebetas for left parasitic
print, text="savebeta, label=rnnipnnplcc, place=mkipnnplcc;";
};

sbr(nn,cc): macro = {
!--- macro to create savebetas for right parasitic
print, text="savebeta, label=rnnipnnprcc, place=mkipnnprcc;";
};

inl(xx,nn,cc): macro = {
!--- macro installing bb and markers for left side parasitic beam-beam
print, text="install, element= xxipnnplcc, at=-cc*b_h_dist, from=ipnn;";
};

inr(xx,nn,cc): macro = {
!--- macro installing bb and markers for right side parasitic beam-beam
print, text="install, element= xxipnnprcc, at=cc*b_h_dist, from=ipnn;";
};

bbl(nn,cc): macro = {
!--- macro defining parasitic beam-beam elements; nn = IP number
print, text="bbipnnplcc: beambeam, sigx=sqrt(rnnipnnplcc->betx*epsx),";
print, text="          sigy=sqrt(rnnipnnplcc->bety*epsy),";
print, text="          xma=rnnipnnplcc->x,yma=rnnipnnplcc->y,";
print, text="          charge:=on_lrnnl;";
};

bbr(nn,cc): macro = {
!--- macro defining parasitic beam-beam elements; nn = IP number
print, text="bbipnnprcc: beambeam, sigx=sqrt(rnnipnnprcc->betx*epsx),";
print, text="          sigy=sqrt(rnnipnnprcc->bety*epsy),";
print, text="          xma=rnnipnnprcc->x,yma=rnnipnnprcc->y,";
print, text="          charge:=on_lrnnr;";
};
\end{verbatim}

%\href{http://www.cern.ch/Hans.Grote/hansg_sign.html}{hansg}, June 17, 2002 


%%%%\title{Range Selection}
%  Changed by: Chris ISELIN, 27-Jan-1997 
%  Changed by: Hans Grote, 16-Jan-2003 

\section{General Control Statements}

\subsection{ASSIGN}
\begin{verbatim}
assign, echo="file_name", truncate;
\end{verbatim} 
where "file\_name" is the name of an output file, or "terminal" and
truncate specifies if the file must be truncated when opened (ignored
for terminal). This allows switching the echo stream to a file or back,
but only for the commands value, show, and print. Allows easy
composition of future MAD-X input files. A real life example (always the
same) is to be found under \href{foot.html}{footprint example}.  

\subsection{CALL}
\begin{verbatim}
call, file = "file_name";
\end{verbatim} 
where "file\_name"  is the name of an input file. This file will be read
until a "return;" statement, or until end\_of\_file; it may contain any
number of calls itself, and so on to any depth.  

%% 2013-Jul-11  17:23:00  ghislain: I propose to move COGUESS to the
%% orbit correction part of the manual
\subsection{COGUESS}
\label{subsec:general_coguess}
\begin{verbatim}
coguess, tolerance = double, 
         x = double, px = double, 
         y = double, py = double, 
         t = double, pt = double;
\end{verbatim} 
sets the required convergence precision in the closed orbit search
("tolerance", see as well Twiss command
\href{../twiss/twiss.html#tolerance}{tolerance}).  

The other parameters define a first guess for all future closed orbit
searches in case they are different from zero.  

\subsection{CREATE}
\begin{verbatim}
create, table = table, column = var1, var2,_name,...;
\end{verbatim} 
creates a table with the specified variables as columns. This table can
then be \hyperlink{fill}{fill}ed, and finally one can
\hyperlink{write}{write} it in TFS format. The attribute "\_name" adds
the element name to the table at the specified column, this replaces the
undocumented "withname" attribute that was not always working properly.  

See the \href{../Introduction/select.html#ucreate}{user table I}
example; 
or an example of joining 2 tables of different length in one table
including the element name:
\href{../Introduction/select.html#screate}{user table II} 

\subsection{DELETE}
\begin{verbatim}
delete, sequence = s_name, table = t_name;
\end{verbatim} 
deletes a sequence with name "s\_name" or a table with name "t\_name"
from memory. The sequence deletion is done without influence on other
sequences that may have elements that were in the deleted sequence.  

%% 2013-Jul-11  17:24:21  ghislain: I propose to move DUMPSEQU to the
%% sequence edition and manipulation part of the manual
\subsection{DUMPSEQU}
\begin{verbatim}
dumpsequ, sequence = s_name, level = integer;
\end{verbatim} 
Actually a debug statement, but it may come handy at certain
occasions. Here "s\_name" is the name of an expanded (i.e. USEd)
sequence. The amount of detail is controlled by "level":  
\begin{verbatim}
level = 0:    print only the cumulative node length = sequence length
      > 0:    print all node (element) names except drifts
      > 2:    print all nodes with their attached parameters
      > 3:    print all nodes, and their elements with all parameters
\end{verbatim}


\subsection{EXEC}
\begin{verbatim}
exec, label;
\end{verbatim} 
Each statement may be preceded by a label; it is then stored and can be
executed again with "exec, label;" any number of times; the executed
statement may be another "exec", etc.; however, the major usage of this
statement is the execution of a \href{special.html#macro}{macro}.  

\subsection{EXIT}
\begin{verbatim}
exit;
\end{verbatim} 
ends the program execution. 

\subsection{FILL} 
Every command 
\begin{verbatim}
fill, table = table;
\end{verbatim} 
adds a new line with the current values of all column variables into the
user table \hyperlink{create}{create}d beforehand. This table one can
then \hyperlink{write}{write} in TFS format.  See as well the
\href{../Introduction/select.html#ucreate}{user table} example.  

\subsection{OPTION}
\label{subsec:general_option}
\begin{verbatim}
option, flag { = true | = false };
option, flag | -flag;
\end{verbatim} 
sets an option as given in "flag"; the part in curly brackets is
optional: if only the name of the option is given, then the option will
be set true (see second line); a "-" sign preceding the name sets it to
"false".  

Example: 
\begin{verbatim}
option, echo = true;
option, echo;
\end{verbatim} 
are identical, ditto 
\begin{verbatim}
option, echo = false;
option, -echo;
\end{verbatim} 

The available options are: 
\begin{verbatim}
  name           default meaning if true
  ====           ======= ===============
  echo            true   echoes the input on the standard output file
  warn            true   issues warnings
  info            true   issues informations
  debug           false  issues debugging information
  trace           false  prints the system time after each command
  verify          false  issues a warning if an undefined variable is used
  tell            false  prints the current value of all options
  reset           false  resets all options to their defaults
  no_fatal_stop   false  Prevents madx from stopping in case of a fatal error. 
                         Use at your own risk.

  rbarc           true   converts the RBEND straight length into the arc 
                         length
  thin_foc        true   if false suppresses the 1(rho**2) focusing of thin 
                         dipoles
  bborbit         false  the closed orbit is modified by beam-beam kicks
  sympl           false  all element matrices are symplectified in Twiss
  twiss_print     true   controls whether the twiss command produces output.
\end{verbatim} 

The option "rbarc" is implemented for backwards compatibility with MAD-8
up to version 8.23.06 included; in this version, the RBEND length was
just taken as the arc length of an SBEND with inclined pole faces,
contrary to the MAD-8 manual.  


\subsection{PRINT}
\begin{verbatim}
print, text = "...";
\end{verbatim} 
prints the text to the current output file (see ASSIGN above). The text
can be edited with the help of a  \href{special.html#macro}{macro
  statement}. For more details, see there.  


\subsection{QUIT}
\begin{verbatim}
quit;
\end{verbatim} 
ends the program execution. 


\subsection{READTABLE}
\begin{verbatim}
readtable, file = "file_name";
\end{verbatim} 
reads a TFS file containing a MAD-X table back into memory. This table
can then be manipulated as any other table, i.e. its values can be
accessed, it can be plotted, written out again etc.  


\subsection{READMYTABLE}
\label{subsec:general_readmy}
\begin{verbatim}
readmytable, file = "file_name", table = internalname;
\end{verbatim} 
reads a TFS file containing a MAD-X table back into memory. This table
can then be manipulated as any other table, i.e. its values can be
accessed, it can be plotted, written out again etc. 

An internal name for
the table can be freely assigned while for the command READTABLE it is
taken from the information section of the table itself. This feature
allows to store multiple tables of the same type in memory without
overwriting existing ones.   


\subsection{REMOVEFILE}
\begin{verbatim}
removefile, file = "file_name";
\end{verbatim} 
remove the file "file\_name" from disk. It is more portable than  
\begin{verbatim}
system("rm filename"); // Unix specific
\end{verbatim}


\subsection{RENAMEFILE}
\begin{verbatim}
renamefile, file = "file_name", name = "new_file_name";
\end{verbatim} 
rename the file "file\_name" to "new\_file\_name" on the disk. It is more
portable than  
\begin{verbatim}
system("mv file_name new_file_name"); // Unix specific
\end{verbatim}

%% 2013-Jul-11  17:24:21  ghislain: I propose to move RESBEAM to the
%% beam declaration part of the manual
\subsection{RESBEAM}
\begin{verbatim}
resbeam, sequence = sequence_name;
\end{verbatim} 
resets the default values of the beam belonging to sequence sequence\_name, or
of the default beam if no sequence is given.  


\subsection{RETURN}
\begin{verbatim}
return;
\end{verbatim} 
ends reading from a "called" file; if encountered in the standard input
file, it ends the program execution.  


\subsection{SAVE}
\label{subsec:general_save}
\begin{verbatim}
save, sequence = sequ1, sequ2, ..., file = "file_name", beam, bare;
\end{verbatim} 
saves the sequence(s) specified with all variables and elements needed
for their expansion, onto the file "file\_name". 

{\bf Warning:} If quotes are used for
the "file\_name", capital and low characters are kept as specified, if they
are omitted the "filename" will have lower characters only. 

Example:
\begin{verbatim}
save, sequence = lhc, file = "Test_One";
\end{verbatim}
saves the lhc sequence to a file name Test\_One on disk, while
\begin{verbatim}
save, sequence = lhc, file = Test_One;
\end{verbatim}
saves the lhc sequence to a file name test\_one on disk.

The optional
flag can have the value "mad8" (without the quotes), in which case the
sequence(s) is/are saved in MAD-8 input format.  

The flag "beam" is optional; when given, all beams belonging to the
sequences specified are saved at the top of the save file.  

The parameter "sequence" is optional; when omitted, all sequences are
saved.  

However, it is not advisable to use "save" without the "sequence" option
unless you know what you are doing. This practice will avoid spurious
saved entries.    Any number of "select,flag=save" commands may precede
the SAVE command. In that case, the names of elements, variables, and
sequences must match the pattern(s) if given, and in addition the
elements must be of the class(es) specified. See here for a
\href{../Introduction/select.html#save_select}{SAVE with SELECT}
example.  

It is important to note that the precision of the output of the save
command depends on the output precision. Details about default
precisions and how to adjust those precisions can be found at the
\href{../Introduction/set.html#Format}{SET Format} instruction page.   
 
The attribute 'bare' allows to save just the sequence without the
element definitions nor beam information. This allows to re-read in a
sequence with might otherwise create a stop of the program. This is
particularly useful to turn a line into a sequence to seqedit
it. 

Example:  
\begin{verbatim}
tl3:line=(ldl6,qtl301,mqn,qtl301,ldl7,qtl302,mqn,qtl302,ldl8,ison);
DLTL3 : LINE=(delay, tl3);
use, period=dltl3;

save,sequence=dltl3,file=t1,bare; // new parameter "bare": only sequ. saved
call,file=t1; // sequence is read in and is now a "real" sequence
// if the two preceding lines are suppressed, seqedit will print a warning
// and else do nothing
use, period=dltl3;
twiss, save, betx=bxa, alfx=alfxa, bety=bya, alfy=alfya;
plot, vaxis=betx, bety, haxis=s, colour:=100;
SEQEDIT, SEQUENCE=dltl3;
  remove,element=cx.bhe0330;
  remove,element=cd.bhe0330;
ENDEDIT;

use, period=dltl3;
twiss, save, betx=bxa, alfx=alfxa, bety=bya, alfy=alfya;
\end{verbatim}


\subsection{SAVEBETA}
\label{subsec:general_savebeta}
\begin{verbatim}
savebeta, label = label, place = place, sequence = sequence_name;
\end{verbatim} 
marks a place named "place" in an expanded sequence "sequence\_name"; 
at the next TWISS command execution, a
\href{../twiss/twiss.html#beta0}{beta0} 
block will be saved at that place with the label "label". This is done
only once; in order to get a new beta0 block there, one has to re-issue
the command. The contents of the beta0 block can then be used in other
commands, e.g. TWISS and MATCH.  

Example (after sequence expansion): 
\begin{verbatim}
savebeta, label = sb1, place = mb[5], sequence = fivecell;
twiss;
show, sb1;
\end{verbatim} 
will save and show the beta0 block parameters at the end (!) of the
fifth element of type mb in the sequence.  


\subsection{SELECT} %select</a}{SELECT}
\begin{verbatim}
select, flag = flag, range = range, class = class, pattern = pattern,
        slice = integer, column =s1, s2, s3,..,sn, sequence=sequence_name,
        full, clear;
\end{verbatim} 
selects one or several elements for special treatment in a subsequent
command. All selections for a given command remain valid until "clear"
is specified; the selection criteria on the same command are logically
ANDed; the selection criteria on different SELECT statements logically
ORed.   

 Example: 
\begin{verbatim}
select, flag = error, class = quadrupole, range = mb[1]/mb[5];
select, flag = error, pattern = "^mqw.*";
\end{verbatim} 
selects all quadrupoles in the range mb[1] to mb[5], and all elements
(in the whole sequence) the name of which starts with "mqw", for
treatment by the error module.  

"flag" can be one of the following: 
\begin{itemize}
   \item seqedit: selection of elements for the
     \href{seqedit.html}{seqedit} module.  
   \item error: selection of elements for the
     \href{../error/error.html}{error} assignment module.  
   \item makethin: selection of elements for the
     \href{../makethin/makethin.html}{makethin} module that
     converts the sequence into one with thin elements only.  
   \item sectormap: selection of elements for the
     \href{../Introduction/sectormap.html}{sectormap} output file
     from the Twiss module.  
   \item table: here "table" is a table name such as twiss, track
     etc., and the rows and columns to be written are selected.  
\end{itemize} 

For the RANGE, CLASS, PATTERN, FULL, and CLEAR parameters
see \href{../Introduction/select.html}{SELECT}.  

"slice" is only used by \href{../makethin/makethin.html}{makethin} and
prescribes the number of slices into which the selected elements have to
be cut (default = 1).  

"column" is only valid for tables and decides the selection of columns
to be written into the TFS file. The "name" argument is special in that
it refers to the actual name of the selected element. For an example,
see \href{../Introduction/select.html}{SELECT}.  


\subsection{SHOW}
\begin{verbatim}
show, command;
\end{verbatim} 
prints the "command" (typically "beam", "beam\%sequ", or an element
name), with the actual value of all its parameters.  


\subsection{STOP}
\begin{verbatim}
stop;
\end{verbatim} 
ends the program execution. 


\subsection{SYSTEM}
\begin{verbatim}
system, "string";
\end{verbatim} 
transfers the string in quotes to the system for execution.  

Example: 
\begin{verbatim}
system,"ln -s /afs/cern.ch/user/u/user/public/some/directory short";
\end{verbatim}


\subsection{TABSTRING}
Note: this is not a command and should appear in the variables section
\begin{verbatim}
tabstring(arg1,arg2,arg3)
\end{verbatim}  
The "string function" tabstring(arg1,arg2,arg3) with exactly  three
arguments; arg1 is a table name (string), arg2 is a column name
(string), arg3 is a row number (integer), count starts at 0. The
function can be used in any context where a string appears; in case
there is no match, it returns \_void\_.  


\subsection{TITLE}
\begin{verbatim}
title, "title";
\end{verbatim} 
inserts the string in quotes as title in various tables and plots.  


\subsection{USE}
\label{subsec:general_use}
\begin{verbatim}
use, period = sequence_name, range = range, survey;
\end{verbatim} 
expands the sequence with name "sequence\_name", or a part of it as specified
in the \href{../Introduction/ranges.html#range}{range}. The
\texttt{survey} option plugs the survey data into the sequence elements
nodes on the first pass (see \href{../survey/survey.html}{survey}).  


\subsection{VALUE}
\begin{verbatim}
value, exp1, exp2,...;
\end{verbatim} 
prints the actual values of the expressions given. 

Example: 
\begin{verbatim}
a = clight/1000.;
value, a, pmass, exp(sqrt(2));
\end{verbatim} 
results in 
\begin{verbatim}
a = 299792.458         ;
pmass = 0.938271998        ;
exp(sqrt(2)) = 4.113250379        ;
\end{verbatim}


\subsection{WRITE}
\label{subsec:general_write}
\begin{verbatim}
write, table = table, file = "file_name";
\end{verbatim} 
writes the table "table" onto the file "file\_name"; only the rows and
columns of a preceding \verb+select, flag = table,...;+ are written. If no select
has been issued for this table, the file will only contain the
header. If the FILE argument is omitted, the table is written to
standard output.  


%\href{http://www.cern.ch/Hans.Grote/hansg_sign.html}{hansg}, June 17, 2002 

%%%%\title{Range Selection}
%  Changed by: Chris ISELIN, 27-Jan-1997 
%  Changed by: Hans Grote, 30-Sep-2002 

\subsection{Program Flow Statements}

\begin{itemize}
	\item IF
\begin{verbatim}

if (logical_expression) {statement 1; statement 2; ...; statement n; }
\end{verbatim}
\href{logical}{ where "logical\_expression" } is one of 
\begin{verbatim}

expr1 oper expr2
expr11 oper1 expr12 && expr21 oper2 expr22
expr11 oper1 expr12 || expr21 oper2 expr22
\end{verbatim} 
and oper one of 
\begin{verbatim}

==          ! equal
<>          ! not equal
<           ! less than
>           ! greater than
<=          ! less than or equal
>=          ! greater than or equal
\end{verbatim} 
The expressions are arithmetic expressions of type real. The statements
in the curly brackets are executed if the logical expression is true.  


	\item ELSEIF%elseif}{ELSEIF}
\begin{verbatim}

elseif (logical_expression) {statement 1; statement 2; ...; statement n; }
\end{verbatim} 
Only possible (in any number) behind an IF, or another ELSEIF; is
executed if  logical\_expression is true, and if none of the preceding
IF or ELSEIF logical conditions was true.  


	\item ELSE%else}{ELSE}
\begin{verbatim}

else {statement 1; statement 2; ...; statement n; }
\end{verbatim} 
Only possible (once) behind an IF, or an ELSEIF; is executed if
logical\_expression is true, and if none of the preceding IF or ELSEIF
logical conditions was true.  

For a real life example, see \href{foot.html}{ELSE example}. 


	\item WHILE
\begin{verbatim}

while (logical_condition){statement 1; statement 2; ...; statement n; }
\end{verbatim}  
executes the statements in curly brackets while the logical\_expression
is true. A simple example (in case you have forgotten the first ten
factorials) would be  
\begin{verbatim}

option,-info;   ! otherwise you get redifiniton warnings
n=1; m=1;
while (n <= 10)
{
  m = m * n;  value, m;
  n = n + 1;
};
\end{verbatim}

For a real life example, see \href{foot.html}{WHILE example}.

	\item MACRO

\begin{verbatim}

label: macro = {statement 1; statement 2; ...; statement n; };
label(arg1,...,argn): macro = {statement 1; statement 2; ...; statement n; };
\end{verbatim} 
The first form allows the execution of a group of statements via a
single command:  
\begin{verbatim}

exec, label;
\end{verbatim} 
will execute the statements in curly brackets exactly once. This command
can be issued any number of times.  

The second form allows to replace strings anywhere inside the statements
in curly brackets by other strings, or integer numbers prior to
execution. This is a powerful construct and should be handled with care.  

Simple example: 
\begin{verbatim}

option,-echo,-info;  ! otherwise the output is somewhat confusing
simple(xx,yy): macro = { xx = yy^2 + xx; value, xx;};
a = 3;
b = 5;
exec, simple(a,b);
\end{verbatim}

Somewhat more tricky (a "\$" in front of an argument means that the
truncated integer value of this argument is used for replacement, rather
than the argument string itself).  
\begin{verbatim}

tricky(xx,yy,zz): macro = {mzz.yy: xx, l = 1.yy, kzz = k.yy;};
n=0;
while (n < 3)
{
  n = n+1;
  exec,tricky(quadrupole,$n,1);
  exec,tricky(sextupole,$n,2);
};
\end{verbatim} 
Whereas the actual use of the preceding example is NOT recommended,
a real life example, showing the full power (!) of macros is to be
found under \href{foot.html}{macro usage} for the usage, and
under \href{foot.html#macro}{macro definition} for the
definition.


Beware of the following rules:

	\item Generally speaking: \textit{ special constructs } like IF, WHILE,
MACRO will only allow one level of inclusion of another \textit{
special construct }.

	\item  Macros must not be called with numbers, but with strings
(i.e. variable names in case of numerical values), i.e. 



NOT

\begin{verbatim}

exec,thismacro($99,$129);
\end{verbatim}
BUT

\begin{verbatim}

n1=99; n2=219;
exec,thismacro($n1,$n2);
\end{verbatim}

\end{itemize}

%\href{http://www.cern.ch/Hans.Grote/hansg_sign.html}{hansg}, June 17, 2002




\chapter{General Control Statements} 

\madx consists of a core program, and modules for specific tasks such as
\hyperref[chap:twiss]{twiss parameter calculation},
\hyperref[chap:match]{matching}, \hyperref[chap:thintrack]{thin lens
  tracking}, \textsl{etc.}   
 
The statements listed here are those executed by the program core.
They deal with the I/O, element and sequence declaration, sequence
manipulation, statement flow control (e.g. \texttt{IF, WHILE}),
\texttt{MACRO} declaration, saving sequences onto files in \madx or
\madeight format, \textsl{etc.}  


%% Moved to TWISS chapter
%% \subsection{COGUESS}
%% \label{subsec:coguess}

%% In order to help the initial finding of the closed orbit by the
%% \texttt{TWISS} module, it is possible to specify an initial guess. 

%% \madbox{
%% COGUESS, \=TOLERANCE=real, \\
%%          \>X=real, PX=real, Y=real, PY=real, T=real, PT=real, \\
%%          \>CLEAR=logical;
%% }
%% sets the required convergence precision in the closed orbit search
%% ("tolerance", see as well Twiss command
%% \href{../twiss/twiss.html#tolerance}{tolerance}).  

%% The other parameters define a first guess for all future closed orbit
%% searches in case they are different from zero.  

%% The clear parameter in the argument list resets the tolerance to its default value 
%% and cancels the effect of coguess in defining a first guess for subsequent 
%% closed orbit searches. \\
%% Default = false, \texttt{clear} alone is equivalent to \texttt{clear=true}


\section{EXIT, QUIT, STOP}
\label{sec:exit}\label{sec:quit}\label{sec:stop}
Any of these three commands ends the execution of \madx:
\madbox{
EXIT;
}
\madbox{
QUIT;
}
\madbox{
STOP;
}


\section{HELP}
\label{sec:help}
The \texttt{HELP} command prints all parameters, and their defaults
values, for the statement given; this includes basic element types.
\madbox{
HELP, statement\_name;
}

\section{SHOW}
\label{sec:show}
The \texttt{SHOW} command prints the \texttt{command} (typically
\texttt{beam}, \texttt{beam\%sequ}, or an element name), with the actual
value of all its parameters.   
\madbox{
SHOW, command;
}

\section{VALUE}
\label{sec:value}
The \texttt{VALUE} command evaluates the current value of all listed
expressions, constants or variables, and prints the result in the form
of \madx statements on the assigned output file. 
\madbox{
VALUE, expression\{, expression\} ;
}

Example: \\
\madxmp{
a = clight/1000.; \\
value, a, pmass, exp(sqrt(2));
}
results in 
\madxmp{
a = 299792.458         ; \\
pmass = 0.938272046        ; \\
exp(sqrt(2)) = 4.113250379        ; \\
}

\section{OPTION}
\label{sec:option}

The \texttt{OPTION} commands sets the logical value of a number of flags
that control the behavior of \madx.

\madbox{
OPTION, flag=logical;
}

Because all attributes of \texttt{OPTION} are logical flags, the
following two statements are identical:
\madxmp{
OPTION, flag = true;\\
OPTION, flag;
}
And the following two statements are also identical:
\madxmp{
OPTION, flag = false;\\
OPTION, -flag;
}

Several flags can be set in a single \texttt{OPTION} command, e.g.
\madxmp{
OPTION, ECHO, WARN=true, -INFO, VERBOSE=false;
}

The available flags, their default values and their effect on \madx when
they are set to \texttt{TRUE} are listed in table \ref{table:options}.

\begin{table}[ht]
  \caption{Flags available to \texttt{OPTION} command}
  \vspace{1ex}
  \centering
  \label{table:options}
  \begin{tabular}{|l|c|l|}
    \hline
    \textbf{FLAG }  & \textbf{default} & \textbf{effect if \texttt{TRUE}} \\
    \hline
    \texttt{ECHO}      & true  & echoes the input on the standard output file \\
    \texttt{WARN}      & true  & issues warning statements\\
    \texttt{INFO}      & true  & issues information statements\\
    \texttt{DEBUG}     & false & issues debugging information \\
    \texttt{ECHOMACRO} & false & issues macro expansion printout for debugging \\
    \texttt{VERBOSE}   & false & issues additional printout in makethin \\
    \texttt{TRACE}     & false & prints the system time after each command \\
    \texttt{VERIFY}    & false & issues a warning if an undefined variable is used 
    \\
    \hline
    \texttt{TELL}      & false & prints the current value of all options \\
    \texttt{RESET}     & false & resets all options to their defaults \\
    \hline
    \texttt{NO\_FATAL\_STOP} & false & Prevents madx from stopping in case of a fatal error \\
    &       & \textbf{Use at your own risk !} \\
    \hline
    \texttt{RBARC}     & true & converts the RBEND straight length into the arc length \\
    \texttt{THIN\_FOC} & true & enables the $1/\rho^2$ focusing of thin dipoles \\
    \texttt{BBORBIT}   & false & the closed orbit is modified by beam-beam kicks \\
    \texttt{SYMPL}     & false & all element matrices are symplectified in Twiss \\
    \texttt{TWISS\_PRINT} & true & controls whether the twiss command produces output \\
    \texttt{THREADER}  & false & enables the threader for closed orbit finding in Twiss \\ 
    &       & (see Twiss module) \\ 
    \hline
  \end{tabular}
\end{table}

The option \texttt{RBARC} is implemented for backwards compatibility
with \madeight up to version 8.23.06 included; in this version, the
\texttt{RBEND} length was just taken as the arc length of an
\texttt{SBEND} with inclined pole faces, contrary to the \madeight manual.  



\section{EXEC}
\label{sec:exec}
Each statement may be preceded by a label, when parsed and executed the
statement is then also stored and can be executed again with
\madbox{
EXEC, label;
}
This mechanism can be invoked any number of times, and the executed
statement may be calling another \texttt{EXEC}, etc. 
\madxmp{
tw: TWISS, FILE, SAVE; ! first execution of TWISS \\
... \\
EXEC, tw; ! second execution of the same TWISS command \\
}
Note however, that the main usage of this \madx construct is the
execution of a \hyperref[sec:macro]{\texttt{MACRO}}.   

\section{SET}
\label{sec:set}
The \texttt{SET} command is used in two forms:
\madbox{
SET, FORMAT=string \{, string\} ;\\
SET, SEQUENCE=string;
}


The first form of the \texttt{SET} command defines the formats for the
output precision that \madx uses with the \texttt{SAVE}, \texttt{SHOW},
\texttt{VALUE} and \texttt{TABLE} commands. The formats can be
given in any order and stay valid until replaced. 

The formats follow the C convention and must be included in double
quotes. The allowed formats are \\
\textit{n}\texttt{d} for integers with a field-width = \textit{n}, \\
\textit{n.m}\texttt{f} or \textit{n.m}\texttt{g} or
\textit{n.m}\texttt{e} for floats with field-width = \textit{n}
and precision = \textit{m}, \\
\textit{n}\texttt{s} for strings with a field-width = \textit{n}.\\
The default is "right adjusted", a "-" changes it to "left adjusted".

\textbf{Example:}\\
\madxmp{
SET, FORMAT="12d", "-18.5e", "25s";
}

%% \begin{verbatim}
%% "nd" for integer with n = field width.
%% \end{verbatim}
%% \begin{verbatim}
%% "m.nf" or "m.ng" or "m.ne" for floating, m field width, n precision.
%% \end{verbatim}
%% \begin{verbatim}
%% "ns" for string output.
%% \end{verbatim} 


The default formats are \texttt{"10d", "18.10g"} and \texttt{"-18s"}.

Example: 
\begin{verbatim}
set, format="22.14e";
\end{verbatim} 
changes the current floating point format to \texttt{22.14e}; the other
formats remain unchanged.  
\begin{verbatim}
set, format="s","d","g";
\end{verbatim} 
sets all formats to automatic adjustment according to C conventions. 

The second form of the \texttt{SET} command allows to select the
current sequence without the \hyperref[sec:use]{\texttt{USE}} command,
which would bring back to a bare lattice without errors. The command
only works 
if the chosen sequence has been previously activated with a
\hyperref[sec:use]{\texttt{USE}} 
command, otherwise a warning is issued and \madx continues with the
unmodified current sequence. This command is particularly useful for
commands that do not have the sequence as an argument like
\hyperref[chap:emit]{\texttt{EMIT}} or \hyperref[chap:ibs]{\texttt{IBS}}. 



\section{SYSTEM}
\label{sec:system}
\madbox{
SYSTEM, "string";
}
transfers the quoted \texttt{string} to the operating system for
execution. The quotes are stripped and no check of the return status is
performed by \madx. 

\textbf{Example:} 
\madxmp{
SYSTEM, "ln -s /afs/cern.ch/user/j/joe/input shortname";
}
makes a local link to an AFS directory with the name \texttt{shortname}
on a \texttt{UNIX} system.  

\textbf{Attention:} Although this command is very convenient, it is
clearly not portable across systems and it should probably be avoided if
one intends to share \madx scripts with collaborators working on other
platforms.  

\section{TITLE}
\label{sec:title}
\madbox{
TITLE, "string";
}
defines a \texttt{string} that is inserted as title in various table
outputs and plot results.  


\section{USE}
\label{sec:use}
\madx operates on beamlines that must be loaded and expanded in memory
before other commands can be invoked. The \texttt{USE} command allows
this loading and expansion.

\madbox{
USE, \=SEQUENCE=sequence\_name, PERIOD=sequence\_name,\\
     \>RANGE=range, \\
     \>SURVEY=logical;
}

The attributes to the \texttt{USE} command are:
\begin{madlist}
  \ttitem{SEQUENCE} name of the sequence to be loaded and expanded. 
  \ttitem{PERIOD} name of the sequence to be loaded and expanded. \\ 
  \texttt{PERIOD} is an alias to \texttt{SEQUENCE} that was kept for
  backwards compatibility with \madeight and only one of them should be
  specified in a \texttt{USE} statement. 
  \ttitem{RANGE} specifies a \hyperref[sec:range]{range}.   
  restriction so that only the specified part of the named sequence is
  loaded and  expanded.
  \ttitem{SURVEY} option to plug the survey data into the sequence elements
  nodes on the first pass (see \hyperref[chap:survey]{\texttt{SURVEY}}).
\end{madlist}

Note that reloading a sequence with the \texttt{USE} command reloads a
bare sequence and that any \hyperref[chap:error]{\texttt{ERROR}} or
orbit correction previously assigned or associated to the sequence are
discarded. A mechanism to select a sequence without this side effect of the 
\texttt{USE} command is provided with the
\hyperref[sec:set]{\texttt{SET, SEQUENCE=...}} command. 


\section{SELECT} 
\label{sec:select}

Some \madx commands can perform specific operations on selected elements
or ranges of elements and can produce specific output for selected
elements or ranges of elements. 

The selection is made through the \texttt{SELECT} command and applies to
subsequent commands.

\madbox{
SELECT, \=FLAG=string, RANGE=string, CLASS=string, PATTERN=string, \\
        \>SEQUENCE=string, FULL=logical, CLEAR=logical,\\
        \>COLUMN=string\{,string\},  SLICE=integer, THICK=logical, \\
        \>STEP=real, AT=\{real, \ldots \};
} 

The attributes to the \texttt{SELECT} command are:
\begin{madlist}
  \ttitem{FLAG} determines the applicability of the \texttt{SELECT}
  statement and the attribute value can be one of the following: 
  \begin{madlist}
    \ttitem{SEQEDIT} selection of elements for the
    \hyperref[sec:seqedit]{\texttt{SEQEDIT}} module.
    
    \ttitem{ERROR} selection of elements for the
    \hyperref[chap:error]{error assignment} module.
    
    \ttitem{MAKETHIN} selection of elements for the
    \hyperref[chap:makethin]{\texttt{MAKETHIN}} command that
    converts the sequence into one with thin elements.
    
    \ttitem{SECTORMAP} selection of elements for the
    \hyperref[sec:sectormap]{\texttt{SECTORMAP}} output file
    from the \hyperref[chap:twiss]{\texttt{TWISS}} module.
    
    \ttitem{SAVE} selection of elements for the
    \hyperref[sec:save]{\texttt{SAVE}} command.

    \ttitem{INTERPOLATE} selection of interpolation points for the
    \hyperref[chap:twiss]{\texttt{TWISS}} command.

    \ttitem{tablename} is a table name such as \texttt{twiss}, 
    \texttt{track} etc., and the rows and columns to be written are
    selected.
  \end{madlist} 
  
  \ttitem{RANGE} the range of elements to be selected as defined in
  section \ref{sec:range} on \hyperref[sec:range]{range} selection.

  \ttitem{CLASS} the class of elements to be selected as defined in
  section \ref{sec:class} on \hyperref[sec:class]{class} selection.

  \ttitem{PATTERN} the regular expression pattern for the element names
  to be selected as defined in section \ref{sec:regex} on selection via
  \hyperref[sec:regex]{regular expressions}. 

  \ttitem{SEQUENCE} the name of a sequence to which the selection is applied.

  \ttitem{FULL} a logical falg to select ALL positions in the sequence
  for the named flag. \\
  For the flag \texttt{TWISS}, this attribute includes all standard
  columns for a \texttt{TWISS} table, and therefore the following two
  statements are equivalent:
  \madxmp{
SELECT, FLAG=twiss, COLUMN= name, s, betx, ..., var1; \\
SELECT, FLAG=twiss, FULL, COLUMN= var1; 
  } 
  \texttt{FULL=true} is the default for the \texttt{MAKETHIN} flag and
  for tables: \textsl{e.g.} \texttt{SELECT, FLAG=makethin;} is
  equivalent to \texttt{SELECT, FLAG=makethin, FULL;}   
  
  \ttitem{CLEAR} deselects ALL positions in the sequence for the flag
  "name". This is the default for \texttt{ERROR} and \texttt{SEQEDIT}
  flags. \\
  \textsl{e.g.} \texttt{SELECT, FLAG=error;} is equivalent to
  \texttt{SELECT, FLAG=error, CLEAR;} 

  \ttitem{COLUMN} is only valid for tables and takes as attribute value
  a list of columns to be written into the TFS file. The special "\_name"
  argument refers to the actual name of the element. 
  %For an example, see \hyperref[sec:select]{\texttt{SELECT}}.  

  \ttitem{SLICE} is the number of slices into which the selected
  elements have to be cut and is only used by
  \hyperref[chap:makethin]{\texttt{MAKETHIN}} and
  \hyperref[chap:twiss]{\texttt{FLAG=INTERPOLATE}}. (Default = 1).

  \ttitem{THICK} is a logical flag to indicate  whether the selected
  elements are treated as thick elements by the
  \hyperref[chap:makethin]{\texttt{MAKETHIN}} command. \\  
  This only applies up to now to
  \hyperref[sec:quadrupole]{\texttt{QUADRUPOLE}}s and
  \hyperref[sec:bend]{\texttt{BEND}}s for which thick maps
  have been explicitely derived. 

  \ttitem{STEP} output intermediate values every \texttt{STEP} meters
  in the \hyperref[chap:twiss]{\texttt{TWISS}} command.

  \ttitem{AT} manual specification of interpolation points for the
  \hyperref[chap:twiss]{\texttt{TWISS}} command. Specified as a
  fraction of the node length, i.e. a value of 0.5 slices at
  the centre of the element.
\end{madlist}

\vskip 5mm
\textbf{Composition of \texttt{SELECT} statements:} \\
The selection criteria provided on a single \texttt{SELECT} statement
are logically \texttt{AND}ed, \textsl{i.e.} selected elements have to
fulfill \textbf{all} provided criteria in the single \texttt{SELECT}
statement. 

The selection criteria on different \texttt{SELECT} statements are
logically \texttt{OR}ed, \textsl{i.e.} selected elements have to fulfill
\textbf{any} of the selection criteria provided by the different
\texttt{SELECT} statements. 

All selections for a given flag remain valid until a \texttt{SELECT}
statement with the \texttt{CLEAR} argument is specified for the same flag.

Note that because of these composition rules, it is considered good
practice to start by clearing the selection for a given flag before
making a new selection, eg: 
\madxmp{ 
SELECT, FLAG=twiss, CLEAR; \\
SELECT, FLAG=twiss, CLASS=MQ; \\
SELECT, FLAG=twiss, RANGE=MQ[5]/MQ[7]; \\
...
}


\vskip 5mm
\textbf{Examples:} 
\madxmp{ 
SELECT, FLAG = ERROR, CLASS = quadrupole, RANGE = mb[1]/mb[5];\\
SELECT, FLAG = ERROR, PATTERN = "\textasciicircum mqw.*";
}
selects all quadrupoles in the range mb[1] to mb[5], as well as all
elements (in the whole sequence) with name starting with "mqw", for 
treatment by the \hyperref[chap:error]{\texttt{ERROR}} module.  

\vskip 5mm
\madxmp{
SELECT, FLAG=SAVE, CLASS=variable, PATTERN="abc.*"; \\
SAVE, FILE=mysave;
}
saves all variables (and sequences) containing "abc" in their name, 
but does not save elements with names containing "abc" since the class
"variable" does not exist.  

\vskip 5mm
\madxmp{
sig1 := sqrt(beam->ex*table(twiss,betx)); \\
SELECT, FLAG=twiss, COLUMN= \_name, s, betx, ..., sig1; ! or equivalently \\
SELECT, FLAG=twiss, FULL, COLUMN= sig1; ! default columns + new
}
writes the current value of ``sig1'' into the \texttt{TWISS} table each
time a new line is added; Note that the values from the same (current)
line can be are accessed by the variable ``sig1''.
The \hyperref[chap:plot]{\texttt{PLOT}} command also accepts the new variable 
in the table.  

%% EOF


