%%\title{PTC\_EPLACEMENT}

\section{PTC\_EPLACEMENT}

\begin{verbatim}
PTC_EPLACEMENT, 
   range = [s, none],
   x     = [r, 0],    y = [r, 0],    z = [r, 0],
   phi   = [r, 0],    theta = [r, 0], 
   onlyposition    = [l, false, true] ,
   onlyorientation = [l, false, true] ,
   autoplacedownstream = [l, true, true] ,
   refframe = [s, gcs] ; 
\end{verbatim}


Places a given element at required position and orientation.  All
rotations are made around the front face of the element.

{\bf  Command parameters and switches }
\begin{itemize}
   \item {\bf range}=string in range format (Default: ???)\\
     Specifies name of the element to be moved.   

   \item {\bf x,y,z}=real (Default: 0.0)\\
     Coordinate of the front face of the magnet.   

   \item {\bf phi, theta}=real (Default: 0.0)\\
     polar (in xz plane, around z axis) and azimuthal (around x axis)
     rotation angles, respectively.   
     
   \item {\bf refframe}=string (Default: gcs)\\
     Defines the coordinate system with respect to which coordinates and
     angles are specified. \\
     Possible values are:       
     \begin{itemize}
        \item[gcs]  global coordinate system 
        \item[current]   current position
        \item[previouselement]  end face of the previous element 
     \end{itemize}

   \item {\bf onlyposition}=logical (Default: .false.)\\
     If true, only translation are performed and orientation of the
     element is not changed.    

   \item {\bf onlyorientation}=logical (Default: .false.)\\
     If true, only rotations are performed and position of the element
     is not changed.    

   \item {\bf autoplacedownstream}=logical (Default: .true.)\\
     if true all elements downstream are placed at default positions
     with respect to the moved element, \\
     if false the rest of the layout stays untouched.    

   \item {\bf surveyall}=logical  (Default: .true.)\\
     If true, survey of all the line is performed after element
     placement at new position and orientation. It is implemented
     mainly for the software debugging purposes. If patching was
     performed correctly, the global survey should not change anything.     
\end{itemize}

\textbf{Example }\\

\href{http://cern.ch/frs/mad-X_examples/ptc_madx_interface/eplacement/chicane.madx}{Dog
  leg chicane}: postion of quadrupoless is matched to obtain required
R566 value.   


{\bf PROGRAMMER'S MANUAL}

The command is implemented pro\_ptc\_eplacement function in madxn.c and
by subroutine ptc\_eplacement() in madx\_ptc\_eplacement.f90.  

Specified range is resolved with help of get\_range command. Number of
the element in the current sequence is resolved and passed as the
parameter to the fortran routine. It allows to resolve uniquely the
corresponding element in the PTC layout.  

TRANSLATE\_Fibre and ROTATE\_Fibre routines of ptc are employed to place
and orient an element in space. These commands adds rotation and
translation from the current position. Hence, if the specified reference
frame is other then "current", the element firstly needs to be placed at
the center of the reference frame and then it is moved about the user
specified coordinates.   

After element placement at new position and orientation patch needs to
be recomputed. If autoplacedownstream is false then patch to the next
element is also recomputed. Otherwise, the layout is surveyed from the
next element on, what places all the elements downstream with default
position with respect to the moved element.  

At the end all the layout is surveyed, if surveyall flag is true, what
normally should always take place.      


