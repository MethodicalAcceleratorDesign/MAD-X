%%\title{Match using PTC knobs}

\section{PTC\_VARYKNOBS: Matching with PTC knobs}

This matching procedure takes advantage of the parametric results that
are accessible with PTC. Namely, parameters occuring in the matching
constrains are obtained as functions (polynomials) of the matching
variables. In other words, each variable is a knob in PTC
calculation. Evaluation of the polynomials is relatively fast comparing
to the regular PTC calculation which makes findinng the minimum with the
parametrized constraints very fast.  

However, the algorithm is not faster in a general case: 
\begin{enumerate}
   \item  The calculation time dramatically increases with the number of
     parameters and at some point penalty rising from this overcomes the
     gain we get from the fast polynomial evaluation.    
   \item  A parametric result is an approximation that is valid only
     around the nominal parameter values.     
\end{enumerate}

The algorithm is described below. \\
 
\begin{verbatim}
MATCH, use_ptcknobs=true;
...
PTC_VARYKNOB: 
  initial = [s, none] , 
  element = [s, none] , 
  kn    = [i, -1], 
  ks    = [i, -1], 
  exactmatch = [l, true, true], 
  trustrange    = [r, 0.1],  
  step     = [r, 0.0], 
  lower    = [r, -1.e20],
  upper    = [r,  1.e20]; 
...
END_MATCH;
\end{verbatim}

For user convenience the limits are specified in the MAD-X units (k1,k2,
etc). This also applies to dipolar field where the user must specify
limits of k0=angle/path\_lengh. This guarantees concistency in
treatment of normal and skew dipole components.   

Important: Note that inside the code skew magnets are represented only
by  normal component and tilt, so the nominal skew component is always
zero.  Inside PTC tilt can not become a knob, while skew component can.
Remember about this fact when setting the limits of skew components in
the matching.  When the final results are exported back to MAD-X, they
are converted back to the "normal" state, so the nominal skew compoment
is zero and tilt and  normal component are modified accordingly.     

trustrange - defines the range the expansion is trusted \\



{\bf Example}\\
\href{http://cern.ch/frs/mad-X_examples/ptc_madx_interface/matchknobs/.madx}{dog leg chicane}.


{\bf Algorithm}\\
\begin{enumerate}
   \item Buffer the key commands (ptc\_varyknob, constraint,
     ptc\_setswitch, ptc\_twiss or ptc\_normal, etc) appearing between 
     match, useptcknobs=true; and any of matching actions calls
     (migrad,lmdif,jacobian, etc) 
   \item  When matching action appears,  
     \begin{enumerate}
       \item set "The Current Variables Values" (TCVV) to zero      
       \item perform THE LOOP, i.e. points 3-17 
     \end{enumerate}
   \item Prepare PTC environment (ptc\_createuniverse,
     ptc\_createlayout)  
   \item Set the user defined knobs (with ptc\_knob).  
   \item Set TCVV using ptc\_setfieldcomp command.  
   \item Run a PTC command (twiss or normal).  
   \item Run a runtime created script that performs a standard matching;
     all the user defined knobs are variables of this matching.  
   \item Evaluate constraints expressions to get the matching function
     vector (I). 
   \item Add the matched values to TCVV. 
   \item End PTC session (run ptc\_end). 
   \item If the matched values are not close enough to zeroes then goto 3.
   \item Prepare PTC environment (ptc\_createuniverse,
     ptc\_createlayout). 
   \item Set TCVV using ptc\_setfieldcomp command.
     \\   ( --- please note that knobs are not set in this case  -- )  
   \item Run a PTC command (twiss or normal).
   \item Evaluate constraints expressions to get the matching function
     vector (II). 
   \item Evaluate a penalty function that compares matching function
     vectors (I) and (II).\\     See points 7 and 14.
   \item If the matching function vectors are not similar to each other
     within requested precision then goto 3. 
   \item Print TCVV, which are the matched values. 
\end{enumerate}


 
% <h3> PROGRAMMERS MANUAL </h3>
% 
% <p> 
% The command is implemented pro_PTC_SETKNOBVALUE function in madxn.c and 
% 
