%%%\title{Sequence Editor}
%  Changed by: Chris ISELIN, 24-Feb-1998 
%  Changed by: Hans Grote, 17-Jun-2002 

\chapter{Sequence Editor}
\label{chap:seqedit}
With the help of a few commands, an existing sequence may be
modified in many ways: in the case of a circular machine, the starting point of 
the sequence can be moved to another place;
the order of elements can be inverted; elements can be inserted one by
one, or as a whole group with one single command; single elements, or
classes thereof can be removed; elements can be replaced by others;
finally, the sequence can be "flattened", i.e. all inserted sequences
are replaced by their actual elements, such that a flattened sequence
contains only elements. 

It is good practice to add a \hyperref[sec:flatten]{\texttt{FLATTEN}}
statement at the end of a \texttt{SEQEDIT} operation to ensure a fully
operational sequence. This is particularly useful for the 
\hyperref[sec:save]{\texttt{SAVE}} command to properly save
\textit{shared sequences} and to write them out in \madeight format. 


%% 2013-Jul-11  17:35:46  ghislain: need a synopsis paragraph similar to
%% what PTC offers and a complement of a few examples using all commands
%% shown here...

\section{SEQEDIT}
\label{sec:seqedit}
\madx provides an environment for the edition of sequences that is invoked with 
the command:
\madbox{SEQEDIT, SEQUENCE=string;}
The only attribute is the name of the sequence to be edited. 

The editing is performed on the sequence as provided by the user and before it 
is expanded with the \hyperref[sec:use]{\texttt{USE}} command. 
At the end of sequence edition, the resulting sequence must be expanded through 
the \hyperref[sec:use]{\texttt{USE}} command as necessary.

\section{FLATTEN}
\label{sec:flatten}
Sequences can be built from elements but also sub-sequences resulting in a 
nested structure (see chapter \ref{chap:sequence} on sequence definition). 
The positioning of elements within a sequence can also be specified with values 
or expressions, and by reference to other elements.

\madx provides a command to resolve these dependencies and transform a complex 
sequence into a simple list of elements with positioning values referring to 
the start of the sequence, discarding the user-specified expressions for the 
positioning.

This command takes no argument: 
\madbox{
FLATTEN;
}

If the sequence being edited contains sub-sequences, \texttt{FLATTEN}
recursively includes all sub-sequences until the sequence is only
composed of a simple list of elements. 

\texttt{FLATTEN} also resolves the positioning of each element within
the sequence to a single value with reference to the start of the
sequence and updates the \texttt{AT} attribute of that element while
also discarding the user-specified expression if present.

The \texttt{FLATTEN} command is recommended at the beginning of sequence
edition as well as at the very end as in
\madxmp{
SEQEDIT, \=SEQUENCE=name; \\
	 \>FLATTEN; \\
	 \>\ldots commands to edit the named sequence\ldots \\
	 \>FLATTEN; \\
ENDEDIT;
}

\section{CYCLE}
\label{sec:cycle}
\madbox{
CYCLE, START=string;
}
This makes the sequence start at the location given as attribute value
of the \texttt{START} attribute. The element named by the \texttt{START}
attribute must be a marker. \\  
In case there is a shared sequence in the used sequence, the
command \texttt{FLATTEN} should be used before the command
\texttt{CYCLE}. \\ 
Example:  
\madxmp{
FLATTEN; \\
CYCLE, START=place; 
}

Note that the \texttt{FLATTEN} command inserts another marker before the
start location, with a name composed of the name of the sequence being
edited, followed by the start location name and the string "\_P\_". 

\section{REFLECT}
\label{sec:reflect}
\madbox{REFLECT;}
This inverts the order of element in the sequence, starting from the
last element. \\ 
If there are shared sequences inside the
\hyperref[sec:use]{\texttt{USE}}d sequence,  
the command \texttt{FLATTEN} must be used before the command
\texttt{REFLECT}.   
Alternatively each shared sequence must first be reflected. Example:   
\madxmp{
FLATTEN;\\
REFLECT; 
}


\section{INSTALL}
\label{sec:install}
\madbox{
  INSTALL, \= ELEMENT=string, CLASS=string, \\
           \> AT=real, FROM=\{string|SELECTED\};
}
where the parameters have the following meaning: 
\begin{madlist}
   \ttitem{ELEMENT} name of the (new) element to be inserted (mandatory) 

   \ttitem{CLASS} class of the new element to be inserted (mandatory) 

   \ttitem{AT} position where the element is to be inserted; if no "from"
     is given,this is relative to the start of the sequence. If "from"
     is given, it is relative to the position specified there. 

   \ttitem{FROM} either a place (i.e. the name(+occurrence count) of an
     element already existing in the sequence, e.g. mb[15], or
     mq.a..i1..4 etc.; or the string \texttt{SELECTED}; in the latter
     case an element of the type specified will be inserted behind all
     elements in the sequence that are currently selected by one or several
     \hyperref[sec:select]{\texttt{SELECT}} commands of the type 
     \madxmp{SELECT, FLAG=seqedit, CLASS=.., PATTERN=.., RANGE=..;}
\end{madlist}

\textbf{Note:} No element definition can occur inside a \texttt{SEQEDIT
  ... ENDEDIT} block.


\section{MOVE}
\label{sec:move}
\madbox{
MOVE, ELEMENT=\{string|SELECTED\}, BY=real, TO=real, FROM=string;
}
\begin{madlist}
  \ttitem{ELEMENT} name of the existing element to be moved, or
  "SELECTED", in which case all elements from existing
  \hyperref[sec:select]{\texttt{SELECT}} commands will be moved;
  in the latter case, the \texttt{BY} attribute must be given.  
  \ttitem{BY} distance by which the element(s) is/are to be moved; no
  \texttt{TO} or \texttt{FROM} attributes should be given in this case.  
  \ttitem{TO} position to which the element has to be moved; if no
  \texttt{FROM} attribute is given, the position is relative to the
  start of the sequence; otherwise, it is relative to the location
  given in the \texttt{FROM} argument   
  \ttitem{FROM} place in the sequence with respect to which the element
  is to be positioned.  
\end{madlist}

\section{REMOVE}
\label{sec:remove}
\madbox{
REMOVE, ELEMENT=\{string|SELECTED\};
}
\begin{madlist}
  \ttitem{ELEMENT} name of existing element(s) to be removed. \\
  The special case \texttt{ELEMENT = SELECTED} removes all elements
  previously selected by \hyperref[sec:select]{\texttt{SELECT}} commands
\end{madlist}

\textbf{Note:} \madx expects to find some special markers in a beam line and it  
is therefore dangerous to remove all markers from a sequence! In particular the 
\texttt{START=...} marker and markers added by a \hyperref[sec:cycle]{\texttt{CYCLE}} 
command must never be removed from a sequence.


\section{REPLACE}
\label{sec:replace}
\madbox{
REPLACE, ELEMENT=\{string|SELECTED\}, BY=string;
}

The parameters are defined as:
\begin{madlist}
  \ttitem{ELEMENT} names the elements to be replaced. \\
  The special case \texttt{ELEMENT = SELECTED} replaces all elements previously 
  selected by \hyperref[sec:select]{\texttt{SELECT}} commands 
  
  \ttitem{BY} names the elements replacing the elements selected for 
  replacement.
\end{madlist}

\section{EXTRACT}
\label{sec:extract}
A new sequence can be extracted as a subset of an existing sequence. The 
extracted sequence is given a new name and can be
\hyperref[sec:use]{\texttt{USE}}d as any user defined sequence.
\madbox{
  EXTRACT, \= SEQUENCE=string, FROM=string, TO=string, \\
           \> NEWNAME=string;
}

The parameters are defined as:
\begin{madlist}
\ttitem{SEQUENCE} the name of the existing sequence from which the new sequence 
is extracted.

\ttitem{FROM} the name of an element in the sequence that becomes the first 
element of the extracted sequence.

\ttitem{TO} the name of an element in the sequence that becomes the last 
element of the extracted sequence. 

\ttitem{NEWNAME} the name of the extracted sequence. 
\end{madlist}


The extracted sequence is declared as \hyperref[chap:sequence]{\texttt{SHARE}}d and
can therefore be combined \textsl{e.g.} into the cycled original sequence.

\textbf{Note:} the element given by the \texttt{FROM} attribute must be
located, in the existing sequence, before the element given by the
\texttt{TO} attribute, or \madx fails with a fatal error. 
In the case of circular sequences, this can be ensured by performing a 
\hyperref[sec:cycle]{\texttt{CYCLE}} of the original sequence with
\texttt{START} pointing to the same element given in the \texttt{FROM}
attribute of the \texttt{EXTRACT} command. 


\section{ENDEDIT}
\label{sec:endedit}
The sequence editing environment is closed with the command
\madbox{ENDEDIT;}
The nodes in the sequence are finally renumbered according to their occurrence 
which might have changed during editing.




\section{SAVE}
\label{sec:save}

The \texttt{SAVE} command saves a sequence to a specified file together with all 
relevant information.

\madbox{
SAVE, \=SEQUENCE=string{,string}, FILE=filename, \\
      \>BEAM=logical, BARE=logical, MAD8=logical, \\
      \>NOEXPR=logical, NEWNAME=string;
}

The parameters are defined as: 
\begin{madlist}
	\ttitem{SEQUENCE} lists the sequences to be saved, separated by commas. 
	This attribute is optional and when omitted, all known 
	sequences are saved. \\
	However, because of internal inconsistencies that can result in spurious 
	entries in the output file, the user is strongly advised to always provide 
	explicitly the names of sequences to be saved.

	\ttitem{FILE} the filename of the output file. (Default: "save")

	\ttitem{BEAM} an optional flag to specify that all beams linked to the
	specified sequences are saved at the top of the output file.

	\ttitem{BARE} an optional flag to save only the sequence without the
	element definitions nor beam information. This allows to re-read in a
	sequence with might otherwise create a stop of the program. This is
	particularly useful to turn a line into a sequence in order to further edit 
	it with \hyperref[sec:seqedit]{\texttt{SEQEDIT}}.

	\ttitem{MAD8} an optional flag to request that the sequences should be  
	saved using \madeight input format.

	\ttitem{NOEXPR} an optional flag to save values of expressions 
	instead of the expressions themselves: the expressions in commands 
	and variables are expanded and evaluated before saving.
	This option must be used with care because the exported values were not deeply
	checked and the code that writes variables and commands is widely spread
	in the internal structure. \\
	This option does not apply only for the saving of sequences in \madeight format. 

	\ttitem {NEWNAME} provides a name for the saved sequence, overriding the 
	original name. (see \hyperref[sec:extract]{\texttt{EXTRACT}} above)
  
  \ttitem{CSAVE} an option that saves the strength of the corrector magnets used
  for an orbit correction.
\end{madlist}

Any number of \texttt{SELECT, FLAG=save, ...} commands may precede
the \texttt{SAVE} command. In that case, the names of elements, variables, and
sequences must match the pattern(s) if given, and in addition the
elements must be of the class(es) specified. 

%% See here for a
%% \href{../Introduction/select.html#save_select}{SAVE with SELECT}
%% example.  

The precision of the output of the \texttt{SAVE} command depends on the
defined output precision for \madx, which can be adjusted with the 
\hyperref[sec:set]{\texttt{SET, FORMAT...}} command.

%Details about default
%precision and how to adjust it can be found at the
%\href{../Introduction/set.html#Format}{SET Format} instruction page.   
 
Note that with \texttt{BARE=false} the saved sequence may contain redundant 
efinitions of elements, \textsl{i.e.} the same element is defined in the
declaration  of elements in the form \texttt{label:\ type...} and in the
sequence itself in the form \texttt{label:\ type, at=...}. This is
flagged by \madx as implicit element redefinition and is ignored but a
warning is issued.  

Example:  
\begin{verbatim}
tl3: LINE = ( ldl6, qtl301, mqn, qtl301, ldl7, qtl302,
                            mqn, qtl302, ldl8, ison);
dltl3: LINE=(delay, tl3);

Use, period=dltl3;

Save, sequence=dltl3, file=t1, bare; // only sequence is saved

Call, file=t1; // sequence is read in and is now a "real" sequence
// if the two preceding lines are suppressed, seqedit will print a warning
// and else do nothing

Use, period=dltl3;

Twiss, save, betx=bxa, alfx=alfxa, bety=bya, alfy=alfya;

Plot, vaxis=betx, bety, haxis=s, colour:=100;

SEQEDIT, SEQUENCE=dltl3;
  remove,element=cx.bhe0330;
  remove,element=cd.bhe0330;
ENDEDIT;

Use, period=dltl3;
Twiss, save, betx=bxa, alfx=alfxa, bety=bya, alfy=alfya;
\end{verbatim}


\section{DUMPSEQU}
\label{sec:dumpsequ}
\madbox{
DUMPSEQU, SEQUENCE=string, LEVEL=integer;
}
This command is actually more of a debug statement, but it may come handy at certain
occasions. The argument of the \texttt{SEQUENCE} attribute is the name of an
already expanded (i.e. \hyperref[sec:use]{\texttt{USE}}d) sequence. The amount of 
detail in the output is controlled by the \texttt{LEVEL} argument:
\begin{itemize}
\item[$=0$ : ]    print only the cumulative node length = sequence length
\item[$>0$ : ]    print all node (element) names except drifts
\item[$>2$ : ]    print all nodes with their attached parameters
\item[$>3$ : ]    print all nodes, and their elements with all parameters
\end{itemize}

\chapter{Save state}
\label{chap:savestate}  
It is sometimes convientine to save the state of a MAD-X script in order to reload it at a later occation or in order to distribute it to another users. 
The \texttt{SAVE\_STATE} command provide the user with the possiblity to save a sequence, the beam information, the errors, and the macros in a single command. 
A MAD-X file is also created in order to load the produced files. The files are saved in a separate folder and in hexadecimal format in order not to lose
any information in the saving.

\madbox{
SAVE\_STATE, \=SEQUENCE=string{,string}, FILE=filenames, \\
      \>BEAM=logical, FOLDER=foldername, csave=logical ;
}

The parameters are defined as: 
\begin{madlist}
  \ttitem{SEQUENCE} lists the sequences to be saved, separated by commas. 
  This attribute is optional and when omitted, all known 
  sequences are saved. \\
  However, because of internal inconsistencies that can result in spurious 
  entries in the output file, the user is strongly advised to always provide 
  explicitly the names of sequences to be saved.

  \ttitem{FILE} the starting of the filenames the output files will have. (Default: "save")

  \ttitem{BEAM} an optional flag to specify that all beams linked to the
  specified sequences are saved at the top of the output file. (Default=TRUE)

  \ttitem{FOLDER} the name of the folder where the files are saved (Default:"save\_state")

  \ttitem{CSAVE} an option that saves the strength of the corrector magnets used
  for an orbit correction (Default=TRUE). 

\end{madlist}




%% EOF
