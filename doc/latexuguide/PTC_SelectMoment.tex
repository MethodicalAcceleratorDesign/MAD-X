%%\title{PTC\_SELECT\_MOMENT}

\section{PTC\_SELECT\_MOMENT}

\begin{verbatim}
PTC_SELECT_MOMENT, 
   table      = [s, none, none], 
   column     = [s, none, none], 
   moment_s   = [s, none] , 
   moment     = [i, {0}] , 
   parametric = [l, false, true], 
\end{verbatim}

Selects a moment to be: 
\begin{enumerate}
  \item {\bf Stored in a user specified table and column.}
    \textit{Table} and \textit{column} must be specified then, and such
    table with such column must exists.  

  \item {\bf Stored as a function (taylor series) of
    \href{PTC_Knob.html}{knobs}, if any is defined.} Then,
    \textit{parametric} should be set to true. 
\end{enumerate}
Both cases can be joined in one command.   


{\bf Command parameters and switches}
\begin{itemize}
   \item {\bf moment\_s}=list of coma separated strings composed of up
     to 6 digits (Default: ???)\\
     Defines moment of the polynomial in PTC nomenclature. String 'ijklmn'
     (where i,j,k,l,m,n are digits from 0 to 9) defines
     $<$x$^i$p$_x$$^j$y$^k$p$_y$$^l$ $\Delta$T$^m$($\Delta$p/p)$^n$$>$. 
     \\ For example, moment\_s=100000 defines $<$x$^1$$>$ 

     Note that for input we always use MAD-X notation where dp/p is always
     the 6th coordinate. Internally to PTC, dp/p is the 5th coordinate. We
     perform automatic conversion that is transparent for the user. As the
     consequence RMS in dp/p is always defined as 000002, even in 5D case.    

     This notations allows to define more then one moment with one
     command. In this case, the corresponding column names are as the passed
     strings with "mu" prefix. However, they are always extended to 6 digits,
     i.e. the trailing 0 are automatically added. For example, if specified
     moment\_s=2, the column name is mu200000.

     This method does not allow to pass bigger numbers then 9. If you need to
     define such a moment, use the command switch below.    
     
   \item {\bf moment}=list of up to 6 coma separated integers (Default: ???)\\
     Defines a moment. For example: moment=2 defines $<$x$^2$$>$ ,
     moment=0,0,2 : $<$y$^2$$>$, moment=0,14,0,2 : $<$px$^{14}$py$^2$$>$,
     etc. 

   \item {\bf table}=string (Default: moments)\\
     Specifies the name of the table where the calculated moments are stored.   

   \item {\bf column}=string (Default: ???)\\
     Ignored if \textit{ moments } is specified. Defines name of the
     column where values should be stored. If not specified then it
     is automatically generated from moment the definition
     $<$x$^i$p$_x$$^j$y$^k$p$_y$$^l$
     $\Delta$T$^m$($\Delta$p/p)$^n$$>$ =$>$ mu\_i\_j\_k\_l\_m\_n
     (numbers separated with underscores).                  

   \item {\bf parametric}=logical (Default: .false.)\\
     If it is true, and any \href{PTC_Knob.html}{knobs} are defined the map
     element is stored as the parametric result.             
\end{itemize}

{\bf Examples}\\
\href{http://cern.ch/frs/mad-X_examples/ptc_madx_interface/moments/moments.madx}{ATF2}

 
% <h3> PROGRAMMERS MANUAL </h3>
% 
% <p> 
% The command is implemented pro_ptc_SELECT function in madxn.c and 
% by subroutine xxxx in madx_ptc_xxx.f90.
% <p>
% Sopecified range is resolved with help of get_range command. Number of the element in the current sequence
% is resolved and passed as the parameter to the fortran routine. It allows to resolve uniquely the corresponding
% element in the PTC layout.
% <p>
