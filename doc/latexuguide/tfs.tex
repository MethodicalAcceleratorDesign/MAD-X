%%\title{the mad program}
%  Changed by: Chris ISELIN, 27-Jan-1997 
%  Changed by: Hans Grote, 10-Jun-2002 
%\href{http://www.cern.ch/Hans.Grote/hansg_sign.html}{hansg}, June 17, 2002 

\chapter{TFS File Format}
\label{chap:tfs}

\texttt{TFS}\cite{TFS} is a an acronym for the ``Table File
System''. \texttt{TFS} files have been used in the LEP control
system. The \mad program knows only coded \texttt{TFS} files. The
\texttt{TFS} format has been chosen for all table output of
\madx. \texttt{TFS} formatted tables can be read back into \madx, and
may then be further processed. 
 
\section{Descriptor Lines}
\label{sec:tfs_desc}

\madx writes the following descriptors in all tables: 
\begin{itemize}
   \item COMMENT: The current title string from the most recent 
   \hyperref[sec:title]{\tt TITLE} command. 
   \item ORIGIN: The version of \madx used. 
   \item DATE: The date of the \madx run. 
   \item TIME: The wall clock time of the \madx run. 
   \item TYPE: The type of the table: e.g. {\tt TWISS} 
\end{itemize} 

Additional descriptors exist in the \href{twiss_desc.html}{Twiss table},
as well as the \href{tables.html#track}{Track tables}.  



\section{Column Formats}
\label{sec:tfs_columns}

The column formats used are listed below: 
%in the  \hyperlink{table}{TFS columns table}.   

\begin{table}[ht]
  \begin{center}
    \caption{ Column Formats used in TFS Tables}    
    \vspace{1ex}
    \begin{tabular}{|l l l|}
      \hline
      \textbf{C format} & \textbf{Meaning} & \textbf{C format} \\ 
      \hline
      \%hd & Short integer & (\%8d) \\ 
      \%le & Long float & (\%-18.10g) \\ 
      \%ks & String of length k & ("$\backslash$"-18s$\backslash$"")\\
      \hline
   \end{tabular}
  \end{center}
\end{table}

Control lines begin with the TFS control character, followed by a
blank. Data lines begin with two blanks. Columns are also separated by
one blank character. The column width is chosen such as to accommodate
the largest of the column name and the width of data values of the column.  



\section{Twiss TFS file header}
\label{sec:tfs_twiss}
 
The format of the twiss table is best illustrated with an
\href{select.html#tfs}{TFS file example}.  

\madx gives access to parameters from {\tt TWISS} and other tables using the
\href{../Introduction/expression.html#table}{table access} function.  

%% EOF

